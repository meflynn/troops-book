\chapter{The Good, Bad, and Ugly American: Variation in Global Anti-US Military Base Protests\index{protest} \label{cha:protest}}

\doublespacing




%\section*{Introduction}
\vspace*{-0.5cm}
\rule{\linewidth}{0.10pt} \\[-1cm]
{\footnotesize\paragraph{Summary:}  Previous chapters examined how military deployments affect beliefs and attitudes. This chapter turns to focus on individuals' behavior. US military deployments have long been cited as causing negative externalities in host countries. These negative events may help to mobilize\index{mobilization} opposition to the US presence. Drawing on new country-level protest\index{protest} data and individual-level survey\index{survey} data, our analyses yield several important findings. First, larger US troop deployments cause more frequent anti-US protest\index{protest} events. Second, our models of individual behavior correctly classify more than 90\% of survey\index{survey} respondents' involvement in anti-US protest\index{protest} activities. These models show that individuals' attitudes and experiences---not simple demographic traits---offer the strongest predictive power in determining who participates in anti-US protest\index{protest} events. Finally, crime\index{crime} victimization, in particular, is a very strong predictor of protest\index{protest} involvement.} 
\\[-0.5cm] 
\rule{\linewidth}{0.10pt}

\vspace*{0.5cm}

In the middle of an extraordinary July heat wave sweeping through continental Europe\index{Europe} in 2019, while locals sought refuge from 40 degrees Celsius temperatures in the few air-conditioned restaurants they could find, we sat in a small office in a nondescript building in Berlin's\index{Germany!Berlin} Mitte\index{Germany!Mitte district} district. ``It's a long story,'' began the German\index{German} peace activist\index{activists}, as he leaned back on his chair with two young interns looking on. 

In the interview\index{interview} that followed, this particular activist\index{activists}, whose organization focuses broadly on expanding the peace movement in Germany\index{Germany} and specifically on protesting\index{protest} against the US Air Force's\index{Air Force, US} Ramstein Air Base\index{Germany!Ramstein Air Base}, explained the history of the German\index{German} protest\index{protest} movement. He discussed how the peace movement peaked in the 1980s and how, afterward, the German\index{German} population did not place as much importance on bases, leading to frustration by local activists\index{activists}. 2005 marked a turning point, as the activist\index{activists} launched a new appeal against Ramstein Air Base\index{Germany!Ramstein Air Base}. Before we could ask, ``Why Ramstein\index{Germany!Ramstein Air Base}?'' (the US, after all, has 87 military facilities in Germany\index{Germany}),\cite{DOD2018} he volunteered the information. Ramstein\index{Germany!Ramstein Air Base} has one specific characteristic that separates it from the others: It is the only air base in Germany\index{Germany} from which military personnel operate remotely piloted aircraft (RPAs\index{Remotely Piloted Aircraft} or drones\index{Remotely Piloted Aircraft}). As the activist\index{activists} noted, it is a key point to transmit the signals from Nevada\index{Nevada} and New Mexico\index{New Mexico} for the drones\index{Remotely Piloted Aircraft} to ``go elsewhere.''\cite{berlinone20190723} 

Like many others we conducted, this interview\index{interview} illustrated how the presence of foreign military personnel is not sufficient to cause host-country citizen grievances against US overseas bases. Mobilizing\index{mobilization} against Ramstein\index{Germany!Ramstein Air Base} is effective because of RPA\index{Remotely Piloted Aircraft} use and not necessarily because of animosity towards the service members themselves. In the case of this activist\index{activists} group, they did not want Germany\index{Germany} to be morally complicit with the US military's drone\index{Remotely Piloted Aircraft} strikes in Central Asia\index{Central Asia}. The actions the military takes at Ramstein Air Base\index{Germany!Ramstein Air Base} (or the belief about what actions take place there) and the resulting activist\index{activists} focus on the base for those actions suggest that the particular behavior at specific bases may motivate opposition to the presence of the US military. 

In 2019, opposition to Ramstein's\index{Germany!Ramstein Air Base} activities escalated from domestic protests\index{protest} to a set of legal challenges in the German\index{German} court system.\cite{Kloeckner2019,Reuters2019} The specificity of the opposition implies variation in the number of protests\index{protest} or mobilization\index{mobilization} that a base can generate. Knowing what conditions encourage activists\index{activists} and others to protest\index{protest} the United States and its military is important to understanding opposition mobilization\index{mobilization}.  Protests\index{protest} against Ramstein\index{Germany!Ramstein Air Base} center around both militarism and drone\index{Remotely Piloted Aircraft} usage. Protesters\index{protest} at Henoko-Oura Bay\index{Japan!Henoko-Oura Bay} in Japan\index{Japan} (in Okinawa\index{Okinawa!Japan}) protest\index{protest} against Ospreys\index{Osprey}, US occupation, offenses by US marines\index{Marines, US}, and the environmental destruction caused by new base construction (especially its effects on the coral reef and the dugong population).\cite{Hibbett2019} At Osan Air Base\index{Japan!Osan Air Base} and Camp Humphreys\index{Camp Humphreys} in South Korea\index{South Korea}, protesters\index{protest} organized against racial injustice and for Black Lives Matter\index{Black Lives Matter} in the United States.\cite{Sisk2020}
		
		A major aim of the German\index{German} activist's\index{activists} peace organization is to disseminate information about the US military bases and their activities to the German\index{German} (and international) public. They expect that as German\index{German} citizens become more aware of the negative effects of the US presence (such as air pollution from jet exhaust) and what Germans\index{German} implicitly condone by hosting the US military, they will become more willing to mobilize\index{mobilization} against it. The group believes that their movement is growing. When we interviewed\index{interview} them, they told us that they had achieved record participation levels that they expected to continue to grow. The German\index{German} activist\index{activists} fondly recalled how, during one year, protesters\index{protest} formed a human chain around the base. In the summer of 2019, the annual protest\index{protest} at Ramstein\index{Germany!Ramstein Air Base} had 5,000 attendees and 50 different workshops and events.  
		
		It is easy to see how a US military base could lead to negative reactions from locals and how larger bases can create more opportunities for negative interactions. Military jets are noisy and pollute the air through high emissions. Fuel waste that seeps into the soil can contaminate local drinking water supplies. US service members sometimes get into fights at local bars or drive drunk.\cite{kasernetwo20190725} An American government\index{government} relations officer at a US base in Germany\index{Germany} corroborated those last two points; when we drove into the base, there was a wrecked car on exhibit near the entrance with a sign warning US service members not to drink and drive. In the UK\index{United Kingdom}, we heard consistent complaints that bases were noisy and worsened traffic\index{traffic}.
		Given that many of a base's negative externalities occur in its immediate proximity, we might expect that it would be those located close to a military installation who would react the most negatively and be most active in opposing the base. But this is not always the case. In comparing the United Kingdom\index{United Kingdom} and Germany\index{Germany} more broadly, it is clear that the amount of protest\index{protest} in each country and the specific issues that trigger them are different. 
		
		This variation presents us with various questions. Why, despite having very similar histories with US military basing, are these two countries so different? At the macro-level, what makes Germany\index{Germany} more or less likely to experience protests\index{protest} than the United Kingdom\index{United Kingdom}? Or, more broadly, what factors make anti-US protest\index{protest} events more likely in some countries and less likely in others? And at the micro-level, does a German\index{German} have a higher or lower threshold of willingness to join a protest\index{protest} to express their dissatisfaction than a Briton\index{British} does? What factors make some individuals more or less likely to participate in protests\index{protest} against the US military?
		
		In our interview\index{interview} with the peace activist\index{activists}, we asked whether he would contact activists\index{activists} in local organizations found in other towns near military bases. In what we found to be a very honest and self-aware response, he noted that in many cases, activists\index{activists} have struggled to build local peace movements because of the deep and long-lasting relationship between the US military and communities close to US bases. Locals simply are less willing to protest\index{protest} the US military. Some of it, of course, relates to the local population's economic dependence on the US military (``bakers, auto shops, and prostitution'' were the examples of businesses reliant on the US military given by the activist\index{activists}). Those who rent housing to US members benefit heavily from the presence. The housing allowance to service members appears to be common knowledge and tends to be more generous than most locals can afford. Local rental prices around military bases tend to match what soldiers can afford and do not reflect local wages. This makes bases popular with property owners. Even beyond economics, the activist\index{activists} noted that the US military and local governments\index{government} would hold common parties, public events, and ``friendship meetings'' to build community relations. ``It makes it not easy [to establish a local protest\index{protest} movement],'' he said. 
		
		
		
		To an outside observer, it may seem puzzling that those the US military presence affects most directly are also the ones least likely to protest\index{protest} it. Yet, Germany\index{Germany} is not an isolated case. Fitz-Henry finds that in the case of the US military base in Manta\index{Ecuador!Manta}, Ecuador\index{Ecuador}, the protest\index{protest} movement was based in the capital city of Quito\index{Ecuador!Quito}, while locals in Manta\index{Ecuador!Manta} reacted not against the base, but against the \textit{activists}.\cite{Fitz2015} Locals may complain about inconveniences when pressed, but those concerns alone are not enough for them to mobilize\index{mobilization} against basing. Minor inconveniences may pale in comparison to the economic benefits of well-payed Americans patronizing local establishments.
		
		
		
		Our argument in Chapter \ref{cha:meth} noted that both economic benefits (such as the military patronizing auto shops) and contact (like the ``friendship meetings'') could create more positive beliefs about the US military. It is not surprising that those closest to the base, those most likely to have contact and receive economic benefits, are also less likely to protest\index{protest}. Yet, it stands out that we also find a significant negative effect of contact on respondent beliefs about US actors. We have argued that while most interactions with the US military tend to be positive  (and somewhat casual), those negative interactions that do occur, particularly ones that harm the local individual in some way, can also create or reinforce negative perceptions, and mobilize\index{mobilization} individuals against the US military. This chapter extends our analysis further to understand another vector of negative perceptions and the conditions that can change negative perceptions into political action. Negative perceptions, or grievances generally, are not sufficient for anti-base or anti-US activism. This chapter's contribution, exploring the link between perceptions and actions, is fundamentally important to academics and policymakers. This chapter explores both protest\index{protest} mobilization\index{mobilization} at the macro-level (``Which countries are more likely to experience protest\index{protest}?'') and the micro-level (``What makes an individual more likely to protest\index{protest}?'') to see how negative experiences and negative perceptions may, or may not, manifest in political behavior. 
		
		%%State importance of research question
		Even if the negative effect of contact with the military on perceptions is smaller than the positive one, it does not require a majority of members of a population to be opposed to a military base to effectively mobilize\index{mobilization}, particularly if those who feel positively about the base have only weakly positive feelings. Activists\index{activists}, those people willing to invest time and effort into organizing for a given cause, are usually in the minority\index{minority} in most societies.\cite{Burstein2002}  This is true of most anti-US base activists\index{activists} as well.\cite{Fitz2015}  At the same time, because activists\index{activists}, similar to lobbyists, care so much about a particular cause that others do not have as strong preferences over, they are willing to expend large amounts of effort and are more willing to incur costs than others would be.  Because of this, activists\index{activists} can obtain their preferred policy outcomes even when they are in the minority\index{minority}. As articulated in the ``3.5 percent rule'' developed by Chenoweth, simply having 3.5 percent of a state's population involved in active, non-violent protest\index{protest} can be enough to lead to a successful outcome and achieve the protesters'\index{protest} aims.\cite{Chenoweth2011} 
		
		
		%%going to link this to the theory chapter. I think we can build up a pretty good argument based on the protest literature that mobilization and perceptions do matter to the regime. 
		%\ref{cha:theory}
		A major point that we made in Chapter \ref{cha:theory} was that public opinion\index{public opinion} matters to regimes and that perception of the US military in host countries could indeed influence the stability of the hierarchical\index{hierarchy} relationship between the host country and the United States. Research shows that protest\index{protest} and mobilization\index{mobilization} can influence political actors at the national level to alter policy actions, even when such actions are controversial, such as granting more rights to minority\index{minority} groups.\cite{Gillion2013,Fassiotto2017} In particular, stronger and clearer expressions of public opinion\index{public opinion} are more likely to influence policy.\cite{Baumgartner2015,Fassiotto2017} When governments\index{government} make decisions on policy, they face an overwhelming amount of information, some of it gathered by the government\index{government} itself (through intelligence agencies, for example). At the same time, journalists or activists\index{activists} can directly transmit some of that information to the government\index{government}.\cite[p. 15]{Baumgartner2015} Protesting\index{protest} is a way in which the public broadcasts its preferences to the government\index{government}, thus influencing their policy choices. 
		
		
		Even though some have discounted the role that public opinion\index{public opinion} plays in leaders' policymaking decisions, existing work shows that leaders care about public opinion\index{public opinion} and are hesitant to engage in policy actions that go against it, fearing potential political costs.\cite{Tomz2018} While there is variation in how responsive different types of governments\index{government} will be to their population's preferences, it is still true that protests\index{protest} are one way in which people communicate information to governments\index{government}.\footnote{We also note that protests\index{protest} have influenced even non-democratic\index{democracy} regimes in leading to policy change. The Arab Spring\index{Arab Spring} protests\index{protest} were an example of a case in which non-violent protest\index{protest} could achieve concessions, if not outright regime change in most circumstances. \cite[See:][]{Chenoweth2013}} Anti-base protests\index{protest} can be an example of how even small groups of activists\index{activists} can obtain their desired outcome (removal of a base, for example) through organization and effective mobilization\index{mobilization}.\cite{Cooley2008}  Even if anti-base protests\index{protest} are rare, it is essential to understand their determinants, as these protests\index{protest} can have a considerable influence on US foreign policy.  This chapter asks what situations are more likely to lead to local populations protesting\index{protest} a US military presence. Likewise, as shown by the case of Ramstein\index{Germany!Ramstein Air Base} in Germany\index{Germany}, this chapter highlights the idea that the United States and its military can take actions that increase or decrease the likelihood of mobilization\index{mobilization} against its presence. If we understand what causes protests\index{protest} to emerge, we may understand what policy options reduce the probability of overseas opposition to peacetime deployments.  
		
		%%Preview our argument (we probably need to add a bit more here on the specific argument once we nail it down a bit more)
		
		Just as crucial as studying when anti-US base protests\index{protest} occur is investigating cases in which they do not occur.  As noted by Fitz-Henry, there are indeed cases in which communities close to US military installations do not only not protest\index{protest} but also actively mobilize\index{mobilization} in favor of the US base.\cite{Fitz2015} South Korea provides examples of both pro- and anti-US military presence mobilizations\index{mobilization} that battle over the positive and negative effects of the bases. Particularly of interest are cases in which local communities are less likely to protest\index{protest} the US military presence than individuals in other parts of the country.  This fits with our argument that individuals who have more direct contact with deployed personnel are more likely to support the US military presence, even in the face of other domestic opposition.  Alternatively (or in addition to this previous explanation), the US may choose to locate some of its military installations in areas whose geographic or demographic characteristics are less likely to mobilize\index{mobilization} opposition into protest\index{protest}. The US may base in more remote locations far away from the urban centers and the populations within them that are more likely to mobilize\index{mobilization}. 
		
		
		%I don't think we're actually studying counter-protests (which would be interesting, but outside our scope, so cutting this out. CM. 
		%Mobilization of any sort is not without costs, so it is especially interesting to also consider what makes people to mobilize against anti-US protests.
		
		%%More macro implications
		
		%%Moving this paragraph into main theory chapter
		%While negative views of the US by host country publics are bad in and of themselves; they are also problematic for US foreign policy. As individuals' negative views towards a US military presence deepen, they will be more likely to mobilize\index{mobilization} (through voting, social media posts, legal action, or, as we explore in this chapter, protest) against the military presence. As we have previously noted, acts of dissent, like protest, can impose costs on host country leadership. In turn, the government\index{government} may try to pass those costs on to the United States as a condition for continuing to host the troops. Popular opinion mobilization\index{mobilization} has removed United States forces from the Philippines and Spain\index{Spain} and limited the functional operation of the United States Air Force in Turkey\index{Turkey} during the build-up to the 2003 Iraq War\index{Iraq War} \cite{cooley2008,Kakizaki2011}. Sustained opposition fed by grievances fundamentally weakens the US position and its ability to maintain its troop presence or at least makes it more costly.
		
		We note the importance of studying specifically what the determinants of both \textit {anti-US} and \textit{anti-US base} protests\index{protest} are, both at the individual and national level. As we discussed in earlier chapters, we need to understand better the degree to which anti-military sentiment affects attitudes and behaviors vis-\'{a}-vis the United States more generally, and vice versa. While the relationship between US military deployments and host-state populations is complex, basic empirical and causal questions still require systematic examination. In particular, we seek to answer the most general question of whether an increase in the number of deployed US military personnel increases the probability of seeing these types of protests\index{protest}. We have previously discussed that deployments and interactions between host-state residents and US personnel can have mixed effects. But we are attempting to assess what the ``net'' effect might remain a challenge. In the following sections, we pursue multiple research goals. First, we look to see a clear causal relationship between the size of a US military presence and protest\index{protest} events against the US and its military personnel. Second, we then examine the characteristics of both countries and individuals that make protest\index{protest} most likely. In general, we expect that a larger military presence will make protests\index{protest} more likely. We also expect that several individual-level characteristics make participation in protests\index{protest} more or less likely and that these factors may vary across countries and regions. 
		
		%This seems to be less of a focus now, so I greyed it out. CM. 
		%It is certainly the case that some areas of the world, or certain subnational areas within countries, are more prone to protest than others.  We are therefore interested in cases of citizens that choose not to protest US bases but do protest other policies, or alternatively, citizens who do not protest other policies but do choose to protest US bases.  Consequently, we study US military bases in a broader context of all protests in our temporal domain, regardless of the target of the protest. 
		
		%%Outline the chapter
		
		This chapter will discuss research on protest\index{protest} and mobilization\index{mobilization} in general and anti-US protests\index{protest} in particular. We then use a cost-benefit analysis\index{cost-benefit analysis} framework to develop theoretical expectations on the determinants of protest\index{protest} at both the state and individual levels, focusing on the causal path that leads from military deployments to protest\index{protest}. Following, we set up our models for our newly collected protest\index{protest} data and the survey\index{survey} data we used throughout the book. We then use both sets of data to create two types of models. The first model examines the causal relationship between troop deployments and protests\index{protest} at the macro-level. The second model uses a predictive model to see how individual characteristics and relationships with the US military may inform us about people's likelihood to attend protest\index{protest} events. We conclude with policy implications focused on the actions that the US military can take to reduce anti-base protests\index{protest} and the actions that activist\index{activists} groups can take to broaden the appeal of their message to local populations. 
		
		\section*{Research on Anti-US Protests\index{protest}}
		
		To better understand protests\index{protest} against the US and its military facilities, we take two different tracks to disentangle the empirical relationships that appear to be either causal of or related to protest\index{protest} behavior. First, we examine protests\index{protest} through a macroscopic lens by considering the likelihood that a country experiences protests\index{protest} in a given year. This allows us to identify structural conditions that make protests\index{protest} more likely. Second, we return to our survey\index{survey} data and study the conditions that correlate with protest\index{protest} behavior at the individual level. What do our surveys\index{survey} say about individual behavior? Are there demographic, ideological, or geographic factors associated with an individual attending an anti-US protest\index{protest}? What predicts the likelihood of an individual engaging in demonstrations against the United States or military bases? To better understand both sets of frames, macro and micro, we first turn to the established literature.
		
		
		\subsection*{Macro-Behavior Theoretical Expectations}
		When explaining anti-US protests\index{protest}, we first discuss the determinants of protest\index{protest} and mobilization\index{mobilization} in general, which are likely to be drivers of anti-US protests\index{protest}.  We conceptualize protest\index{protest} as a form of collective action\index{collective action}.  Like other forms of collective action\index{collective action}, such as rebellion, protest\index{protest} can run into free-riding problems.\cite{Olson1965,Lohmann1993}  If others protest\index{protest}, a given individual can reap the benefits of the protest\index{protest} (policy change, for example) without incurring any of the costs (lost time, possible arrest, suffering violence, etc.).  Much like rebellion, or even non-violent actions such as voting, a single individual is unlikely to make a difference in the probability of a movement succeeding. Still, if enough individuals follow the same logic and attempt to free-ride off others' participation, the movement can fail.\cite{lichbach1993}  Therefore, we focus on the type of situations that make collective action\index{collective action}, in this case, anti-US protests\index{protest}, more likely to take place.  
		
		One strand of explanations of collective action\index{collective action} and mobilization\index{mobilization} focuses on relative deprivation; Gurr provided much of the foundational work on relative deprivation in the early literature on the causes of civil war\index{civil war}.\cite{Gurr1968}  The idea behind this theory is that the societies that are most vulnerable to domestic political violence are not those that have the poorest people in the world, but those in which individuals are relatively worse off when compared to others within their society. When individuals perceive that they are not receiving the benefits they believe they deserve (as they can observe others in their society receiving), they will be more likely to experience anger\index{anger}. This anger\index{anger} becomes the basis for their mobilization into action.  In other words, if there is a dissonance between what people expect and what people receive (more colloquially, those that have and those that have-not), individuals will be more likely to overcome collective action\index{collective action} problems and mobilize\index{mobilization} into action. 
		
		In the case of anti-base protests\index{protest}, we should expect more anti-US protests\index{protest} in cases where the population is expected to receive greater benefits from a US military presence. Instead, the negative externalities have outweighed the benefits. This situation will be even more at risk for protests\index{protest} when there is a separation between those that receive the benefits of military deployments (such as landlords, business owners, and people in urban centers) and those that bear the burden of the costs of deployment (such as marginalized peoples, rural communities, and minority\index{minority} groups). We can thus think of the population's decision calculus in determining whether to participate in protests\index{protest} as a cost-benefit analysis\index{cost-benefit analysis}. They weigh the benefits that they obtain from the US military presence against the costs associated with it. The greater the costs are relative to the benefits, the more likely we are to observe populations mobilizing\index{mobilization} to attempt to influence the host government\index{government} to remove the US military. 
		
		There is likely variation in the specific factors or behaviors members of the host population are likely to find objectionable. That said, we argue that the larger the US military presence is, all else being equal, the greater the \textit{opportunity} for locals to have grievances about the US military deployments. More specifically, larger deployments should lead to increases in many specific factors that appear likely to stoke anti-base sentiments, such as crime\index{crime}, environmental degradation, increased traffic\index{traffic}, noise pollution\index{noise pollution}, industrial accidents, and more. Accordingly, we begin by exploring the simple question of whether or not larger US military deployments cause an increase in protest\index{protest} activity. Simply put, assuming that some proportion of US military members will engage in objectionable behavior, having a greater number of forces present increases the probability of negative interactions that can mobilize\index{mobilization} the population into protest\index{protest}.\cite{allenandflynn2013}
		
		\begin{hyp}
			All else being equal, larger US military deployments in a country should lead to a higher likelihood of anti-US protests\index{protest} within that country.
		\end{hyp}
		
		This will be our primary motivating hypothesis for our initial set of country-level models. Several other factors will influence protests\index{protest} as well, and we include them in our models. Though we do not offer explicit hypotheses about these factors, it is important that we take them into account to adjust for their influence and accurately predict instances of protest\index{protest}. While we expect troop increases to lead to an increase in protest\index{protest}, the environment in which troops operate will moderate the size of this effect. This leads us to expect variations in how likely different countries are to protest\index{protest} a US military presence.  
		
		%There is, of course, variation in how likely populations are to protest a US military presence, regardless of the size of the deployment. 
		
		When considering what makes countries more likely to experience a protest\index{protest} against the US military presence in a cost-benefit analysis\index{cost-benefit analysis} context, we can think about the kinds of settings in which the benefits of the military presence accrue to all members of the population, not just a specific group (such as supporters of a specific government\index{government}). The cases in which benefits accrue to (almost) all members of the population are less likely to experience protest\index{protest}. Even if people associate costs with the military presence, the benefits of the presence can offset the costs.\cite{Bitar2016}
		
		We thus argue that when the military provides public goods, the entire population will benefit from them.  When the goods supplied by the military presence are private, the host state government\index{government} will distribute these benefits to only its supporters.\cite{demesquita2005} In the setting of military deployments, we can think of the economic benefits that a military presence brings with it (such as contracts given to local contractors) as private goods and the security provided by the troops (such as deterring attacks against the host country) as public goods. We argue that in those states in which the military provides security to the population, we will observe fewer protests\index{protest}, as security, unlike economic benefits, tends to be a public good that everyone, not just government\index{government} supporters, can benefit from. In theory, any country with a US military presence receives security, but the utility to the population can vary. In particular, we argue that in countries experiencing internal or external threats; the general population will derive more utility from the US military presence. Thus, we expect that states facing higher levels of internal or external threats will be less likely to experience anti-US protests\index{protest}.
		
		The literature on the ``rally round the flag'' effect also supports this expectation. States facing internal threats (like terrorism\index{terrorism}) or external ones (like aggression from a foreign rival) garner bumps in leader credibility and unification of opposition parties with the governing party.\cite{Chowanietz2011} Internally, public opinion\index{public opinion} shifts in favor of executives when a new crisis emerges.\cite{lee1977,Norrander1993}  Researchers have not deeply delved into whether support for alliances increases when countries find themselves under threat, but there is evidence host states tend to match US expenditures in their territory when they face a collective threat.\cite{allenetal2016} Regardless, if the argument is generalizable, support for the government\index{government} in the face of a security threat should increase support for the state's security commitments as well.
		
		%\begin{hyp}
			%All else being equal, anti-US base protests are less likely to occur when there are high levels of external threat against the host country.
			%\end{hyp}
		
		
		
		Related to the idea of relative deprivation, when people are doing better economically than they expected to, they should be less likely to want to engage in mobilization\index{mobilization}. For example, a State Department\index{State Department} Regional Analyst\index{regional analyst} at the US embassy\index{embassy} in Panama\index{Panama} noted that locals felt positively towards the US because of the strong Panamanian\index{Panama} economy and low levels of unemployment that they attributed in part to US investment in Panama\index{Panama}.\cite{embthree20180712} In contrast, when people are experiencing economic hardship, they might become dissatisfied with the American military presence.\cite[Though we note that we may see less mobilization\index{mobilization} at extreme levels of economic downturn, which would lead to extreme poverty. As pointed out by an interview subject, ``When you have to fight for your day to day surviving \text{[}sic\text{]}, you cannot be active \text{[}against bases\text{]}.''][]{berlinone20190723}  This may be due to individuals believing that they, in particular, should benefit economically from the base but are not able to gain that sought-after benefit while seeing others enjoying their desired economic fortune.  If they are not receiving such benefits, anger\index{anger} may manifest from dissatisfaction, and protesting\index{protest} becomes an outlet for that dissatisfaction. We thus expect an economic downturn in a country to correlate with more anti-US protests\index{protest}.
		
		%%expand on this and cite more stuff here. CM.
		
		
		%\begin{hyp}
			%All else being equal, anti-US base protests are more likely to occur when there is economic downturn in the host country.
			%\end{hyp}
		
		%Work by \citeasnoun{Yeo2011} also shows that successful anti-base mobilization efforts are produced by two things. First, the country must have a fractured elite security consensus. This fracture allows a protest\index{protest} movement "air to breathe" without being completely denied by the entirety of the country's establishment. Since much of the public will take their cues from elites, many people will avoid protests\index{protest} in the presence of an elite structure that stands united behind the idea that the country shares security threats with the United States, that the country needs help from the US, and that American forces are present for these purposes. The opposite is also true, in the presence of prominent elites who question such tenets, individuals will be more likely to question the presence of American forces and join protest\index{protest} movements. 
		
		%Second, Yeo explains that successful protest movements include broad sections of the host state's society. This includes variation across ethnicity\index{ethnic!minority}, gender, income, region, and more. Protest movements that are confined to single demographics are unlikely to succeed, as there is less incentive for host governments to respond. When host governments are faced with broad coalitions of anti-base protesters, their political survival is at stake without being able to lean heavily on other demographics. It also signals the importance of the issue, in that the anti-base grievances are not confined to a single group. 
		
		%While we do not purport to explain the success of protest movements, theories from \citeasnoun{Yeo2011} help us make predictions about the propensity of protests. Since successful protest movements are likely to be larger and last longer, and since successful protests stem from fractures in the security consensus and broad-based support within protests movements, we can make predictions about the likelihood of protests. 
		%** NOTE: These hypotheses seem endogenous at best. Thoughts? - AS **
		
		%\begin{hyp}
			%Higher rates of negative opinions toward the US military presence will result in higher protest rates.
			%\end{hyp}
		
		%\begin{hyp}
			%Higher rates of negative opinions toward the US military presence will result in larger protests.
			%\end{hyp}
		
		%\begin{hyp}
			%More demographically cross-cutting levels of protest participation will be correlated with higher protest rates.
			%\end{hyp}
		
		%\begin{hyp}
			%More demographically cross-cutting levels of protest participation will be correlated with larger protests.
			%\end{hyp}
		
		%**NOTE: We would have to use aggregate numbers from the survey\index{survey} instead of individual responses to test each of these hypotheses. It limits the sample to only survey\index{survey}ed countries and only for the years we have survey\index{survey}s. Not sure if that will be sufficient to find worthwhile results.**
		
		
		%\ref{cha:theory}
		Finally, we note that the decision calculus that populations engage in when deciding whether to protest\index{protest} a US military installation also involves the success of the protest\index{protest}. Even if there are many costs associated with a US military presence, if the population does not believe that it will be likely to achieve its aims (of removing or modifying the presence) through protest\index{protest}, it will be unlikely to mobilize\index{mobilization}. This relates to the broader point that we make in Chapter \ref{cha:theory} about why public opinion\index{public opinion} affects host-country governments\index{government} and how the presence of US troops may mobiliz\index{mobilization} their populations.
		
		%this used to say that regime type is an indicator of protest success, but i changed it to just protest, since we don't actually measure protest success. CM. 
		We argue that the host country's regime type will be a strong indicator of protest\index{protest}. As discussed by Murdie et al., the relationship between protest\index{protest} and the openness of regime type is curvilinear.\cite{Murdie2015}  In the most open regime types, it is easiest to organize a protest\index{protest} without fear of government\index{government} repression.  At the same time, in these systems, there are alternate, legitimate avenues through which grievances can be expressed, such as a functioning judicial\index{judicial} system through which the political opposition can challenge the US military presence.\cite{Bitar2016} In contrast, in the most closed systems, the protest\index{protest} is less likely to produce results and is also more likely to lead to repression by the government\index{government}. We, therefore, expect there to be more anti-US protests\index{protest} in states with medium levels of openness.
		
		This fits with observed trends in opposition to US military installations. Bitar argues that the US has shifted its basing approach in Latin America\index{Latin America} to informal bases (rather than formal ones) because opposition to the US military presence has grown in these countries as they have democratized\index{democracy} in recent years, but have yet to achieve full democracy\index{democracy} status.\cite{Bitar2016} Cooley also argues that political officials of anocratic states are more likely to politicize US military bases as a means of garnering mass levels of support.\cite{Cooley2008} When a state's institutional structures are in flux, this type of politicization can activate latent nationalism within the public. With the base used as a point of friction within the electorate and as an example of the leader's nationalist bona fides, protest\index{protest} movements can erupt in support of nationalist candidates and in opposition to US basing, as they did in Uzbekistan\index{Uzbekistan}, the Philippines\index{Philippines}, and Spain\index{Spain} during anocratic periods. Without the institutional structures that regularize the basing relationship, this politicization can make the basing contract uncertain.\cite{stravers2018}
		
		%This paragraph seems repetitive, so I cut it for now. 
		%Democratic countries have institutional structures in place that incorporate the preferences of the population into government action, and the government can then act on these preferences within the normal course of government and the legal structure. Autocratic states also tend to have more developed institutional foundations than in anocratic states, though these structures often simply stifle the preferences of the public and do not allow for protests to occur and protest movements to develop. Anocracies lie between these two extremes, with insufficient preferences aggregation combined with often personalistic leadership styles along with sufficient openness to allow protests (oftentimes spurred by the government itself) to occur.
		
		%Even today, we see similar uses of US basing by leaders in Turkey\index{Turkey} and the Philippines during periods of democratic backsliding. However, despite numerous historical examples of this sort, there has been little systematic work done to examine whether such regimes actually produce more protest activity, though they have tested regime type's relationship to different aspects of basing . We thus derive our next hypothesis:  
		
		%**Note - was this Bitar portion completed? I jumped to the next paragraph in case it wasn't.** It ws def not complete, rigth now it's all splashing ideas at the page, feel free to edit/add/etc CM.
		
		
		%\begin{hyp}
			%Anti-US base protests are more likely to occur in states that have medium levels of political openness. 
			%\end{hyp}
		
		
		Finally, when considering the cost-benefit analysis\index{cost-benefit analysis} that determines whether host country populations engage in anti-base protests\index{protest}, we also consider the situations in which some of the costs of protest\index{protest} are sunk costs. An insight from our qualitative interviews\index{interview} is that embassy\index{embassy} personnel tend to be careful about planning military exercises when other, unrelated protests\index{protest} are occurring (or expected to occur) in the area.  They noted that if there was already a group of people mobilized and protesting\index{protest}, and they found a grievance against the US military, it was effortless for the protest\index{protest} to turn into an anti-base protest\index{protest}.\cite{embone20180712,embthree20180712}  Given that one of the major challenges of collective action\index{collective action} is the act of mobilization\index{mobilization}, we expect that once individuals have been mobilized for a protest\index{protest} (once the initial costs of protest\index{protest} have been sunk), even if it is for a different cause, it becomes easier to organize a protest\index{protest} against US military installations. Thus, we expect countries that experience more protests\index{protest}, in general, also to be more likely to have anti-US and anti-base protests\index{protest}. 
		
		Again, we want to emphasize that our current approach focuses primarily on identifying the causal effect of military deployments on protest\index{protest} at the country level. To do this, we require a solid theoretical framework to help us develop a model capable of doing so. This set of expectations helps us form the necessary components to create a reasonable quantitative model for predicting protests\index{protest} by country. We will return to this topic later. We now turn to the relevant literature and theoretical expectations for our micro-behavioral model. 
		
		%\begin{hyp}
			%All else being equal, anti-US base protests are less likely to occur when there are there are high levels of other, unrelated protests occurring in the host country.
			%\end{hyp}
		
		
		
		
		
		%\begin{hyp}
			%Anti-US base protests are less likely to occur when members of the US military are engaged in humanitarian and development work with local communities. 
			%\end{hyp}
		
		%**NOTE** This last hypothesis will be really hard to test, since it will be hard to know when the military is engaging the local community, but I'm leaving it in here for now just in case.  Might be something we poke into in the future (related to the work Mike A and I did with Ali). 
		
		
		
		
		
		
		%Note sure we're actually going to get to this, so greying it out for now.CM. 
		
		%It is important to distinguish between different types of protest.  We note that some protests are more fleeting, where they may last only a day and not mobilize further support or lead to significant change.  There are also protests that are part of a larger social movement that contain a civil-society component that provides them with better organization.  These are the types of protests that would be considered civil resistance \cite{Chenoweth2013}. In addition, protests that are connected to domestic or international organizations are able to draw more individuals to the protest through their connection with the organization and its network \cite{Bell2014,Murdie2011}.
		
		%NOTE: Erica Chenoweth has a data project where they count crowds at protests, but I don't think they've published anything academic from it, just Monkey cage blog posts. I did cite her stuff on the ``3.5 percent rule''
		
		%NOTE: We want to think about the distinction between violent and non-violent protest and look at what leads some protests to turn violent. The Chenoweth and Stephan book finds that non-violent protests are more effective than violent ones (can be more inclusive, can involve more kinds of people, do not need guns, can be spoken about more openly).
		
		
		\subsection*{Micro-Behavior Theoretical Expectations}
		%\ref{cha:meth}
		Predicting protests\index{protest} at a  macro level suffers some shortcomings for concluding the relationship between military deployments and the act of protest\index{protest}. There is an ecological inference issue where assuming individual motivations and actions from higher levels of aggregate information could be erroneous.\cite{King2004} We cannot be sure that information at the national level, such as the number of troops in a country, actually affects the population we seek to study (those that choose to protest\index{protest}). While we can make reasonable inferences that connect these concepts, we would be relying upon conclusions that our data do not directly represent. Consequently, we also analyze the microfoundations of protest\index{protest}. There will likely be a difference in how individuals decide to participate in protest\index{protest} versus how protests\index{protest} are explained at the national level. To better understand individual attitudes toward protest\index{protest}, we again use our survey\index{survey} data (as detailed in Chapter \ref{cha:meth}). 
		
		Before considering individuals' protest\index{protest} cost-benefit analysis\index{cost-benefit analysis}, we first note that in many cases, participation in an anti-base protest\index{protest} involves coming into contact with members of the US military, as protests\index{protest} often occur at the site of the military installations themselves (or if this is not allowed, as close to them as possible). Thus, while we do not imply causation between these two variables, we expect a positive correlation between protest\index{protest} participation and contact with members of the US military.\footnote{To try to separate those individuals who come into contact with the US military because they participated in an anti-US military protest\index{protest} from those who have everyday contact with the US military in their communities, we will also test a hypothesis on the relationship between protest\index{protest} behavior and living in a province that houses a US military installation.} Analogous to our first hypothesis about states that have a greater US military presence being more likely to experience protest\index{protest}, we derive our first individual-level hypothesis:
		
		\begin{hyp}
			All else being equal, individuals who have personal or network contact with the US military are more likely to participate in an anti-base protest\index{protest}. 
		\end{hyp}
		
		Beyond exploring the role of contact, several other attributes may associate with the likelihood of attending an anti-US presence protest\index{protest}. When analyzing protest\index{protest} at the individual level, we can focus on the factors that influence decisions to protest\index{protest}. While we do not suggest causal relationships for these variables, we can find the conditions that better facilitate the ability for individuals to overcome collective action\index{collective action} problems and mobilize in protest\index{protest}. As previously mentioned, the decision to engage in protest\index{protest} is essentially a cost-benefit analysis\index{cost-benefit analysis}, where individuals consider the costs and benefits of participating in a protest\index{protest} and the probability of the protest\index{protest} succeeding.
		
		Given this micro focus, we can focus on ideas, feelings, or experiences that we can incorporate into the rational choice framework we are operating under.\cite{Elster1999} For example, emotions can help the individual decide rationality by affecting preferences over outcomes and determining what is most salient to the individual at a given point in time.\cite{deSousa1990,Pearlman2013} Using this framework, we explore how anger\index{anger} can affect individual decisions to engage in protest\index{protest}. As discussed in the macro-level theory section, research on mobilization\index{mobilization} shows that anger\index{anger} can make individuals more likely to mobilize\index{mobilization}, even when there are costs involved in doing so \cite{Gurr1968,Goodwin2009}. In turn, the motivating emotion can then transform into a collective emotion of solidarity that is shared among a group and keeps the protest\index{protest} movement going.\cite{Goodwin2009} Anger\index{anger} is of particular relevance to mobilization\index{mobilization} because, as Pearlman argues, anger\index{anger} is an ``emboldening emotion''.\cite[p. 388]{Pearlman2013} This means that anger\index{anger} influences individuals' cost-benefit analysis\index{cost-benefit analysis} by making them more optimistic about their possibility of success, in this case of achieving their aim through protest\index{protest}. Anger\index{anger} is an emotion that drives individuals to act to correct a perceived wrong.\cite{Carver2009} This is important because the costs and benefits of each expected outcome have to be weighed by the possibility of each outcome occurring, in this case, the probability of success in achieving the aim of removing (or modifying) the US military presence through protest\index{protest}. When individuals experience anger\index{anger}, they can be driven to mobilize\index{mobilization}, even if the probability of success, and therefore the utility of protesting\index{protest}, would otherwise be low.  
		
		%Note: The Garcia & Young piece notes that violent crime is more likely to provoke anger, so if we want to distinguish across different kinds of crimes, we can always cite that as justification
		
		Of course, emotions are challenging to measure. Even assuming that individuals respond sincerely when asked about emotion in a survey\index{survey}, levels of emotion such as anger\index{anger} vary in time. The anger\index{anger} that made an individual willing to mobilize\index{mobilization} in the past may no longer be present at the time of surveying\index{survey}. Therefore, we focus on particularly traumatic events that are highly likely to be correlated with anger\index{anger}; in this case, we are the victim of a crime\index{crime}. Previous research finds that individuals who are the victims of a crime\index{crime}, particularly ones who perceive themselves as innocent, are more likely to experience anger\index{anger} (and in turn to engage in mobilization\index{mobilization} against governments\index{government} who have failed to protect them from crime\index{crime}).\cite{Garland2012,Garcia2019} We assume that if an individual were the victim of a crime\index{crime} perpetrated by a US service member, then they would likely fault the US military, at some level, for its perpetration. We expect that those individuals will be more likely to participate in anti-US protests\index{protest}. 
		
		Crime\index{crime} victimization may serve as an added, direct route towards mobilization\index{mobilization}. Specifically, by being a victim of a crime\index{crime}, the survey\index{survey} respondent has experienced a specific, direct cost of the military presence. The negative externality of US troop deployments concentrates on the individual. Given their experience, they are more likely to see the troops as a net negative in their community, and engaging in protest\index{protest} is one way for them to express their preferences over the presence of the troops. We believe that those individuals who have been the victim of a crime\index{crime} perpetrated by a US service member will be more likely to respond to mobilization\index{mobilization} efforts to oppose the United States' military presence. 
		
		
		
		%\begin{hyp}
			%All else being equal, individuals who have been criminally victimized by the US military are more likely to participate in an anti-base protest. 
			%\end{hyp}
		
		As we have mentioned before, a significant part of the individual's cost-benefit analysis\index{cost-benefit analysis} used in determining whether to participate in a protest\index{protest} is calculating the probability of success; deciding to participate in a demonstration is a costly act (it requires time, transportation, and represents a host of opportunity costs) and engaging in protest\index{protest} with little likelihood of success would not be an attractive opportunity. Consequently, we expect that potential protesters\index{protest} care about the effect of their actions. Much of the probability of success of public demonstrations depends on the potential protesters'\index{protest} capacities. 
		
		In the context of mobilization\index{mobilization} theories, a second strand of explanations focuses on the resources that individuals have available to them. While grievance-based explanations are common for civil wars\index{civil war}, they are less satisfactory in explaining the likelihood of conflict for a few reasons. First, grievances often present collective action\index{collective action} problems for mobilizers\index{mobilization}. A grievance is not sufficient for them to get people to mobilize\index{mobilization} around unless the direct cost to the individual is significant. Even then, free-riding on the efforts of others tends to be a more tenable option than risking the individual's resources or life in some cases.\cite{lichbach1993} Second, grievances are ubiquitous across societies as there are always issues that people want to be remedied. Social entrepreneurs who engage in mobilization\index{mobilization} can capitalize on particular grievances by offering selective incentives, but this suggests that grievances themselves are not sufficient.  Consequently, mobilization\index{mobilization} will only occur when groups have resources available to them that allow them to mobilize\index{mobilization}, provide selective incentives, or lower participation costs.\cite{Olson1965,Tilly1973,Khawaja1994} For example, in the case of the Arab Spring\index{Arab Spring} protests\index{protest}, many pointed to the availability of mobile phones and social media, resources which facilitated coordination, as facilitating mobilization itself.\cite[][Similarly, during the Iranian Revolution, opportunities for social gathering also led to increased protest mobilization.]{Hussain2013,Rasler1996}
		
		
		We expect that situations in which resources are available for coordination will ease protest\index{protest} organization in anti-US base protests\index{protest}. Of course, what types of resources an individual needs to mobilize\index{mobilization} will vary from person to person. One commonality across individuals, though, is that greater proximity to others and ease of communications should decrease the cost of mobilization\index{mobilization}. To better account for this, we consider whether individuals live in urban areas, which are likely to have greater access not only to other individuals (because of higher population densities), but also more reliable telecommunications networks and public transport, which should ease the ability to coordinate and travel to protest\index{protest} sites. 
		
		In addition, those living in large urban areas (such as capital cities) are more likely to contact transnational anti-basing activist\index{activists} organizations that will work to organize locals against a military presence.\cite{Murdie2011,Murdie2015,Kiyani2020} Prior work on protest\index{protest} and mobilization\index{mobilization} shows that having access to transnational activist\index{activists} networks, which already have in place the resources and infrastructure to facilitate protest\index{protest}, facilitates mobilization\index{mobilization} and leads to an increased probability of protest\index{protest}. For example, the German\index{German} peace activist\index{activists} interviewed\index{interview} for this project was based in the capital city of Berlin\index{Germany!Berlin}, and he talked at length about the international connections to other activist\index{activists} groups he had. He noted that his organization had recently held an international conference against bases and war, with representatives from 40 different countries attending, and noting that he often gets invited to anti-base events in Japan\index{Japan}.\cite[][He also highlighted the importance to mobilization of having an activism infrastructure in place, noting that the German peace movement has a very good peace infrastructure.]{berlinone20190723} In addition, exposure to transnational activism may lead to individuals associating the American base with a broader network of American imperialism.\cite{Immerwahr2019}  
		
		Further, urban areas are also more likely to house those groups that are more predisposed to protest\index{protest}. A common theme in our interviews\index{interview}, whether we were talking to the US or host-country government\index{government} officials, was that other demographics are more likely to protest\index{protest}, similar to how certain demographic groups are predisposed to feel more positively towards the US military. For example, there seemed to be agreement that students and labor unions\index{labor union} were more likely to mobilize\index{mobilization} against the US military presence; this is a point echoed in the existing base protest\index{protest} literature.\cite{Fitz2015,embone20180712,berlinone20190723,journ20180712,journ20180713,Allen2020} Activists\index{activists} themselves seem to be aware of this because of failed recruitment efforts. Regarding the city of Wiesbaden\index{Wiesbaden}, which is near the Clay Kaserne Army Garrison\index{Clay Kaserne Army Post}, the German\index{German} peace activist\index{activists} noted, ``It is a horrible city to start a movement. Demographics are difficult. It is where the Russian Czars\index{Russia} would visit. Lots of rich pensioners live there. Wiesbaden\index{Wiesbaden} has a nice historical past, but the demographics are difficult for a peace movement.''\cite{berlinone20190723}
		
		Related to this point, and as discussed in the introduction to this chapter, we believe that individuals who are part of a community that regularly interacts with the US military will be less likely to protest\index{protest} the US military presence. In agreement with previous work, in Chapter \ref{cha:meth} we have found that even though contact with the US military correlates with increased probabilities of both positive and negative views of the US military, the positive effect is stronger than the negative one.\cite{Allen2020} We have argued that an essential reason for this is the fact that American personnel often become a part of local communities, creating both ``bridging'' and ``bonding'' opportunities between US personnel and host-state citizens.\cite{Woolcock2000}  This creates a sense of shared identity or experience that allows locals to overcome negative stereotypes about the US military. 
		
		Local residents are therefore more likely to support bases than faraway residents removed from the actual troops.\cite{Fitz2015,Flynn2018} Residents are more likely to feel a sense of affinity with the US forces. Over and over again during our interviews\index{interview}, regardless of which country we conducted them in, we heard reinforcing evidence for the idea that individuals whose communities were more proximate to US deployments were more likely to view the US military positively. This was true whether the interviewee\index{interview} viewed this relationship positively or not. 
		
		For example, at the US embassy\index{embassy} in Panama\index{Panama}, a variety of individuals we met with referred to anti-US protesters\index{protest} as ``paid protesters\index{protest}'' and noted that most Panamanians\index{Panama} felt positively towards the US because  of the long-term nature of the US presence and the ``strong cultural ties formed between American and Panamanian\index{Panama} people.''\cite{embone20180712} They specifically noted the high frequency of marriages between US service members and Panamanians\index{Panama}. This point was confirmed by an interview\index{interview} with a former Panamanian\index{Panama} President, who noted that US Panamanian\index{Panama} relations ``have been going very well.''\cite[][The President noted that intermarriage also created some problems, as Panamanians who married American service members often became stateless after renouncing their Panamanian citizenship but before acquiring American citizenship.]{embthree20180712,pres20180714} The same former President, when we asked him about more recent deployments that provide humanitarian\index{humanitarian} assistance (such as Beyond the Horizon\index{Beyond the Horizon} and New Horizons\index{New Horizons}) said that they are viewed positively by the people who receive them, who perceive the economic benefits that they receive from them.\cite{pres20180714}
		
		From the opposite perspective, a journalist in Panama City\index{Panama!Panama City} who noted that he had previously participated in anti-US marches made no secret of his suspicions over humanitarian\index{humanitarian} outreach carried out by deployed US military personnel. As he talked to us at a mostly empty grilled meat restaurant he had asked us to meet at, he expressed concerns that the US had these programs in place to spy on Latin America\index{Latin America}; that the humanitarian\index{humanitarian} help is just a disguise. He noted that the people who benefit from these programs (most of whom reside in rural areas) had ``low education levels and critical analysis skills'' and did not analyze the ``political implications'' of the help they received. Yet, he acknowledged that they felt positively towards the deployments.\cite{journ20180713} A former Panamanian\index{Panama} Cabinet Member, this one speaking to us at a much higher-end restaurant, was much less suspicious about the secrecy of US deployments. At the same time, she was also unenthusiastic about humanitarian-oriented\index{humanitarian} deployments, noting that civilian personnel should deliver US aid. However, she also noted that Panamanians\index{Panama} who had regular interactions with deployed personnel had positive views of the US military. Her particular example involved sex workers\index{sex workers}, who she said preferred US military clients, as they were considered more physically fit and attractive.\cite{journ20180712} 
		
		Given this, we expect that those living in areas where the US military incorporates into society will be less likely to participate in anti-base protests\index{protest}. Even though we note in our initial hypothesis that contact with the US military makes protest\index{protest} more likely, by the simple logic that protest\index{protest} itself in many cases involves coming into contact with US service members, we also expect that individuals living in provinces that house a US military installation will be less likely to participate in an anti-base protest\index{protest}. 
		
		%[NOTE: Add more here if needed. We can draw from interviews, etc. This is where we are linking directly to our theory from Ch. 2, so it should feel like it's a strong part of the chapter]
		
		%\begin{hyp}
			%All else being equal, individuals living in provinces that house a US military installation are less likely to participate in an anti-base protest. 
			%\end{hyp}
		
		%%%This might be too hard to do, not erasing it, but this seems like too hard of a road to go down**
		%*****Note: we should have a hypothesis about the role of transnational activist\index{activists} groups.  A transnational activist\index{activists} presence should make protest more likely, but the transnational activists\index{activists} have to be perceived as sincere and credible, or otherwise they may draw backlash against them (Fitz-Henry 2015). 
		
		%As state above, and as argued by \citeasnoun{Fitz2015}, the association of a military base with a broader idea of American imperialism, rather than with a local, bilateral and consensual agreement between the US military and the local government, is more likely to lead to opposition to the base.  Quasi-bases, which are ``foreign military bases that are not supported by a formal agreement'' are easier for governments to hide from potentially hostile opposition and civil society actors\cite{Bitar2016} (p. 50). For example, in Latin America, as countries democratize and leaders face more political backlash from hosting US military bases that do not provide widespread benefits that include political opposition, the US has come to depend on informal agreements under which US military personnel deploy to existing military installations rather than building new American bases \cite{Bitar2016}. 
		
		%Thus, we expect that US military bases will experience less protest in cases in which they are housed within an existing military installation that also houses the host country's military.  As \citeasnoun{Fitz2015} notes, bases that are framed as forward operating locations rather than bases per se (which are viewed as more permanent and intrusive), will be more likely to draw negative attention and thus experience protests. 
		
		
		%**NOTE: The USAF base in Manta is one example; I'm guessing that Lakenheath will also be a good example of this.  Hopefully we can insert a good anecdote about it here.  Bases in Germany and Japan are examples of the opposite.CM.**
		
		%\begin{hyp}
			%All else being equal, anti-US base protests are less likely to occur when US troops are housed alongside host country troops in an existing military installation. 
			%\end{hyp}
		
		%\begin{hyp}
			%All else being equal, anti-US base protests are less likely to occur when US troops are present as part of a forward operating location rather than a permanent base. 
			%\end{hyp}
		
		%NOTE: add transition
		
		While some of our interview\index{interview} subjects, particularly those who represented the US government\index{government} and military, referred to US military spending and investment as contributing to pro-US views, we do not expect protest\index{protest} to correlate with pure measures of military expenditures in the country or the area.\cite[For example, an interview subject at the US embassy in Panama attributed the high level of acceptance for the US in Panama to long-term and widespread US investment in Panamanian infrastructure.][]{embthree20180712}  As we have argued in previous work and as other studies have found (see, for example, Fitz-Henry on the case of the city of Manta\index{Ecuador!Manta} in Ecuador\index{Ecuador}),\cite{Fitz2015} economic motivation and direct financial gains only go so far in creating a sense of acceptance.\cite{Allen2020}  In fact, excessive spending can even lead to negative externalities such as inflation in the area and locals not being able to afford their homes.\cite{Hohn2010,Fitz2015}  We instead rely on an argument that focuses on negative and positive views of bases being what leads to protests\index{protest}, rather than purely financial motivations.
		Though we do not expect aggregate military expenditures in an area to influence protest\index{protest}, we believe that economic benefits received from the US military may be related to protest\index{protest}participation at the individual level. In Chapter \ref{cha:meth} and our previous work, we show that personal or network economic benefits from the US military can lead to more positive views of US actors and the US presence in the host country.\cite{Allen2020} As we have noted earlier in this chapter, we believe that those individuals who perceive the US military more positively will be less likely to participate in anti-US military protests\index{protest}. Unlike contact with US military members, economic benefits are not something that we expect to result from participation in an anti-base protest\index{protest}. 
		
		%For this research, we are less concerned about the direction of causality in this case and derive our last micro-level hypothesis: 
		
		
		%\begin{hyp}
			%All else being equal, individuals who receive, or whose network receives, economic benefits from the US military are less likely to participate in an anti-base protest. 
			%\end{hyp}
		
		%I figured we should go with one or the other, so probably economic benefits being negatively correlated to protest behavior? 
		
		%\begin{hyp}
			%All else being equal, individuals who receive, or whose network receives, economic benefits from the US military are more likely to participate in an anti-base protest. 
			%\end{hyp}
		
		
		
		%aid-provision by the US military, in particular humanitarian work such as investing in infrastructure and other community engagement activities, is likely to lead to a lower incidence of protest.  Having the military engage in activities that provide needed projects to local communities can lead members to view the military installation as a more fair exchange between them and the American military. Previous work shows that deployments that specifically engage in development-oriented work, such as building or renovating schools and hospitals and providing immunizations and medical care are related with more positive perceptions of the US military \cite{Flynn2018}. In this way, deployments that are present for a non-development-oriented mission may be able to avoid the negative backlash against them by providing projects such as infrastructure development or other forms of engagement with the local population.  
		
		\section*{Research Methodology}
		
		We have two primary aims for our analysis in this chapter. First, we are interested in evaluating the relationships between different predictor variables and protest\index{protest} events at the country level. We first present a series of simple count models where we predict the number of protest\index{protest} events in a country--year as a function of several predictor variables. More specifically, we are also interested in the causal effect US military deployments have on protest\index{protest} activity at the country level. To address this more specific question, we use more advanced techniques to identify this specific causal relationship. Second, we are also interested in understanding the factors that correlate with individual-level protest\index{protest} involvement. We build a series of predictive models of individual protest\index{protest} participation. Accordingly, our analysis in this chapter has four primary parts, and we begin by focusing on protest\index{protest} events at the country level, followed by protest\index{protest} participation at the individual level. 
		
		This section focuses on 1) understanding the factors that correlate with country-level protest\index{protest} events and the causal effects of US military deployments on the occurrence of protest\index{protest} events across countries, more specifically; and 2) generating a basic predictive model to help us understand the different types of people who are likely to participate in anti-US protest\index{protest} events, and to examine comparisons between different groups and their likelihood of engaging in protest\index{protest} activities.
		
		
		
		\subsection*{Macro-Behavior Models: What Causes Anti-US Protest\index{protest} Events?}
		
		Above, we discuss our theoretical expectations concerning the link between US military deployments and macro-level protest\index{protest} events. In this section will discuss the data collection, operationalization, and estimation strategies we adopt to assess the causal effects of our variables of interest. 
		
		This chapter uses two country-level measures of anti-US protest\index{protest}-- protests\index{protest} against the United States, in general, and protests\index{protest} against the US military, specifically. These variables are counts of the number of protest\index{protest} events in a given country in a given year. We use events data on worldwide protest\index{protest} events between 1990 and 2018 gathered from the Integrated Data for Events Analysis (IDEA)\index{Integrated Data for Events Analysis} dataset to generate these variables.\cite{Bond2003} The dataset compiles events from a range of global news sources. Through a mixture of automated parsing procedures and manual data review, we refined the general search to focus on various types of protest\index{protest} events that involved 1) the mass physical mobilization\index{mobilization} of individuals and 2) protest\index{protest} events that aimed at the United States government\index{government}, in general, or the United States military, specifically.\footnote{We selected all instances of protest\index{protest} events, which include protest\index{protest} demonstrations (event code 181, ``All protest\index{protest} demonstrations not otherwise specified''), protest\index{protest} obstruction (event code 1811, ``sit-ins and other non-military occupation protests\index{protest}''), protest\index{protest} procession (event code 1812, ``picketing and other parading protests\index{protest}''), protest\index{protest} defacement (event code 1813, ``damage, sabotage and the use of graffiti to desecrate property and symbols''), and protest\index{protest} altruism (event code 1814, ``protest\index{protest} demonstrations that place the source [protestor\index{protest}] at risk for the sake of unity with the target'').}
		
		%\footnote{Interview subjects across different countries noted that there had been an increase in anti-US protests since Donald Trump became President of the United States in 2016, while we cannot control for the ``Trump effect'' in our micro level analysis, as interviews were conducted in 2018, 2019, and 2020, in our macro-level analysis we account for this by taking annual measures into account \textbf{stuff for flynn to fill in} \cite{journ20180713,mpone20190717}.}
		
		The IDEA\index{Integrated Data for Events Analysis} dataset identifies protest\index{protest} events as an action from one actor (the sender) towards another actor (the target). Both senders and targets can be parsed to varying levels of specificity. For our purposes, we focus on two specific sets of actors, as discussed above. First, we selected all cases where the database identifies the target country as the United States, regardless of the more specific target (such as civilian government\index{government} officials or military). For example, protests\index{protest} of the election of US President Donald Trump would be in this category. Our sample includes 531 cases of anti-US protests\index{protest} distributed throughout time and space.\footnote{In some cases, the events data lists the location of the protest\index{protest} broadly as a region such as ``Western Africa\index{Africa},'' or even more broadly, as ``World.'' Because the data does not include enough detail for us to identify the specific country or countries in which these protests\index{protest} occurred, we drop these cases from the sample.}
		
		For our second measure, we further narrow our focus to protests\index{protest} that target the US military. In this case, we take the sample of anti-US protests\index{protest} and select the circumstances in which the data identifies the target as the military; the data defines this as ``official armed forces, peacekeepers, astronauts, military police, military justice officers, military academies, and border guards.''\cite{Bond2003} This gives us a total of 29 anti-US military protest\index{protest} observations, with Afghanistan\index{Afghanistan}, Australia\index{Australia}, Iraq\index{Iraq}, and South Korea\index{South Korea} having the highest numbers of anti-military protests\index{protest}. As with the more general protest\index{protest} measure, we omit cases that occur within the United States.
		
		While our outcome measures focus on protests\index{protest} directed against the United States, the data collection process also measured total protests\index{protest} in each country-year. This offers a baseline of comparison for anti-US protests\index{protest} as it is likely that people in some countries are more likely to protest\index{protest} in general than others. We note that the data set collects the protest\index{protest} data at the country-month level (meaning the data set codes events for every month). However, in our analysis, we aggregate it to the country-year level given that our other variables of interest are at the country-year level of analysis.  This allows us to calculate a variable that is the number of anti-US protests\index{protest} in a country year as a proportion of total protests\index{protest}, thus allowing us to both control for greater inclination towards protest\index{protest} in some countries as well as to study whether there is something specific about anti-US protests\index{protest} that makes their determinants different from protests\index{protest} overall. In addition, we create a metric that measures anti-US military protests\index{protest} as a proportion of all anti-US protests\index{protest}; this again allows us to test whether different factors determine whether people are likely to protest\index{protest} against the US military in particular or whether these are just a subset of anti-US protests\index{protest} that are determined by the same factors as non-military focused protests\index{protest}.           
		
		
		\subsubsection*{Model Specification and Identification}
		
		%Political openness (squared term to capture curvilinear relationship)
		
		We approach our analysis of country-level protest\index{protest} events in two steps. First, drawing on the theoretical arguments outlined in the previous section, we present four Bayesian\index{Bayesian} multilevel negative binomial count models. We model protest\index{protest} events as a function of several predictor variables for the 1990--2018 period. These initial models will provide us with a baseline set of relationships to better understand the correlates of protest\index{protest} events against the United States. 
		
		We intend the second approach to examine the more complicated question of identifying the causal effect of US military deployments on protest\index{protest} activity. Several factors complicate the accurate assessment of the \textit{causal} effect of deployments of protest\index{protest}. Traditional regression approaches are limited in their ability to estimate causal effects for many reasons. One of the most challenging problems to deal with is that the history of military deployments is likely to matter as much as short-term deployments. Still, basic regression models typically only provide estimates of short-term effects. Further, these histories are highly variable across countries. This can severely bias the estimates from our model, as not every country has an equal opportunity of experiencing US military deployments-- both in terms of hosting them and the size of those deployments. Finally, as is often the case, the relationship between military deployments and protest\index{protest} events may suffer from confounding effects. Other variables may exert a causal effect on both of our variables of interest. Accordingly, modeling protest\index{protest} events as a function of military deployments and adjusting for many other variables will not yield results that have any meaningful interpretation as causal effects. 
		
		To address these problems, we estimate a marginal structural model (MSM).\cite{Robinsetal2000,BlackwellGlynn2018} These models have a relatively long history of use in fields like biostatistics and epidemiology and can be used to help estimate causal effects in observational studies where exposure to treatments varies across space and time. We can also use this to help us estimate the contemporaneous effect of treatment and the more general effect of a particular treatment history. However, MSMs require us to do some additional work before estimating the effects of troop deployments and their histories on the outcome of interest. There will likely be systematic differences between the individual countries that receive troop deployments or larger deployments and those that receive no or smaller deployments.
		Additionally, troop deployments in a given period (what we refer to as time $t$) are heavily dependent on deployments in the previous period (time $t-1$). Deployments at time $t-1$ are also likely to affect other predictor variables in subsequent periods. We must estimate a series of structural weights for each observation for the final regression model to resolve this. We calculate these weights using the formula found in the Appendix section \ref{eq:structuralweights}.
		
		
		
		More simply, the weights shown in Equation \ref{eq:structuralweights} are a form of propensity score that we can use to re-balance the observations in our data. This is simply a variant of the inverse probability of treatment (IPTW) weighting method.\cite[For more information on estimating propensity scores see:][Chapter 13]{ImbensRubin2015} Normally, estimating propensity scores is a relatively straightforward process as it is commonly used to estimate scores in data where the treatment, and often the outcome, are binary (meaning 0 or 1) variables. The fact that our treatment (the number of US troops present in a country) is continuous (meaning that it can, in theory, take on any integer value that is 0 or greater) complicates this process slightly.\cite[For a fuller discussion of estimating the inverse probability of treatment weights for MSMs, estimating these weights for continuous treatment variables, or estimating weights for a continuous treatment in the presence of multilevel/grouped data using multilevel models see the following works:][]{ColeHernan2008,vanderwalGEskus2011,Naimietal2014,SchulerChuCoffman2016} To generate these weights we first have to estimate two separate models wherein troop deployments themselves are the outcome of interest. The first of these models (the numerator) is relatively straightforward to estimate. Here we are running a regression model wherein we predict the size of the troop deployments in country $i$ at time $t$ as a function of a one-unit lag of the troop deployment variable, the cumulative sum of troop deployments in the country $i$ from the beginning of its series up through time $t-2$, and a range of time-invariant covariates, $Z$. In cases where researchers are working with just a binary treatment variable and only a single period, using ``1'' as the numerator is often all that is required. However, since we are dealing with a continuous treatment variable, generating the numerator using this regression-based approach can help to stabilize the resulting weights by ensuring that the ratios are not enormous.\cite{ColeHernan2008}
		
		The second model (the denominator) is more complicated. In most respects, it is identical to the numerator, except the gamma ($\gamma$) term, which denotes a vector of covariates---the key here is that these covariates have to be sufficient to meet the sequential ignorability criteria. In essence, this requires that there is no unmeasured confounding between the treatment (troops) and the outcome (protests\index{protest}).\cite[The language on this point can be confusing given that these techniques were developed across several different literatures and disciplines and compounded by the fact that they often did not speak to one another. To put it in slightly different terms using language from Pearl, this set of covariates must ensure that Pr$ (Y_{it} \Perp X_{it} | \gamma, Z_{i}) $.][That is, the outcome is conditionally independent of the treatment, conditional upon the time-varying covariates ($\gamma$) and the time-invariant covariates ($Z$).]{Pearl2009} We will return to this below, but once we have both of these models, we then generate two sets of predicted values and residuals for each observation in the data set---one using the numerator and one using the denominator. For each set of predictions and residuals, we calculate the probability of the actual observed value of troops in country $i$ at time $t$ using the predictions and the standard deviation of the residuals as the mean and standard deviation of a normal probability density function. We then divide the values generated using the numerator figures by the values generated by the numerator. The actual structural weight for any given observation is the cumulative product of these ratios for country $i$ from the beginning of that country's series up through time $t$. 
		
		This represents a brief overview of the process of estimating the structural weights. But, as we note above, estimating the second model is complicated because we need a set of covariates that ensure there is no unmeasured confounding between the treatment and outcome variables. To aid in this process, we turn to another tool---directed acyclic graphs (DAGs). Like with MSMs, scholars from other disciplines have more commonly used DAGs (such as computer science and epidemiology), and they serve a variety of useful purposes. First, they help make explicit the wider array of theorized causal relationships between various predictor variables. Second, assuming a well-developed theoretical model, they can also help to identify sources of confounding and bias.\cite[See:][Keele et al. offer a fuller discussion of DAGs and their applications.]{Pearl2009,MorganWinship2015,KeeleStevensonElwert2020} Related, they can be used to identify which variables need to be adjusted for in a statistical model to close off the ``back-door'' paths between the treatment variable and the outcome that serve as the sources of confounding in the model, but also to ensure that we do not adjust for variables that may open up such informational pathways between the treatment and outcome and introduce additional sources of bias (i.e. collider bias).\cite[][we use the {\tt dagitty} package in {\tt R } to build our DAG and to identify possible adjustment sets.]{Pearl2009,Textoretal2016}
		
		To estimate these models, we first begin by constructing a theoretical model wherein we treat protest\index{protest} events as the outcome of interest and military deployments as our key predictor, or ``treatment'' variable. These variables themselves are embedded in a larger theoretical causal framework where they are themselves caused by, and cause, a variety of other variables. Figure \ref{fig:dagtroops} presents a directed acyclic graph (DAG) depicting the basic contemporaneous relationships between the variables of interest and their relationships between two time periods, $t$ and $t+1$ (which is the period directly following t). To ease interpretation, we present a simplified version of the full DAG. We condense most of the time-variant and time-invariant predictor variables down to the time-invariant covariates ($Z$ terms). We also only present two periods, but the chain of events theoretically runs back to the first observation of $t$ in the series. The light blue node represents the outcome of interest (protests\index{protest}), while the light green node represents the treatment of interest (troops). Any node connected to another indicates a proposed causal relationship, and the direction of the arrow represents the direction of that relationship.
		
		\begin{figure}[t]
			\centering\includegraphics[scale=0.8]{../../Figures/Chapter-Protests/fig-dag-protest-simple.png}
			\caption{Directed Acyclic Graph (DAG) showing the theoretical model of protest\index{protest} across time periods.}
			\label{fig:dagtroops}
		\end{figure}
		
		Using this model as our starting point, we can generate several adjustment sets-- variables that will ensure there is no unmeasured confounding when included in the model. Notably, this rests entirely on the notion that we have the ``right'' theoretical model. We can use many possible adjustment sets given the current model, but they all accomplish the same basic task. If we have included causal relationships in our theoretical model that do not exist, or if we have omitted key causal relationships that \textit{do} exist, then the implications of the model and our ability to isolate causal effects may change. How these changes would affect our ability to assess the causal relationships we are proposing depends entirely on the specific connections that would change. Assuming for present purposes that our model is sufficient, we can now estimate the denominator described in Equation \ref{eq:structuralweights}.
		
		Once we have calculated the structural weights using this method, we can now estimate the average treatment effect (ATE) of troop deployments on protest\index{protest} activity. To do so, we estimate a series of multilevel negative binomial regressions predicting the number of protests\index{protest} in a country year using the troop deployment variable and treatment history (i.e. the cumulative sum of deployments through time $t-2$) as predictor variables.\cite[For more information on the use of multilevel models for causal inference see:][]{Hill2013}  Notably, these models weight the observations according to the structural weights we calculated above. US military deployments are generally stable \textit{within} countries, with substantial variation \textit{between} countries. For example, Germany's\index{Germany} troop levels are high and relatively stable over the period that we study. In other cases, troop levels are low but also relatively stable. There are, however, some cases where troop levels increase and decrease radically at various points throughout 1990--2018. These sudden changes are typically associated with large military operations, like the 1999 war in Kosovo\index{Kosovo} or the 2003 Iraq War\index{Iraq War}. This extreme variability makes estimating the outcome models somewhat difficult because we have extremely large weights for a small set of observations. To address this problem, we estimate the models multiple times using several different truncation points for the weights to better assess the sensitivity of our estimates to the weighted sample. The weights effectively create a pseudo sample of data by replicating observations according to their score. For example, an observation with a score of 4 would be copied four times when we run the model. Given these cases where we see very rapid and extreme changes in US military deployment levels, we encounter a situation where the weights are often enormous (such as weighting scores of 50,000$+$). There is some question about the proper size of the weights and how to deal with very large or very small weights. Because we are dealing with some exceptionally large weights, we estimate our outcome model multiple times using different truncation points to cut down on the number of extreme values and to assess how sensitive our estimates of the ATE are to changes in the weights.\footnote{There is no clearly ``correct'' way to deal with extreme weight values. Cole and Hernan note that in the context of binary treatment variables, the average of the weights should be approximately 1. Still, beyond this, there is relatively little guidance of which we are aware \cite{ColeHernan2008}. One approach is to trim weight scores, which we employ here. Specifically, we set multiple thresholds at 10, 50, 500, 1,000, 5,000, and 10,000. Observations, where the calculated weight falls above the threshold reset to the maximum threshold. For example, when estimating our models using the 500 threshold, a structural weight calculated to be 1,000 would be re-coded to be 500. This approach is useful in that it does not require us to throw away data. There is a clear gain in efficiency due to increasing the effective number of observations, but too few or too many pseudo observations can bias the estimates. Hence, we present all of these points to assess bias and sensitivity.} We display this information below in the results section.
			
			
			
			
			\subsection*{Micro-Behavior Models}
			
			%At the micro level, we use survey\index{survey} data from surveys carried out in 14 different countries with a high level of US presence, with approximately 1,000 respondents per survey\index{survey}.\footnote{The countries covered in the survey\index{survey} were the same ones covered in previous chapters: Philippines, Poland, Australia, Portugal, Netherlands, Belgium, Turkey\index{Turkey}, Kuwait, Spain\index{Spain}, United Kingdom\index{United Kingdom}, and Italy.} We conducted the survey\index{survey}s from early September until early November of 2018 and 2019. The specific question that referred to protest behavior asked: ``Have you ever attended a protest event against a US military base?'' The response options were the following: Never, one time, two times, three times, more than three times, and don't know/decline to answer.
			
			This section shifts our attention back to our public opinion\index{public opinion} data to examine individual-level participation in protest\index{protest} events targeting the United States in general and the United States military more specifically. 
			
			
			\subsubsection*{What Causes Individuals to Participate in Protest\index{protest} Activity?}
			
			As noted earlier in this chapter, individuals conduct a cost-benefit analysis\index{cost-benefit analysis} to determine whether to join an anti-base protest\index{protest}. To begin with, individuals must have some grievance that drives them to protest\index{protest}. In some cases, individuals will benefit from participating in the demonstrations and expressing their frustrations publicly. Of course, the protest\index{protest} is usually a means to an end, meaning that the protest\index{protest} is meant to change US policy or eject the US military from their country. The larger the probability of successfully achieving their desired outcome, the more likely individuals are to protest\index{protest}. Finally, the costlier protest\index{protest} participation is, either through the opportunity cost of participation or because of its consequences, such as repression by the government\index{government} the less likely individuals are to protest\index{protest}. 
			
			
			
			\subsubsection*{Predicting Protest\index{protest} Activity}
			
			The goal of this section is twofold: First, we generally aim to predict individual involvement in protests\index{protest} against the United States military. Our survey\index{survey} questionnaire specifically asks individuals about their history of attending protest\index{protest} events. Using this information, we construct a series of predictive models better to understand the occurrence of protest\index{protest} activity among foreign publics. Unlike our country-level models of protest\index{protest}, these models do not explore the causal effects of various predictor variables, and the coefficients generally do not have clear causal interpretations. Instead, we first focus here on constructing predictive models using available covariates and then assessing the model's overall performance in accurately predicting whether an individual will participate in a protest\index{protest} or not. 
			
			The second goal will be to use this information to shed light on some of the individual-level characteristics that correlate with protest\index{protest} activity. Although we cannot offer more concrete causal interpretations given the structure of our survey\index{survey}, we can use the predictive models to generate useful descriptive comparisons across groups. This is effectively what we did in chapter \ref{cha:min} on minority\index{minority} views of the US military and other actors. In this case, the coefficients are more usefully understood as providing information on differences in protest\index{protest} behavior between groups once we adjust for various other predictors of interest.
			
			\subsubsection*{Model Specification and Estimation}
			
			
			Our survey\index{survey} asked individuals about their involvement in anti-US protest\index{protest} events. It asked them how many protests\index{protest} they had participated in. To simplify the prediction process, we turn what is a count (of the number of protests\index{protest} attended) variable into a binary variable that simply indicates whether a respondent has previously attended an anti-US protest\index{protest} event or not ($1 = Yes; 0 = No$). 
			
			We estimate multiple models to compare how well our models perform, beginning with a basic Bayesian\index{Bayesian} multilevel logistic regression predicting protest\index{protest} participation, leaving the model unspecified except for the population-level intercepts, country-level error, and yearly error. In other words, this is a model that does not consider any individual characteristics, it just tells us how likely an individual is to protest\index{protest} in a given year. This serves as our baseline model of protest\index{protest} involvement. Next, we estimate a similar model, but we include several individual-level demographic variables as reported by the respondents. This model includes only the individuals' characteristics and does consider their attitudes and views. Third, we estimate a fuller model including the individual-level demographic variables listed above, along with individual-level attitudes and experiences, which we ask about on a range of questions. Fourth, we estimate the previous model with demographic and attitudinal information again, but we add country-level variables including gross domestic product (GDP), population, the number of bases in the respondent's province, and the size of the US military deployment in that country.\footnote{Note that although we have data at the province level across all countries, we are often lacking in sufficiently large sample sizes at the province level to estimate reliable effects. Since we are grouping respondents according to the country and year of the survey\index{survey}, the estimates for the base count variable are likely to be biased.} In other words, the fifth model allows for country-level characteristics to influence protest\index{protest} participation predictions as well. The fifth model builds on Model 4 by using the same predictor variables but adds varying coefficients for the age, gender, ideology, and base count variables. Although the individual-level predictors are identical in models 3, 4, and 5, each model builds upon the previous in important ways. For example, by adding group-level variables in Model 4 we can account for systematic differences between the different countries included in our data. Similarly, what we are doing by ``varying coefficients'' in Model 5 is to allow for the possibility that different factors (such as, for example, gender) may influence protest\index{protest} behavior differently in different countries.  
			This means that we are relaxing the assumption that there is a constant relationship between predictor factors and protest\index{protest} countries.\footnote{We add an additional layer of flexibility to the models by allowing correlation between all of the varying or ``random'' to vary as well. This relaxes the assumption that the correlation between these effects is 0 and allows the model to estimate each instead.} Table \ref{tab:protestpredictionvariables} provides an overview of the specifications of these five models. The varying coefficients vary by country only (that is, not by year). 
			
			\input{../Tables/Chapter-Protests/model-protest-prediction-varlist.tex}
			
			In using these models, we must assess how good they are at predicting actual protest\index{protest} behavior. To do this, we divide our data into two samples: a training sample and a test one. This step helps to ensure that we are not overfitting our model to the particular data that we have collected. In other words, it ensures that the model is indeed a general one and not just one that works only with the data from the specific individuals we surveyed\index{survey}. This step also allows us to assess how well our model can predict protest\index{protest} when we test it out on ``new'' data that were not used to estimate the model (the ``test sample'' that we referred to earlier). We group our opinion data according to the individual representative characteristics (age, income, and gender). We then randomly assign 80\% of the observations to the training data and the remaining 20\% to the test data. This ensures that both the training data and the test data are representative of these same variables. We use the training data to fit the models we present below. Once we fit the models, we use the test data to assess how accurate the models are at predicting protest\index{protest}.
			
			In the pages that follow, we present the basic model-level results followed by diagnostic information to assess how including different categories of variables affects the model's overall performance. 
			
			\section*{Results}
			
			This section presents our models' results. First, we review the results of our models that estimate the number of observed protest\index{protest} events in countries across time. Our main finding is that larger US military deployments do seem to cause protests\index{protest} against both the US and the US military in particular. As expected, military deployments have a larger positive effect on anti-US military protests\index{protest} than they do on broader anti-US protests\index{protest}.
			
			Next, we review the results of the models that focus on individual-level protest\index{protest} involvement as reported by the respondents in our cross-national survey\index{survey} data. In general, we find that the greatest gains in predictive accuracy occur when we include individuals' experiences and attitudes, not just demographics, in the models. This finding carries significant weight for policy, as policymakers can more easily influence experiences than demographic attributes. 
			
			\subsection*{Macro-Behavior Results}
			
			The appendix contains table \ref{tab:antiusprotesteventmodels} and shows the results for the four models of country-year protest\index{protest} events. To ease interpretation, we also present a coefficient plot in Figure \ref{fig:coefplotcountryprotestmodels}. Models 1 and 3 present the basic models, while Models 2 and 4 include interaction terms designed to capture the expected conditional relationships between troop deployment size and protest\index{protest} activity. Across all four models, we find that there is a large positive correlation between the size of the US military deployment in a country and the number of anti-US and anti-US military protest\index{protest} events. Notably, the magnitude of this coefficient is roughly 3.5 times larger in Models 3 and 4 as compared to Models 1 and 2, suggesting that countries that host larger US military deployments are more likely to see a higher number of protest\index{protest} events in general, but that the difference is even greater for protest\index{protest} events aimed specifically at the US military. We will return to a fuller evaluation of the causal nature of this relationship below.
			
			
			%fix appendix references
			
			
			We find that several other variables appear to correlate with anti-US protest\index{protest} events. Although relatively rare during this period, countries involved as primary participants in a war that also involves the US tend to have a higher expected protest\index{protest} count than countries not at war with the US.\cite[Note that we use the countries listed in the PRIO/UCDP data's ``gw\_loc'' field. This lists the countries central to the incompatibility at the center of the conflict. This is a slightly broader way to group countries than looking only at countries where the US has invaded, but narrower than including all countries who participate in broad military operations like the ones we see in Afghanistan and Iraq.][]{Gleditschetal2002,Pettersson2019} Still, the models indicate that a US war leads to a fairly sizable increase in the expected count of protest\index{protest} events. This difference applies to both the general anti-US protest\index{protest} models and the models focusing only on protests\index{protest} against the US military. As with the troops coefficient, the coefficient is larger for the military protest\index{protest} models than the more general protest\index{protest} models. However, US wars in the region surrounding the country in question do not appear to correlate with a similar increase in protest\index{protest} events. More generally, domestic conflict within a state also appears to correlate positively with anti-US military protests\index{protest} during this period. However, we find no differences between conflict and non-conflict states regarding the number of more general anti-US protest\index{protest} events. This is likely due to the strong relationship between US wars and domestic conflict during this period ---in our data, there are 0 observations where a state is at war with the United States and \textit{not} coded as experiencing a domestic conflict as well. 
			
			The variable for the broader protest\index{protest} environment in a country also produces a positive coefficient. This coefficient indicates that countries, where protest\index{protest} in general is more common, also tend to have a higher number of anti-US protests\index{protest}. However, the magnitude here is fairly small, and we only find a clear positive coefficient for the more general anti--US protest\index{protest} models. The coefficients from the anti-US military protest\index{protest} models are close to zero, and a large share of the distribution of the coefficients falls above and below 0. Though a more precise causal analysis of this particular claim is beyond this project's scope, this lends some support to the claim that a more robust protest\index{protest} environment can be leveraged to support anti-US protest\index{protest} movements. Similarly, we find some evidence of spatial clustering of anti-US protests\index{protest}. Countries with a higher number of anti-US protests\index{protest} in the surrounding region also tend to have a higher expected count of anti-US protest\index{protest} events.
			
			
			\begin{figure}[t]
				\centering\includegraphics[scale=0.7]{../../Figures/Chapter-Protests/fig-country-coefficients.png}
				\caption{Coefficient plots for predictor variables of different protest\index{protest} events. Points represent the median point highest density posterior estimate and 50\%, 80\%, and 95\% highest density intervals. The histograms represent the distribution of the simulated draws from the model.}
				\label{fig:coefplotcountryprotestmodels}
			\end{figure}
			
			There are also some structural features of countries that correlate with higher expected protest\index{protest} counts. Countries with larger populations tend to have a higher expected count of anti-US protests\index{protest} than less populous countries. However, we find that this is limited to the more general anti-US protest\index{protest} models and find some evidence that these more populous countries simultaneously see \textit{lower counts} of protests\index{protest} against the US military. We also find some moderate evidence that more strongly democratic\index{democracy} and more strongly autocratic\index{autocracy} regimes tend to see a higher count of protests\index{protest} against the US and US military. The posterior distribution for these coefficients does overlap with 0, denoting some uncertainty around these results, but a large portion of the distribution falls on the positive side. 
			
			\subsubsection{Assessing the Causal Effect of Troop Deployments on Protest\index{protest} Events}
			
			The results presented in the previous section point to a positive relationship between the presence and size of US troop deployments in a country and protest\index{protest} events against the United States. As we discuss above, we take additional steps to explore the degree to which this relationship may be causal, meaning that the US deployments are causing the increase in protest\index{protest}, not just that the two are correlated. Using the marginal structural models we discuss in the research design, Figure \ref{fig:atetroops} shows the estimated average treatment effects (ATE), and also the effect of the treatment history of troop deployments for the two sets of models we estimate. The results of our marginal structural models indicate that US military deployments do, on average, have a positive contemporaneous effect on both anti-US protest\index{protest} events and anti-US military protest\index{protest} events more specifically. The ATE estimates for the anti-US protest\index{protest} model appear in the bottom two panels, and the estimates for the anti-US military protest\index{protest} model appear in the top two panels. In each panel, we display point estimates and credible intervals for the ATE, along with histograms showing the distribution of the ATE based on 10,000 simulations for each iteration of the model. As we noted in the previous section, US military deployments can increase or decrease sharply in a particular set of cases makes estimating the propensity scores and treatment weights difficult. To assess how sensitive our results are to the choice of truncation points for the weights, we present the estimates for each of six separate truncation points. 
			%%This is something for later, but we probably will want to have a couple of sentences explaining what the truncation points actually mean, for our less methodsy readers. CM.
			
			
			\begin{figure}[t]
				\centering\includegraphics[scale=0.7]{../../Figures/Chapter-Protests/figure-ate-troops.pdf}
				\caption{Plots show the predicted average treatment effect of US military deployments on protest\index{protest} events. Points represent the median point estimate and 50\%, 80\%, and 95\% highest density intervals. The histograms represent the distribution of the simulated draws from the model.}
				\label{fig:atetroops}
			\end{figure}
			
			Briefly, we find evidence of a positive average treatment effect (ATE) for the troop deployment variable in both models, indicating that countries that see an increase in the size of US military deployments tend to see an increase in the likelihood of an anti-US protest\index{protest} event, but also protests\index{protest} against the US military, more specifically. Notably, the multiple iterations of the model also show that the ATE estimate is indeed sensitive to the choice of truncation points, but the evidence generally indicates a positive effect. For example, in the general anti-US protests\index{protest} we find average treatment effect coefficient estimates ranging from approximately $-$0.06 to $+$1.24. However, the only negative value appears in the model run using the 10,000 IPTW truncation point, which is extremely large. The IPTW truncation point of 10 produces a mean IPTW score of just over 1, which is relatively close to the guidance given by Cole and Hernan.\cite{ColeHernan2008} Otherwise, we find that the models using the IPTW truncation points of 10, 50, 500, and 1,000 all produce relatively similar positive estimates, and we generally find that as we increase the truncation point, the error around the ATE estimates narrows and the coefficient decreases slightly. 
			
			When we look at the models predicting anti-US military protests\index{protest}, we see that the ATE is estimated to be positive across all of the IPTW truncation points, ranging from 2.68 at the lowest truncation point to 3.85 at the highest truncation point. As in the previous model, we find some evidence that the dispersion around the ATE estimates shrinks as the truncation point increases, but nowhere near as dramatically as in the more general protest\index{protest} model. It is also notable that the estimates are considerably larger for protests\index{protest} against the US military than the more general anti-US protest\index{protest} model.  And while we do see some shift in the median ATE estimate across the various models, there is a considerable amount of stability among the ATE estimates for the four highest truncation points. This is because there is far less variation in the anti-US military protest\index{protest} variable as compared to the more general anti-US protest\index{protest} variables, as can be seen in Figure \ref{fig:protesthistograms}. The maximum number of anti-US military protests\index{protest} is 4 in a given country-year, while we observe multiple cases where there are 5 or more protest\index{protest} events in a given country-year. Recalling the research design, we calculate the structural weights based on the treatment variable. These essentially tell the model how many copies of a given observation to make when generating the pseudo data for the model. This tells us that, after a certain point, the introduction of additional observations from the pseudo data does not fundamentally change the relationships between the predictor and the outcome variable (protest\index{protest}).
			
			\begin{figure}[t]
				\centering\includegraphics[scale=0.7]{../../Figures/Chapter-Protests/fig-histogram-protests.png}
				\caption{Histograms showing the counts of anti-US and anti-US military protest\index{protest} events, 1990--2018.}
				\label{fig:protesthistograms}
			\end{figure}
			
			
			Notably, the treatment histories differ slightly in their expected effects. Larger deployment histories appear to cause a reduction in the expected number of anti-US protest\index{protest} events in a given country-year. As with the contemporaneous effects, higher IPTW truncation points produce estimates with less dispersion, but the median point estimates themselves are fairly stable. Alternatively, we find that evidence is more mixed regarding anti-US military protest\index{protest} events, where we find some indication of a positive effect. At the lowest IPTW truncation point we find the coefficient value is quite close to 0 with about 50\% of the posterior for the ATE estimate falling above and below 0. For the other IPTW truncation points, we find roughly an 87\% chance of a positive effect.
			Regarding the more general protest\index{protest} model, this negative effect of the treatment history makes some intuitive sense. Cases like Germany\index{Germany}, South Korea\index{South Korea}, and Japan\index{Japan} have long histories of hosting large US military deployments. After some time, these deployments may provoke less general hostility towards the United States. However, the possibility of a positive effect of treatment history on protests\index{protest} against the US military also makes intuitive sense. In these same cases, the presence of the military itself frequently remains highly contentious---even as individuals themselves frequently have close personal and professional relationships with US service personnel in and around base communities. Even as individuals may become more familiar with Americans, reducing more general hostility, the presence of a long-term military facility suggests continued exposure to various negative externalities of the sort we have discussed, including pollution, crime\index{crime}, environmental degradation, and more. These stimuli are understandably likely to provoke a continued reaction among the host-nation's population. However, the results also suggest that this dynamic is more uncertain than the more general protest\index{protest} model.
			
			%Add something like this to explain what the ATEs mean: In other words, an increase of in troop deployments would lead to an increase of .21 to .27 protests in a given country-year.
			
			\subsection*{Micro-Behavior Results}
			
			This section reviews the results of our predictive models of individual protest\index{protest} participation. The appendix includes table \ref{tab:predictiveopinionmodels} and demonstrates the coefficient estimates and some model fit statistics for our models of protest\index{protest} involvement. First, we assess the general predictive power of the models. Then we provide a brief discussion of select inter-group comparisons we discuss in the theoretical section of this chapter. Specifically, we focus on the coefficients for personal contact, network contact, and personal experiences with crime\index{crime}. Our earlier chapter discussed how minority\index{minority} populations within the host country might bear a disproportionate share of the burdens and costs associated with hosting a US military facility or large deployment. Accordingly, we also look at how minority\index{minority} respondents compare to non-minority\index{minority} respondents with respect to their likelihood of participating in anti-US protest\index{protest} activity.
			
			
			We take multiple steps to assess the predictive power of the models. To provide a quick initial visual assessment, we first present a series of separation plots in Figure \ref{fig:sepplotcombined}, each of which corresponds to one of the five models we discussed above. Greenhill et al. present the separation plot as a means of visually assessing model performance.\cite{GreenhillWardSacks2011} After estimating the models using the training data, we use the models to generate predicted probabilities for each observation in the test data. We then order the observations according to those predicted probabilities, with observations receiving the lowest predicted probabilities appearing on the left side of the plot and those observations with the highest predicted probabilities appearing on the right. Finally, we color-coded each observation according to whether the binary response category for whether the individual participated in an anti-US protest\index{protest} is a yes (1) or a no (0).  Red indicates cases where the response was ``Yes'' and blue responses that were ``No''. The better the model performs the more red (``Yes'') observations we should see clustered on the right side of the panel, corresponding to higher predicted probability values. The five panels of Figure \ref{fig:sepplotcombined} correspond to the models shown in Appendix table \ref{tab:predictiveopinionmodels}. The ``Intercept Only'' model contains only the intercept values with no predictor variables. The model on the bottom is the full model, including the individual and group-level predictors, along with the varying coefficient estimates.
			
			\begin{figure}[t]
				\centering\includegraphics[scale=0.7]{../../Figures/Chapter-Protests/figure-sep-plot-combined.pdf}
				\caption{Separation plot from various models of individual-level protest\index{protest} behavior. The black line shows the in-sample predicted probabilities of protest\index{protest} involvement. We ordered the observations from low to high predicted values. The vertical red lines indicate observed ``Yes'' responses from the respondents. More red clustered towards the right side of the figure indicates stronger predictive performance.}
				\label{fig:sepplotcombined}
			\end{figure}
			
			From this figure, we can see that the predictive performance of the models generally increases across the first three models. In the Intercept Only model, while there is obvious clustering towards the right side of the graph, there are also many red observations scattered across the rest of the graph space. As represented by the black line, the predicted values are largely flat across the range of the graph, increasing only step-wise at the far right end of the figure. We see some improvement once we move to the basic demographic model in the second panel, though there is still a considerable dispersion of ``Yes'' values across the figure. The model, including both demographic and attitudinal/experiential variables, produces some fairly significant gains. Here we see both clustering of red observations on the right side and more confidence in the predicted probability values towards the right side of the panel. Adding the group-level variables produces some additional gains, as does the inclusion of the varying coefficients in the final model. In both cases, we see some additional reduction in the number of ``Yes'' observations scattered across the panel, with a greater share of protest\index{protest} attendees clustered towards the right side.  
			
			\input{../Tables/Chapter-Protests/model-protest-predict-stats.tex}
			
			In general, we can see that these models become progressively better at correctly classifying observations in the test data. However, the separation plot is still a blunt instrument, and our ability to more precisely discriminate between the predictive performance of each model is limited. To supplement the information from the separation plots, Table \ref{tab:predictivecheck} contains additional statistics on the models' performance and predictive power. The first column shows the percent of cases that the model predicts correctly---that is, the model predicts protest\index{protest} involvement where a respondent reports protest\index{protest} involvement and no protest\index{protest} involvement where a respondent does not report protest\index{protest} involvement.\footnote{Note that we use predicted probability values of 0.5 as the threshold for classifying predictions. Where $Pr(Protest | X_{ij}, Z_j)$ $\geq$ $0.5 = 1$ or ``Yes'', and $Pr(Protest | X_{ij}, Z_j) < 0.5 = 0$ or ``No''. This simply means that where, conditional upon the values for our set of predictors for individual $i$ in country $j$, $X_{ij}$, and the country-level variables for country $j$, we observed a predicted probability of 0.5 or greater, we classify that individual as having a predicted value of ``Yes'', and ``No'' otherwise.} At first glance, the models all appear to perform relatively well, with all models predicting at least 90\% of the cases correctly. However, given the overwhelming prevalence of ``No'' responses to the protest\index{protest} question in the data, this statistic alone can be misleading. Table \ref{tab:predictivecheck} contains several other statistics to help us understand their strengths and weaknesses. The false-positive rate indicates how the model incorrectly classified many individuals who replied ``No'' to the protest\index{protest} question as having been involved in protest\index{protest} activity of some sort. In general, the models all have relatively low false-positive rates, suggesting that we are not misclassifying too many ``No'' respondents. However, we do see some small increases in false positives as the models increase in complexity.
			
			The false-negative rate tells us how many ``Yes'' respondents the model misclassifies as not participating in protest\index{protest} activity. Here we start to understand better the models' weaknesses and the sources of divergence in their performance. For example, the false-negative rate of 82\% for the Intercept Only model means that the base model incorrectly classifies 82\% of respondents who self-reported having been involved in an anti-US protest\index{protest}. To rephrase slightly, if 100 individuals reported ``Yes'' when asked if they had previously attended an anti-US protest\index{protest}, the model would incorrectly predict 20 of these cases as ``No'' responses. We do see a reduction in the false-negative rate as the models increase in complexity, but the lowest score we achieve is $\sim$51\%, which is still not very strong.
			
			The sensitivity and specificity figures provide similar information. While the names may be less than clear, their meanings are perhaps slightly more intuitive than the false negative and false positive metrics. These metrics are complements, summing to 100\% (specifically, false-positive and specificity, and false negatives and sensitivity). Sensitivity tells us what percentage of ``Yes'' cases the model accurately predicts. Specificity tells us what percentage of ``No'' cases the model accurately predicts. A low sensitivity score means the model is doing a poor job of accurately predicting protest\index{protest}.
			In contrast, a high sensitivity score means it is doing a better job of accurately predicting it. This statistic is appealing because it more directly speaks to the outcome of interest, protest\index{protest}, and the model's ability to recognize it when it occurs. Related, high specificity scores mean that the model is doing a better job of accurately predicting the \emph{non-events}---those cases where protest\index{protest} involvement did not happen. 
			
			Predictably, the Intercept Only model does the poorest overall job of accurately classifying cases. The sensitivity score of 17\% indicates that the model correctly classifies only a small number of individuals who reported participating in anti-US protests\index{protest}. Alternatively, we can see that the specificity score of 99\% shows that the model did a good job accurately classifying those who reported never having attended an anti-US protest\index{protest}. However, such examples highlight the problems of relying solely on the percent correctly predicted metric---because the data are overwhelmingly ``No'' responses, it is fairly easy to classify those cases correctly. But the model does a poor job of recognizing ``Yes'' responses when it sees them. If we think of the model as akin to a genetic test---say one looking for some sort of chromosomal anomaly in a fetus---these results indicate that the test would correctly flag less than 20\% of such cases.
			
			The models do better as we add more individual-level variables and as we introduce more flexibility. That said, there are some diminishing returns. The largest increases in sensitivity come with adding the demographic variables and then again when we add the attitudinal and experiential variables (models 2 and 3). In each case, we see an increase of about 15 percentage points in the sensitivity score. This comes at the expense of specificity, which decreases slightly compared to the base model. Ultimately the most complex model only achieves a specificity score of approximately 49\%, meaning it is only correctly classifying about half of the instances where individuals have reported attending an anti-US protest\index{protest} event.
			
			There are a few things to note here. To start, these scores are sensitive to the choice of cutpoints for classifying cases. In table \ref{tab:predictivecheck} we classify cases according to a fairly na\"{i}ve predicted probability threshold of 0.50. There is an infinite number of alternative cutpoints that could be employed, but these entail trade-offs. Different predictive thresholds will affect sensitivity, specificity, and the overall share of accurate predictions. There are two ways that we can better understand the models' performance across these probability thresholds. First, Figure \ref{fig:protest-roc-plot} shows a receiver operating curve (ROC) plot for the five models we present. The ROC compares the true positive rate (specificity) and false positive rate ($1-$ specificity) for different probability thresholds. Curves closer to the upper left corner are generally better at correctly classifying observations (such as pairing a ``Yes'' prediction with an individual who responded ``Yes'' to the question about attending anti-US protest\index{protest} events). The idea here is that there is an inherent tradeoff in classification---we can correctly classify all ``Yes'' cases if we use a very low probability threshold and classify every observation as a ``Yes'', but this necessarily means we end up with an enormous number of false positives (meaning that we classify observations as ``Yes'' when they are really a ``No''). Curves that are closer to the dashed reference line are equivalent to a random classification scheme.
			
			
			\begin{figure}[t]
				\centering\includegraphics[scale=0.8]{../../Figures/Chapter-Protests/fig-roc-plot.png}
				\caption{Receiver operating curve (ROC) plot for the models shown in Table \ref{tab:predictiveopinionmodels}). This plot shows the percent of accurate ``Yes'' classifications relative to the percent of false positives (i.e. incorrect ``Yes'' classifications across a range of several different classification probability thresholds. Curves closer to the upper left represent better predictive accuracy.)}
				\label{fig:protest-roc-plot}
			\end{figure}
			
			From the curves presented in Figure \ref{fig:protest-roc-plot} we can see that there is indeed substantial variation in model performance when we account for the full range of probability thresholds. Ideally, we would want to see a model's curve spike up to 100\% immediately on the left side of the figure, but none of the models come close to this ideal type. The Intercept Only model performs the worst of the five, as we expect. Again, however, we see substantial improvements as we add demographic variables, and then demographic, attitudinal, and experiential variables. Ss this figure helps to make clearer, the relative performance of the three more complex models---including the one allowing for varying coefficient estimates---is nearly identical, with a very slight edge to the varying coefficient model. Ultimately, all of the models perform better than a simple random assignment mechanism. Still, some are clearly doing a better job of accurately classifying cases of protest\index{protest} involvement while sacrificing less in generating a larger share of false positives.
			
			\begin{figure}[t]
				\centering\includegraphics[scale=0.8]{../../Figures/Chapter-Protests/fig-pvalue-comparison.png}
				\caption{Comparison of sensitivity, specificity, and correct predictions across the full range of probability threshold values.)}
				\label{fig:protest-pvalue-compare}
			\end{figure}
			
			The disadvantage of the ROC plot is that it obscures the specific probability values and the trade-off between the different aspects of model performance that we seek to maximize. Figure \ref{fig:protest-pvalue-compare} plots the sensitivity, specificity, and percent predicted correctly figures across the range of possible classification probability thresholds. In general, higher probability thresholds make it more difficult for a given observation to make it into the ``Yes'' category. Conversely, lower thresholds mean that we are more likely to classify observations as ``Yes''. This is a seemingly trivial point, but it is worth reiterating because there is nothing special about the 0.50 threshold we use above---it is simply a convention of sorts and seemingly ``neutral''. But as we can see from Figure \ref{fig:protest-pvalue-compare} the trade-offs associated with moving between various probability thresholds are not all equal. In this case, our initial na\"{i}ve threshold of 0.50 sacrifices a considerable amount of performance on the sensitivity metric for relatively little gain on the specificity and overall correct prediction metrics. The percent of cases predicted correctly actually starts to decline slightly as we increase the probability threshold.
			To put it differently, setting too high a barrier for classification is also problematic as it means we miss an increasingly large share of cases. There are practical considerations in devising such a threshold as well. For example, if we found that the true probability of protesting\index{protest} were to be a 10\% chance, then we would be wrong nine of ten times. However, our model provides information that these cases stand out from even lower-probability observations. Part of the decision calculus in choosing a threshold is judging the utility of having more false-positive or false-negatives and how that informs decision-making or policymaking. If a policymaker wants to guard against protests\index{protest}, then increasing the number of false positives to capture the true positives is worthwhile. However, if the event itself is non-consequential or involves a severe policy response, then we should up our threshold and accept a higher number of false negatives. For example, suppose individuals who did not protest\index{protest} were subject to intensive surveillance or harassment because authorities \textit{believed} they were involved in protests\index{protest}. In that case, this sort of policy response could easily backfire and become a self-fulfilling prophecy that makes protest\index{protest} behavior \textit{more likely}. Returning to the table, we can see that lower classification thresholds generate relatively comparable figures for specificity and overall correct predictions across all five models while yielding much better performance on the sensitivity metric. Referring to Figure \ref{fig:sepplotcombined} helps to clarify this point, as the vast majority of cases have very low predicted probability values assigned to them. Using a lower probability threshold for determining which observations fall into the ``Yes'' category can thus improve overall predictive accuracy.
			
			
			\input{../Tables/Chapter-Protests/model-protest-predict-stats-targeted.tex}
			
			Table \ref{tab:predictivechecktargeted} shows the models' predictive accuracy when we use 0.10 as the probability threshold for predicting individuals' protest\index{protest} experience. In general, the overall percent of observations correctly predicted remains fairly high but has declined slightly. We also see the false positive rate has increased, which we should expect from having a much lower threshold. More importantly, the sensitivity score has increased substantially across each model. The minimum here is approximately 41\%, with a high of 77\%. In other words, using a lower probability threshold, our final model is correctly classifying three-quarters of cases where individuals report having participated in anti-US protest\index{protest} events. Again, this increase does come with a cost. The lower overall rate of correct predictions results from the slight decrease in the correct predictions of ``No'' responses. The increased false-positive rate also reflects this. 
			
			As one last check on our models' performance, Figure \ref{fig:ppcheck-individual-protest} shows the results of a set of posterior predictive checks using the five protest\index{protest} models discussed in this section. We generate 1,000 simulated data sets for each model, each containing simulated predicted Yes/No values for the outcome variable. For each data, set we take the mean value of these predictions, giving us the proportion of each simulated data set where the model predicts individuals to have responded ``Yes'' to the outcome variable question. The light blue histograms in each panel show the distribution of these predicted proportion values. The dark line shows the actual proportion of ``Yes'' responses in the real survey\index{survey} data. The narrow bands on the X-axis somewhat mask the fact that the actual distribution of simulated values is fairly narrow in every case. All five models produce reasonably close approximations of the real data, though the mean values are slightly below the actual mean value in every panel except the Intercept Only model. The model including only respondents' demographic characteristics produces the closest set of simulated values, though with slightly greater dispersion than the other more complex and fully specified models. 
			
			Ultimately the models all produce fairly accurate aggregate-level predictions regarding individual-level classifications. The appropriate balance for selecting probability thresholds is somewhat subjective. It depends upon the specific goals one has in mind. Still, these comparisons help show that our models are reasonably flexible and accurate when predicting individuals' likelihood of participating in protests\index{protest} against the United States. 
			
			
			
			\begin{figure}[t]
				\centering\includegraphics[scale=0.8]{../../Figures/Chapter-Protests/fig-opinion-ppcheck-plots.png}
				\caption{Posterior predictive check on individual-level protest\index{protest} models. The black bar represents the proportion of ``Yes'' responses to the protest\index{protest} attendance question in the test data (i.e. the number of people who say they've been to an anti-US protest\index{protest} event. The light blue area represents the distribution of the estimated proportion of protest\index{protest} attendees based on 1,000 simulations from the training data.).}
				\label{fig:ppcheck-individual-protest}
			\end{figure}
			
			
			In addition to the predictive accuracy of the models, we are also interested in assessing comparisons between groups. The individual coefficient estimates for a few key variables can help us to understand these patterns better. Figure \ref{fig:coefplot-population-compare} shows the coefficients for the minority\index{minority} status, personal contact, network contact, and experience with crime\index{crime} variables from four of the models (except the Intercept Only model). The estimates are all positive, and all fairly consistent across models. The coefficient on the variable capturing reported experiences with crime\index{crime} produces the most variation of the four. Individuals who have experienced a crime\index{crime} involving US military personnel are far more likely to be involved in anti-US protest\index{protest} events than individuals who lack such experiences. The coefficient values here are approximately 2, indicating that individuals who are victims of a crime\index{crime} perpetrated by a US service member are approximately 50\% more likely to participate in anti-US protest\index{protest} activity than people who responded ``No'' to this question. Contrast this with the minority\index{minority} and contact coefficients---0.85 and 0.62, respectively---indicating a roughly 20\% and 15\% higher probability of participating in anti-US protests\index{protest} than non-minority\index{minority} respondents and those who report not having contact with US service personnel. Individuals who have experienced a crime\index{crime} involving a US service member are much more likely to mobilize\index{mobilization} politically against the United States military than those who have not. 
			
			
			\begin{figure}[t]
				\centering\includegraphics[scale=0.65]{../../Figures/Chapter-Protests/fig-coefplot-population-opinion-model-compare.png}
				\caption{Coefficient estimates for models 1, 2, 3, and 4 presented above. These coefficient estimates exclude those for the varying coefficients model, which we include in a separate figure. 50\%, 80\% and 95\% credible intervals shown around median highest density probability prediction.}
				\label{fig:coefplot-population-compare}
			\end{figure}
			
			Finally, Figure \ref{fig:coefplot-varying-coefficients} shows the coefficient estimates from the varying coefficient model broken down by country. Though the general patterns look similar to those we see for the population-level estimates, there are notable differences. First, as with the population-level coefficients, we again see that individuals reporting having experienced a crime\index{crime} involving US service personnel are far more likely to report protest\index{protest} participation than those who have not had these experiences. While this difference is the largest across all 14 countries, there is some noteworthy variation in just how big these comparisons are. The differences between groups tend to be smallest in Kuwait\index{Kuwait}, which stands to reason given its authoritarian government\index{government} and the difficulty of protesting\index{protest} under such conditions.  The median estimate for Japan\index{Japan} is the largest of all countries, but the dispersion around the point prediction is quite larger. This results from only having 21 respondents out of $\sim$3,000 total respondents in Japan\index{Japan} reported having experienced such a crime\index{crime}. This is not the case in every country, however. In Italy\index{Italy}, 90 individuals reported such experiences with crime\index{crime}. In Portugal\index{Portugal}, 116 individuals reported experiencing a crime\index{crime} involving US service personnel. In Kuwait\index{Kuwait}, 837. We provide a fuller review of the relationship between US military deployments and crime\index{crime} in the previous chapter. Still, we should reiterate two points: although these cases are a clear minority\index{minority} in our data, we have a sufficient number of observations to generate reasonable coefficient estimates. The relatively small number of observations does lead to an increase in the error around the predicted estimates. Even allowing for this larger error, though, the entire range of these coefficient estimates tends to be quite a bit larger than the other coefficient estimates. 
			
			
			
			
			\begin{figure}[t]
				\centering\includegraphics[scale=0.65]{../../Figures/Chapter-Protests/fig-coefplot-varyingeffect-opinion-model-5.png}
				\caption{Country level coefficient estimates for the varying coefficient models of individual-level protest\index{protest} behavior. 50\%, 80\% and 95\% credible intervals shown around median highest density probability prediction.}
				\label{fig:coefplot-varying-coefficients}
			\end{figure}
			
			Second, there may be under-reporting on questions such as this one. We cannot currently say how this under-reporting may affect these inter-group comparisons. Individuals who choose not to report experiencing such crimes\index{crime} may also not wish to engage in protest\index{protest} activity against the US Survivors of sexual assault\index{crime!sexual assault} or other violent crimes\index{crime} may struggle with trauma in the aftermath of their experiences, seeking to avoid situations or activities that require them to interact with US service personnel or expose themselves to harm. The risk of violent confrontation with authorities, or even co-nationals, is higher at protest\index{protest} events, so individuals may choose not to participate. In such cases, we might expect the treatment effect of crime\index{crime} to vary based on other conditioning factors. For example, as we note in this paragraph, the type of crime\index{crime} may determine how individuals choose to mobilize\index{mobilization} politically against the US military.
			
			
			\section*{Conclusions}
			
			
			
			%%NOTE: Add something about the "base in your area"' variable once we have that in the models.CM.
			We opened this chapter with a conversation with a German\index{German} peace activist\index{activists} who recounted the struggles and successes the German\index{German} peace movement has faced in mobilizing\index{mobilization} the population against the US military presence. Somewhat counter-intuitively, he noted that while two-thirds of protesters\index{protest} mobilized\index{mobilization} by his organization are regional, ``the number of locals involved in the movement is growing, but not fast enough.''\cite{berlinone20190723} While it may seem surprising that activist\index{activists} organizations would have a hard time recruiting from those communities that are closest to a US military base and, therefore, more likely to feel the negative effects of it, our analysis finds that some of these observations can be explained by both demographics and by individuals' experiences and perceptions.    
			
			%discussing the successes and limitations of organizing anti-base activity in Germany. In Germany, anecdotally, opposition to bases is strongest in urban centers and not near the bases themselves. 
			%Additionally, particular policies of the bases may drive organization while economic ties and good relations with service members may decrease that opposition.
			
			It is important to note that engaging in protest\index{protest} tends to be a higher-commitment form of political expression. Not everyone who dislikes the military presence will mobilize\index{mobilization} and protest\index{protest} against it.  The activist\index{activists} told us that he thinks there must be 500,000 people adversely affected by US bases, yet the maximum turnout they can get at protests\index{protest} is 5,000 (``This gap is our challenge,'' he noted).\cite{berlinone20190723} While he expressed some bafflement at the fact that ``they are unhappy, but they are not acting,'' he also seemed to have a strong understanding of why this is the case. The collective action\index{collective action} problem is difficult for opposition groups that want to increase their numbers while the German\index{German} and local governments\index{government} and the United States are simultaneously working to obtain support for their positions. Despite the 75-year history of US bases in Germany\index{Germany}, the German\index{German} anti-base movement still faces challenges to overcome the collective action\index{collective action} problem. Some of them stem from demographics, but others from the interactions between locals and the US military.
			
			%Given the infancy of the anti-base movement, despite the 75 year history of US bases in Germany, suggests that overcoming that organizing to protest\index{protest} requires a substantial effort of organizers. At least that is the case in Germany. 
			
			
			
			%After survey\index{survey}ing the literature, our expectations remained that those who have stronger incentives to protests were more likely to do so. Like previous literature, the conditions that favor reward participation in political action make protest more likely.
			
			This chapter looked beyond individuals' preferences and views on the US military to see how those views manifest in political action. Considering both events-based and survey\index{survey} data, we identified the conditions that predict anti-US military and anti-US protests\index{protest} and protest\index{protest} participation. We find evidence that there is indeed a causal relationship between the number of US troops deployed to a foreign country and the number of protests\index{protest} in that country against the US and the US military more specifically. While this may seem obvious, establishing causality is vital because of the substantial policy implications it carries. When deciding whether to deploy more troops abroad, the US government\index{government} should be aware that the increase in troops, even when there is already an existing deployment in place, will likely lead to a rise in anti-base protests\index{protest}. In addition, this effect extends to more general anti-American protests\index{protest}, not just those focused on the military presence. A more prominent military presence can result in the expression of more anti-American sentiment.
			
			%\ref{cha:meth}
			Of course, reducing its military deployments abroad is not always a realistic option for the United States government\index{government}. We argue that the effect of the military presence on protest\index{protest} can be attenuated by various other factors, many of them within the control of the US government\index{government} and military. In general, as we found in 3, contact and economic relationships can reduce the negative perceptions of the United States military. These interactions could produce basing situations that would be less likely to experience anti-US protest\index{protest} as a reaction to a military deployment and thus could create better hosting environments for the US military. As found in this chapter, if the contact is negative, that can make people more likely to want to protest\index{protest} against the US military. 
			
			%Though these are predictive, not causal models, they do provide information about the settings in which anti-US protests are most likely to occur in. 
			
			There is also variation in whether individuals will be more or less likely to want to protest\index{protest} against the US military presence within countries. Some of this propensity is related to demographic characteristics. People who are poorer and in democratic\index{democracy} countries are more likely to self-report participating in an anti-US military protest\index{protest}. In Chapter \ref{cha:min}, we discussed the unique place minority\index{minority} members in a host state have relative to bases. They are more likely to pay the costs of the base (through location, environmental consequences, and negative interactions) while reaping less of the benefits (national security\index{national security}, direct economic ties). We find continued support for this argument in that self-described minority\index{minority} status correlates with an increased likelihood of participating in protests\index{protest}.
			
			
			Beyond demographics, people's experiences correlate with the probability of responding yes to the question about having participated in anti-US military protests\index{protest}. The roles of contact, economic reliance, and respondent social network continued to play important roles as they did when we focused on the perception of US actors. While the models for the survey\index{survey} data were correlational and predictive, we did see that those with more contact (exposure) were more likely to protest\index{protest} a military base. Likewise, economic reliance correlated with increased protest\index{protest} attendance. Exceptionally negative experiences, such as being the victim of a crime\index{crime} committed by a US military member or knowing someone who had been the victim of such a crime\index{crime}, also correlated with individuals being more likely to attend more anti-base protests\index{protest}. This is an essential point related to policy since the US cannot change host country demographics, but it can influence the types of interactions that service members have with the host country publics. It is thus cause for optimism to know that effective policy can affect many of the determinants of protest\index{protest} participation.
			
			As the United States enters an era that requires more consent in maintaining its basing agreements and hopes to continue to received substantial burden-sharing from allied states, the consent and support of the host country population will become increasingly important. Even though individual members of the population do not make hosting decisions themselves, their protest\index{protest} activity can influence their governments'\index{government} policies towards the United States. When protests\index{protest} are successful and influence public opinion\index{public opinion}, they will make the basing more costly for the military. 
			
			%loop back to the supra argument
			
			
			
			
			
			\begin{comment}
			


%\section*{Introduction}
\vspace*{-0.5cm}
\rule{\linewidth}{0.10pt} \\[-1cm]
{\footnotesize\paragraph{Summary:}  Previous chapters examined how military deployments affect beliefs and attitudes. This chapter turns to focusing on individuals' behavior. US military deployments have long been cited as causing negative externalities in host countries. These negative events may help to mobilize opposition to the US presence. Drawing on new country-level protest data and individual-level survey data our analyses yield a number of important findings. First, larger US troop deployments cause more frequent anti-US protest events. Second, our models of individual behavior correctly classify more than 90\% of survey respondents' involvement in anti-US protest activities. These models show that individuals' attitudes and experiences---not simple demographic traits---offer the strongest predictive power in determining who participates in anti-US protest events. Finally, crime victimization in particular is a very strong predictor of protest involvement.} 
\\[-0.5cm] 
\rule{\linewidth}{0.10pt}

\vspace*{0.5cm}

In the middle of an extraordinary July heat wave that was sweeping through continental Europe in 2019, while locals sought refuge from 40 degree Celsius temperatures in the few air-conditioned restaurants they could find, we sat in a small office in a nondescript building in Berlin's Mitte district. ``It's a long story,'' began the German peace activist, as he leaned back on his chair with two young interns looking on. 

In the interview that followed, this particular activist, whose organization focuses broadly on expanding the peace movement in Germany and specifically on protesting against the US Air Force's Ramstein Air Base, explained the history of the German protest movement. He discussed how the peace movement peaked in the 1980s and how, afterwards, the German population did not place as much importance on bases, leading to frustration by local activists. 2005 marked a turning point, as the activist launched a new appeal against Ramstein Air Base. Before we could ask, ``Why Ramstein?'' (the US, after all, has 87 military facilities in Germany \cite{DOD2018}), he volunteered the information. Ramstein has one specific characteristic that separates it from the others: It is the only air base in Germany from which military personnel operate remotely piloted aircraft (also known as Unmanned Aerial Vehicles, UAVs, or drones). As the activist noted, it is a key point to transmit the signals from Nevada and New Mexico for the drones to ``go elsewhere'' \cite{berlinone20190723}. 

This interview, like many others we conducted, illustrated how the presence of foreign military personnel is not itself sufficient to cause host-country citizen grievances against US overseas bases. That is, mobilizing against Ramstein is effective because of the remotely piloted aircraft use and not necessarily because of animosity towards the service members themselves. In the case of this activist's group, they did not want Germany to be morally complicit with the US military's drone strikes in Central Asia. The actions the military takes at Ramstein Air Base (or the belief about what actions take place there) and the resulting activist focus on the base for those actions suggest that the particular behavior at specific bases may motivate opposition to the presence of the US military. 

In 2019, opposition to the Ramstein's activities escalated from domestic protests to a set of legal challenges in the German court system \cite{Kloeckner2019,Reuters2019}. The specificity of the opposition implies variation in the number of protest or mobilization that a base can generate. Knowing what conditions encourage activists and others to protest the United States and its military is important to understanding opposition mobilization.  Protests against Ramstein center around both militarism and drone usage. Protesters at Henoko-Oura Bay in Japan (in Okinawa) protest against Ospreys, US occupation, offenses by US marines, and the environmental destruction caused by new base construction (especially its effects on the coral reef and the dugong population) \cite{Hibbett2019}. At Osan Air Base and Camp Humphreys in South Korea, protestors organized against racial injustice and for Black Lives Matter in the United States \cite{Sisk2020}. 
%I thought that this was a key point to highlight, that it's not just a simplistic "anti-base"' thing, that the actions that the bases take can actually affect the amount of protest that happens. This is kind of like the point that Cliff Morgan brought up at Peace Science, that we should focus on the stuff that the US can actually change. 

%%Relate vignette to quetions
A major aim of the German activist's peace organization is to disseminate information about the US military bases and their activities to the German (and international) public. Their expectation is that as German citizens become more aware of the negative effects of the US presence (such as air pollution from jet exhaust), and what Germans implicitly condone by hosting the US military, they will become more willing to mobilize against it. The group believes that their movement is growing. When we interviewed them, they told us that they had achieved record participation levels that they expected to continue to grow. The German activist recalled fondly how, during one year, protesters formed a human chain around the base and that, in the summer of 2019, the annual protest at Ramstein had 5,000 attendees and 50 different workshops and events.  

%%Set up puzzle
It is easy to see how a US military base could lead to negative reactions from locals and how larger bases can create more opportunities for negative interactions. Military jets are noisy and pollute the air through high emissions. Fuel waste that seeps into the soil can contaminate local drinking water supplies. US service members sometimes get into fights at local bars or drive drunk \cite{kasernetwo20190725}. An American government relations officer at a US base in Germany corroborated those last two points; when we drove into the base there was a wrecked car on exhibit near the entrance with a sign warning US service members to not drink and drive. In the UK, we heard consistent complaints that bases were noisy and worsened traffic. Given that many of a base's negative externalities occur in its immediate proximity, we might expect that it would be those located close to a military installation who would react the most negatively and be most active in opposing the base. But this is not always the case. In comparing the United Kingdom and Germany more broadly, it is clear that the amount of protest in each country, as well as the specific issues that trigger them, are different. 

This variation presents us with various questions. Why, despite having very similar histories with US military basing, are these two countries so different? At the macro-level, what makes Germany more or less likely to experience protests than the United Kingdom? Or, more broadly, what factors make anti-US protest events more likely in some countries and less likely in others? And at the micro-level, does a German have a higher or lower threshold of willingness to join a protest to express their dissatisfaction than a Briton does? What factors make some individuals more or less likely to participate in protests against the US military?

In our interview with the peace activist, we asked whether he would put us in contact with activists in local organizations found in other towns near military bases. In what we found to be a very honest and self-aware response, he noted that in many cases activists have struggled to build local peace movements because of the deep and long-lasting relationship between the US military and communities close to US bases. Locals simply are less willing to protest the US military. Some of it of course relates to the local population's economic dependence on the US military (``bakers, auto shops, and prostitution'' were the examples of businesses reliant on the US military given by the activist). Those who rent housing to US members benefit heavily from the presence. The housing allowance to service members appears to be common knowledge and tends be more generous than what most locals can afford. Local rental prices around military bases tend to match what soldiers can afford and not reflect local wages. This makes bases popular with property owners. Even beyond economics, the activist noted that the US military and local governments will hold common parties, public events, and ``friendship meetings'' to build community relations. ``It makes it not easy [to establish a local protest movement],'' he said. 

%%State chapter's research question

%switched "negatively"' to "`directly'' , since we just mentioned positive externalities too
To an outside observer, it may seem puzzling that those the US military presence affects most directly are also the ones least likely to protest it. Yet, Germany is not an isolated case. \citeasnoun{Fitz2015} finds that in the case of the US military base in Manta, Ecuador, the protest movement was based in the capital city of Quito, while locals in Manta reacted not against the base, but against the \textit{activists}. Locals may complain about inconveniences when pressed, but those concerns alone are not enough for them to mobilize against basing. Minor inconveniences may pale in comparison to the economic benefits of well-payed Americans patronizing local establishments.

%\ref{cha:meth}

Our argument in Chapter \ref{cha:meth} noted that both economic benefits (such as the military patronizing auto shops) and contact (like the ``friendship meetings'') can create more positive beliefs about the US military. It is not surprising that those closest to the base, those most likely to have contact and receive economic benefits, are also less likely to protest. Yet, it stands out that we also find a significant negative effect of contact on respondent beliefs about US actors. We have argued that while most interactions with the US military tend to be positive  (and somewhat casual), those negative interactions that do occur, particularly ones that harm the local individual in some way, can also create or reinforce negative perceptions, and mobilize individuals against the US military. This chapter extends our analysis further to not only understand another vector of negative perceptions, but also the conditions that can change negative perceptions into political action. Negative perceptions, or grievances generally, are not sufficient for anti-base or anti-US activism. This chapter's contribution, exploring the link between perceptions and actions, is fundamentally important to both academics as well as policymakers. This chapter explores both protest mobilization at the macro-level (``Which countries are more likely to experience protest?'') and the micro-level (``What makes an individual more likely to protest?'') to see how negative experiences and negative perceptions may, or may not, manifest in political behavior. 

%%State importance of research question
Even if the negative effect of contact with the military on perceptions is smaller than the positive one, it does not require a majority of members of a population to be opposed to a military base to effectively mobilize, particularly if those who feel positively about the base have only weakly positive feelings. Activists, those people willing to invest time and effort into organizing for a given cause, are usually in the minority in most societies \cite{Burstein2002}.  This is true of most anti-US base activists as well \cite{Fitz2015}.  At the same time, because activists, similar to lobbyists, care so much about a particular cause that others do not have as strong preferences over, they are willing to expend large amounts of effort and are more willing to incur costs than others would be.  Because of this, activists can obtain their preferred policy outcomes even when they are in the minority. As articulated in the ``3.5 percent rule'' developed by \citeasnoun{Chenoweth2011}, simply having 3.5 percent of a state's population involved in active, non-violent protest can be enough to lead to a successful outcome and achieve the protesters' aims. 


%%going to link this to the theory chapter. I think we can build up a pretty good argument based on the protest literature that mobilization and perceptions do matter to the regime. 
%\ref{cha:theory}
A major point that we made in Chapter 2 was that public opinion matters to regimes and that perceptions of the US military in host countries can indeed influence the stability of the hierarchical relationship between the host country and the United States. Research shows that protest and mobilization can influence political actors at the national level to alter policy actions, even when such actions are controversial, such as granting more rights to minority groups \cite{Gillion2013,Fassiotto2017}. In particular, stronger and clearer expressions of public opinion are more likely to influence policy \cite{Baumgartner2015,Fassiotto2017}. When governments make decisions on policy, they face an overwhelming amount of information, some of it gathered by the government itself (through intelligence agencies, for example). At the same time, journalists or activists can directly transmit some of that information to the government \cite[p. 15]{Baumgartner2015}. Protesting is a way in which the public broadcasts its preferences to the government, thus influencing their policy choices. 


Even though some have discounted the role that public opinion plays in leaders' policymaking decisions, existing work shows that leaders do care about public opinion, and are hesitant to engage in policy actions that go against it, in fear of potential political costs \cite{Tomz2018}. While there is variation in how responsive  different types of governments will be to their population's preferences, it is still true that protests are one way in which people communicate information to governments.\footnote{We also note that protests have influenced even non-democratic regimes in leading to policy change. The Arab Spring protests were an example of a case in which non-violent protest was able to achieve concessions, if not outright regime change in most cases \cite{Chenoweth2013}.} Anti-base protests can be an example of how even small groups of activists can obtain their desired outcome (removal of a base, for example) through organization and effective mobilization \cite{cooley2008}.  Even if anti-base protests are rare, it is important to understand their determinants, as these protests can have a large influence on US foreign policy.  This chapter asks the question of what situations are more likely to lead to local populations protesting a US military presence. Likewise, as shown by the case of Ramstein in Germany, this chapter highlights the idea that the United States and its military can take actions that increase or decrease the likelihood of mobilization against its presence. If we understand what causes protests to emerge, we may be able to understand what policy options reduce the likelihood of overseas opposition to peacetime deployments.  

%%Preview our argument (we probably need to add a bit more here on the specific argument once we nail it down a bit more)

Just as important as studying when anti-US base protests occur is studying cases in which they do not occur.  As noted by \citeasnoun{Fitz2015}, there are indeed cases in which communities in close proximity to US military installations do not only not protest, but actively mobilize in favor of the US base. South Korea provides examples of both pro- and anti-US military presence mobilizations that battle over the positive and negative effects of the bases. Particularly of interest are cases in which local communities are less likely to protest the US military presence than individuals in other parts of the country.  This fits with our argument that individuals who have more direct contact  with deployed personnel are more likely to support the US military presence, even in the face of other domestic opposition.  Alternatively (or in addition to this previous explanation), it is possible that the US is choosing to locate some of its military installations in areas whose geographic or demographic characteristics are less likely to mobilize opposition into protest; the US may base in more remote locations far away from the urban centers and the populations within them that are more likely to mobilize. 


%I don't think we're actually studying counter-protests (which would be interesting, but outside our scope, so cutting this out. CM. 
%Mobilization of any sort is not without costs, so it is especially interesting to also consider what makes people to mobilize against anti-US protests.

%%More macro implications

%%Moving this paragraph into main theory chapter
%While negative views of the US by host country publics are bad in and of themselves, they are also problematic for US foreign policy. As individuals' negative views towards a US military presence deepen, they will be more likely to mobilize (be that through voting, social media posts, legal action, or, as we explore in this chapter, protest) against the military presence. As we have previously noted, acts of dissent, like protest, have the potential to impose costs on host country leadership. The government may in turn try to pass those costs on to the United States as a condition for continuing to host the troops. Popular opinion mobilization has removed United States forces from the Philippines and Spain and limited the functional operation of the United States Air Force in Turkey during the build up to the 2003 Iraq War \cite{cooley2008,Kakizaki2011}. Sustained opposition fed by grievances fundamentally weakens the US position and its ability to maintain its troop presence, or at least makes it more costly.

We note the importance of studying specifically what the determinants of both \textit {anti-US} and \textit{anti-US base} protests are, both at the individual and national level. As we discuss in earlier chapters, we need to better understand the degree to which anti-military sentiment affects attitudes and behaviors vis-\'{a}-vis the United States more generally, and vice versa. While the relationship between US military deployments and host-state populations is complex, there are basic empirical and causal questions that still require systematic examination. In particular, we seek to answer the most general question of whether an increase in the number of deployed US military personnel increases the probability of seeing these types of protests. We have previously discussed the idea that deployments and interactions between host-state residents and US personnel can have mixed effects. But attempting to assess what the ``net'' effect might be remains a challenge. In the following sections we pursue multiple research goals. First, we look to see if there is a clear causal relationship between the size of a US military presence and protest events against the US and its military personnel. Second, we then examine the characteristics of both countries and individuals that make protest most likely. In general, we expect that a larger military presence will make protests more likely. We also expect that there are several individual-level characteristics that make participation in protests more or less likely, and that these factors may vary across countries and regions. 

%This seems to be less of a focus now, so I greyed it out. CM. 
%It is certainly the case that some areas of the world, or certain subnational areas within countries, are more prone to protest than others.  We are therefore interested in cases of citizens that choose not to protest US bases but do protest other policies, or alternatively, citizens who do not protest other policies but do choose to protest US bases.  Consequently, we study US military bases in a broader context of all protests in our temporal domain, regardless of the target of the protest. 

%%Outline the chapter

In what follows of this chapter we will discuss research on protest and mobilization in general and anti-US protests in particular. We then use a cost-benefit analysis framework to develop theoretical expectations on the determinants of protest at both the state and individual levels, focusing on the causal path that leads from military deployments to protest. Following, we set up our models for both our newly collected protest data and the survey data we have used throughout the book. We then use both sets of data to create two types of models. The first model examines the causal relationship between troop deployments and protests at the macro-level. The second model uses a predictive model to see how individual characteristics and relationships with the US military may inform us about people's likelihood to attend protest events. We conclude with policy implications focused on the actions that the US military can take to reduce anti-base protests as well as the actions that activist groups can take to broaden the appeal of their message to local populations. %surveys will be 2018-2020 eventually.
% We will also describe our newly-collected data, which includes both self-reported, individual-level data on protest participation drawn from surveys conducted in 2018, and events data on anti-US military protests. Not sure about this last point, but figured it was a way to make it sound like we're not just writing a manual for the US military. CM. 
			
\section*{Research on Anti-US Protests}

To better understand protests against the US and its military facilities, we take two different tracks to disentangle the empirical relationships that appear to be either causal of or related to protest behavior. First, we examine protests through a macroscopic lens by considering the likelihood that a country experiences protests in a given year. This allows us to identify structural conditions that make protest more likely. Second, we return to our survey data and study the conditions that correlate with protest behavior at the individual level. What do our surveys say about individual behavior? Are there demographic, ideological, or geographic factors that are associated with an individual attending an anti-US protest? What predicts the likelihood of an individual engaging in protests against the United States or military bases? To better understand both sets of frames, macro and micro, we first turn to the established literature.


\subsection*{Macro-Behavior Theoretical Expectations}
 
%%Moved these two paragraphs to theory chapter

%There is a significant amount of work, particularly from a qualitative perspective, which has studied how it is that anti-US base protests can be successful in influencing host country policies. Work by Yeo \citeyear{Yeo2011} shows that successful anti-base mobilization efforts stem from two processes. First, the country must have a fractured elite security consensus. This fracture allows a protest movement ``air to breathe'' without the entirety of the country's elite establishment denying it. Since much of the public will take their cues from elites, many will avoid protests in the presence of an elite structure that stands united behind the idea that the host country shares security threats with the United States, that the country needs help from the US, and that American forces are present for these purposes. The opposite is also true: in the presence of prominent elites who question such tenets, individuals will be more likely to question the presence of American forces and join protest movements. 

%Second, Yeo explains that successful protest movements include broad sections of the host state's society. This includes variation across ethnicity, gender, income, region, and more. Protest movements confined to single demographics are unlikely to succeed, as there is less incentive for host governments to respond to a fraction of the population. When host governments face broad coalitions of anti-base protesters, their political survival is at stake and it is difficult for them to build a counter-coalition. A broad-based movement also signals the salience of an issue by cutting across demographic lines that would normally remain isolated or even opposed to each other. While we do not purport to explain the success of protest movements, theories from \citeasnoun{Yeo2011} stress the importance of studying the determinants of protest. If certain factors make individuals more likely to engage in protest, and these protests can indeed be successful in changing host country policies on US  military bases, then protest movements can fundamentally weaken the United States' international position and its ability to maintain a global military presence.  

%%%

%for the theory, situate it in the context of bad contact = dissent/opposition and fundamentally weakens the US' position/ability to stay

When explaining anti-US protests, we first discuss the determinants of protest and mobilization in general which are likely to also be drivers of anti-US protest.  We conceptualize protest as a form of collective action.  Like other forms of collective action, such as rebellion, protest can run into free-riding problems \cite{Olson1965,Lohmann1993}.  If others protest, a given individual can reap the benefits of the protest (policy change, for example) without incurring any of the costs (lost time, possible arrest, suffering violence, etc.).  Much like rebellion, or even non-violent actions such as voting, a single individual is unlikely to make a difference in the probability of a movement succeeding, but if enough individuals follow the same logic and attempt to free-ride off others' participation, the movement can fail \cite{lichbach1993}.  Therefore, we focus on the type of situations that make collective action, in this case anti-US protests, more likely to take place.  

One strand of explanations of collective action and mobilization focuses on relative deprivation; \citeasnoun{Gurr1968} provided much of the foundational work on relative deprivation in the early literature on the causes of civil war.  The idea behind this theory is that the societies that are most vulnerable to domestic political violence are not those that have the poorest people in the world, but those in which individuals are relatively worse off when compared to others within their own society. When individuals perceive that they are not receiving the benefits that they believe they deserve (as they can observe others in their society receiving) they will be more likely to experience anger. This anger becomes the basis for their mobilization into action.  In other words, if there is a dissonance between what people expect and what people receive (more colloquially, those that have and those that have-not), individuals will be more likely to overcome collective action problems and mobilize into action. 

In the case of anti-base protests, we should expect to see more anti-US protests in cases where the population expected to receive greater benefits from a US military presence but instead the negative externalities have outweighed the benefits. This situation will be even more dire when there is a separation between those that receive the benefits of military deployments (e.g. landlords, business owners, and people in urban centers) and those that bear the burden of the costs of deployment (e.g. marginalized peoples, rural communities, minority groups, etc.). We can thus think of the population's decision calculus in determining whether to take part in protests as a cost-benefit analysis where they weigh the benefits that they obtain from the US military presence against the costs associated with it. The greater the costs are compared to the benefits, the more likely we are to observe populations mobilizing in order to attempt to influence the host government to remove the US military. 

There is likely to be variation in the specific factors or behaviors members of the host population are likely to find objectionable. That said, we argue that the larger the US military presence is, all else being equal, the greater the \textit{opportunity} for locals to have grievances about the US military deployments. More specifically, larger deployments should lead to increases in many specific factors that appear likely to stoke anti-base sentiment, such as crime, environmental degradation, increased traffic, noise pollution, industrial accidents, and more. Accordingly, we begin by exploring the simple question or whether or not larger US military deployments cause an increase in protest activity. Simply put, assuming that there is some proportion of members of the US military who will engage in objectionable behavior, having a greater number of forces present increases the probability of negative interactions that can mobilize the population into protest \cite{allenandflynn2013}.

%Andy, I'm guessing you added this hypothesis below. Does the explanation above fit with what you were thinking? CM. 

 %more stuff on grievances

\begin{hyp}
	All else being equal, larger US military deployments in a country should lead to a higher likelihood of anti-US protests within that country.
\end{hyp}

This will be our primary motivating hypothesis for our initial set of country-level models. There are several other factors that will influence protests as well, and we include them in our models. Though we do not offer explicit hypotheses about these factors, it is important that we take them into account to adjust for their influence and accurately predict instances of protest. While we expect troop increases to lead to an increase in protest, the size of this effect will be moderated by the environment in which troops operate. This leads us to expect there to be variation in how likely different countries are to protest a US military presence.  

%There is, of course, variation in how likely populations are to protest a US military presence, regardless of the size of the deployment. 

When considering what makes countries more likely to experience a protest against the US military presence in a cost-benefit analysis context, we can think about the kinds of settings in which the benefits of the military presence accrue to all members of the population, not just a specific group (such as supporters of a specific government). The cases in which benefits accrue to (almost) all members of the population are less likely to experience protest, as even if costs associate with the military presence, they can be offset by the benefits provided \cite{Bitar2016}. 

We thus argue that when the military provides public goods the entire population will benefit from them.  When the goods provided by the military presence are private ones, the host state government will distribute these benefits to only its supporters \cite{demesquita2005}. In the setting of military deployments, we can think of the economic benefits that a military presence brings with it (such as contracts given to local contractors) as private goods and the security provided by the troops (such as deterring attacks against the host country) as public goods. We argue that in those states in which the military is providing security to the population we will observe fewer protests, as security, unlike economic benefits, tends to be a public good that everyone, not just supporters of the government, can benefit from. In theory any country that has a US military presence receives security, but the utility of that security to the population can vary. In particular, we argue that in countries experiencing internal or external threats, the general population will derive more utility from the US military presence. We thus expect that states that are facing higher levels of internal or external threats will be less likely to experience anti-US protests.



%This example doesn't really fit anymore, so i greyed it out for now
%In the case of the US military base in Manta, Ecuador, which the US gave up in 2009 after pressure from the Ecuadorean government and civil society, crime in the area surrounding the base actually increased after the departure of US troops \cite{Fitz2015}. 


% (NOTE: maybe?  I'm just kind of making stuff of now as I think through it, will come back through many times later and clean all of this up)

%NOTE: I changed this to internal and external threats; I think this makes more sense. We need to expand on the threat part, though. We can just poach from the bit in the original paper which says that countries that face higher external threat are more likely to have favorable opinions of the US military



%\begin{hyp}
%All else being equal, anti-US base protests are less likely to occur when there is an internal rebellion in the host country.
%\end{hyp}

The literature on the ``rally round the flag'' effect also supports this expectation. States facing either internal threats (like terrorism) or external ones (like aggression form a foreign rival), garner bumps in leader credibility and unification of opposition parties with the governing party \cite{Chowanietz2011}. Internally, public opinion shifts in favor of executives when a new crisis emerges \cite{lee1977,Norrander1993}.  Researchers have not deeply delved into whether support for alliances increases when countries find themselves under threat, but there is evidence host states tend to match US expenditures in their territory when they face a collective threat \cite{allenetal2016}. Regardless, if the argument is generalizable, support for the government in the face of a security threat should increase support for the state's security commitments as well.

%\begin{hyp}
%All else being equal, anti-US base protests are less likely to occur when there are high levels of external threat against the host country.
%\end{hyp}



Related to the idea of relative deprivation, when people are doing better economically than they expected to, they should be less likely to want to engage in mobilization. For example, a State Department Regional Analyst at the US embassy in Panama noted that locals felt positively towards the US because of the strong Panamanian economy and low levels of unemployment that they attributed in part to US investment in Panama \cite{embthree20180712}. In contrast, when people are experiencing economic hardship, they might become dissatisfied at the American military presence.\footnote{Though we note that at extreme levels of economic downturn, which would lead to extreme poverty, we may see less mobilization. As noted by an interview subject, ``When you have to fight for your day to day surviving [sic], you cannot be active [against bases]'' \cite{berlinone20190723}.}  This may be due to individuals believing that they, in particular, should benefit economically from the base but are not able to gain that sought-after benefit while seeing others enjoying their desired economic fortune.  If they are not receiving any such benefits, anger may manifest from dissatisfaction and protesting becomes an outlet for that dissatisfaction. We thus expect economic downturn in a country to correlate with more anti-US protest.

%%expand on this and cite more stuff here. CM.


%\begin{hyp}
%All else being equal, anti-US base protests are more likely to occur when there is economic downturn in the host country.
%\end{hyp}

%Work by \citeasnoun{Yeo2011} also shows that successful anti-base mobilization efforts are produced by two things. First, the country must have a fractured elite security consensus. This fracture allows a protest movement "air to breathe" without being completely denied by the entirety of the country's establishment. Since much of the public will take their cues from elites, many people will avoid protests in the presence of an elite structure that stands united behind the idea that the country shares security threats with the United States, that the country needs help from the US, and that American forces are present for these purposes. The opposite is also true, in the presence of prominent elites who question such tenets, individuals will be more likely to question the presence of American forces and join protest movements. 

%Second, Yeo explains that successful protest movements include broad sections of the host state's society. This includes variation across ethnicity, gender, income, region, and more. Protest movements that are confined to single demographics are unlikely to succeed, as there is less incentive for host governments to respond. When host governments are faced with broad coalitions of anti-base protesters, their political survival is at stake without being able to lean heavily on other demographics. It also signals the importance of the issue, in that the anti-base grievances are not confined to a single group. 

%While we do not purport to explain the success of protest movements, theories from \citeasnoun{Yeo2011} help us make predictions about the propensity of protests. Since successful protest movements are likely to be larger and last longer, and since successful protests stem from fractures in the security consensus and broad-based support within protests movements, we can make predictions about the likelihood of protests. 
%** NOTE: These hypotheses seem endogenous at best. Thoughts? - AS **

%\begin{hyp}
%Higher rates of negative opinions toward the US military presence will result in higher protest rates.
%\end{hyp}

%\begin{hyp}
%Higher rates of negative opinions toward the US military presence will result in larger protests.
%\end{hyp}

%\begin{hyp}
%More demographically cross-cutting levels of protest participation will be correlated with higher protest rates.
%\end{hyp}

%\begin{hyp}
%More demographically cross-cutting levels of protest participation will be correlated with larger protests.
%\end{hyp}

%**NOTE: We would have to use aggregate numbers from the survey instead of individual responses to test each of these hypotheses. It limits the sample to only surveyed countries and only for the years we have surveys. Not sure if that will be sufficient to find worthwhile results.**


%\ref{cha:theory}
Finally, we note that the decision calculus that populations engage in when deciding whether to protest a US military installation also involves the probability of the protest being successful. Even if there are many costs associated with a US military presence, if the population does not believe that it will be likely to achieve its aims (of removing or modifying the presence) through protest, it will be unlikely to mobilize. This relates to the broader point that we make in Chapter \ref{cha:theory} about why public opinion affects host-country governments and how the presence of US troops may mobilize their populations.

%this used to say that regime type is an indicator of protest success, but i changed it to just protest, since we don't actually measure protest success. CM. 
We argue that the host country's regime type will be a strong indicator of protest. As discussed by \citeasnoun{Murdie2015}, the relationship between protest and the openness of regime type is curvilinear.  In the most open regime types, it is easiest to organize a protest without fear of government repression.  At the same time, in these systems there are alternate, legitimate avenues through which grievances can be expressed, such as a functioning judicial system through which the political opposition can challenge the US military presence \cite{Bitar2016}. In contrast, in the most closed systems the protest is less likely to produce results and is also more likely to lead to repression by the government. We therefore expect there to be more anti-US protests in states with medium levels of openness.

This fits with observed trends in opposition to US military installations. \citeasnoun{Bitar2016} argue that the US has shifted its basing approach in Latin America to informal bases (rather than formal ones) because opposition to the US military presence has grown in these countries as they have democratized in recent years, but have yet to achieve full democracy status. \citeasnoun{cooley2008} also argues that political officials of anocratic states are more likely to politicize US military bases as a means of garnering mass levels of support. When a state's institutional structures are in flux, this type of politicization can activate latent nationalism within the public. With the base used as a point of friction within the electorate and as an example of the leader's nationalist bona fides, protest movements can erupt in support of nationalist candidates and in opposition to US basing, as they did in Uzbekistan, the Philippines, and Spain during anocratic periods. Without the institutional structures that regularize the basing relationship, this politization can make the basing contract uncertain \cite{stravers2018}. 

%This paragraph seems repetitive, so I cut it for now. 
%Democratic countries have institutional structures in place that incorporate the preferences of the population into government action, and the government can then act on these preferences within the normal course of government and the legal structure. Autocratic states also tend to have more developed institutional foundations than in anocratic states, though these structures often simply stifle the preferences of the public and do not allow for protests to occur and protest movements to develop. Anocracies lie between these two extremes, with insufficient preferences aggregation combined with often personalistic leadership styles along with sufficient openness to allow protests (oftentimes spurred by the government itself) to occur.

 %Even today, we see similar uses of US basing by leaders in Turkey and the Philippines during periods of democratic backsliding. However, despite numerous historical examples of this sort, there has been little systematic work done to examine whether such regimes actually produce more protest activity, though they have tested regime type's relationship to different aspects of basing . We thus derive our next hypothesis:  
 
%**Note - was this Bitar portion completed? I jumped to the next paragraph in case it wasn't.** It ws def not complete, rigth now it's all splashing ideas at the page, feel free to edit/add/etc CM.


%\begin{hyp}
%Anti-US base protests are more likely to occur in states that have medium levels of political openness. 
%\end{hyp}


Finally, when considering the cost-benefit analysis that determines whether host country populations engage in anti-base protests, we also consider the situations in which some of the costs of protest are sunk costs. An insight from our qualitative interviews is that embassy personnel tend to be careful about planning military exercises when there are other, unrelated protests occurring (or expected to occur) in the area.  They noted that if there was already a group of people mobilized and protesting, and they found a grievance against the US military, it was very easy for the protest to turn into an anti-base protest \cite{embone20180712,embthree20180712}.  Given that one of the major challenges of collective action is the act of mobilization, we expect that once individuals have been mobilized for a protest (once the initial costs of protest have been sunk), even if it is for a different cause, it becomes easier to organize a protest against US military installations. Thus, we expect countries that experience more protest in general to also be more likely to have anti-US and anti-base protests. 

We want to again emphasize that our current approach focuses primarily on identifying the causal effect of military deployments on protest at the country level. To do this we require a solid theoretical framework to help us develop a model capable of doing so. This set of expectations helps us form the necessary components to create a reasonable quantitative model for predicting protests by country. We will return to this topic later. We now turn to the relevant literature and theoretical expectations for our micro-behavioral model. 

%\begin{hyp}
%All else being equal, anti-US base protests are less likely to occur when there are there are high levels of other, unrelated protests occurring in the host country.
%\end{hyp}





%\begin{hyp}
%Anti-US base protests are less likely to occur when members of the US military are engaged in humanitarian and development work with local communities. 
%\end{hyp}

%**NOTE** This last hypothesis will be really hard to test, since it will be hard to know when the military is engaging the local community, but I'm leaving it in here for now just in case.  Might be something we poke into in the future (related to the work Mike A and I did with Ali). 





 
%Note sure we're actually going to get to this, so greying it out for now.CM. 

%It is important to distinguish between different types of protest.  We note that some protests are more fleeting, where they may last only a day and not mobilize further support or lead to significant change.  There are also protests that are part of a larger social movement that contain a civil-society component that provides them with better organization.  These are the types of protests that would be considered civil resistance \cite{Chenoweth2013}. In addition, protests that are connected to domestic or international organizations are able to draw more individuals to the protest through their connection with the organization and its network \cite{Bell2014,Murdie2011}.

%NOTE: Erica Chenoweth has a data project where they count crowds at protests, but I don't think they've published anything academic from it, just Monkey cage blog posts. I did cite her stuff on the ``3.5 percent rule''

%NOTE: We want to think about the distinction between violent and non-violent protest and look at what leads some protests to turn violent. The Chenoweth and Stephan book finds that non-violent protests are more effective than violent ones (can be more inclusive, can involve more kinds of people, do not need guns, can be spoken about more openly).



			\subsection*{Micro-Behavior Theoretical Expectations}
%\ref{cha:meth}
Predicting protests at a  macro level suffers some shortcomings for drawing conclusions about the relationship between military deployments and the act of protest. Primarily, there is an ecological inference issue where assuming individual motivations and actions from higher levels of aggregate information could be erroneous \cite{King2004}. We cannot be certain that information at the national level, such as the number of troops in a country, actually affects the population we seek to study (those that choose to protest). While we can make reasonable inferences that connect these concepts, we would be relying upon conclusions that our data do not directly speak to. Consequently, we also analyze the microfoundations of protest, as there will likely be a difference in how individuals make the decision to participate in protest versus how protests are explained at the national level. To better understand individual attitudes toward protest, we again use our survey data (as detailed in Chapter \ref{cha:meth}). 

Before considering individuals' protest cost-benefit analysis, we first note that in many cases, participation in an anti-base protest involves coming into contact with members of the US military, as protests often occur at the site of the military installations themselves (or if this is not allowed, as close to them as possible). Thus, while we do not imply causation between these two variables, we do expect there to be a positive correlation between protest participation and contact with members of the US military.\footnote{To try to separate those individuals who come into contact with the US military because they participated in an anti-US military protest from those who have everyday contact with the US military in their communities we will also test a hypothesis on the relationship between protest behavior and living in a province that houses a US military installation.} Analogous to our first hypothesis about states that have a greater US military presence being more likely to experience protest, we derive our first individual-level hypothesis:

\begin{hyp}
	All else being equal, individuals who have personal or network contact with the US military are more likely to participate in an anti-base protest. 
\end{hyp}

Beyond exploring the role of contact, there are several other attributes that may associate with the likelihood of attending an anti-US presence protest. When analyzing protest at the individual level, we can focus on the factors that influence decisions to protest. While we do not suggest causal relationships for these variables, we can find the conditions that better facilitate the ability for individuals to overcome collective action problems and mobilize in protest. As previously mentioned, the decision to engage in protest is essentially a cost benefit analysis, where individuals consider the costs and benefits of participating in a protest, as well as the probability of the protest succeeding.
 
Given this micro focus, we can focus on ideas, feelings, or experiences that we can incorporate into the rational choice framework we are operating under \cite{Elster1999}. For example, emotions can help the individual make a decision under rationality by affecting preferences over outcomes and determining what is most salient to the individual at a given point in time \cite{deSousa1990,Pearlman2013}. Using this framework, we explore how anger can affect individual decisions to engage in protest. As we discussed in the macro-level theory section, research on mobilization shows that anger can make individuals more likely to mobilize, even when there are costs involved in doing so \cite{Gurr1968,Goodwin2009}. In turn, the motivating emotion can then transform into a collective emotion of solidarity that is shared among a group and keeps the protest movement going \cite{Goodwin2009}. Anger is of particular relevance to mobilization because as \citeasnoun[p. 388]{Pearlman2013} argues, anger is an ``emboldening emotion''. This means that anger influences individuals' cost-benefit analysis by making them more optimistic about their possibility of success, in this case of achieving their aim through protest. Anger is an emotion that drives individuals to action in order to right a perceived wrong \cite{Carver2009}. This is important, because the costs and benefits of each expected outcome have to be weighed by the possibility of each outcome occurring, in this case the probability of success in achieving the aim of removing (or modifying) the US military presence through protest. When individuals experience anger, they can be driven to mobilize, even if the probability of success, and therefore the utility of protesting, would otherwise be low.  

%Note: The Garcia & Young piece notes that violent crime is more likely to provoke anger, so if we want to distinguish across different kinds of crimes, we can always cite that as justification

Of course, emotions are particularly difficult to measure, as even assuming that individuals respond sincerely when asked about an emotion in a survey, levels of emotion such as anger vary in time. The anger that made an individual willing to mobilize in the past may no longer be present at the time of surveying. We therefore focus on particularly traumatic events that are highly likely to be correlated with anger; in this case being the victim of a crime. Previous research finds that individuals who are the victims of a crime, particularly ones who perceive themselves as innocent, are more likely to experience anger (and in turn to engage in mobilization against governments who have failed to protect them from crime) \cite{Garland2012,Garcia2019}. We assume that if an individual was the victim of a crime perpetrated by a US service member, then they would likely fault the US military, at some level, for its perpetration. We expect that those individuals will be more likely to participate in anti-US protests. 

Crime victimization may serve an added, direct route towards mobilization. Specifically, by being a victim of a crime, the survey respondent has experienced a specific, direct cost of the military presence. The negative externality of US troop deployments concentrates on the individual. Given their experience, they are more likely to see the troops as a net negative in their community and engaging in protest is one way for them to express their preferences over the presence of the troops. We believe that those individuals who have been the victim of a crime perpetrated by a US service member will be more likely to respond to mobilization efforts to oppose the United States' military presence. 

 

%\begin{hyp}
	%All else being equal, individuals who have been criminally victimized by the US military are more likely to participate in an anti-base protest. 
%\end{hyp}

As we have mentioned before, a significant part of the individual's cost benefit analysis used in determining whether to participate in a protest is calculating the probability of success; deciding to participate in a demonstration is a costly act (it requires time, transportation, and represents a host of opportunity costs) and engaging in protest with little likelihood of success would not be an attractive opportunity. Consequently, we expect that potential protesters care about the effect of their actions. Much of the probability of success of public demonstration depends on the potential protesters' capacities. 

In the context of theories of mobilization, a second strand of explanations focuses on the resources that individuals have available to them. While grievance-based explanations are common for civil wars, they are less satisfactory in explaining the likelihood of conflict for a few reasons. First, grievances often present collective action problems for mobilizers and the existence of a grievance is not sufficient for them to get people to mobilize around unless the direct cost to the individual is significant. Even then, free riding on the efforts of others tends to be a more tenable option than risking the individual's own resources or life in some cases \cite{lichbach1993}. Second, grievances are ubiquitous across societies as there are always issues that people want remedied. Social entrepreneurs that engage in mobilization can certainly capitalize on particular grievances by offering selective incentives, but this suggests that grievances by themselves are not sufficient.  Consequently, mobilization will only occur when groups have resources available to them that allow them to mobilize, provide selective incentives, or lower the costs of participation \cite{Olson1965,Tilly1973,Khawaja1994}. For example, in the case of the Arab Spring protests, many pointed to the availability of mobile phones and social media, resources which facilitated coordination, as facilitating mobilization itself \cite{Hussain2013}.\footnote{Similarly, during the Iranian Revolution, opportunities for social gathering also led to increased protest mobilization \cite{Rasler1996}}


We expect that situations in which resources are available for coordination will ease protest organization in the case of anti-US base protests. Of course, what types of resources an individual needs to mobilize will vary from person to person. One commonality across individuals, though, is that greater proximity to others and ease of communications should decrease the cost of mobilization. To better account for this, we consider whether individuals live in urban areas, which are likely to have greater access not only to other individuals (because of higher population densities), but also more reliable telecommunications networks and public transport, which should ease the ability to coordinate and travel to protest sites. 

In addition, those living in large urban areas (such as capital cities) are more likely to come into contact with transnational anti-basing activist organizations that will work to organize locals against a military presence \cite{Murdie2011,Murdie2015,Kiyani2020}. Prior work on protest and mobilization shows that having access to transnational activist networks, which already have in place the resources and infrastructure to facilitate protest, facilitates mobilization and leads to increased probability of protest. For example, that the German peace activist interviewed for this project was based in the capital city of Berlin, and he talked at length about the international connections to other activist groups he had. He noted that his organization had recently held an international conference against bases and war, with representatives from 40 different countries attending, and noting that he often gets invited to anti-base events in Japan \cite{berlinone20190723}.\footnote{He also highlighted the importance to mobilization of having an activism infrastructure in place, noting that the German peace movement has a very good peace infrastructure.} In addition, exposure to transnational activism may lead to individuals associating the American base with a broader network of American imperialism \cite{Immerwahr2019}.  

Further, urban areas are also more likely to house those groups that are more predisposed to protest. A common theme in our interviews, whether we were talking to US or host-country government officials, was that, similarly to how certain demographic groups are predisposed to feel more positively towards the US military, other demographics are more likely to protest. For example, there seemed to be agreement that students and labor unions were more likely to mobilize against the US military presence, a point echoed in the existing base protest literature \cite{Fitz2015,embone20180712,berlinone20190723,journ20180712,journ20180713,Allen2020}. Activists themselves seem to be aware of this because of failed recruitment efforts. Regarding the city of Wiesbaden, which is near the Clay Kaserne Army Garrison, the German peace activist noted, ``It is a horrible city to start a movement. Demographics are difficult. It is where the Russian Czars would visit. Lots of rich pensioners live there. Wiesbaden has a nice historical past, but the demographics are difficult for a peace movement'' \cite{berlinone20190723}.

%We are not using the base itself as a unit of analysis, so taking this out. CM. 
%In particular, we expect that deployments which are located in more heavily populated, urban areas will be more likely to draw protest because of the ease of protesting close to a US military installation. Of course, individuals do not have to protest at a base location in order to protest against a deployment, but we believe that the physical presence of troops is more likely to draw protesters than more remote deployments
%This presents us with the following hypothesis:

%cut urban hypothesis
%\begin{hyp}
%All else being equal, individuals living in urban areas are more likely to participate in an anti-base protest. 
%\end{hyp}

%\ref{cha:meth}
Related to this point, and as we discussed in the introduction to this chapter, we believe that individuals who are part of a community that interacts regularly with the US military will be less likely to protest the US military presence. In agreement with previous work, in Chapter \ref{cha:meth} we have found that even though contact with the US military is correlated with increased probabilities of both positive and negative views of the US military, the positive effect is stronger than the negative one \cite{Allen2020}. We have argued that an important reason for this is the fact that American personnel often become a part of local communities, creating both ``bridging'' and ``bonding'' opportunities between US personnel and host-state citizens \cite{Woolcock2000}.  This creates a sense of shared identity or experience that allows locals to overcome negative stereotypes about the US military. 

%Changing this up a bit to make sure we are differentiating from the contact hypothesis that we start the chapter with. This is about being part of a community that is close to a base.
Local residents are therefore more likely to support bases than more faraway residents who are more removed from the actual troops \cite{Fitz2015,Flynn2018}. Residents are more likely to feel a sense of affinity with the US forces. Over and over again during our interviews, regardless of which country we conducted them in, we heard reinforcing evidence for the idea that individuals whose communities were more proximate to US deployments were more likely to view the US military positively. This was true whether the interviewee viewed this relationship positively or not. 

%As noted in Chapter \ref{cha:meth} and in our previous work, this is an effect that is particularly strong in communities that are proximate to US bases or deployments, as there is also a network effect. Even if an individual has not had direct contact with a member of the US military others in their social network might. Individuals will be able to gather information about the US military from their social network \cite{Mcclurg2006,Huckfeldt2001}, having others in their social network interact with the US military leads to a tolerant environment  in which prejudice against the US military decreases and positive views of the US military increase \cite{Wright1997,Liebkind1999,Pettigrew2007}

For example, at the US embassy in Panama, a variety of individuals we met with referred to anti-US protesters as ``paid protesters'' and noted that most Panamanians felt positively towards the US because  of the long-term nature of the US presence and the ``strong cultural ties formed between American and Panamanian people'' \cite{embone20180712}. They specifically noted the high frequency of marriages between US service members and Panamanians. This point was confirmed by an interview with a former Panamanian President, who also noted that US Panamanian relations ``have been going very well'' \cite{embthree20180712,pres20180714}.\footnote{The President noted that intermarriage also created some problems, as Panamanians who married American service members often became stateless after renouncing their Panamanian citizenship but before acquiring American citizenship \cite{pres20180714}.} The same former President, when we asked him about more recent deployments that provide humanitarian assistance (such as Beyond the Horizon and New Horizons) said that they are viewed positively by the people who receive them, who perceive the economic benefits that they receive from them \cite{pres20180714}.

From the opposite perspective, a journalist in Panama City who noted that he had previously participated in anti-US marches made no secret of his suspicions over humanitarian outreach carried out by deployed US military personnel. As he talked to us at a mostly empty grilled meat restaurant he had asked us to meet at, he expressed concerns that the US had these programs in place to spy on Latin America; that the humanitarian help is just a disguise. He noted that the people who benefit from these programs (most of whom reside in rural areas) had ``low education levels and critical analysis skills'' and did not analyze the ``political implications'' of the help they received. Yet he acknowledged that they felt positively towards the deployments \cite{journ20180713}. A former Panamanian Cabinet Member, this one speaking to us at a much higher-end restaurant, was much less suspicious about the secrecy of US deployments. At the same time, she was also unenthusiastic about humanitarian-oriented deployments, noting that civilian personnel should deliver US aid. Yet she also noted that Panamanians who had regular interactions with deployed personnel had positive views of the US military. Her particular example involved sex workers, who she said had a preference for US military clients, as they were considered to be more physically fit and attractive \cite{journ20180712}. 

Given this, we expect that those individuals living in areas in which the US military is incorporated into society will be less likely to participate in anti-base protests. Even though we note in our initial hypothesis that contact with the US military makes protest more likely, by the simple logic that protest itself in many cases involves coming into contact with US service members, we also expect that individuals living in provinces that house a US military installation will be less likely to participate in an anti-base protest. 

%[NOTE: Add more here if needed. We can draw from interviews, etc. This is where we are linking directly to our theory from Ch. 2, so it should feel like it's a strong part of the chapter]

%\begin{hyp}
%All else being equal, individuals living in provinces that house a US military installation are less likely to participate in an anti-base protest. 
%\end{hyp}

%%%This might be too hard to do, not erasing it, but this seems like too hard of a road to go down**
%*****Note: we should have a hypothesis about the role of transnational activist groups.  A transnational activist presence should make protest more likely, but the transnational activists have to be perceived as sincere and credible, or otherwise they may draw backlash against them (Fitz-Henry 2015). 

%As state above, and as argued by \citeasnoun{Fitz2015}, the association of a military base with a broader idea of American imperialism, rather than with a local, bilateral and consensual agreement between the US military and the local government, is more likely to lead to opposition to the base.  Quasi-bases, which are ``foreign military bases that are not supported by a formal agreement'' are easier for governments to hide from potentially hostile opposition and civil society actors\cite{Bitar2016} (p. 50). For example, in Latin America, as countries democratize and leaders face more political backlash from hosting US military bases that do not provide widespread benefits that include political opposition, the US has come to depend on informal agreements under which US military personnel deploy to existing military installations rather than building new American bases \cite{Bitar2016}. 

%Thus, we expect that US military bases will experience less protest in cases in which they are housed within an existing military installation that also houses the host country's military.  As \citeasnoun{Fitz2015} notes, bases that are framed as forward operating locations rather than bases per se (which are viewed as more permanent and intrusive), will be more likely to draw negative attention and thus experience protests. 



 

%**NOTE: The USAF base in Manta is one example; I'm guessing that Lakenheath will also be a good example of this.  Hopefully we can insert a good anecdote about it here.  Bases in Germany and Japan are examples of the opposite.CM.**

%\begin{hyp}
%All else being equal, anti-US base protests are less likely to occur when US troops are housed alongside host country troops in an existing military installation. 
%\end{hyp}

%\begin{hyp}
%All else being equal, anti-US base protests are less likely to occur when US troops are present as part of a forward operating location rather than a permanent base. 
%\end{hyp}

%NOTE: add transition

While some of our interview subjects, particularly those who represented the US government and military, referred to US military spending and investment as contributing to pro-US views, we do not expect protest to correlate with pure measures of military expenditures in the country or in the area.\footnote{For example, an interview subject at the US embassy in Panama attributed the high level of acceptance for the US in Panama to long-term and widespread US investment in Panamanian infrastructure \cite{embthree20180712}.}  As we have argued in previous work and as other studies have found (see for example \citeasnoun{Fitz2015} on the case of the city of Manta in Ecuador), economic motivation and direct financial gains only go so far in creating a sense of acceptance \cite{Allen2020}.  In fact, excessive spending can even lead to negative externalities such as inflation in the area and locals being priced out of their homes \cite{Hohn2010,Fitz2015}.  We instead rely on an argument that focuses on negative and positive views of bases being what leads to protests, rather than purely financial motivations.
%\ref{cha:meth}
Though we do not expect aggregate military expenditures in an area to influence protest, we do believe that at the individual level, economic benefits received from the US military may be related to protest participation. In Chapter \ref{cha:meth} and in our previous work, we show that personal or network economic benefits from the US military can lead to more positive views of US actors and the US presence in the host country \cite{Allen2020}. As we have noted earlier in this chapter, we believe that those individuals who perceive the US military more positively will be less likely to participate in anti-US military protests. Unlike contact with members of the US military, economic benefits are not something that we expect to result from participation in an anti-base protest. 

%For this research, we are less concerned about the direction of causality in this case and derive our last micro-level hypothesis: 


%\begin{hyp}
	%All else being equal, individuals who receive, or whose network receives, economic benefits from the US military are less likely to participate in an anti-base protest. 
%\end{hyp}

%I figured we should go with one or the other, so probably economic benefits being negatively correlated to protest behavior? 

%\begin{hyp}
	%All else being equal, individuals who receive, or whose network receives, economic benefits from the US military are more likely to participate in an anti-base protest. 
%\end{hyp}



%aid-provision by the US military, in particular humanitarian work such as investing in infrastructure and other community engagement activities, is likely to lead to a lower incidence of protest.  Having the military engage in activities that provide needed projects to local communities can lead members of these communities to view the military  installation as a more fair exchange between them and the American military. Previous work shows that deployments that specifically engage in development-oriented work, such as building or renovating schools and hospitals and providing immunizations and medical care are related with more positive perceptions of the US military \cite{Flynn2018}. In this way, deployments that are present for a non-development-oriented mission may be able to avoid the negative backlash against them by providing projects such as infrastructure development or other forms of engagement with the local population.  
			\section*{Research Methodology}

We have two primary aims for our analysis in this chapter. First, we are interested in evaluating the relationships between different predictor variables and the occurrence of protest events at the country level. To do this we first present a series of simple count models where we predict the number of protest events in a country--year as a function of several predictor variables. More specifically, we are also interested in what, if any, causal effect US military deployments have on protest activity at the country level. To address this more specific question we use a number of more advanced techniques to identify this specific causal relationship. Second, we are also interested understanding the factors that correlate with individual-level protest involvement. We build a series of predictive models of individual protest participation. Accordingly, our analysis in this chapter has four primary parts and we begin by focusing on protest events at the country level, followed by protest participation at the individual level. 

This section focuses on 1) understanding the factors that correlate with country-level protest events, and the causal effects of US military deployments on the occurrence of protest events across countries, more specifically; and 2) generating a basic predictive model to help us understand the different types of people who are likely to participate in anti-US protest events, and to examine comparisons between different groups and their likelihood of engaging in protest activities.



\subsection*{Macro-Behavior Models: What Causes Anti-US Protest Events?}

Above we discuss our theoretical expectations concerning the link between US military deployments and macro-level protest events. In this section will discuss the data collection, operationalization, and estimation strategies we adopt to assess the causal effects of our variables of interest. 

In this chapter, we use two country-level measures of anti-US protest---protests against the United States, in general, and protests against the US military, specifically. These variables are counts of the number of protest events that occur in a given country in a given year. To generate these variables, we use events data on worldwide protest events between 1990 and 2018 gathered from the Integrated Data for Events Analysis (IDEA) dataset \cite{Bond2003}. The dataset compiles events from a range of global news sources. Through a mixture of automated parsing procedures and manual data review, we refined the general search to focus on various types of protest events that involved 1) the mass physical mobilization of individuals, and 2) protest events that were aimed at the United States government, in general, or the United States military, specifically.\footnote{We selected all instances of protest events, which include protest demonstrations (event code 181, ``All protest demonstrations not otherwise specified''), protest obstruction (event code 1811, ``sit-ins and other non-military occupation protests''), protest procession (event code 1812, ``picketing and other parading protests''), protest defacement (event code 1813, ``damage, sabotage and the use of graffiti to desecrate property and symbols''), and protest altruism (event code 1814, ``protest demonstrations that place the source [protestor] at risk for the sake of unity with the target'').}

%\footnote{Interview subjects across different countries noted that there had been an increase in anti-US protests since Donald Trump became President of the United States in 2016, while we cannot control for the ``Trump effect'' in our micro level analysis, as interviews were conducted in 2018, 2019, and 2020, in our macro-level analysis we account for this by taking annual measures into account \textbf{stuff for flynn to fill in} \cite{journ20180713,mpone20190717}.}

The IDEA dataset identifies protest events as an action taken by one actor (the sender) towards another actor (the target). Both senders and targets can be parsed to varying levels of specificity. For our purposes we adopt focus on two specific sets of actors, as discussed above. First, we selected all cases in which the database identifies the target country as the United States, regardless of the more specific target (i.e. civilian government officials, military, etc.). For example, protests of the election of US President Donald Trump would be in this category. Our sample includes 531 cases of anti-US protests, distributed throughout time and space.\footnote{In some cases the events data lists the location of the protest broadly as a region such as ``Western Africa,'' or even more broadly, as ``World.'' Because the data does not include enough detail for us to identify the specific country or countries in which these protests occurred, we drop these cases from the sample.}

For our second measure, we further narrow our focus to protests that target the US military. In this case, we take the sample of anti-US protests and select the cases in which the target is identified as being the military; the data defines this as, ``official armed forces, peacekeepers, astronauts, military police, military justice officers, military academies, and border guards'' \cite{Bond2003}. This gives us a total of 29 anti-US military protest observations, with Afghanistan, Australia, Iraq, and South Korea having the highest numbers of anti-military protests. As with the more general protest measure, we omit cases that occur within the United States.

While our outcome measures focus on protests directed against the United States, the data collection process also produced a measure of total protests in each country-year. This offers a baseline of comparison for anti-US protests as it is likely the case that people in some countries are more likely to protest in general than they are in others. We note that the data set collects the protest data at the country-month level (meaning events are coded for every month), though in our analysis we aggregate it to the country-year level given that our other variables of interest are at the country-year level of analysis.  This allows us to calculate a variable that is the number of anti-US protests in a country year as a proportion of total protests, thus allowing us to both control for greater inclination towards protest in some countries as well as to study whether there is something specific about anti-US protests that makes their determinants different from protests overall. In addition, we create a metric that measures anti-US military protests as a proportion of all anti-US protests; this again allows us to test whether different factors determine whether people are likely to protest against the US military in particular or whether these are just a subset of anti-US protests that are determined by the same factors as non-military focused protests.           


\subsubsection*{Model Specification and Identification}

%Political openness (squared term to capture curvilinear relationship)

We approach our analysis of country-level protest events in two steps. First, drawing on the theoretical arguments outlined in the previous section we present four Bayesian multilevel negative binomial count models where we model protest events as a function of several predictor variables for the 1990--2018 period. These initial models will provide us with a baseline set of relationships to better inform our understanding of the correlates of protest events against the United States. 

We intend the second approach to examine the more complicated question of identifying the causal effect of US military deployments on protest activity. Several factors complicate the accurate assessment of the \textit{causal} effect of deployments of protest. Traditional regression approaches are limited in their ability to estimate causal effects for a number of reasons. One of the most difficult problems to deal with is that the history of military deployments is likely to matter as much as short-term deployments, but basic regression models typically only provide estimates of short-term effects. Further, these histories are highly variable across countries. This can severely bias the estimates from our model, as not every country has an equal opportunity of experiencing US military deployments---both in terms of hosting them, but also with respect to the size of those deployments. Finally, as is often the case, the relationship between military deployments and protest events may suffer from confounding effects as other variables may exert a causal effect on both of our variables of interest. Accordingly, modeling protest events as a function of military deployments and adjusting for many other variables will not yield results that have any meaningful interpretation as causal effects. 

To address these problems, we estimate a marginal structural model (MSM) \cite{Robinsetal2000,BlackwellGlynn2018}. These models have a fairly long history of use in fields like biostatistics and epidemiology and can be used to help estimate causal effects in observational studies where exposure to treatments varies across space and time. They can also be used to help us estimate the contemporaneous effect of a treatment, as well as the more general effect of a particular treatment history. However, MSMs require us to do some additional work before we can estimate the effects of troop deployments and their histories on the outcome of interest. There are likely to be systematic differences between the individual countries that receive troop deployments, or larger deployments, and those that receive no, or smaller deployments. Additionally, troop deployments in a given period (what we refer to as time $t$) are heavily dependent on deployments in the previous period (time $t-1$). Deployments at time $t-1$ are also likely to affect other predictor variables in subsequent periods. To resolve this, we must estimate a series of structural weights for each observation to be used in the final regression model. We calculate these weights using the formula found in Appendix section \ref{eq:structuralweights}.



More simply, the weights shown in Equation \ref{eq:structuralweights} are a form of propensity score that we can use to re-balance the observations in our data. This is simply a variant of the inverse probability of treatment (IPTW) weighting method.\footnote{See \citeasnoun[Chapter 13]{ImbensRubin2015} for more information on estimating propensity scores.} Normally estimating propensity scores is a relatively straightforward process as it is commonly used to estimate scores in data where the treatment, and often the outcome, are binary (meaning 0 or 1) variables. The fact that our treatment (the number of US troops present in a country) is continuous (meaning that it can in theory take on any integer value that is 0 or greater) complicates this process slightly.\footnote{For a fuller discussion of estimating inverse probability of treatment weights for MSMs see \citeasnoun{ColeHernan2008}. For a discussion of estimating these weights for continuous treatment variables see \citeasnoun{vanderwalGEskus2011} and \citeasnoun{Naimietal2014}. Finally, for a discussion of estimating weights for a continuous treatment in the presence of multilevel/grouped data using multilevel models, see \citeasnoun{SchulerChuCoffman2016}.} To generate these weights we first have to estimate two separate models wherein troop deployments themselves are the outcome of interest. The first of these models (the numerator) is relatively straightforward to estimate. Here we are running a regression model wherein we predict the size of the troop deployments in country $i$ at time $t$ as a function of a one-unit lag of the troop deployment variable, the cumulative sum of troop deployments in country $i$ from the beginning of its series up through time $t-2$, and a range of time-invariant covariates, $Z$. In cases where researchers are working with just a binary treatment variable and only a single period, using ``1'' as the numerator is often all that is required. However, given that we are dealing with a continuous treatment variable, generating the numerator using this regression-based approach can help to stabilize the resulting weights by ensuring that the ratios are not enormous (see \citeasnoun{ColeHernan2008}).

The second model (the denominator) is more complicated. In most respects it is identical to the numerator, with the exception of the gamma ($\gamma$) term, which denotes a vector of covariates---the key here is that these covariates have to be sufficient to meet the sequential ignorability criteria. In essence, this is requires that there is no unmeasured confounding between the treatment (troops) and the outcome (protests).\footnote{The language on this point can be confusing given that these techniques were developed across several different literatures and disciplines and compounded by the fact that they often did not speak to one another. To put it in slightly different terms using language from \citeasnoun{Pearl2009}, this set of covariates must ensure that Pr$ (Y_{it} \Perp X_{it} | \gamma, Z_{i}) $. That is, the outcome is conditionally independent of the treatment, conditional upon the time-varying covariates ($\gamma$) and the time-invariant covariates ($Z$).} We will return to this below, but once we have both of these models we then generate two sets of predicted values and residuals for each observation in the data set---one using the numerator and one using the denominator. For each set of predictions and residuals we calculate the probability of the actual observed value of troops in country $i$ at time $t$ using the predictions and the standard deviation of the residuals as the mean and standard deviation of a normal probability density function. We then divide the values generated using the numerator figures by the values generated by the numerator. The actual structural weight for any given observation is the cumulative product of these ratios for country $i$ from the beginning of that country's series up through time $t$. 

This represents a brief overview of the process of estimating the structural weights. But, as we note above, estimating the second model is complicated by the fact that we need a set of covariates that ensure there is no unmeasured confounding between the treatment and outcome variables. To aid in this process we turn to another tool---directed acyclic graphs (DAGs). Like with MSMs, scholars from other disciplines have more commonly used DAGs (such as computer science and epidemiology) and they serve a variety of useful purposes. First, they help make explicit the wider array of theorized causal relationships between various predictor variables. Second, assuming a well-developed theoretical model, they can also help to identify sources of confounding and bias.\footnote{See \citeasnoun{Pearl2009}, \citeasnoun{MorganWinship2015}, and \citeasnoun{KeeleStevensonElwert2020} for a fuller discussion of DAGs and their applications.} Related, they can be used to identify which variables need to be adjusted for in a statistical model to close off the ``back-door'' paths between the treatment variable and the outcome that serve as the sources of confounding in the model, but also to ensure that we do not adjust for variables that may open up such informational pathways between the treatment and outcome and introduce additional sources of bias (i.e. collider bias) \cite{Pearl2009}.\footnote{We use the {\tt dagitty} package in {\tt R } to build our DAG and to identify possible adjustment sets \cite{Textoretal2016}.}

To estimate these models, we first begin by constructing a theoretical model wherein we treat protest events as the outcome of interest and military deployments as our key predictor, or ``treatment'' variable. These variables themselves are embedded in a larger theoretical causal framework where they are themselves caused by, and cause, a variety of other variables. Figure \ref{fig:dagtroops} presents a directed acyclic graph (DAG) depicting the basic contemporaneous relationships between the variables of interest and their relationships between two time periods, $t$ and $t+1$ (which is the period directly following t). To ease interpretation, we present a simplified version of the full DAG in which we condense most of the time-variant and time-invariant predictor variables down to the time-invariant covariates ($Z$ terms). We also only present two periods, but the chain of events theoretically runs back to the first observation of $t$ in the series. The light blue node represents the outcome of interest (protests), while the light green node represents the treatment of interest (troops). Any node connected to another indicates a proposed causal relationship, and the direction of the arrow represents the direction of that relationship.

\begin{figure}[t]
	\centering\includegraphics[scale=0.8]{../../Figures/Chapter-Protests/fig-dag-protest-simple.png}
	\caption{Directed Acyclic Graph (DAG) showing the theoretical model of protest across time periods.}
	\label{fig:dagtroops}
\end{figure}

Using this model as our starting point, we are able to generate a number of adjustment sets---variables that, when included in the model, will ensure there is no unmeasured confounding. Notably, this rests entirely on the notion that we have the ``right'' theoretical model. There are a number of possible adjustment sets we can use given the current model, but they all accomplish the same basic task. If we have included causal relationships in our theoretical model that do not actually exist, or if we have omitted key causal relationships that \textit{do} actually exist, then the implications of the model and our ability to isolate causal effects may change. The degree to which these changes would affect our ability to assess the causal relationships we are proposing depends entirely on the specific connections that would change. Assuming for present purposes that our model is sufficient, we can now estimate the denominator described in Equation \ref{eq:structuralweights}.

Once we have calculated the structural weights using this method, we can now proceed to estimate the average treatment effect (ATE) of troop deployments on protest activity. To do so we estimate a series of multilevel negative binomial regressions predicting the number of protests in a country year using the troop deployment variable and treatment history (i.e. the cumulative sum of deployments through time $t-2$) as predictor variables.\footnote{See \citeasnoun{Hill2013} for more information on the use of multilevel models for causal inference.}  Notably, these models weight the observations according to the structural weights we calculated above. US military deployments are generally stable \textit{within} countries, with substantial variation \textit{between} countries. For example, Germany's troop levels are high and relatively stable over the period that we study. In other cases, troop level are low, but also relatively stable. There are, however, some cases where troop levels increase and decrease radically at various points throughout the 1990--2018 period. These sudden changes are typically associated with large military operations, like the 1999 war in Kosovo or the 2003 Iraq War. This extreme variability makes estimating the outcome models somewhat difficult because we end up with extremely large weights for a small set of observations. To address this problem, we estimate the models multiple times using several different truncation points for the weights to better assess the sensitivity of our estimates to the weighted sample. The weights effectively create a pseudo sample of data by replicating observations according to their score. For example, an observation with a score of 4 would be copied four times when we run the model. Given these cases where we see very rapid and extreme changes in US military deployment levels we encounter a situation where the weights are often enormous (e.g. weighting scores of 50,000$+$). There is some question about the proper size of the weights and how to deal with very large or very small weights. Because we are dealing with some exceptionally large weights, we estimate our outcome model multiple times using different truncation points to cut down on the number of extreme values, and to assess how sensitive our estimates of the ATE are to changes in the weights.\footnote{There is no clearly ``correct'' way to deal with extreme weight values. \citeasnoun{ColeHernan2008} note that in the context of binary treatment variables the average of the weights should be approximately 1, but beyond this there is relatively little guidance of which we are aware. One approach is to trim weight scores, which we employ here. Specifically, we set multiple thresholds at 10, 50, 500, 1,000, 5,000, and 10,000. Observations where the calculated weight falls above the threshold are reset to the maximum threshold. For example, when estimating our models using the 500 threshold, a structural weight that is calculated to be 1,000 would be re-coded to be 500. This approach is useful in that it does not require us to throw away data. There is a clear gain in efficiency as a result of increasing the effective number of observations, but too few or too many pseudo observations can bias the estimates. Hence, we present all of these points to assess bias and sensitivity.} We display this information below in the results section.




			
\subsection*{Micro-Behavior Models}

%At the micro level, we use survey data from surveys carried out in 14 different countries with a high level of US presence, with approximately 1,000 respondents per survey.\footnote{The countries covered in the survey were the same ones covered in previous chapters: Philippines, Poland, Australia, Portugal, Netherlands, Belgium, Turkey, Kuwait, Spain, United Kingdom, and Italy.} We conducted the surveys from early September until early November of 2018 and 2019. The specific question that referred to protest behavior asked: ``Have you ever attended a protest event against a US military base?'' The response options were the following: Never, one time, two times, three times, more than three times, and don't know/decline to answer.

In this section, we shift our attention back to our public opinion data to examine individual-level participation in protest events targeting the United States in general and the United States military more specifically. 


\subsubsection*{What Causes Individuals to Participate in Protest Activity?}

As noted earlier in this chapter, individuals carry out a cost-benefit analysis in determining whether to join an anti-base protest. To begin with, individuals must have some grievance that drives them to protest. In some cases, individuals will derive a benefit from participating in the protests itself and expressing their frustrations publicly. Of course, the protest is usually a means to an end, meaning that the protest is meant to change US policy or eject the US military from their country. The larger the probability of successfully achieving their desired outcome, the more likely individuals are to protest. Finally, the costlier protest participation is, either through the opportunity cost of participation or because of the consequences of it, such as repression by the government, the less likely individuals are to protest. 



\subsubsection*{Predicting Protest Activity}

The goal of this section  is twofold: First, we generally aim to predict individual involvement in protests against the United States military. Our survey questionnaire specifically asks individuals about their history of attending protest events. Using this information, we construct a series of predictive models to better understand the occurrence of protest activity among foreign publics. Unlike our country-level models of protest, these models do not explore the causal effects of various predictor variables and the coefficients for these variables generally do not have clear causal interpretations. Instead, we first focus here on constructing predictive models using available covariates and then assessing the model's overall performance in accurately predicting whether an individual will participate in a protest or not. 

The second goal will be to use this information to shed light on some of the individual-level characteristics that correlate with protest activity. Although we cannot offer more concrete causal interpretations given the structure of our survey, we can use the predictive models to generate useful descriptive comparisons across groups. This is effectively what we did in chapter \ref{cha:min} on minority views of the US military and other actors. In this case the coefficients are more usefully understood as providing information on differences in protest behavior between groups, once we adjust for various other predictors of interest.

\subsubsection*{Model Specification and Estimation}


Our survey asked individuals about their involvement in anti-US protest events. It asked them how many protests they had participated in. To simplify the prediction process we turn what is a count (of the number of protests attended) variable into a binary variable that simply indicates whether a respondent has previously attended an anti-US protest event or not ($1 = Yes; 0 = No$). 

We estimate multiple models to compare how well our models perform, beginning with a basic Bayesian multilevel logistic regression predicting protest participation, leaving the model unspecified except for the population-level intercepts, country-level error, and yearly error. In other words, this is a model that does not consider any individual characteristics, just tells us how likely an individual is to protest in a given year. This serves as our baseline model of protest involvement. Next, we estimate a similar model, but this time we include several individual-level demographic variables as reported by the respondents. This model includes only the individuals' characteristics and does consider their attitudes and views. Third, we estimate a fuller model including the individual-level demographic variables listed above, along with individual-level attitudes and experiences, which we ask about on a range of questions. Fourth, we estimate the previous model with demographic and attitudinal information again, but we add country-level variables including gross domestic product (GDP), population, the number of bases in the respondent's province, and the size of the US military deployment in that country.\footnote{Note that although we have data at the province level across all countries, we are often lacking in sufficiently large sample sizes at the province level to estimate reliable effects. Since we are grouping respondents according to country and year of the survey, the estimates for the base count variable are likely to be biased.} In other words, the fifth model allows for country-level characteristics to influence protest participation predictions as well. The fifth model builds on Model 4 by using all of the same predictor variables but adds varying coefficients for the age, gender, and ideology, and base count variables. Although the individual-level predictors are identical in models 3, 4, and 5, each model builds upon the previous in important ways. For example, by adding group-level variables in Model 4 we can account for systematic differences between the different countries included in our data. Similarly, what we are doing by ``varying coefficients'' in Model 5 is to allow for the possibility that different factors (such as, for example, gender) may influence protest behavior differently in different countries.  
This means that we are relaxing the assumption that there is a constant relationship between predictor factors and protest countries.\footnote{We add an additional layer of flexibility to the models by allowing the correlation between all of the varying or ``random'' to vary as well. This relaxes the assumption that the correlation between these effects is 0 and allows the model to estimate each instead.} Table \ref{tab:protestpredictionvariables} provides an overview of the specifications of these five models. The varying coefficients vary by country only (i.e. not by year). 

\input{../Tables/Chapter-Protests/model-protest-prediction-varlist.tex}

In using these types of models, it is important that we assess how good they are at predicting actual protest behavior. In order to do this, we divide our data into two samples: a training sample and a test one. This step helps to ensure that we are not overfitting our model to the particular data that we have collected. In other words, it ensures that the model is indeed a general one and not just one that works only with the data from the specific individuals we surveyed. This step also allows us to assess how well our model is able to predict protest when we test it out on ``new'' data that were not used to estimate the model (the ``test sample'' that we referred to earlier). To do this we group our opinion data according to the representative individual characteristics (which are age, income, and gender). We then randomly assign 80\% of the observations to the training data and the remaining 20\% to the test data. This ensures that both the training data and the test data are representative on these same variables. We use the training data to fit the models we present below. Once we fit the models, we use the test data to assess how accurate the models actually are at predicting protest.

In the pages that follow, we present the basic model-level results followed by diagnostic information to assess how including different categories of variables affects the model's overall performance. 
			\section*{Results}

In this section we present our models' results. First, we review the results of our models that estimate the number of observed protest events in countries across time. Our main finding is that larger US military deployments do seem to cause protests against both the US and the US military in particular. As is to be expected, military deployments have a larger positive effect on anti-US military protests than they do on broader anti-US protests.

Next, we review the results of the models that focus on individual-level protest involvement as reported by the respondents in our cross-national survey data. In general we find that the greatest gains in predictive accuracy occur when we include individuals' experiences and attitudes, not just demographics, in the models. This finding carries significant weight for policy, as policymakers can more easily influence experiences than demographic attributes. 

\subsection*{Macro-Behavior Results}

The appendix contains table \ref{tab:antiusprotesteventmodels} and shows the results for the four models of country-year protest events. To ease interpretation, we also present a coefficient plot in Figure \ref{fig:coefplotcountryprotestmodels}. Models 1 and 3 present the basic models, while Models 2 and 4 include interaction terms designed to capture the expected conditional relationships between troop deployment size and protest activity. Across all four models we find that there is a large positive correlation between the size of the US military deployment in a country and the number of anti-US and anti-US military protest events. Notably the magnitude of this coefficient is roughly 3.5 times larger in Models 3 and 4 as compared to Models 1 and 2, suggesting that countries that  host larger US military deployments are more likely to see an higher number of protest events in general, but that the difference is even greater for protest events aimed specifically at the US military. We will return to a fuller evaluation of the causal nature of this relationship below.


%fix appendix references


We find that several other variables appear to correlate with anti-US protest events. Although relatively rare during this time period, countries involved as primary participants in a war that also involves the US tend to have a higher expected protest count than countries not at war with the US\footnote{Note that we use the countries listed in the PRIO/UCDP data's ``gw\_loc'' field \cite{Gleditschetal2002,Pettersson2019}. This lists the countries central to the incompatibility at the center of the conflict. This is a slightly broader way to group countries than looking only at countries where the US has invaded, but narrower than including all countries who participate in broad military operations like the ones we see in Afghanistan and Iraq.} Still, the models indicate that a US war leads to a fairly sizable increase in the expected count of protest events. This difference applies to both the general anti-US protest models and the models focusing only on protests against the US military. As with the troops coefficient, the coefficient is larger for the military protest models as compared to the more general protest models. US wars in the region surrounding the country in question, however, do not appear to correlate with a similar increase in protest events. More generally, domestic conflict within a state also appears to correlate positively with anti-US military protests during this period. However, we find no differences between conflict and non-conflict states in terms of the number of more general anti-US protest events. This is likely due to the strong relationship between US wars and domestic conflict during this time period---in our data, there are 0 observations where a state is at war with the United States and \textit{not} coded as experiencing a domestic conflict as well. 

The variable for the broader protest environment in a country also produces a positive coefficient. This coefficient indicates that countries where protest in general is more common tend to also have a higher number of anti-US protests. The magnitude here is fairly small, however, and we only find a clear positive coefficient for the more general anti--US protest models. The coefficients from the anti-US military protest models are close to zero and a large share of distribution of the coefficients falls above and below 0. Though a more precise causal analysis of this particular claim is beyond the scope of this project, this lends some support to the claim that a more robust protest environment can be leveraged to support anti-US protest movements. Similarly, we find some evidence of spatial clustering of anti-US protests. Countries with a higher number of anti-US protests in the surrounding region also tend to have a higher expected count of anti-US protest events.


\begin{figure}[t]
	\centering\includegraphics[scale=0.7]{../../Figures/Chapter-Protests/fig-country-coefficients.png}
	\caption{Coefficient plots for predictor variables of different protest events. Points represent the median point highest density posterior estimate and 50\%, 80\%, and 95\% highest density intervals. The histograms represent the distribution of the simulated draws from the model.}
	\label{fig:coefplotcountryprotestmodels}
\end{figure}

There are also some structural features of countries that correlate with higher expected protest counts. Countries with larger populations tend to have a higher expected count of anti-US protests than less populous countries. However, we find that this is limited to the more general anti-US protest models and find some evidence that these more populous countries simultaneously see \textit{lower counts} of protests against the US military. We also find some moderate evidence that more strongly democratic and more strongly autocratic regimes both tend to see a higher count of protests against the US and US military. The posterior distribution for these coefficients does overlap with 0, denoting some uncertainty around these results, but a large portion of the distribution falls on the positive size. 

%%Note: I think we need some summary
%%added a bit above.CM 

\subsubsection{Assessing the Causal Effect of Troop Deployments on Protest Events}

The results presented in the previous section point to a positive relationship between the presence and size of US troop deployments in a country and protest events against the United States. As we discuss above, we take additional steps to explore the degree to which this relationship may be causal, meaning that the US deployments are actually causing the increase in protest, not just that the two are correlated. Using the marginal structural models we discuss in the research design, Figure \ref{fig:atetroops} shows the estimated average treatment effects (ATE), and also the effect of the treatment history of troop deployments, for the two sets of models we estimate. The results of our marginal structural models indicate that US military deployments do, on average, have a positive contemporaneous effect on both anti-US protest events, and anti-US military protest events more specifically. The ATE estimates for the anti-US protest model appear in the bottom two panels and the estimates for the anti-US military protest model appear in the top two panels. In each panel, we display point estimates and credible intervals for the ATE, along with histograms showing the distribution of the ATE based on 10,000 simulations for each iteration of the model. As we note in the previous section, the fact that US military deployments can increase or decrease sharply in a particular set of cases makes estimating the propensity scores and treatment weights difficult. To assess how sensitive our results are to the choice of truncation points for the weights we present the estimates for each of six separate truncation points. 
%%This is something for later, but we probably will want to have a couple of sentences explaining what the truncation points actually mean, for our less methodsy readers. CM.


\begin{figure}[t]
	\centering\includegraphics[scale=0.7]{../../Figures/Chapter-Protests/figure-ate-troops.pdf}
	\caption{Plots show the predicted average treatment effect of US military deployments on protest events. Points represent the median point estimate and 50\%, 80\%, and 95\% highest density intervals. The histograms represent the distribution of the simulated draws from the model.}
	\label{fig:atetroops}
\end{figure}

Briefly, we find evidence of a positive average treatment effect (ATE) for the troop deployment variable in both models, indicating that countries that see an increase in the size of US military deployments tend to see an increase in the likelihood of an anti-US protest event, but also protests against the US military, more specifically. Notably, the multiple iterations of the model also show that the ATE estimate is indeed sensitive to the choice of truncation points, but the evidence generally indicates a positive effect. For example, in the case of the general anti-US protests we find average treatment effect coefficient estimates ranging from approximately $-$0.06 to $+$1.24. However, the only negative value appears in the model run using the 10,000 IPTW truncation point, which is extremely large. The IPTW truncation point of 10 produces a mean IPTW score of just over 1, which is relatively close to the guidance given by \citeasnoun{ColeHernan2008}. Otherwise we find that the models using the IPTW truncation points of 10, 50, 500, and 1,000 all produce relatively similar positive estimates, and we generally find that as we increase the truncation point the error around the ATE estimates narrows and the coefficient decreases slightly. 

When we look at the models predicting anti-US military protests, we see that across all of the IPTW truncation points the ATE is estimated to be positive, ranging from 2.68 at the lowest truncation point to 3.85 at the highest truncation point. As in the previous model we find some evidence that the dispersion around the ATE estimates shrinks as the truncation point increases, but nowhere near as dramatically as in the more general protest model. It is also notable that the estimates are considerably larger for protests against the US military as compared to the more general anti-US protest model.  And while we do see some shift in the median ATE estimate across the various models, there is a considerable amount of stability among the ATE estimates for the four highest truncation points. This is because there is far less variation in the anti-US military protest variable as compared to the more general anti-US protest variables, as can be seen in Figure \ref{fig:protesthistograms}. The maximum number of anti-US military protests is 4 in a given country-year, while we observe multiple cases where there are 5 or more protest events in a given country-year. Recalling the research design, the structural weights are calculated on the basis of the treatment variable, and these essentially tell the model how many copies of a given observation to make when generating the pseudo data for the model. What this tells us is that, after a certain point, the introduction of additional observations from the pseudo data does not fundamentally change the relationships between the predictor and the outcome variable (protest).

\begin{figure}[t]
	\centering\includegraphics[scale=0.7]{../../Figures/Chapter-Protests/fig-histogram-protests.png}
	\caption{Histograms showing the counts of anti-US and anti-US military protest events, 1990--2018.}
	\label{fig:protesthistograms}
\end{figure}


Notably, the treatment histories differ slightly in their expected effects. Larger deployment histories appear to cause a reduction in the expected number of anti-US protest events in a given country-year. As with the contemporaneous effects, higher IPTW truncation points produce estimates with less dispersion, but the median point estimates themselves are fairly stable. Alternatively, we find that evidence is more mixed when it comes to anti-US military protest events, where we find some indication of a positive effect. At the lowest IPTW truncation point we find the coefficient value is quite close to 0 with about 50\% of the posterior for the ATE estimate falling above and below 0. For the other IPTW truncation points, we find roughly an 87\% chance of a positive effect. Regarding the more general protest model, this negative effect of the treatment history makes some intuitive sense. Cases like Germany, South Korea, and Japan have long histories of hosting large US military deployments. After a period of time these deployments may provoke less general hostility towards the United States. However, the possibility of a positive effect of treatment history on protests against the US military also makes intuitive sense. In these same cases the presence of the military itself frequently remains highly contentious---even as individuals themselves frequently have close personal and professional relationships with US service personnel in and around base communities. Even as individuals may come to be more familiar with Americans, reducing more general hostility, the presence of a long-term military facility suggests continued exposure to various negative externalities of the sort we have discussed, including pollution, crime, environmental degradation, and more. These stimuli are understandably likely to provoke a continued reaction among the host-nation's population. However, the results also suggest that this dynamic is more uncertain than the more general protest model.

%Add something like this to explain what the ATEs mean: In other words, an increase of in troop deployments would lead to an increase of .21 to .27 protests in a given country-year.


			\subsection*{Micro-Behavior Results}

This section reviews the results of our predictive models of individual protest participation. The appendix includes table \ref{tab:predictiveopinionmodels} and demonstrates the coefficient estimates and some model fit statistics for our models of protest involvement. First, we assess the general predictive power of the models. Then we provide a brief discussion of select inter-group comparisons we discuss in the theoretical section of this chapter. Specifically, we focus on the coefficients for personal contact, network contact, and personal experiences with crime. In our earlier chapter, we discussed how minority populations within the host country may bear a disproportionate share of the burdens and costs associated with hosting a US military facility or large deployment. Accordingly, we also look at how minority respondents compare to non-minority respondents with respect to their likelihood of participating in anti-US protest activity.


%fix appendix references


We take multiple steps to assess the predictive power of the models. To provide a quick initial visual assessment we first present a series of separation plots in Figure \ref{fig:sepplotcombined}, each of which corresponds to one of the five models we discuss above. \citeasnoun{GreenhillWardSacks2011} present the separation plot as a means of visually assessing model performance. After estimating the models using the training data, we use the models to generate predicted probabilities for each observation in the test data. We then order the observations according to those predicted probabilities, with observations receiving the lowest predicted probabilities appearing on the left side of the plot and those observations with the highest predicted probabilities appearing on the right. Finally, we color coded each observation according to whether the binary response category for whether the individual participated in an anti-US protest is a yes (1) or a no (0).  Red indicates cases where the response was ``Yes'' and blue responses that were ``No''. The better the model performs the more red (``Yes'') observations we should see clustered on the right side of the panel, corresponding to higher predicted probability values. The five panels of Figure \ref{fig:sepplotcombined} correspond to the models shown in Appendix table \ref{tab:predictiveopinionmodels}. The ``Intercept Only'' model contains only the intercept values with no predictor variables. The model on the bottom is the full model, including the individual and group-level predictors, along with the varying coefficient estimates.

\begin{figure}[t]
	\centering\includegraphics[scale=0.7]{../../Figures/Chapter-Protests/figure-sep-plot-combined.pdf}
	\caption{Separation plot from various models of individual-level protest behavior. The black line shows the in-sample predicted probabilities of protest-involvement. We ordered the observations from low to high predicted values. The vertical red lines indicate observed ``Yes'' responses from the respondents. More red clustered towards the right side of the figure indicates stronger predictive performance.}
	\label{fig:sepplotcombined}
\end{figure}

From this figure we can see that the predictive performance of the models generally increases across the first three models. In the Intercept Only model, while there is obvious clustering towards the right side of the graph, there are also a large number of red observations scattered across the rest of graph space. The predicted values, as represented by the black line, are largely flat across the range of the graph, increasing only in a step-wise fashion at the far right end of the figure. We see some improvement once we move to the basic demographic model in the second panel, though there is still a considerable dispersion of ``Yes'' values across the figure. The model including both demographic and attitudinal/experiential variables produce some fairly significant gains. Here we see both clustering of red observations on the right side, and also more confidence in the predicted probability values towards the right side of the panel. Adding the group-level variables produces some additional gains, as does the inclusion of the varying coefficients in the final model. In both cases, we see some additional reduction in the number of ``Yes'' observations scattered across the panel, with a greater share of protest attendees clustered towards the right side.  

\input{../Tables/Chapter-Protests/model-protest-predict-stats.tex}

In general, we can see that these models become progressively better at correctly classifying observations in the test data. However, the separation plot is still a blunt instrument, and our ability to more precisely discriminate between the predictive performance of each model is limited. To supplement the information from the separation plots, Table \ref{tab:predictivecheck} contains additional statistics on the models' performance and predictive power. The first column shows the percent of cases that the model predicts correctly---that is, the model predicts protest involvement where a respondent reports protest involvement, and no protest involvement where a respondent does not report protest involvement.\footnote{Note that we use predicted probability values of 0.5 as the threshold for classifying predictions. Where $Pr(Protest | X_{ij}, Z_j) \geq 0.5 = 1$ or ``Yes'', and $Pr(Protest | X_{ij}, Z_j) < 0.5 = 0$ or ``No''. This simply means that where, conditional upon the values for our set of predictors for individual $i$ in country $j$, $X_{ij}$, and the country-level variables for country $j$, we observed a predicted probability of 0.5 or greater, we classify that individual as having a predicted value of ``Yes'', and ``No'' otherwise.} At first glance the models all appear to perform relatively well, with all models predicting at least 90\% of the cases correctly. However, given the overwhelming prevalence of ``No'' responses to the protest question in the data, this statistic alone can be misleading. Table \ref{tab:predictivecheck} contains several other statistics to help us understand their strengths and weaknesses. The false positive rate gives us an indication of how many individuals who replied ``No'' to the protest question were incorrectly classified by the model as having been involved in protest activity of some sort. In general, the models all have relatively low false positive rates, suggesting that we are not misclassifying too many ``No'' respondents. However, we do see some small increase in false positives as the models increase in complexity.

The false negative rate tells us how many ``Yes'' respondents the model misclassifies as not having been involved in protest activity. Here we start to get a better picture of the models' weaknesses and the sources of divergence in their performance. For example, the false negative rate of 82\% for the Intercept Only model means that the base model is incorrectly classifying 82\% of respondents who self-reported having been involved in anti-US protest. To rephrase slightly, if 100 individuals reported ``Yes'' when asked if they had previously attended an anti-US protest, the model would incorrectly predict 20 of these cases as ``No'' responses. We do see a reduction in the false negative rate as the models increase in complexity, but the lowest score we achieve is $\sim$51\%, which is still not very strong.

The sensitivity and specificity figures provide similar information, and while the names may be less than clear, their meanings are perhaps slightly more intuitive than the false negative and false positive metrics. Notably, these metrics are complements, summing to 100\% (i.e. false positive and specificity, and false negatives and sensitivity). Sensitivity tells us what percentage of ``Yes'' cases the model accurately predicts. Specificity tells us what percentage of ``No'' cases the model accurately predicts. A low sensitivity score means the model is doing a poor job of accurately predicting protest, whereas a high sensitivity score means that it is doing a better job of accurately predicting it. This statistic is appealing because it more directly speaks to the outcome of interest, protest, and the model's ability to recognize it when it occurs. Related, high specificity scores mean that the model is doing a better job of accurately predicting the \emph{non-events}---those cases where protest involvement did not happen. 

Predictably, the Intercept Only model does the poorest overall job of accurately classifying cases. The sensitivity score of 17\% indicates that only a small number of individuals who reported participating in anti-US protests are correctly classified as such. Alternatively, we can see that the specificity score of 99\% shows that the model did a good job of accurately classifying those individuals who reported never having attended an anti-US protest. However, such examples highlight the problems of relying solely on the percent correctly predicted metric---because the data are overwhelmingly ``No'' responses it is fairly easy to correctly classify those cases. But the model does a poor job of recognizing ``Yes'' responses when it sees them. If we think of the model as being akin to a genetic test---say one looking for some sort of chromosomal anomaly in a fetus---these results indicate that the test would correctly flag less than 20\% of such cases.

The models do better as we add more individual-level variables, and as we introduce more flexibility. That said, there are some diminishing returns. The largest increases in sensitivity come with the addition of the demographic variables, and then again when we add the attitudinal and experiential variables (i.e. models 2 and 3). In each case we see an increase of about 15 percentage points in the sensitivity score. This comes at the expense of specificity, which decreases slightly compared to the base model. Ultimately the most complex model only achieves a specificity score of approximately 49\%, meaning it is only correctly classifying about half of the instances where individuals have reported attending an anti-US protest event.

There are a few things to note here. To start, these scores are sensitive to the choice of cutpoints for classifying cases. In table \ref{tab:predictivecheck} we classify cases according to a fairly na\"{i}ve predicted probability threshold of 0.50. There are obviously an infinite number of alternative cutpoints that could be employed, but these entail trade-offs. Different predictive thresholds will affect sensitivity, specificity, and the overall share of accurate predictions. There are two ways that we can better understand the models' performance across these probability thresholds. First, Figure \ref{fig:protest-roc-plot} shows a receiver operating curve (ROC) plot for the five models we present. The ROC compares the true positive rate (specificity) and false positive rate ($1-$ specificity) for many different probability thresholds. Curves closer to the upper left corner are generally better at correctly classifying observations (such as pairing a ``Yes'' prediction with an individual who responded ``Yes'' to the question about attending anti-US protest events). The idea here is that there is an inherent tradeoff in classification---we can correctly classify all ``Yes'' cases if we use a very low probability threshold and classify every observation as a ``Yes'', but this necessarily means we end up with an enormous number of false positives (meaning that we classify observations as ``Yes'' when they are really a ``No''). Curves that are closer to the dashed reference line are equivalent to a random classification scheme.


\begin{figure}[t]
	\centering\includegraphics[scale=0.8]{../../Figures/Chapter-Protests/fig-roc-plot.png}
	\caption{Receiver operating curve (ROC) plot for the models shown in Table \ref{tab:predictiveopinionmodels}). This plot shows the percent of accurate ``Yes'' classifications relative to the percent of false positives (i.e. incorrect ``Yes'' classifications across a range of several different classification probability thresholds. Curves closer to the upper left represent better predictive accuracy.)}
	\label{fig:protest-roc-plot}
\end{figure}

From the curves presented in Figure \ref{fig:protest-roc-plot} we can see that there is indeed substantial variation in model performance when we account for the full range of probability thresholds. Ideally, we would want to see a model's curve spike up to 100\% immediately on the left side of the figure, but none of the models come close to this ideal type. The Intercept Only model performs the worst of the five, as we expect. Again, however, we see substantial improvements as we add demographic variables, and then demographic, attitudinal, and experiential variables. Ss this figure helps to make clearer, the relative performance of the three more complex models---including the one allowing for varying coefficient estimates---is nearly identical, with a very slight edge to the varying coefficient model. Ultimately, all of the models perform better than a simple random assignment mechanism, but some are clearly doing a better job of accurately classifying cases of protest involvement while sacrificing less in terms of generating a larger share of false positives.

\begin{figure}[t]
	\centering\includegraphics[scale=0.8]{../../Figures/Chapter-Protests/fig-pvalue-comparison.png}
	\caption{Comparison of sensitivity, specificity, and correct predictions across the full range of probability threshold values.)}
	\label{fig:protest-pvalue-compare}
\end{figure}

The disadvantage of the ROC plot is that it obscures the specific probability values and the trade-off between the different aspects of model performance that we seek to maximize. Figure \ref{fig:protest-pvalue-compare} plots the sensitivity, specificity, and percent predicted correctly figures across the range of possible classification probability thresholds. In general, higher probability thresholds make it more difficult for a given observation to make it into the ``Yes'' category. Conversely, lower thresholds mean that we are more likely to classify observations as ``Yes''. This is a seemingly trivial point but it is one worth reiterating because there is nothing special about the 0.50 threshold we use above---it is simply a convention of sorts, and one that is seemingly ``neutral''. But as we can see from Figure \ref{fig:protest-pvalue-compare} the trade-offs associated from moving between various probability thresholds are not all equal. In this case, our initial na\"{i}ve threshold of 0.50 sacrifices a considerable amount of performance on the sensitivity metric for relatively little gain on the specificity and overall correct prediction metrics. In fact, the percent of cases predicted correctly actually starts to decline slightly as we increase the probability threshold. To put it differently, setting too high a barrier for classification is also problematic as it means we miss an increasingly large share of cases.There are practical considerations in devising such a threshold as well. For example, if we found that the true probability of protesting were to be a 10\% chance, then we would be wrong nine of ten times. However, our model is providing information that these cases stand out from even lower-probability observations. Part of the decision calculus in choosing a threshold is judging the utility of having more false-positive or false-negatives and how that informs decision-making or policymaking. If a policymaker wants to guard against protests, then increasing the number of false positives to ensure that we capture the true positives is worthwhile. However, if the event itself is non-consequential or involves a severe policy response, then we should up our threshold and accept a higher number of false negatives. For example, if individuals who did not protest were subject to intensive surveillance or harassment because authorities \textit{believed} they were involved in protests, this sort of policy response could easily backfire and become a self-fulfilling prophecy that makes protest behavior \textit{more likely}. Returning to the table, across all five models we can see that lower classification thresholds generate relatively comparable figures for specificity and overall correct predictions, while yielding much better performance on the sensitivity metric. Referring to Figure \ref{fig:sepplotcombined} helps to clarify this point, as the vast majority of cases have very low predicted probability values assigned to them. Using a lower probability threshold for determining which observations fall into the ``Yes'' category can thus improve overall predictive accuracy.


\input{../Tables/Chapter-Protests/model-protest-predict-stats-targeted.tex}

Table \ref{tab:predictivechecktargeted} shows the models' predictive accuracy when we use 0.10 as the probability threshold for predicting individuals' protest experience. In general, the overall percent of observations correctly predicted remains fairly high but has declined slightly. We also see the false positive rate has increased, which we should expect from having a much lower threshold. More importantly, the sensitivity score has increased substantially across each model. The minimum here is approximately 41\%, with a high of 77\%. In other words, using a lower probability threshold our final model is correctly classifying three-quarters of cases where individuals report having participated in anti-US protest events. Again, this increase does come with a cost. The lower overall rate of correct predictions results from the slight decrease in the correct predictions of ``No'' responses. The increased false positive rate also reflects this. 

As one last check on our models' performance, Figure \ref{fig:ppcheck-individual-protest} shows the results of a set of posterior predictive checks using the five protest models discussed in this section. For each model, we generate 1,000 simulated data sets, each containing simulated predicted Yes/No values for the outcome variable. For each data, set we take the mean value of these predictions, giving us the proportion of each simulated data set where the model predicts individuals to have responded ``Yes'' to the outcome variable question. The light blue histograms in each panel shows the distribution of these predicted proportion values. The dark line shows the actual proportion of ``Yes'' responses in the real survey data. The narrow bands on the X-axis somewhat mask the fact that the actual distribution of simulated values is fairly narrow in every case. All five models produce reasonably close approximations of the real data, though the mean values are slightly below the actual mean value in every panel except the Intercept Only model. The model including only respondents' demographic characteristics actually produces the closest set of simulated values, though with slightly greater dispersion than the other more complex and fully specified models. 

Ultimately the models all produce fairly accurate aggregate-level predictions. Regarding individual-level classifications. The appropriate balance for selecting probability thresholds is somewhat subjective and depends upon the specific goals one has in mind, but these comparisons help to show that our models are reasonably flexible and accurate when it comes to predicting individuals' likelihood of participating in protests against the United States. 



\begin{figure}[t]
	\centering\includegraphics[scale=0.8]{../../Figures/Chapter-Protests/fig-opinion-ppcheck-plots.png}
	\caption{Posterior predictive check on individual-level protest models. The black bar represents the proportion of ``Yes'' responses to the protest attendance question in the test data (i.e. the number of people who say they've been to an anti-US protest event. The light blue area represents the distribution of the estimated proportion of protest attendees based on 1,000 simulations from the training data.).}
	\label{fig:ppcheck-individual-protest}
\end{figure}


In addition to the predictive accuracy of the models we are also interested in assessing comparisons between groups. The individual coefficient estimates for a few key variables can help us to better understand these patterns. Figure \ref{fig:coefplot-population-compare} shows the coefficients for the minority status, personal contact, network contact, and experience with crime variables from four of the models (except the Intercept Only model). The estimates are all positive, and all fairly consistent across models. The coefficient on the variable capturing reported experiences with crime produce the most variation of the four. Individuals who have experienced a crime involving US military personnel are far more likely to be involved in anti-US protest events than individuals who lack such experiences. The coefficient values here are approximately 2, indicating that individuals who are the victim of a crime perpetrated by a US service member are approximately 50\% more likely to participate in anti-US protest activity than people who responded ``No'' to this question. Contrast this with the minority and contact coefficients---0.85 and 0.62, respectively---indicating a roughly 20\% and 15\% higher probability of participating in anti-US protests than non-minority respondents, and those who report not having contact with US service personnel. Individuals who have experienced a crime involving a US service member are much more likely to mobilize politically against the United States military than those who have not. 


\begin{figure}[t]
	\centering\includegraphics[scale=0.65]{../../Figures/Chapter-Protests/fig-coefplot-population-opinion-model-compare.png}
	\caption{Coefficient estimates for models 1, 2, 3, and 4 presented above. These coefficient estimates exclude those for the varying coefficients model, which we include in a separate figure. 50\%, 80\% and 95\% credible intervals shown around median highest density probability prediction.}
	\label{fig:coefplot-population-compare}
\end{figure}

Finally, Figure \ref{fig:coefplot-varying-coefficients} shows the coefficient estimates from the varying coefficient model broken down by country. Though the general patterns look similar to those we see for the population-level estimates, there are some notable differences. First, as with the population-level coefficients, we again see that individuals reporting having experienced a crime involving US service personnel are far more likely to report protest participation than those who have not had these experiences. While this difference is by far the largest across all 14 countries, there is some noteworthy variation in just how big these comparisons are. The differences between groups tend to be smallest in Kuwait, which stands to reason given its authoritarian government and the difficulty of protesting under such conditions.  The median estimate for Japan is the largest of all of the countries, but the dispersion around the point prediction is quite larger. This results from only having 21 respondents out of $\sim$3,000 total respondents in Japan reported having experienced such a crime. This is not the case in every country, however. In Italy, 90 individuals reported such experiences with crime. In Portugal, 116 individuals reported experiencing a crime involving US service personnel. In Kuwait, 837. We provide a fuller review of the relationship between US military deployments and crime in the previous chapter, but we should reiterate two points here: although these cases are a clear minority in our data, we have a sufficient number of observations to generate reasonable coefficient estimates. The relatively small number of observations does lead to an increase in the error around the predicted estimates. Even allowing for this larger error, though, the entirely range of these coefficient estimates tends to be quite a bit larger than the other coefficient estimates. 




\begin{figure}[t]
	\centering\includegraphics[scale=0.65]{../../Figures/Chapter-Protests/fig-coefplot-varyingeffect-opinion-model-5.png}
	\caption{Country level coefficient estimates for the varying coefficient models of individual-level protest behavior. 50\%, 80\% and 95\% credible intervals shown around median highest density probability prediction.}
	\label{fig:coefplot-varying-coefficients}
\end{figure}

Second, it is entirely possible that there is under-reporting on questions such as this one. We cannot currently say how this under-reporting may affect these inter-group comparisons. It is entirely possible that individuals who choose to not report experiencing such crimes may also not wish to engage in protest activity against the US Survivors of sexual assault, or other violent crimes, may struggle with trauma in the aftermath of their experiences, seeking to avoid situations or activities that require them to interact with US service personnel or expose themselves to harm. Given that the risk of violent confrontation with authorities, or even co-nationals, is higher at protest events, individuals may choose to not participate. In such cases, we might expect the treatment effect of crime to vary based on other conditioning factors. For example, as we note in this paragraph, the type of crime may matter in determining how, and if, individuals choose to mobilize politically against the US military.









			\section*{Conclusions}



%%NOTE: Add something about the "base in your area"' variable once we have that in the models.CM.
We opened this chapter with a conversation with a German peace activist who recounted both the struggles and successes the German peace movement has faced in mobilizing the population against the US military presence. Somewhat counter-intuitively, he noted that while two-thirds of protesters mobilized by his organization are regional, ``the number of locals involved in the movement is growing, but not fast enough'' \cite{berlinone20190723}. While it may seem surprising that activist organizations would have a hard time recruiting from those communities that are closest to a US military base and, therefore, more likely to feel the negative effects of it, our analysis finds that some of these observations can be explained by both demographics and by individuals' experiences and perceptions.    

%discussing the successes and limitations of organizing anti-base activity in Germany. In Germany, anecdotally, opposition to bases is strongest in urban centers and not near the bases themselves. 
%Additionally, particular policies of the bases may drive organization while economic ties and good relations with service members may decrease that opposition.

It is important to note that engaging in protest tends to be a higher-commitment form of political expression. Not everyone who dislikes the military presence will actually mobilize and protest against it.  The activist told us that he thinks there must be 500,000 people adversely affected by US bases, yet the maximum turnout they can get at protests is 5,000 (``This gap is our challenge,'' he noted) \cite{berlinone20190723}. While he expressed some bafflement at the fact that ``they are unhappy, but they are not acting,'' he also seemed to have a strong understanding of why this is the case. The collective action problem is difficult to overcome for opposition groups that want to increase their numbers while the German and local governments and the United States are simultaneously working to obtain support for their positions. Despite the 75-year history of US bases in Germany, the German anti-base movement still faces challenges to overcome the collective action problem. Some of them stem from demographics, but others from the types of interactions that occur between locals and the US military.

%Given the infancy of the anti-base movement, despite the 75 year history of US bases in Germany, suggests that overcoming that organizing to protest requires a substantial effort of organizers. At least that is the case in Germany. 



%After surveying the literature, our expectations remained that those who have stronger incentives to protests were more likely to do so. Like previous literature, the conditions that favor reward participation in political action make protest more likely.

In this chapter, we looked beyond individuals' preferences and views on the US military to see how those views manifest in political action. Considering both events-based and survey data, we identified the conditions that predict anti-US military and anti-US protests and protest participation. We find evidence that there is indeed a causal relationship between the number of US troops deployed to foreign country and the number of protests in that country against the US and the US military more specifically. While this may seem obvious, establishing causality is important because of the strong policy implications it carries. When deciding whether to deploy more troops abroad, the US government should be aware that the increase in troops, even when there is already an existing deployment in place, will be likely to lead to increase in anti-base protests. In addition, this effect extends to more general anti-American protests, not just those focused on the military presence. A larger military presence can result in the expression of more anti-American sentiment.

%\ref{cha:meth}
Of course, reducing the size of its military deployments abroad is not always a realistic option for the United States government. We argue that the effect of the military presence on protest can be attenuated by a variety of other factors, many of them within the control of the US government and military. In general, as we found in 3, contact and economic relationships can reduce the negative perceptions of the United States military. These interactions could produce basing situations that would be less likely to experience anti-US protest as a reaction to a military deployment, and thus could create better hosting environments for the US military. If the contact is negative, that can actually make people more likely to want to protest against the US military, as we find in this chapter. 

%Though these are predictive, not causal models, they do provide information about the settings in which anti-US protests are most likely to occur in. 

Within countries there is also variation in whether individuals will be more or less likely to want to protest against the US military presence. Some of this propensity is related to demographic characteristics. People who are poorer and in democratic countries are more likely to self-report having participated in an anti-US military protest. In Chapter \ref{cha:min}, we discussed the unique place minority members in a host state have relative to bases. They are more likely to pay the costs of the base (through location, environmental consequences, and negative interactions) while reaping less of the benefits (national security, direct economic ties). We find continued support for this argument in that self-described minority status correlates with an increased likelihood of participating in protests.


Beyond demographics, people's experiences are also correlated with the probability that they will respond yes to the question about having participated in anti-US military protests. The roles of contact, economic reliance, and respondent social network continued to play important roles as they did when we focused on perception of US actors. While the models for the survey data were correlational and predictive, we did see that those with more contact (exposure) were more likely to protest a military base. Likewise, economic reliance correlated with increased protest attendance. Exceptionally negative experiences, such as being the victim of a crime committed by a member of the US military, or knowing someone who had been victim of such a crime, also correlated with individuals being more likely to attend more anti-base protests. This is particularly important point as it relates to policy, since the US cannot change host country demographics, but it can influence the types of interactions that service members have with host country publics. It is thus cause for optimism to know that effective policy can affect many of the determinants of protest participation.

As the United States enters an era that requires more consent in maintaining its basing agreements and hopes to continue to received substantial burden sharing from allied states, the consent and support of the host country population will become increasingly important. Even though individual members of the population do not make hosting decisions themselves, their protest activity can indeed influence their governments' policies towards the United States. When protests are successful and influence public opinion, they will make the basing more costly for the military. 

%loop back to the supra argument




			\end{comment}
			
			
