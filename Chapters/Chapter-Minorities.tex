\chapter{Bearing the Burden: The US Military and Marginalized Minority\index{minority} Groups in Host States \label{cha:min}}


%\setcounter{page}{1}
\doublespacing


%NOTE: Now that this chapter has more of a negative spin maybe we should think of a different title for this chapter

%\section*{Introduction}
\vspace*{-0.85cm}
\rule{\linewidth}{0.10pt} \\[-1.25cm]
{\footnotesize\paragraph{Summary:} Conventional accounts hold that minority\index{minority} communities often bear the brunt of the negative externalities of US military bases. Others suggest that US military bases can provide social and economic mobility opportunities that may not otherwise be present for minority\index{minority} groups in some host countries. We show that the relationships between minority\index{minority} populations and their attitudes towards US actors are complex. First, minority\index{minority} self-identification alone correlates with a higher likelihood of expressing a positive view of US actors. Second, once we adjust for demographic and attitudinal variables that correlate with minority\index{minority} status, like income and education, this relationship disappears in the population-level effects, and minority\index{minority} self-identification correlates with less positive views of all three groups. Finally, we find that there is substantial local variability---adjusting for other individual-level factors, the difference between minority\index{minority} and non-minority\index{minority} views of US actors depends on the national context, as well as individual experiences.} 
\\[-0.5cm] 
\rule{\linewidth}{0.10pt}

\vspace*{0.5cm}

%%%%NOTE: Just went through the interview with the Wiesbaden mayor's office liaison and there's a ton about ethnic minorities that we haven't used yet. Let's make sure and go through and add some of that to this chapter. CM. 


``Okinawan\index{Japan!Okinawa} people have a strong feeling that they bear too much of the cost,'' said the retired Japanese\index{Japanese} ambassador, sitting in his home in Tokyo\index{Japan!Tokyo}.\cite{tokyoone20200427} In a polite, diplomatic\index{diplomats} fashion that is typical of his profession, and that on our third year of fieldwork we were starting to become familiar with, he further explained this sentiment to us, ``I have no intention of criticizing the US forces, but many crimes\index{crime} have happened because of the US soldiers.'' As we will discuss in more detail in Chapter \ref{cha:protest}, he noted how, among the general population, a single crime\index{crime} could undo years of public diplomacy\index{public diplomacy} efforts. He referenced the example of the 1995 rape\index{crime!rape} of an Okinawan\index{Japan!Okinawa} schoolgirl by three American service members. He noted that even though the rape\index{crime!rape} ``has nothing to do with the US security situation,''  it still harms Okinawans'\index{Japan!Okinawa} perceptions of the US military. 

The US provides external security for Japan\index{Japan}, yet 70 percent of US bases are in Okinawa\index{Japan!Okinawa}, taking up 15 percent of the island's territory (Okinawa\index{Japan!Okinawa} is only 0.6 percent of Japan's\index{Japan} total territory).\cite{JPTimes2020,tokyoone20200427} The sentiment among many Okinawans\index{Japan!Okinawa} is that if US forces offer security to all of Japan\index{Japan}, then other parts of the country should carry a more proportional share of the burden. In this respect, the experience of Okinawans\index{Japan!Okinawa} is not much different from that of other minority\index{minority} populations around the world. The asymmetry between diffuse, collective goods and the small sector of the population bearing the burden for their provision is common in overseas basing for a variety of reasons, including geography, logistics, and national policy.

Okinawans\index{Japan!Okinawa} have long had grievances about discrimination by the Japanese\index{Japanese} government\index{government}. The location of US bases in their region and the problems they bring are crucial points of contention between many in the region and the government\index{government}. Okinawans\index{Japan!Okinawa} are a minority\index{minority} ethnic\index{minority!ethnic} group whose home island was annexed by Japan\index{Japan} in 1879.\cite[Though suppressed by the Japanese\index{Japanese}, the Ryukyuan\index{Japan!Ryukyu} languages (one of which is Okinawan\index{Japan!Okinawa}) are distinct from Japanese\index{Japanese}, and those looking to preserve Okinawan\index{Japan!Okinawa} identity are engaging in concerted efforts to teach the languages to the new generations to prevent their extinction.][]{Heinrich2004,Fifield2014,UNESCO2010} The topic of Okinawa's\index{Japan!Okinawa} disproportionate burden remains relevant, with Okinawa's\index{Japan!Okinawa} Governor Denny Tamaki\index{Tamaki, Denny} noting in May of 2020, as annual anti-US military base protests\index{protest} had to be canceled due to the COVID-19\index{COVID-19} outbreak, ``I will completely devote myself to resolving issues [in Okinawa\index{Japan!Okinawa}] including the heavy burden [of hosting US bases].''\cite{JPTimes2020}

Though the United States encouraged the Ryukyuan\index{Japan!Ryukyu} language and independence during American rule over the Ryukyu\index{Japan!Ryukyu} Islands from 1945 to 1972, the US occupation increased Japanese\index{Japanese} nationalism in this area. It is of note that the Ryukyu\index{Japan!Ryukyu} islands saw bloody battles against American forces on their territory.\cite{tokyoone20200427} The territory that the Americans occupied was one in which the virtual totality of the population became displaced and impoverished.\cite{Heinrich2004} Thus, rather than Americans being viewed positively for their encouragement of Ryukyuan\index{Japan!Ryukyu} independence and culture\index{culture}, nationalized Japanese\index{Japanese} language and identity continued to expand and, after the occupation, tensions between Okinawa\index{Japan!Okinawa} and Tokyo\index{Japan!Tokyo} arose over the continued US military presence in Okinawa\index{Japan!Okinawa}. Especially heinous offenses by service members against Okinawan\index{Japan!Okinawa} civilians sparked protests\index{protest} and turned public opinion\index{public opinion} against the US presence. In our sampling of reported criminal offenses by service members against civilians, 37.5\% of reported events occurred in Okinawa\index{Japan!Okinawa}. These offenses include DUIs\index{crime!DUI}, drug-related incidents\index{crime!drug-related}, manslaughter\index{crime!manslaughter}, kidnapping\index{crime!kidnapping}, and sexual assault\index{crime!sexual assault}.

In stark contrast to the case of Okinawa\index{Japan!Okinawa}, we also talked to US service members and locals at the RAF\index{Royal Air Force} Lakenheath\index{United Kingdom!Lakenheath} military base in Suffolk\index{United Kingdom!Suffolk} in the United Kingdom\index{United Kingdom}. In their glowing descriptions of each other, the English\index{English} locals and the American service members cited a shared language and culture\index{culture} as part of what enables that positive connection. Over coffee at a series of discussions held at the base with American non-commissioned officers and two Britons\index{British} representing the Ministry of Defense\index{Ministry of Defense} and the Royal Air Force\index{Royal Air Force}, we heard a stream of fondly-recalled positive anecdotes: ``I live in government\index{government} housing in the village. Local kids come through for trick or treating because they know they'll get American candy.''\cite{rafsix20190719} ``Oddly enough, the locals like to celebrate the 4th of July with us. Any chance to drink.''\cite{rafeight20190719} ``On the macro level, the base is well received in the local area. Why wouldn’t it be? It's been here so long.''\cite{rafthree20190719} ``There is a sushi place by my house that I religiously eat at. One day they were getting their daily fresh fish run. I made a joke about whether they had any extra, the owner went back inside and gave me a whole lobster tail, and three pounds of shrimp. He said it was to thank me for everything the US does.''\cite{raffive20190719} The locals were no less effusive, noting the importance of the shared language and culture\index{culture} between the British\index{British} and Americans, highlighting the stark contrast between the relationship with the local population in England\index{England} and in Okinawa\index{Japan!Okinawa}.\cite{councilone20190718}

Beyond the shared culture\index{culture}, interviewed\index{interview} service members considered England\index{England} a desirable assignment in part because locals are friendly regardless of the service members' race, ethnicity\index{minority!ethnic}, gender, or sexual orientation. During our discussions, several service members mentioned that in England\index{England}, ``they treat everyone exactly the same.''\cite{raffive20190719}  There was a clear consensus that the type of discrimination that service members faced when deployed to other locations was uncommon in England\index{England}. But what about \textit{local} minority\index{minority} groups?  Could the positive perceptions we observed in England\index{England} be related to the fact that US service members interact mostly with members of the majority ethnic\index{minority!ethnic} group, as opposed to Okinawa\index{Japan!Okinawa}, where many of the locals interacting with the US military belong to an ethnic\index{minority!ethnic} minority\index{minority}? Would ethnic\index{minority!ethnic} minorities in England\index{England} perceive a US military presence in their area differently from majority groups? Given that most of the locations that the US deploys also have significant minority\index{minority} populations, this becomes a key question to explore in understanding relations between the US military and local host populations. 

%Add this back in somewhere for book version.
%A lesbian service member noted that she walks around holding her wife's hand and does not get treated differently at all, observing that the British\index{British} are actually more accepting of being gay than the Americans are. She contrasted this with her experience being stationed in South Korea, where she noted that locals often misgendered her, calling her ``sir'' and changing her name to a male name \cite{rafone20190719}. Another service member recalled more overt racism in Eastern Europe, recounting that going to dinner with African-American friends was ``difficult'' \cite{rafeight20190719}.

%I'm not sure about this paragraph, since it seems to be about some negative local attitudes toward American minorities, not necessarily minority attitudes toward the American presence. I'm not sure the local attitudes described here have any real relationship to the minority populations within the country either, especially in the SK or Eastern Europe examples. Might be good to save for something more about American experiences across contexts rather than minority vs. majority on the local end. - AS

It is of note that we conducted the Lakenheath\index{United Kingdom!Lakenheath} interviews\index{interview} in the aftermath of Britain's\index{British} vote to leave the European Union\index{European Union} in 2016 (``Brexit\index{Brexit}''), as the country negotiated the terms of Britain's\index{British} exit, and just days before Boris Johnson's\index{Johnson, Boris} election as Prime Minister\index{Prime Minister} in July 2019.\footnote{At this point in the process, people widely assumed that Johnson\index{Johnson, Boris} would become the next Prime Minister\index{Prime Minister}, and the country was just awaiting the formal institutional processes to formalize the decision.} Voters in Great Britain\index{Great Britain} voted to exit in large part because of concerns about immigration\index{immigration} from the European\index{Europe} continent (especially Central\index{Central Europe} and Eastern Europe\index{Eastern Europe}) and changes to British\index{British} identity.\cite{Goodwin2016,Goodwin2017} The large influx of refugees\index{refugee} into Europe\index{Europe} in 2015 exacerbated these concerns as people sought asylum from, among other places, the conflicts in Syria\index{Syria}, Iraq\index{Iraq}, and Afghanistan\index{Afghanistan} (and broadly from Africa\index{Africa}, the Middle East\index{Middle East} and South Asia\index{South Asia}).\cite{Chan2015}. Research on the Brexit\index{Brexit} vote found that areas experiencing the highest increases in immigration\index{immigration} in the years before the vote were most likely to vote to leave. This was particularly true in areas that had been previously ethnically\index{minority!ethnic} homogenous and experienced sudden demographic changes due to immigration\index{immigration}. Britons'\index{British} fear of how these sudden changes would alter British\index{British} identity affected their likelihood of voting ``Leave.''\cite{Goodwin2017}

We confirmed some of these observations when talking to British\index{British} locals in both London\index{United Kingdom!London} and Suffolk\index{United Kingdom!Suffolk}. A Lakenheath\index{United Kingdom!Lakenheath} Council Member\index{Council Member} told us that he had voted ``Leave'' in the referendum, though he was starting to change his mind on it as he saw the economic impact Brexit\index{Brexit} was having on Britain\index{British}. The town of Lakenheath\index{United Kingdom!Lakenheath} voted for ``Leave'' (18,160 votes in favor of leaving, 65 percent of the vote, with 72.5 percent turnout). When we asked what it was that had driven the high turnout and ``Leave'' vote, he responded, without hesitation, that it was immigration\index{immigration}. He recounted how in Brandon\index{United Kingdom!Brandon}, a town about 6 miles away from Lakenheath\index{United Kingdom!Lakenheath}, immigrants\index{immigrant} (especially Eastern European\index{Eastern Europe} and Portuguese\index{Portugal} immigrants\index{immigrant}) were more likely to frequent the local libraries than the English\index{English}. He noted, ``wanting to avoid racial overtones; it throws people because it's a change in culture\index{culture}. I'm in the supermarket, and I can't hear English\index{English} being spoken.'' He was quick to clarify that ``it's not bad that [demographic change] happens, it's that it happens too fast,'' summarizing the sentiment as ``it made people feel like England\index{England} wasn't England\index{England} anymore.''\cite{councilone20190718}

The Council Member\index{Council Member} contrasted interactions with immigrants\index{immigrant} and refugees\index{refugee} with the American military personnel, noting that when it comes to the Americans, ``because our cultures\index{culture} are similar, it is not as big of an issue. The fact that our cultures\index{culture} are very similar helps. It would be different without that.'' He also highlighted the importance of the common language in building positive relations between locals and the US military base. Yet, when we asked him about relations between the base and minority\index{minority} populations in the area, he just said that ``It hasn't come up.''\cite[][Despite many active efforts to reach out and interview civil society leaders for minority groups in the area, we received no response. The lack of response is perhaps telling in and of itself, as we identified ourselves in all of our correspondence and calls as researchers based at US universities studying perceptions of the US military abroad.]{councilone20190718}

The interview\index{interview} subjects at RAF\index{Royal Air Force} Lakenheath\index{United Kingdom!Lakenheath} were equally ambiguous when we asked them about relations between the base and minority\index{minority} populations in the region. We found that very few of them had interacted with members of local minority\index{minority} communities and that they mostly held positive but vague views of them. For example, when we asked about their interactions with Portuguese\index{Portugal} or Eastern European\index{Eastern Europe} locals, only one out of six service members volunteered an anecdote, noting that her ``lash lady'' (meaning the technician who applies fake eyelashes) was Portuguese/index{Portugal} and was ``very nice,'' and that the ``night life is diverse.'' She recalled having friendly conversations with immigrants\index{immigrant} while out at nightclubs and bars.\cite[This individual was the youngest in the group of service members we interviewed\index{interview} at Lakenheath\index{United Kingdom!Lakenheath}, describing herself as ``mid-20s'' and self-identified as an ethnic\index{minority!ethnic} minority\index{minority} in the United States (``I'm half Filipina'').][]{raffive20190719} 
%One of the Britons interviewed at the base noted that the immigrants mostly do agricultural work in the area and told us about how in his village once a year they have a yard sale and ``usually the rubbish gets bought by the Eastern Europeans'' \cite{rafthree20190719}.
%NOTE: I went back and forth on that last quote about the rubbish. It's very telling of attitudes towards minorities, but the commander is very easily identifiable and it makes him seem kind of shitty. I'm leaning towards scrapping it to not burn any bridges.CM.
%NOTE: I agree. I'd like to keep it but I can't think of a way to refer to it without referencing him or the specific thing he said, since it's pretty specific. - AS

People living in Japan\index{Japan} and England\index{England} had very different experiences in their interactions with American service members, but there was a clear common thread across both sets of interviews\index{interview}: The experiences that ethnic\index{minority!ethnic} minorities\index{minority} had in dealing with the US military were vastly different from those that dominant ethnic\index{minority!ethnic} groups held. Generally, minorities seemed to be having more negative, or at least less positive, interactions. However, the few comments from American service members about local minorities were largely positive. This highlights the importance of not homogenizing local experiences with the US military and acknowledging that an individual's perception may be significantly influenced by whether they see themselves as part of a state's dominant ethnic\index{minority!ethnic} group or belonging to a minority\index{minority}. 

%We don't actually test anything related to the ethnicities of service members, so taking this out. 
%Given that some positive comments about relationships with local minority populations also came from service members who considered themselves minorities, the presence of minority groups within populations of US personnel on installations may also be an important factor in determining local minority views. 




\section*{Ethnic\index{minority!ethnic} Minorities\index{minority} and the US Military}

Research on the relationship between the US military and local minority\index{minority} populations is relatively thin, especially with regards to quantitative, large-sample research. The work in this area largely consists of episodes in which minority\index{minority} populations have played a key role in American combat operations somewhere in the world, theories on how minority\index{minority} populations can use third-party support to accomplish their goals, or case study research on specific episodes in US-host state relations. The specific work on the US military's treatment of minorities in the context of basing focuses on individual cases, such as those previously mentioned in South Korea\index{South Korea} and Saudi Arabia\index{Saudi Arabia}. 

Katherine McCaffrey has studied this issue on the island of Vieques\index{Vieques}, Puerto Rico\index{Puerto Rico}, where the US military maintained a bombing range until recently.\cite{Mccaffrey2002} McCaffrey states in her work that ``bases are frequently established on the political margins of national territory, on lands occupied by ethnic\index{minority!ethnic} or cultural minorities\index{minority} or otherwise disadvantaged populations''\cite[p 9-10]{Mccaffrey2002} As examples, she points to the US military bases in the Philippines\index{Philippines} that were located on land reserved for indigenous populations, along with Okinawa\index{Japan!Okinawa}. She describes how the local minority\index{minority} populations often had to comb through military trash to survive after their governments\index{government} evicted them from their land. While these are certainly powerful examples, to our knowledge, there are no systematic and cross-national studies that test whether the dynamics illustrated by these examples are consistent and widespread across contexts. Taking it a step further, we ask whether these examples form a representative sample of how minority\index{minority} populations view an American military presence or whether these stark cases of mistreatment are the exception to the rule of generally positive relationships. 

Counter to these negative precedents, the United States also has a long history of intervening in civil wars\index{civil war} and on the side of minority\index{minority} populations around the world. Particularly since World War II\index{World War II} and the realization of the horrors of the Holocaust\index{Holocaust}, there has been a growing consensus within the American foreign policy community that the United States has the ability and, therefore, the responsibility to protect populations worldwide from extreme abuses their governments\index{government}. This idea culminated in the ``Responsibility to Protect\index{Responsibility to Protect}'' (R2P) doctrine, endorsed by the United States and United Nations\index{United Nations} in 2005, after American interventions in the Balkans\index{Balkans}, along with the establishment of no-fly zones to protect Kurdish\index{Kurds} populations in Northern Iraq\index{Iraq} and the protection of rebel groups from government\index{government} forces in Libya\index{Libya}. While those episodes were primarily seen as successfully protecting minority\index{minority} groups, the failure to intervene in Rwanda\index{Rwanda} and the resulting genocide\index{genocide} also stood as a strong counterexample of what could happen in the absence of American military involvement. 

Suppose we extrapolate from these interactions that take place in the context of conflict. In that case, we might conclude that minority\index{minority} ethnic\index{minority!ethnic} groups will likely have positive perceptions of a peacetime US military presence in their home country.  An initial expectation might be that minority\index{minority} groups are likely to view the US positively because the US military presence will make the host government\index{government} less likely to repress them. If the US actively promotes human rights\index{human rights} abroad and is active in naming and shaming human rights\index{human rights} violators and pressuring them to change their repressive behaviors, then minority\index{minority} groups, which majority governments\index{government} often try to repressed, should be glad to have the US military present to act as a restraining influence on the host government\index{government}. After all, previous work finds a positive correlation between a US military presence and respect for physical integrity rights in the host country.\cite{bell2017} The presence of US forces could act as an implicit threat against defection from the US hierarchical\index{hierarchy} system, which endorses respect for human rights\index{human rights}. As occurred in Panama\index{Panama} in 1989, these troops can be used against the host government\index{government} if it is seen as acting against the wishes of the United States. This type of implicit threat can keep host state behavior within the bounds of what is considered acceptable by norms established by the United States, as the leader of a global hierarchical\index{hierarchy} system.\cite{Towns2012} 

%The United States can be a positive influence on central governments in their treatment of minorities.

Yet, this dynamic may not play out in practice in countries where the United States has a large military presence. The decrease in human rights\index{human rights} violations that accompanies a US military presence only occurs in states not considered strategically important to the United States.\cite{bell2017} In fact, Stravers and El Kurd find that a US military presence in autocratic\index{autocracy} states that are also strategically important correlates with increased autocratization\index{autocracy}.\cite{StraversElKurd2018} A good example of this is in the aforementioned case of Bahrain\index{Bahrain} during the Arab Spring\index{Arab Spring} protests\index{protest}, when the presence of American military forces did not restrain the government\index{government} from violating the human rights\index{human rights} of Shia\index{Shia} protesters\index{protest}. With these theories and types of cases in mind, minorities may not only have their everyday lives disrupted by the US military presence but may be made to feel less secure through the US support of a government\index{government} that engages in discrimination against them. Through the need to stay in the good graces of a government\index{government} led by the more powerful majority to maintain the basing apparatus, the United States often turns a blind eye to discrimination in host countries and, in some cases, contributes to it. This process may result in a systematically more negative view of the US military presence among minority\index{minority} populations. 

First, we argue that the US military presence will be more disruptive for minority\index{minority} members' everyday lives than the rest of the population. While the host country gains economic, security, and political benefits from the US military presence, it can also face domestic political costs. In particular, military installations close to major cities and population centers are most likely to encounter opposition. These installations are more likely to serve as a constant reminder to the population of the host's subordinate role relative to the United States and of the United States' ``imperialism.''\cite{Cooley2008} As we note in Chapter \ref{cha:protest}, cities are also more likely to facilitate anti-base activists'\index{activists} coordination capabilities, with more access to transnational anti-basing organizations and demographic groups that are predisposed to protest\index{protest}, such as students. This is in addition to the fact that coordination is easier in cities, which helps reduce collective action\index{collective action} problems involved in protest\index{protest}. Thus, host states will often choose to locate American bases in less contentious, more remote locations.\cite{Cooley2008}

The case of the Thule Air Base\index{Greenland!Thule Air Base} in northern Greenland\index{Greenland}, established in 1951 and expanded to add missile defense in 2004, is an example of this type of dynamic. Thule's strategic location close to the Arctic is, of course, the major part of its appeal to the Americans as a site for missile defense, yet Greenland's\index{Greenland} position within the Danish\index{Denmark} government\index{government} also makes it a politically advantageous location from Denmark's\index{Denmark} perspective. Greenland\index{Greenland} was a Danish\index{Denmark} colony until 1953, when it became a Danish\index{Denmark} county, with representation in the Danish\index{Denmark} Parliament. Though Greenland\index{Greenland} has expanded its self-rule in 2008 and 2009, many aspects of Greenlandic\index{Greenland} policy remain under Danish\index{Denmark} control.\cite{Dragsdahl2005} 

Ethnically\index{minority!ethnic}, Greenland's\index{Greenland} population of 57,000 is 89\% Greenlandic\index{Greenland} Inughuit (Inuit\index{Inuit}). This contrasts with Denmark's\index{Denmark} population, of which 83\% are ethnically\index{minority!ethnic} Danish\index{Denmark} of Nordic European\index{Nordic European} ancestry. Thule Air Base has had a disproportionately negative effect on the Inughuit\index{Inuit} people's livelihood while benefiting the general Danish\index{Denmark} population. The US established the Air Base in 1951 in Thule\index{Greenland!Thule Air Base} and, when the military added anti-aircraft guns in 1953, the Danish\index{Denmark} government\index{government} forcibly relocated the Inughuit\index{Inuit} people of Thule\index{Greenland!Thule Air Base} to the nearby town Qaanaaq\index{Greenland!Qaanaaq}.\cite{Spiermann2004} In 2004, the United States added missile defense systems to the base, increasing concerns by the Inughuit population of being targeted by nuclear weapons. This added to existing grievances of the base harming the environment\index{environment} through toxic waste and affecting the hunting fields traditionally used by the Inughuit.\cite{Dragsdahl2005}

In May of 2004, the Inughuit group Hingitaq 53 (meaning ``The Expelled of 53'') brought a case to the European Court of Human Rights\index{European Court of Human Rights} asking for the authority to return to their original home in Thule\index{Greenland!Thule Air Base}. While the court did not grant them the ability to return, it did order the Danish\index{Denmark} Prime Minister\index{Prime Minister} to compensate the Thule Tribe\index{Greenland!Thule Tribe}.\cite{Spiermann2004} Meanwhile, Denmark\index{Denmark}, knowing the strategic importance of Thule Air Base\index{Greenland!Thule Air Base}, has been able to use the base as a bargaining chip with the United States. As a NATO\index{North Atlantic Treaty Organization} member, Denmark\index{Denmark} often uses the base as evidence of its contribution to the alliance.\cite{Dragsdahl2005} Thus, the benefits from the base distribute to the Danish\index{Denmark} population generally, while the costs are borne disproportionately by an ethnic\index{minority!ethnic} minority\index{minority} that lacks political power. This dynamic has continued up to the present day, with the President of the United States publicly proposing that the United States purchase Greenland\index{Greenland}. Senator Tom Cotton\index{Cotton, Tom}, a close administration ally, discussed the idea with the Danish\index{Denmark} ambassador to the United States.\cite{hart2019,wu2019} While the Danish\index{Denmark} government's\index{government} response was swift and dismissive, this series of events took place without the United States ever involving the local population in the decision-making process, which highlights the degree to which minority\index{minority} populations can often be sidelined. 

%%This sounds more like actually emphasizing Greenlandic autonomy, which is not really the point we're making, so I'm commenting it out for now.
%Even the Prime Minister of Denmark insinuated as much about the American approach, saying ``Greenland is not Danish, Greenland is Greenlandic \cite{jorgensen2019}.'' One of the two Greenlandic representatives in the Danish Parliament supported this sentiment by commenting that ``Greenland is not a commodity which can just be sold'' \cite{thelocal}. 

%and ``I'd be concerned about the type of society we'd have if Greenland becomes American rather than Danish." 

This example highlights the role of host governments\index{government} in basing access and its relationship to minority\index{minority} populations. When host governments\index{government} negotiate with the United States over base access, they are doing so on two levels; directly with the United States government\index{government} and needing to gain enough support from their selectorate\index{selectorate} at home.\cite{Putnam1988,demesquita2005,Cooley2008} They thus need to obtain not only benefits from the United States but also secure the support of key political players at home. Leaders can do this by maximizing the net benefits received by the leader's winning coalitions (maximizing their gains while minimizing their costs).\cite{demesquita2005} In determining where to place US military facilities, a common solution is to have them in remote locations that are often homes to politically powerless ethnic\index{minority!ethnic} minorities\index{minority}. This allows the dominant populations in the country to receive the benefits of a US military presence without experiencing the full weight of its negative consequences.

Okinawa\index{Japan!Okinawa} exemplifies another asymmetry between the collective benefit of national security versus the target costs of hosting bases in a territory with a concentrated minority\index{minority} group. The movement against bases in Okinawa\index{Japan!Okinawa} started in the Cold War\index{Cold War}, and negative interactions between troops and civilians built support for protests\index{protest} and national action against the heavy Marine presence in the region. In classic Olsonian collective action\index{collective action} problems, concentrating the costs onto a smaller population makes collective action\index{collective action} against the presence of bases more likely than a strategy of diffusing bases nationally and, thereby, diffusing the costs of hosting bases in Japan\index{Japan}.\cite{Olson1965} However, as we will discuss in further detail in the chapter on protest\index{protest}, this also potentially limits how cross-cutting protest\index{protest} demographics are, which reduces the likelihood of their long-term success.\cite{Yeo2011}

Second, the US has a troubled history with domestic ethnic\index{minority!ethnic} relations and government\index{government} respect for rights among minority\index{minority} groups. While domestic issues do not always translate into international behavior, the United States has exported its domestic turmoil to other countries it bases in. US President Harry Truman\index{Truman, Harry} formally integrated the armed forces in 1948, yet this did not prevent soldiers' ideas about segregation from being perpetuated in other countries.\footnote{The various branches up through World War II\index{World War II} had differing policies on racial segregation. Truman's\index{Truman, Harry} executive order 9981 uniformly removed segregation across the branches.}  Moon goes into detail about how US racial beliefs conditioned how South Korean\index{South Korea} business establishments treated white and Black soldiers differently in the post-Korean War\index{Korean War} era.\cite{Moon1997} Businesses would exclusively serve white soldiers, play music more specifically targeting white clientele, and create conditions that were generally hostile to Black soldiers. White soldiers would reinforce these conditions both in how they patronized establishments as well as in high-profile cases of soldier behavior amplifying such messages. In one particular case, a white soldier murdered\index{crime!murder} a sex worker after seeing her working with both white and Black clientele.\cite{Moon1997} 

The US propagation of racial attitudes directly negatively impacted US operations, and the de facto segregation encouraged in the 1950s and 1960s required decades of dedicated work to begin to reverse. These attitudes were not simply confined to how US personnel and host populations related to each other. Instead, they became pervasive in host societies and frequently altered majority relations with minority\index{minority} populations, providing a template for racial and ethnic\index{minority!ethnic} discrimination. As such, individual soldier attitudes, even if they do not reflect government\index{government} policy (but especially when they do), can do further damage to minority\index{minority} perceptions about the United States' mission in their country by impacting not only their perceptions of US and US military attitudes toward minorities, but also affecting their place within their own societies. 

Third, by virtue of the military presence in a host state, the US government\index{government} and the US military have a preexisting, institutional, and strategic relationship with the host state. Given the basing relationship, the United States and the central government\index{government} in host states are already on friendly terms. The presence of US forces also signals that the location or country is strategically valuable to the United States and that maintaining basing access will likely be a priority of US policy.\cite{StraversElKurd2018} These dynamics show that the starting point in any tension between substate actors and the central government\index{government} will see bias toward the central government\index{government} from the United States. While minority\index{minority} populations repressed by the host government\index{government} could certainly draw US support under certain circumstances, it is key to note that the United States is not a neutral party. 

When the US places troops in a foreign country, it is making an implicit (or explicit) commitment to the security of that country. Whether US troops are placed there to directly defend the host country or serve as a trip-wire mechanism that would trigger a US intervention if the host country were to be attacked and US troops threatened, troops stationed consensually abroad protect the host country from external threats.\cite{Schelling1966}  Though in most cases US troops are not there to protect the host country from internal threats, the very fact that US troops are providing security from external threats means that the host country can redirect its own resources from protecting against external threats to protecting against internal threats.\cite{machainandmorgan2013,allenetal2016,allenetal2017}

In addition, a US military presence often comes with tangible material benefits for the government\index{government}. For instance, a basing relationship with the United States routinely brings increased bilateral military training exercises that increase the coercive capability of the central government\index{government}.\cite{Ruby2010} The United States also tends to provide more access to foreign military sales for countries that host US military forces, along with sharing existing technology that allows for the growth of a domestic arms industry. These kinds of training, arms transfer, and technology benefits also empower the central government\index{government} in relation to the rest of the country and potentially make challenges from substate actors easier to manage. 

If the host country views minority\index{minority} groups as an internal threat (something that is common among states with significant minority\index{minority} populations), then it is likely to use repression as a way to protect itself from the perceived internal threat.\cite{Regan2005,Jakobsen2009,Brathwaite2014,Hendrix2019}  If the central government\index{government} has more resources available to it (because it is now spending less on external security or because the US is directly providing it with resources), then it can devote more of those resources to repression.  A basing relationship with the United States can then have the effect of materially enhancing the power of the central government\index{government} in relation to substate actors and therefore make challenges to the central government\index{government} less likely. Thus, a US military presence is increasing the resources available for the repression of minorities.  This will naturally condition minority\index{minority} groups to be less supportive of a US military presence that legitimizes the central government\index{government} and allows for the expansion of its repressive power.

Furthermore, the mere presence of US military forces on host territory provides legitimacy to the central government\index{government} from the most powerful country in the international system. Lending legitimacy to central governments\index{government} can have profound effects that further empower the central government\index{government} by backing the idea that the government\index{government} is the rightful power in the country and dissuade challenges to its authority.  For example, the US placing military forces in Spain\index{Spain} through the 1953 Madrid Pact\index{1953 Madrid Pact} legitimized General Francisco Franco's\index{Franco, Francisco} rule in Spain\index{Spain} when the international community had generally ostracized him.\cite{Cooley2008} The presence of US forces also gives a sense that there is a significant external security concern for the country, which may be necessary to defend against before challenging the central government\index{government} directly. This is particularly important in cases where a minority\index{minority} group feels that a central government\index{government} is mistreating them. Other groups within the country and even within the minority\index{minority} group itself will see the legitimacy and authority conferred on the host government\index{government} via US military forces and be less likely to challenge the central government's\index{government} authority. 

By enhancing the central government's\index{government} power, the US basing apparatus may further centralize power in host states. The United States deals mainly with the central governments\index{government} in host state relationships, not regional or local governments\index{government}. For example, when determining where to send humanitarian\index{humanitarian} deployments within a country, the US consults first with the central government\index{government}, which presents a series of options.\cite{Flynn2018} By dealing directly with it, the US not only enhances the power of the central government\index{government} concerning minority\index{minority} groups, but it also enhances the power of the central government\index{government} in relation to regional or local governments\index{government}. Decentralization reduces ethnic\index{minority!ethnic} conflict and secessionism by increasing opportunities for minority\index{minority} groups to participate in government\index{government}.\cite{Brancati2006} By empowering the central government\index{government} both materially and in its legitimacy, the presence of US forces may inadvertently centralize power in a way that disempowers minority\index{minority} groups from effectively engaging in politics. 

Even in democratic\index{democracy} countries, minority\index{minority} groups may see themselves at odds with the majority groups that control the government\index{government} and view a US military presence as supporting the majority population rather than the whole population equally. They may not always be wrong to assume this. While the United States may not support repression and discrimination against ethnic\index{minority!ethnic} minorities\index{minority}, US military deployments' efforts to build goodwill often focuses on those members of society whose support is needed to maintain the presence. This usually is the ethnic\index{minority!ethnic} majority, which controls the government\index{government} and to whom the government\index{government} answers.

Many of these dynamics came to light when we visited the Clay Kaserne Army Post\index{Clay Kaserne Army Post} in Wiesbaden\index{Germany!Wiesbaden}, Germany\index{Germany}, where we heard about outreach efforts targeted at local government\index{government} officials. The base's Government Relations Officer\index{Government Relations Officer} noted that she had already met with every mayor of each town surrounding the base in less than a year since she had taken that position. The mayors all get permanent passes that allow them to drive onto the base without an escort. The Government Relations Officer\index{Government Relations Officer} also offers tours of the base for local council members\index{Council Member} and collaborating with the police and fire departments in neighboring towns. She noted the importance of cultivating relationships with local governments\index{government} and making the relationship feel less transactional for them. A gregarious and extroverted individual, she knew that she was well-suited for the job: ``You have to stroke and smooch them,'' she said to us in a cheerful tone, ``and I am a good smoocher.''\cite{kaserneone20190725}   

Given the very different views towards ethnic\index{minority!ethnic} minorities\index{minority} and majorities that we had seen in Lakenheath\index{United Kingdom!Lakenheath}, we were particularly eager to hear about the base's relations with and outreach efforts towards refugee\index{refugee} and immigrant\index{immigrant} communities in Germany\index{Germany}, which had dramatically increased in size since the country began accepting large numbers of refugees\index{refugee} from the wars in Syria\index{Syria} and Iraq\index{Iraq}. The topic came up on its own when we asked the Government Relations Officer\index{Government Relations Officer} about locals' views of economic benefits from the base. She told us that while there had been a trend of younger German\index{Germany} generations (who had not lived through the Marshall Plan\index{Marshall Plan}) becoming increasingly frustrated with American service members for not learning German\index{German}, the trend was somewhat reversing with the recent influx of refugees\index{refugee}. She noted that Germany\index{Germany} was becoming more like the United States and experiencing ``crime\index{crime}, rape\index{crime!rape}, and pedophilia\index{crime!pedophilia}.''\cite{kaserneone20190725} 


Research on in-group/out-group dynamics shows that crises and the introductions of an outside threat can redefine in-groups and out-groups.\cite{Coser1998,Levy1989,Simmel2010} This dynamic was clearly at play in Germany\index{Germany}. The Government Relations Officer\index{Government Relations Officer} put it plainly, ``Before the refugees\index{refugee}, Americans were seen as outsiders.'' As an illustration, she noted that, as refugee\index{refugee} populations have increased, ``You don't see as much `US Go Home,' but instead `Country X go home''' (with ``Country X'' referring to the home countries of refugees\index{refugee}).\cite{kaserneone20190725} There is thus a redefinition of the outgroup occurring with the native Germans\index{German} viewing the Americans as part of their in-group and perceiving refugees\index{refugee} and immigrants\index{immigrant} as outgroup members.

From the American perspective, this dynamic is also at play. While the Americans are unlikely to feel threatened by the influx of refugees\index{refugee} in the Germans'\index{German} way, they also know that policy concessions will come from those Germans\index{German} who control government\index{government} institutions, not the politically powerless refugees\index{refugee} or other ethnic\index{minority!ethnic} minorities\index{minority}. We specifically asked if the base did any outreach directed at the civil society\index{civil society} groups of different ethnic\index{minority!ethnic} populations, similarly to how it reaches out to the mayors and council members\index{Council Member} of surrounding towns. The government relations officer\index{Government Relations Officer} noted that such groups ``probably exist, but I don't meet with them.'' Regarding the Turkish\index{Turkey} community, she noted, ``They are not my priority to reach out to. Our primary target is the city and mayors.''\cite{kaserneone20190725}

As far as the refugees\index{refugee} go, she anecdotally confirmed that they are less likely to feel positively towards the United States than other Germans\index{German}: ``The refugees\index{refugee} that come from the countries that we are at war with, they don't like us.'' Interactions between refugees\index{refugee} and US service members are thus more likely to be negative ones. She gave the example of young people going to clubs, getting drunk, and fights breaking out between US service members and ``a foreign national'' (meaning a non-German\index{German}) after people from countries that the US is in conflict with say things like ``You killed my parents, you bombed my town.'' The alcohol consumption and the Americans being ``somewhat arrogant'' and patriotic exacerbates the confrontations. 

Previously we have shown that direct and indirect contact\index{contact} with the US military in a non-combat deployment setting can lead to both a positive and a negative effect on perceptions of US actors (with the positive effect generally being more extensive than the negative one).  It is thus of interest to us to understand the determinants of the direction of the impact. As noted by Cooley, a US military presence in a host country can mean different things to different actors within society.\cite{Cooley2008} In this case, we believe that an ethnic\index{minority!ethnic} minority\index{minority} identification will be one of the factors that make it more likely to observe the negative, instead of positive, the effect of interactions on perceptions of US actors for the theoretical reasons described above, along with supporting testimony from local populations and US officials. 

Given both direct and indirect mechanisms, we draw the following hypothesis:

\begin{hyp}
	Minority\index{minority} populations in states that host US military forces will be more likely to have negative perceptions of US military forces, the United States government\index{government}, and the United States people. 
\end{hyp}

% I don't think we need this. If we can reject the null in a positive direction, we can then also reject the first hypothesis. - AS
%The start of this section noted that there is an outstanding belief that the United States is a force for minority empowerment and protection globally. As such, we allow for the opposing hypotheses as well. Theoretically, we are less convinced by the arguments supporting it, but include it for completeness of the theoretical record:

%\begin{hyp}
%	Minority populations in states that host US military forces will be more likely to have positive perceptions of US military forces, the United States government, and the United States people.
%\end{hyp}

Combining our research in chapter \ref{cha:meth} about the role of contact\index{contact}, interactions with minority\index{minority} members of the community is likely to have a uniquely different effect on perceptions alone. While we include contact\index{contact} in the models, we also include an interactive effect between minority\index{minority} respondents that report contact\index{contact}. Given that negative contact\index{contact} with troops generates movements against the US presence in places like Okinawa\index{Japan!Okinawa}, we expect that this type of contact\index{contact} will amplify the negative view that minority\index{minority} populations have of the US presence and make it even more negative.

\begin{hyp}
	Minority\index{minority} populations in states that host US military forces will be more likely to have more negative perceptions of US military forces, the United States government\index{government}, and the United States people due to interacting with the US military. 
\end{hyp}

%Same thing here. I don't think we need the reverse hypothesis. Can deduce it from what we find. - AS
% However, both the positive view of the United States and the social theory we express in contact, it is possible that contact has a mitigating effect on people's perception. As such, we test whether contact reduces the expected negative perception. 

%\begin{hyp}
%	Minority populations in states that host US military forces will be more likely to have less negative perceptions of US military forces, the United States government, and the United States people as a result of interacting with the US military. 
%\end{hyp}

%NOTE: Do we think this effect will carry over to perceptions of the US government and US people? Or is it just about the US military being there and disrupting their lives? Maybe the direct effect of having the military disrupting your day to day life affects your views of the military but not the other two, whereas the indirect effect of having the US support the oppressive state also affects views of the US govt and people? Is there a way we can think of to distinguish between the two when testing? Maybe look at the base locations? If you have a base in your subnational unit you're more likely to face everyday disruptions?

%add additional hypotheses here.


\subsection*{Expectations}

Ethnic\index{minority!ethnic} and minority\index{minority} group relations related to US military deployments abroad have been a concern since the US first started deploying its troops abroad as a major power. The United States often portrays itself as an ally of minority\index{minority} groups abroad. For example, through its interventions in the Balkans\index{Balkans}, in different parts of Africa\index{Africa}, and its support of the Kurds\index{Kurds} in Iraq\index{Iraq}, American military power has been brought to bear to protect minority\index{minority} groups from hostile and repressive governments\index{government}. Though these interventions have sometimes been long-term and costly, they have helped some groups realize ambitions for an independent national state; for others, it has led to the cessation of brutal oppression by majority-led governments\index{government}. Other times, the United States has used its non-military power to press governments\index{government} worldwide for increased respect for human rights\index{human rights}, particularly in the treatment of minority\index{minority} populations. In fact, states who host US military forces have seen an increased respect for human rights\index{human rights}.\cite{bell2017}

However, the United States itself has also had problems with violating the rights of ethnic\index{minority!ethnic} minorities\index{minority} within the United States and has struggled with systematic discrimination against ethnic\index{minority!ethnic} minorities\index{minority}.\cite{Williamson2018} Beyond its borders, scholars, activists\index{activists}, and journalists\index{journalists} have accused the United States of exporting attitudes of racial inferiority, stereotypes, and hierarchies\index{hierarchy} into foreign states. In particular, South Korea\index{South Korea} and Saudi Arabia\index{Saudi Arabia}, two states that have hosted large numbers of US military forces, demonstrate the proliferation of US racial attitudes.\cite{Moon1997,Vitalis2007} Furthermore, after conducting long wars in Iraq\index{Iraq} and Afghanistan\index{Afghanistan} and recently curtailing immigration\index{immigration} and refugee\index{refugee} flows from predominantly Islamic\index{Islam} countries, Muslims\index{Muslim}, whether in majority- or minority-Muslim\index{Muslim} contexts, may not see the United States as an ally against oppression. 
%The history of Diego Garcia, as covered in Chapter 1, gives further credence to the idea that the United States would rather not deal with human populations at all if it can be avoided entirely.

These two trends and narratives leave us with a puzzle. Do minority\index{minority} populations in states that host the US military see it as a friend or a foe? There are several mechanisms by which either of these two narratives could be true. Despite US efforts, long-term deployments in peaceful countries could create direct and indirect paths toward minority\index{minority} discontent with the United States. Directly, Status of Forces Agreements\index{Status of Forces Agreements} between host states and the United States place some of the basing burdens on minority\index{minority} groups not well represented by host-state governments\index{government}. Okinawa\index{Japan!Okinawa}, which in a 2019 referendum voted 72\% against the construction\index{construction} of a new US military base on the Okinawan\index{Japan!Okinawa} town of Henoko\index{Japan!Henoko} (a project that involved various environmental\index{environment} concerns) is an example of this dynamic.\cite{tokyoone20200427,McCurry2019}

%changing language to be more hypothetical and still open to adjudication

Indirectly, the very act of basing also sends credible signals about US commitment to the host state and could empower it to maintain status quo policies towards disenfranchised groups. As an example of this, in Bahrain\index{Bahrain}, where the United States has a significant basing presence by housing both the United States Naval Forces Central Command\index{Naval Forces Central Command} (NAVCENT) and the 5th fleet\index{5th Fleet}, the US seemed to turn a blind eye to Shia\index{Shia}-led protests\index{protest} during the Arab Spring\index{Arab Spring}, which the Sunni\index{Sunni} government\index{government} put down violently.\cite{McDaniel2013} While the Shia\index{Shia} population is not a numerical minority\index{minority} in Bahrain\index{Bahrain}, they are the non-dominant ethnic\index{minority!ethnic} group. These combined direct and indirect mechanisms could compel minority\index{minority} groups to see the United States in less positive terms than the dominant group. Contrarily, the United States spent decades in Iraq\index{Iraq} protecting Kurdish\index{Kurds} populations from Saddam Hussein's\index{Hussein, Saddam} central government\index{government}. The US also attempted to restrain Turkish\index{Turkey} cross-border incursions into Kurdish\index{Kurds} areas. Overall, the United States went to great lengths to shield this minority\index{minority} population from adversarial governments\index{government} and groups throughout the region. 

This chapter thus focuses on studying the differences between the views of individuals who self-identify as ethnic\index{minority!ethnic} minorities\index{minority} in the 14 countries in our survey\index{survey} sample from the rest of the population to adjudicate between these two potential lines of thinking. This chapter goes further than previous ones by contextualizing the contact\index{contact} hypotheses from chapter \ref{cha:meth}. Specifically, we expect that contact\index{contact} will be conditionally different for minority\index{minority} groups relative to the general population. Given the different relationships between minority\index{minority} and majority populations and their governments\index{government}, and the potentially systematically distinct views of the role the United States plays in both the host country and the international system, we expect that those self-identifying as minorities\index{minority} will have consistently different attitudes toward the US military across host nations. 

%Expand a bit on the Kuwait stuff later.   

%\footnote{We address the case of Kuwait\index{Kuwait} in more detail in our research design, as this is also a case in which the dominant ethnic\index{minority!ethnic} group, ethnic\index{minority!ethnic} Kuwaitis\index{Kuwait}, is a numeric minority\index{minority}. Ethnic\index{minority!ethnic} Kuwaitis\index{Kuwait}, the dominant ethnic\index{minority!ethnic} group, make up 30.4\% of the population, compared to 40.3\% Asians\index{Asians}, many of whom are immigrant labourers. This makes inference of our variables related to being in a ``minority\index{minority}'' challenging. All other cases consist of countries where the ethnic\index{minority!ethnic} majority is also the dominant group, necessitating that we consider Kuwait\index{Kuwait} separately.} 


% We expect, conditionally, that contact with minority members of the population will moderate the positive effects of contact and have a distinctly more negative effect than traditional contact with majority groups.  %change amplify/moderate and negative/positive.

%[ADD a summary of the findings when we have them]
%This chapter's findings thus help to add nuance to policy decisions regarding military engagement with communities around military installations. Briefly, across the countries in our survey sample we find that there is a general pattern of self-identified minority group members holding less positive views of the US than non-minority individuals. However, when we look at the countries individually we find that context matters, and the relationships between minority self-identification and views of the US vary across countries. In some cases minority groups appear to hold more favorable views of the US, while in other cases minority groups hold less positive views of those same actors. In still other cases, we find that that is a stronger tendency towards neutral views among members of minority groups.

%Taken together our quantitative and qualitative findings demonstrate the importance of context-specific factors related to individual host societies, and highlight the importance of reaching out to minority groups who will often perceive the United States as supporting a system that discriminates against or represses them. It also shows that generally positive perceptions of the US military in the host country may obscure systematic differences across different racial, ethnic\index{minority!ethnic}, or religious groups. Because minority groups tend to hold less political power in their home countries, it is up to the US and its military to create positive relations with these communities and in some cases intercede on their behalf when dealing with host governments\index{government}. Particularly in cases where the US messaging system toward a host country is run at the national level, through embassy staff, or through the national host government\index{government}, information systems related to the US presence may not be adequately reaching minority populations. This may both reduce the effect of messaging about the purpose and opportunities associated with the US presence and also limit the potential for fruitful interpersonal connections and economic relationships with minority populations that exist with the rest of the host country.

%To test our hypothesis, we conducted a survey in 14 different countries with approximately 1,000 respondents in each country. Using a multilevel fixed-effects logistic regression, we find that minority population support for US troops to be less than majoritarian populations. However, this effect does not hold when we ask about minority perceptions about US people or the government\index{government}. Such findings for research on United States Foreign Policy, Human Rights\index{human rights} and Democracy\index{democracy} promotion, and for policymaking directly. US decision-making in the security realm affects US objectives in other areas and may undermine efforts towards liberal institution building. 


\section*{Research Methodology}
%\ref{cha:theory} and \ref{cha:meth}
Our research design follows the basic approach we discuss in chapters 2 and 3, but here we shift the focus to thinking more specifically about how minority\index{minority} populations within each country view various US actors. We begin by estimating a series of multilevel Bayesian\index{Bayesian} categorical logit models with varying intercepts by country, which allows us to consider the possibility of different dynamics in each country. The analysis in this section proceeds in three stages. First, we estimate a set of models using a single predictor variable---minority\index{minority} self-identification. Minority\index{minority} status is potentially determined by factors unrelated to other individual-level characteristics that we include in our data (meaning that minority\index{minority} status is \textit{exogenous} to other individual characteristics). Accordingly, it can be informative to begin by looking simply at basic differences between individuals who self-identify as some racial, ethnic\index{minority!ethnic}, or religious minority\index{minority} and those who do not. 

Second, because minority\index{minority} status likely correlates with many other individual-level and group-level traits, we estimate another set of models to adjust for some of these other considerations. For example, members of minority\index{minority} groups may have less access to education, lower incomes, or may be geographically concentrated in areas that are more vulnerable to the negative externalities of US basing, like noise, crime\index{crime}, or pollution. The aim here is to understand better how the conditions of self-identified minorities differ from other individuals concerning attitudes towards US basing and troops and what additional information an individual's minority\index{minority} status conveys once we've adjusted for these other considerations. 

Third, we estimate these same fuller model specifications again. Still, this time allows the effect of the minority\index{minority} self-identification variable to vary across countries (by allowing its coefficient to vary across countries). This will provide us with some insight into whether the differences between self-identified minorities and non-minorities are similar, or if they vary, across countries. As we discuss above, relations between US service members and minority\index{minority} populations may be dependent upon various contextual factors. In countries where minority\index{minority} populations disproportionately experience the negative effects of US deployments, we might expect to see more negative attitudes among minority\index{minority} groups towards the US military presence in their country. However, if some countries are home to minority\index{minority} groups who have greater equality in terms of income, education, and geographic mobility, we might expect more positive attitudes among those minority\index{minority} populations, or even no discernible difference from the rest of the population with respect to their views of US military personnel and other US actors. 

\subsection*{Outcome Variable}

As we outline in Chapter \ref{cha:meth}, we have three outcome variables of interest: Individuals' views of the United States military, the United States government\index{government}, and the United States people. We collapse the survey\index{survey} responses into four categories to estimate our models: 1) Positive, 2) Negative, 3) Neutral, and 4) Don't know/Decline to answer. Our primary focus here is on how minority\index{minority} self-identification affects individual attitudes about the US military presence in each country. However, we are also interested in how these relationships compare across different groups, and so we run two additional sets of models using views of the US government\index{government} and US people as outcome variables. Using these additional outcomes will help us determine how general the effects of the key predictor variables are and will help us better understand the contours of attitudes towards the US in general. 

\subsection*{Predictor Variables}

Our main predictor variable is the respondent's self-identification as a member of a minority\index{minority} group within the referent country. Our survey\index{survey} asked the question, ``Do you identify as a racial, ethnic\index{minority!ethnic}, or religious minority\index{minority}?'' Respondents could answer ``Yes,'' ``No,'' or ``Don't Know/Decline to Answer.'' Out of our full sample of 41,545 respondents, 22\% percent responded ``Yes'' to identifying as a minority\index{minority}.  

We emphasize two vital details here. First, readers should note that this includes racial, ethnic\index{minority!ethnic}, and religious groups. Accordingly, this question casts a broad net when considering an individual's status as a minority\index{minority} group. Second, this represents the respondent's self-assessment of their status within a given country. The implications of this are important in this context---the rates of minority\index{minority} self-identification in our survey\index{survey} data may differ from various official categorization schemes as embodied in national census counts. For example, 16.2\% of Japanese\index{Japanese} respondents self-identify as a minority\index{minority} of the type mentioned above in our survey\index{survey} data. At first glance, this seems at odds with common stereotypes of Japanese\index{Japanese} homogeneity. However, some authors have discussed that such ideas are often based on oversimplifications that mask variation in group identities.\cite{Johnson2019} While a fuller discussion of this topic is beyond the scope of this book, readers should keep this point in mind as they proceed through the rest of the chapter. 

% Check this. We probably want to add more details here about demographics as we get them.
Table \ref{tab:minoritypercent} shows the percentage of individuals who self-identified as belonging to a minority group in each country across the three years of our survey. Most countries in our sample tend to cluster around a 10--20\% minority self-identification rate. Germany\index{Germany} is considerably lower across all three years of our survey. In contrast, the Philippines\index{Philippines} and Kuwait\index{Kuwait} are consistently among the highest. The Philippines\index{Philippines} consistently returns 70--84\% minority self-identification rates. Kuwait\index{Kuwait} exhibits the most variation in our sample, with a low of 30\% in 2018 and a high of 98\% in 2020. We expect this to result from a relatively large population of migrant workers in Kuwait\index{Kuwait} and a bias in the survey firm's ability to reach more affluent Arab Kuwaiti\index{Kuwait} citizens as respondents. Notably, this also highlights one of the benefits of using the multilevel modeling framework---outliers like Kuwait\index{Kuwait} are less likely to exert considerable influence on the general results, particularly when we allow effects to vary across countries. That said, we are still sensitive to the possibility that the data collection process for Kuwait\index{Kuwait} is particularly skewed, so we run a series of robustness checks to further assess its influence on the results.


\input{../Tables/Chapter-Minority/table-minority-percent.tex}

%%%NOTE check the sentence below, it said no before, but i think it needed to say yes 

In general, we expect that individuals who respond ``yes'' to whether they identify as a minority\index{minority} will be more likely to have a negative view of the United States military. Beyond that, given the theoretical interest and empirical support for ``contact''\index{contact} thus far, we expect an interactive effect between contact\index{contact} and self-reported minority\index{minority} status. We include the contact\index{contact} variable (see chapters \ref{cha:theory} and \ref{cha:meth}) as a constituent variable as well as a multiplicative interaction term. Jointly, these variables will amplify or dampen the effects of the constituent variables on the dependent variable (views on various actors). Likewise, we maintain the previous control variables discussed in previous chapters. 

%add more if we add anything specific. 

\section*{Results}

First, we discuss the results for the population-level effects from our most basic model that includes only the minority\index{minority} self-identification variable, shown in Table \ref{tab:minoritybase} in the Appendix. These basic models give us a first look at how minorities compare to non-minority\index{minority} populations without adjusting for other factors. To ease interpretation of the results throughout this chapter, we include figures showing the predicted coefficient values. Figure \ref{fig:minoritycoefbase} shows the results of this basic model. This plots the predicted coefficient of minority\index{minority} self-identification on attitudes towards each of the three groups. Given the data we have, each coefficient tells us how strongly a variable correlates with the outcome and whether that correlation is positive or negative. For each coefficient, our model then generates a range of possible more or less likely values. Following a typical bell curve, values closer to the middle are more likely, while values at the edges are less likely. The three panels each represent the effects associated with the three outcome variables. Each row represents the effect on the possible responses (for example, a positive response, a negative response, or a don't know/decline response). Each estimate that we depict shows the median point estimate and the 90\% and 95\% high posterior density credible intervals from the posterior distribution\footnote{The posterior predictive distribution refers to the probability of an individual giving a particular response after we have taken other relevant information into account. See Chapter \ref{cha:meth} for a more detailed discussion of this statistical method.}. 

In these basic models, we see that, compared to non-minority\index{minority} individuals, those identifying as belonging to a minority\index{minority} group tend to have a higher probability of expressing a positive view of all three US groups. Alternatively, self-identified minorities have a lower probability of expressing a negative attitude of US troops in the host country and the US government\index{government}. However, there is some evidence that they also have a higher likelihood of expressing a negative view of the US people---81\% of the probability mass of this coefficient's distribution falls above 0 (a zero coefficient is essentially like a 50/50 coin flip, in terms of probability). Finally, minority\index{minority} respondents tend to have a lower probability of responding with ``Don't know/Decline to answer'' than non-minorities. This holds except for the question on views of the US people in general.

\begin{figure}[t]
	\includegraphics[scale=0.8]{../Figures/Chapter-Minority/fig-coefplot-base.png}
	\caption{Coefficient estimates for minority\index{minority} self-identification and views of US actors. Credible intervals show the 50\%, 80\%, and 95\% highest posterior density intervals around the median point estimate. }
	\label{fig:minoritycoefbase}
\end{figure}


As we note above, these results present an initial look at exploring how minority\index{minority} groups view US military deployments and other US actors and groups. Minority\index{minority} status often correlates with a range of other important characteristics that are also likely to influence views and personal experiences. Once we adjust for these other factors, we may find that minority\index{minority} experience itself yields a different picture of individuals' attitudes. The models presented in Table \ref{tab:minorityfull} in the appendix replicates the first model we estimate, but this time we include several of these other individual-level and group-level variables. As we do above, Figure \ref{fig:minoritycoeffull} plots the coefficient for the minority\index{minority} self-identification variable along with a few other demographic variables to ease interpretation and comparisons across models. 


\begin{figure}[t]
	\includegraphics[scale=0.8]{../Figures/Chapter-Minority/fig-coefplot-full.controls.png}
	\caption{Coefficient estimates for minority\index{minority} self-identification and views of US actors. Models include various individual demographic and attitudinal variables. Credible intervals show the 50\%, 80\%, and 95\% highest posterior density intervals around the median point estimate.}
	\label{fig:minoritycoeffull}
\end{figure}



There are some important changes when we include these other variables in our models. The bottom row in Figure \ref{fig:minoritycoeffull} shows the coefficients on the minority\index{minority} variable, with the different colors representing the three outcomes (for example, positive, negative, or don't know). Across all three models, we see that after adjusting for these other variables, individuals who self-identify as belonging to a minority\index{minority} group are less likely to report a positive view of US military personnel deployed to their country, the US government\index{government}, and the US people in general. They are also less likely to report a \textit{negative} view of the US government\index{government}. Overall the results suggest that minority\index{minority} populations are less likely than non-minorities to see US actors positively, but this does not necessarily translate into more negative attitudes. Across all three reference groups, it appears that the lower likelihood of positive views is more offset by an increase in the propensity to have neutral attitudes. 


Comparing the minority\index{minority} variable coefficients with coefficients for the other variables is informative. First, looking at the age variables, we can see that the older age cohort tends to have a higher likelihood of expressing a positive view of all three US groups than younger age cohorts, and a lower probability of responding with ``I don't know''. Older age cohorts also appear to be less likely to express a negative view of US actors than younger groups.  The reference category here is the 18--24 age cohort, meaning that we should evaluate all of the coefficients concerning that baseline. However, we can still compare the coefficient values of the other groups but need to be cautious where there is a significant overlap between the posterior distributions. 

Individuals with more education appear to be more likely to express both positive and negative views of US actors. However, there is no clear indication that more education correlates with larger differences in negative versus positive views. As we discuss in Chapter \ref{cha:theory}, this likely reflects a similar process whereby individuals with more education simply have more informed opinions, one way or the other. 

%fix references to the appendix

Regarding gender, individuals identifying as female tend to be less likely to express positive views than those who identify as male. Female respondents also appear more likely than male respondents to give a ``Don't know/Decline to Answer'' response when asked their views of these three groups.

Finally, when comparing across income quintiles, we find some interesting patterns. First, the baseline category here is the lowest income grouping---the 0--20\% quintile. Compared to this group, there is some indication that higher income groups tend to have slightly less favorable views of US military personnel, but these differences are very small. We find a similar pattern when looking at views of the US government\index{government}, where only the highest income group shows a slightly positive coefficient value. When looking at views of the US people, groups above the 40\% income group appear to have a slightly more positive probability of expressing positive views than the bottom two income categories. We find somewhat similar patterns when looking at negative views. However, we do see larger differences in the probability of expressing negative views among the highest income groups when looking at views of the US government\index{government}. Perhaps the most notable feature here is that income appears to correlate quite strongly with a lower probability of respondents saying ``Don't Know/Decline to Answer'' when asked about any of the three US groups. Only 2.7\% of the highest income groups across all 14 countries replied ``Don't Know/Decline to Answer'' when asked about the US military presence in their country. On the other hand, 7.7\% of respondents in the lowest income group gave the same response. Like education, income appears to correlate more strongly with more informed opinions instead of clearly lining up with more positive or more negative attitudes. 

Next, we discuss the results of a model that is identical to the one we developed in Table \ref{tab:minorityfull}, but this time we allow the coefficient on the minority\index{minority} variable to vary across countries. We show the results from this model in Appendix Table \ref{tab:minorityvarying} and we plot the minority\index{minority} variable coefficient in Figure \ref{fig:minoritycoefvarying}. This figure shows that there is considerable variation in the coefficient on the minority\index{minority} variable across countries. Put differently, the differences between minority\index{minority} and non-minority\index{minority} populations regarding their views of US actors are highly context-dependent. 

Beginning with the model predicting views of US troops, minority\index{minority} self-identification correlates with a higher probability of a positive response in Turkey\index{Turkey}, Portugal\index{Portugal}, and Italy. The posterior distributions for the minority\index{minority} coefficient for Spain\index{Spain} and South Korea\index{South Korea} are largely positive, with a 93\% and 92\% chance of a positive coefficient given the data. Alternatively, we see minority\index{minority} self-identification correlates with a higher probability of a \textit{negative} response in the United Kingdom\index{United Kingdom}, the Netherlands\index{Netherlands}, and Belgium, with slightly more mixed evidence in South Korea\index{South Korea}, the Philippines\index{Philippines}, and Australia\index{Australia}. In the case of South Korea\index{South Korea}, there's an 83\% chance of a positive effect given the data.

%fix references to the appendix

In the model predicting attitudes towards the US government\index{government}, we find that minority\index{minority} self-identification leads to an increase in the probability of a positive response in several countries, including Turkey\index{Turkey}, Spain\index{Spain}, South Korea\index{South Korea}, and Belgium\index{Belgium}. We see slightly more limited evidence of this effect in Italy and Germany\index{Germany} as well. There is also evidence that minority\index{minority} self-identification leads to a \textit{decrease} in the probability of a respondent expressing a \textit{negative view} of the US government\index{government}. Individuals are more likely to report that they do not know, decline to answer, or have a positive view. This is clearest in Spain\index{Spain}, Portugal\index{Portugal}, the Netherlands\index{Netherlands}, and Australia\index{Australia}, with largely negative (but slightly more mixed) coefficients in Belgium\index{Belgium}, Italy\index{Italy}, the Netherlands\index{Netherlands}, and Turkey\index{Turkey}. 

Looking at the attitudes towards US people, minority\index{minority} status correlates with a higher probability of a positive response in Turkey\index{Turkey} and Portugal\index{Portugal}, with some evidence of a smaller effect in South Korea\index{South Korea}, Belgium\index{Belgium}, and Germany\index{Germany}. We see evidence of a higher probability of a negative response across multiple countries, but fairly sizable portions of the posterior distributions overlap with 0 in many of these cases, indicating slightly more uncertainty over the direction of the effect on negative responses. The clearest cases of a higher probability of negative views among minority\index{minority} communities are in Belgium\index{Belgium}, Germany\index{Germany}, Japan\index{Japan}, Kuwait\index{Kuwait}, the Netherlands\index{Netherlands}, the Philippines\index{Philippines}, South Korea\index{South Korea}, and the United Kingdom\index{United Kingdom}. These countries all yield posterior distributions for the minority\index{minority} coefficient greater than or equal to $70\%$.  Further, we again see evidence of minority\index{minority} self-identification correlating with a lower probability of positive responses, as in the United Kingdom\index{United Kingdom}, Spain\index{Spain}, the Philippines\index{Philippines}, the Netherlands\index{Netherlands}, Kuwait\index{Kuwait}, and Australia\index{Australia}.

\begin{figure}[t]
	\includegraphics[scale=0.8]{../Figures/Chapter-Minority/fig-coefplot-varying.png}
	\caption{Coefficient estimates for minority\index{minority} self-identification and views of US actors across countries. Models include various individual demographic and attitudinal variables. Credible intervals show the 50\%, 80\%, and 95\% highest posterior density intervals around the median point estimate.}
	\label{fig:minoritycoefvarying}
\end{figure}


Taken together, these models highlight a few key points. First, the way that minority\index{minority} communities view various US actors compared to non-minority\index{minority} communities varies across countries and can be quite more complicated than our basic expectations suggest. This cross-national variation is interesting not just because minority\index{minority} groups in some countries have more favorable attitudes of US actors than minority\index{minority} groups in other countries, but because minority\index{minority} self-identification does not appear to cause a clear linear increase or decrease in views of the United States. In some cases, like in Portugal\index{Portugal}, we see clearer evidence of minority\index{minority} self-identification leading to more positive views and less negative views than non-minority]index{minority} respondents. In the United Kingdom\index{United Kingdom}, we see the exact opposite pattern, with minority\index{minority} respondents more likely to express negative views of US military personnel and less likely to express positive views. However, in South Korea\index{South Korea}, we see some evidence that minority\index{minority} status correlates with increases in \textit{both} positive and negative views, suggesting that minority\index{minority} respondents have stronger opinions about US military personnel and the US people, and fewer individuals expressing neutral views, as compared to non-minority\index{minority} individuals. In still other cases, like Poland\index{Poland}, we find evidence that minority\index{minority} respondents are less likely to express positive views of US actors as compared to non-minority\index{minority} respondents. Still, minority\index{minority} status does not appear to affect negative views in the same way. 


\input{../Tables/Chapter-Minority/table-varying-coefficient-posterior-percent.tex}


Thus far, we have only examined the effect of the minority\index{minority} self-identification variable alone. However, we also expect that differences between minority\index{minority} and non-minority\index{minority} groups may also be conditional upon the level of interpersonal contact\index{contact} individuals report. Specifically, we believe there is reason to expect that minority\index{minority} groups will be more likely to hold negative attitudes towards US groups if they report more interactions with US personnel. As we discussed above, it is often the case that minority\index{minority} groups are exposed to more of the negative externalities associated with US basing and military deployments, so the nature of their interpersonal contacts\index{contact} may also be different, on average, as compared to other groups. 


\begin{sidewaysfigure}[h!p]
	\centering\includegraphics[scale=0.48]{../Figures/Chapter-Minority/figure-posterior-predictive-varying.png}
	\caption{Posterior predictive check for varying effect models. Darkened dots with credible intervals show the mean predicted count of the outcome categories based on 1,000 simulations from the model. Light blue vars show the actual count of each outcome category as observed in the data. Better fitting models should produce simulated values that are closer to the actual observed data.}
	\label{fig:minorityvaryingppcheck}
\end{sidewaysfigure}



We present the results of these models in Figure \ref{fig:minorityinteractionvaryingeffect}.\footnote{We generate these comparisons by holding other variables constant at specific categories and values. Age is set to ``35--45 years'', Income to the ``41$^{st}$--60$^{th}$'' percentile, Gender to ``Female'', Personal Benefits to ``No'', Experience with Crime\index{crime} to ``No'', the Amount of American Influence\index{American influence} in the referent state to ``Some'', the Quality of American Influence\index{American influence} to ``Neither positive nor negative'', and the remaining variables---education, ideology, the count of US bases, GDP, population, and troop deployment size---to their mean values.} For the sake of simplicity, we present only a figure showing the primary comparisons of interest for the positive and negative outcome equations. Here we seek to illustrate two primary comparisons: 1) how minority\index{minority} respondents reporting interpersonal contact\index{contact} with US personnel compare to minority\index{minority} respondents who do not report such contacts\index{contact}, and 2) how minority\index{minority} respondents compare to non-minority\index{minority} respondents among those who report having interpersonal contacts\index{contact} with US military personnel. We do this by generating simulated posterior predictions of the probability of a positive or negative response for each group and then comparing the posterior distributions across different groups. These models are the most flexible concerning allowing the coefficients to vary across countries and factor groupings and therefore allow for the most variation in inter-group comparisons. Accordingly, we focus on what we think are some of the more notable points but cannot claim to provide an exhaustive overview of all possible comparisons.

Most generally, it is worth observing that the combination of interpersonal contact\index{contact} and minority\index{minority} self-identification can. It often does generate substantively large changes in the predicted probabilities across all three models. We see the most variation in the US troops model. Personal contact\index{contact} with US military personnel among self-identified minority\index{minority} populations tends to produce larger positive changes in the predicted probability of a positive response. In contrast, when contact\index{contact} is held constant at a ``Yes'' value, toggling minority\index{minority} self-identification on tends to produce more mixed changes, with several countries seeing some reduction in the predicted probability of a positive response, but a few cases, like Turkey\index{Turkey}, Spain\index{Spain}, Portugal\index{Portugal}, and Italy yielding increases in the probability of a positive response. 

Looking at the negative responses, we find a mixture in how minority\index{minority} respondents compare to non-minority\index{minority} respondents when both groups report interpersonal contact\index{contact} with US personnel. In some countries like Turkey\index{Turkey}, Spain\index{Spain}, and Italy\index{Italy}, minority\index{minority} respondents tend to be less likely to give a negative response. In contrast, in other countries, like Poland\index{Poland}, the Philippines\index{Philippines}, and to a lesser extent the Netherlands\index{Netherlands} and Belgium\index{Belgium}, we find that minority\index{minority} respondents are slightly more likely to give a negative response. When we focus on minority\index{minority} respondents and compare those reporting contact\index{contact} versus those not reporting contact\index{contact}, we find that the contact\index{contact} group has a lower probability of giving a negative response in several countries, including the United Kingdom\index{United Kingdom}, Turkey\index{Turkey}, Spain\index{Spain}, and Belgium\index{Belgium}. We find smaller differences in Italy and Australia\index{Australia}, where the comparisons of the posterior distributions do not uniformly show the contact\index{contact} group to be less likely to give a negative response.

\begin{figure}[t!]
	\centering\includegraphics[scale=0.7]{../Figures/Chapter-Minority/fig-contact-minority-interaction.png}
	\caption{Figure shows the change in the predicted probability of positive and negative responses according to changes in the personal contact and minority\index{minority} self-identification variables. These changes were derived from varying effects models identical to those shown in Table \ref{tab:minorityvarying}, except we now include an interaction term between minority\index{minority} self-identification in the population-level equation and allow the effects of these three terms to vary across countries. 50\%, 80\%, and 95\% credible intervals shown around the predicted values.}
	\label{fig:minorityinteractionvaryingeffect}
\end{figure}

When we look at the US Government\index{government} models, we see more consistent increases in the probability of a positive response when we toggle both variables ``on'' while holding the other constant in the ``on'' position across most countries in our data. Poland\index{Poland}, the Philippines\index{Philippines}, and Kuwait\index{Kuwait} are the notable exceptions here. As we have discussed elsewhere, however, Kuwait\index{Kuwait} and the Philippines\index{Philippines} are significant outliers concerning the size of the minority\index{minority} populations in our survey\index{survey} data. While this over-representation is not necessarily a problem in and of itself, the extreme differences we observe in the case of Kuwait\index{Kuwait} may reflect the more unique status of the large migrant worker communities rather than ``domestic'' minority\index{minority} communities.

We find slightly more variation when we look at the negative response comparisons. In general we find that, among those reporting interpersonal contacts\index{contact} with US personnel, minority\index{minority} respondents are often less likely to give a negative response than non-minority\index{minority} respondents. These differences are most evident in the United Kingdom\index{United Kingdom}, Spain\index{Spain}, Portugal\index{Portugal}l, the Netherlands\index{Netherlands}, Germany\index{Germany}, and Belgium. Among minority\index{minority} respondents only, we find that those reporting interpersonal contacts\index{contact} with US service personnel are less likely to respond negatively in the United Kingdom\index{United Kingdom}, Portugal\index{Portugal}, and Belgium\index{Belgium}, with more limited evidence of similar differences in countries like Japan\index{Japan} and Germany\index{Germany}.

Finally, when we look at the ``US people'' models, we find a more consistent pattern of minority\index{minority} respondents with personal contact\index{contact} reporting a higher probability of a positive response in most countries. We find these differences are more muted in Poland\index{Poland}, Japan\index{Japan}, and Italy. In general, the differences between minority\index{minority} respondents who report contact\index{contact} and those who do not are largest in Australia\index{Australia}, and Belgium\index{Belgium}, with Turkey\index{Turkey}, Portugal\index{Portugal}, and Kuwait\index{Kuwait} showing changes of approximately 10--15 percentage points in the predicted probability of a positive response. And of those individuals reporting having had personal contact\index{contact} with US military personnel, we find that minority\index{minority} respondents tend to have a lower probability of expressing a positive view of the US people in most countries. The notable exceptions here are Turkey\index{Turkey}, Portugal\index{Portugal}, Germany\index{Germany}, and Belgium\index{Belgium}, where these groups' differences are more muted.

The comparisons are decidedly different when we look at the negative response models. To facilitate comparisons across response equations and models we held the range of X-axis values constant across panels in the figure. We are using the same scale for the negative models for comparison even though this compresses the perceivable range of the values for these models.  We do indeed find that there are differences between minority\index{minority} and non-minority\index{minority} respondents who report contacts\index{contact}, as well as among minority\index{minority} respondents who do and do not report interpersonal contacts\index{contact}, but the magnitude of these differences is considerably smaller in this model and equation. For example, we find that minority\index{minority} respondents are generally less likely to give a negative response in Turkey\index{Turkey} and Portugal\index{Portugal}, with some more limited differences in Belgium\index{Belgium}, and to a lesser extent Germany\index{Germany}. Among minority\index{minority} respondents, we also find that those reporting interpersonal contacts\index{contact} are less likely to respond negatively in Turkey\index{Turkey}, Portugal\index{Portugal}, Belgium\index{Belgium}, and Australia\index{Australia}. Again, we find more limited differences in the United Kingdom\index{United Kingdom}, the Netherlands\index{Netherlands}, Spain\index{Spain}, Kuwait\index{Kuwait}, and Germany\index{Germany}. Importantly, while we observe differences, the magnitude of these differences compared to other models and outcomes is much smaller. In general the differences we see here are in the 3--6 percentage point range. 

Regarding broader model performance, the varying coefficient models generally perform well. Figure \ref{fig:minorityvaryingppcheck} shows the results of posterior predictive checks on the three sets of varying coefficient models shown in Appendix Table \ref{tab:minorityvarying}, broken down by the country groups in each model. These checks function by using the models to generate several simulated ``data sets'' for the outcome variable. For example, we use the model to generate $\sim$ 38,000 simulated response values for each respondent in the data, and then we do this iteratively 1,000 times. We can compare the relative frequency of these simulated values to the actual frequency of the outcome variable categories as observed in the actual data. In this plot, the dark dots represent the mean of 1,000 different simulations and 95\% credible intervals. The lighter blue bars represent the actual count of each category observed in the data for a given country. Overall the models perform well and the simulated counts fall very close to the observed counts for all levels of the outcome across all countries.




\subsection*{A look at Japan\index{Japan}}

Japan\index{Japan} is well-known for the hotly contested presence of US military facilities, particularly in areas like Okinawa\index{Japan!Okinawa}. However, US military facilities exist throughout the country, so the relationship between minority\index{minority} groups and US facilities may vary depending on geographic context. For example, while many people associate minority-military interactions with Okinawa\index{Japan!Okinawa}, there is often overlap between minority\index{minority} communities and US military facilities in general. That is, aggregating to the regional or province-level and looking for relationships between the location of US facilities and regions with high minority\index{minority} populations can be misleading. 

Figure \ref{fig:japanminoritymap} shows the breakdown of minority\index{minority} respondents across regions within Japan\index{Japan} (shaded green), and the distribution of US military bases (blue dots).\footnote{Note that this map aggregates Okinawa\index{Japan!Okinawa} into the Kyushu\index{Japan!Kyushu} region. In the analysis that follows, we break Okinawa\index{Japan!Okinawa} out into a separate region given its historical importance. Unfortunately, we only have data parsing Okinawa\index{Japan!Okinawa} out in this way for two of the three years of our survey\index{survey}.} As the map shows, the largest proportion of self-identified minority\index{minority} group members in our survey\index{survey} are located in the Kanto\index{Japan!Kanto} Region of Japan\index{Japan}, which includes Tokyo\index{Japan!Tokyo} as well as some key US military installations. While our survey\index{survey} is not designed to be geographically representative there is still an important point to be made here---while there is often a correlation between geographic region and minority\index{minority} status, these two variables do not always map neatly on to one another. Even in regions that are not normally synonymous with particular minority\index{minority} groups there may still be large minority\index{minority} populations. Those populations may have very different interactions with US military personnel than individuals who do not belong to a minority\index{minority} group—for example, occupying space close to bases or spaces exposed to the negative externalities of bases.


\begin{figure}[t]
	\centering\includegraphics[trim = 1.75cm 1cm 1cm 1cm, scale=0.85]{../Figures/Chapter-Minority/fig-map-japan-min-location.png}
	\caption{The distribution of Japanese\index{Japanese} respondents who self-identified as belonging to a minority\index{minority} group and the location of US military bases. The shading intensity represents the share of all minority\index{minority} respondents present in each region.}
	\label{fig:japanminoritymap}
\end{figure}


To examine how minorities compare to non-minority\index{minority} groups in Japan\index{Japan}, we estimate a model identical to the one shown in Appendix Table \ref{tab:minorityvarying}, but this time include only Japan\index{Japan}. As in the full cross-national version of this model, we include several predictor variables associated with minority\index{minority} status, allowing the minority\index{minority} self-identification variable to vary across regions. Figure \ref{fig:coefplot-japan} shows the coefficients for the minority\index{minority} self-identification variables for the Positive, Negative, and Don't know/Decline to answer equations for the three models and reference groups.

Across all three models, we find little difference between minority\index{minority} and non-minority\index{minority} respondents with respect to the probability of a respondent expressing a positive view of US actors. The predicted coefficient value is extremely close to zero, and a substantial amount of the posterior distribution falls on either side of 0. Only in the Kanto\index{Japan!Kanto} region is there some limited evidence of minority\index{minority} respondents having a slightly higher probability of expressing a positive view of the US people, specifically. 

We find slightly more evidence that minority\index{minority} respondents differ from non-minority\index{minority} respondents when we look at the negative outcome equations in each model. We also see some evidence that the differences between minority\index{minority} and non-minority\index{minority} respondents are partly conditional upon the respondents' region. Most of the median point predictions for the posterior distributions fall below 0, and in some cases fairly large proportions of those distributions fall below 0. In Kanto\index{Japan!Kanto}, for example, over 96\% of the posterior distribution falls below 0, indicating that minority\index{minority} respondents there are slightly less likely to express a \textit{negative} view of US troops as compared to non-minority\index{minority} respondents, once we adjust for various other individual-level and environmental\index{environment} characteristics. Similarly, approximately 90\% and 92\% of the posterior falls below 0 in the Chuugoku\index{Japan!Chuugoku} and Hokkaido\index{Japan!Hokkaido} regions, respectively. Alternatively, in the Kinki\index{Japan!Kinki} region, we find that approximately 94\% of the posterior falls \textit{above} 0, indicating that minority\index{minority} populations there are slightly more likely to express a negative view of US military personnel than non-minority\index{minority} groups.


\begin{figure}[t]
	\centering\includegraphics[scale=0.75]{../Figures/Chapter-Minority/fig-coefplot-2-japan.png}
	\caption{Varying coefficient estimates for minority\index{minority} self-identification and views of US actors in Japan\index{Japan}. Coefficients vary across regions. 50\%, 80\%, and 95\% credible intervals shown around point estimate. Credible intervals are calculated using the highest density posterior interval.}
	\label{fig:coefplot-japan}
\end{figure}

Shifting to US government\index{government} views, the results look very similar to the model predicting views of US military personnel with respect to the differences between minority\index{minority} and non-minority\index{minority} groups in giving positive responses. For all regions, the median posterior prediction is close to zero, with substantial portions of the distribution falling on either size of 0. Ultimately this indicates that there is little difference between minority\index{minority} and non-minority\index{minority} groups concerning positive responses. Shifting to negative views, we get a slightly more homogeneous pattern wherein minority\index{minority} groups generally appear \textit{more} likely than non-minorities to give a negative response when asked their views of the US government\index{government}. In Kinki, we again find the most apparent evidence that minority\index{minority} groups tend to be more likely to express a negative view than non-minorities, with approximately 95\% of the posterior distribution falling above 0. However, in Chubu\index{Japan!Chubu}, Kyushu\index{Japan!Kyushu}, Shikoku\index{Japan!Shikoku}, and Tohoku\index{Japan!Tohoku}, we find 75\% or more of the posterior falling above 0 in each case.

Finally, looking at the model predicting attitudes of US people, we find slightly stronger evidence of a generally positive trend, wherein minority\index{minority} respondents are more likely to express a favorable view than non-minority\index{minority} respondents. This is clearest in the Kanto\index{Japan!Kanto} region, where approximately 90\% of the posterior distribution for the coefficient falls above 0. In nearly every other region, we find 70\% or more of the posterior distribution falling above 0, with only Kinki\index{Japan!Kinki} and Tohoku\index{Japan!Tohoku} having values in the 60\% range. Interestingly, we also find a more consistent pattern indicating that minority\index{minority} respondents are more likely to express a negative view of the US people. The posterior distribution on the minority\index{minority} coefficient is positive across every region, but in Chubu\index{Japan!Chubu}, Chugoku\index{Japan!Chuugoku}, and Tohoku\index{Japan!Tohoku} we find 90\% or more posterior falling above 0. We find slightly more muted patterns in Hokkaido\index{Japan!Hokkaido}, Kanto\index{Japan!Kanto}, Kinki\index{Japan!Kinki}, Okinawa\index{Japan!Okinawa}, and Shikoku\index{Japan!Shikoku}, where approximately 70\%--83\% of the posterior falls above 0.

Overall, it is somewhat surprising that we do not find more consistent relationships between minority\index{minority} self-identification and anti-US attitudes. This is what many historical narratives and anti-base activists\index{activists} would suggest we should expect to find. Some of the considerable variation (credible intervals) may be due to the relatively small size of the minority\index{minority} sample---with only about 500 Japanese\index{Japanese} respondents identifying as belonging to a minority\index{minority} group and many regions not having large minority\index{minority} respondent rates, we must allow for the possibility that some of variation we see here is mostly noise rather than genuine variability in effects. One advantage of the multilevel modeling approach is that it allows us to draw information from the general population to help inform estimates when we have fewer observations for particular groups. While this is certainly an advantage compared to other statistical approaches, like using binary fixed effects to capture differences in group-level intercepts, we are cautious given that we expect the differences between minority\index{minority} and non-minority\index{minority} respondents to vary across groups. The shrinkage induced by our approach here may also compress some of the between-group variations we expect to see where we lack more observations.

\section*{Conclusions}

When conducting interviews\index{interview}, we spoke with an anti-base activist\index{activists} in Berlin\index{Germany!Berlin}. During the conversation, he made it clear that minority\index{minority} populations, particularly immigrants\index{immigrant} and refugees\index{refugee}, were largely an afterthought in the dynamics of basing: ``They are not playing a key role in the base discussion. The Germans\index{German} and the US have security concerns about hiring immigrants\index{immigrant} and refugees\index{refugee} to work on the base. They are afraid to hire immigrants\index{immigrant}. They have strong security checks and controls. They are afraid to hire someone from al Qaeda\index{Al Qaeda}.''\cite{berlinone20190723} At the beginning of the chapter, we discussed the idea that many minority\index{minority} populations may see US military installations positively through the potential economic opportunity that they provide. However, as both this anti-base activist\index{activists} and the Government Relations Officer\index{Government Relations Officer} at the US installation in Wiesbaden\index{Germany!Wiesbaden} attest, there was little consideration given to local minority\index{minority} populations at all, let alone any concerted efforts undertaken to reach out to such communities. 

%To the extent that they consider minority populations, it is to sideline them from the types of employment otherwise available to local people on the base.  - This statement seems strong, do we have evidence of active sidelining for employment?

The data that we collected from countries around the world offers a glimpse into how these issues present themselves in public opinion\index{public opinion}. In our most general models, minority\index{minority} groups are less likely to report positive views of US actors than majority groups. When we allow the effect of minority\index{minority} self-identification to vary across countries, we find that the effect of minority\index{minority} status on views of the US is more mixed. In many countries, minority\index{minority} groups are less likely to report positive views of US actors than non-minority\index{minority} respondents, but often this simply translates into more neutral views, while in other cases, there is a clearer increase in actual negative sentiment. This complexity holds true when we look specifically at the effect of interpersonal contact\index{contact}, wherein some countries, self-identified minorities who report interpersonal contact\index{contact} with US forces have more positive views, and in others, they have more negative views. 


%\ref{cha:meth}
We have demonstrated in this chapter that it is important for decision-makers to see host nation populations in more complexity. For example, contrary to many popular assumptions, Johnson describes the relationships between the Okinawan\index{Japan!Okinawa} population and the US military personnel as a deep, multilayered subject, characterized by affection and nostalgia and resentment and anger.\cite{Johnson2019} The people in host nations do not have uniform experiences of the US military, and they do not have uniform views. As discussed in Chapter \ref{cha:meth}, views can depend on whether individuals have had interpersonal contact\index{contact} with members of the US military, and in this chapter, we have uncovered another complicating factor --- minority\index{minority} status. Minority\index{minority} communities in some cases may be on the receiving end of more negative externalities than majority communities. In such cases, the negative relationship between minorities and the US military is direct. In more indirect relationships like the one described in Germany\index{Germany}, minority\index{minority} populations are given very little consideration in the dynamics of military basing. In such circumstances, a more circuitous causal mechanism is likely at work, in which the American relationship with the host central government\index{government} empowers both the government\index{government} and the dominant ethnic\index{minority!ethnic} group. By intertwining itself with the ethnic-majority\index{minority!ethnic} central government\index{government}, the United States becomes conflated with groups contributing to minority\index{minority} discrimination and reducing pathways to minority\index{minority} representation through the centralization of power. 

Much as we would expect from public opinion\index{public opinion} polling in the United States, an individual's identity matters greatly in how they view the world. This is because an individual's experiences will influence their views. The same is true in other states, and researchers should consider it in analyzing how the United States interacts with societies in the international arena. While the US military may see a high degree of support from majority groups in some countries, not all populations have uniform experiences and views of the American military presence. Unique experiences with the US military, the host government\index{government}, and the majority population in the host country will influence individuals' views. From a policy perspective, this issue is vital to understand and apply to the US military's relations with local populations in host countries around the world. Even in societies that see high levels of favorable views toward the US military, negative views among a cohesive minority\index{minority} population can cause any number of issues --- from security concerns to land use issues, and threaten the stability of political support for a US presence. In previous chapters, we argue that outreach to surrounding communities is how the USmilitary can improve relations with surrounding communities. Extending these outreach efforts to minority\index{minority} groups, and specifically tailoring to their wants and needs, would be a way to build better relationships with minority\index{minority} populations and promote better relations between them and central governments\index{government}.  

%outreach recommendation might depend on what the contact variable looks like for minorities. - AS
These results call those engaging in policymaking surrounding US military bases and their relationship to ethnic\index{minority!ethnic} minorities\index{minority} to more deeply understand the communities engaged through the base-making and base-sustaining processes. Many national governments\index{government} have vastly different relationships with minority\index{minority} populations within their borders than others. Many ethnic\index{minority!ethnic} minority\index{minority} groups within the same country can relate very differently with their national governments\index{government} and an outside military force. While some will see an American presence as a sign of potential economic prosperity, others may see it as an unwelcome intrusion. Policymakers must deeply engage in the history and preferences of the individual countries and groups involved in basing decisions to understand whether and how to engage in outreach efforts. Potential basing relationships between the United States and minority\index{minority} populations can be healthy and productive.





\begin{comment}

%NOTE: Now that this chapter has more of a negative spin maybe we should think of a different title for this chapter

%\section*{Introduction}
\vspace*{-0.85cm}
\rule{\linewidth}{0.10pt} \\[-1.25cm]
{\footnotesize\paragraph{Summary:} Conventional accounts hold that minority communities often bear the brunt of the negative externalities of US military bases. Others suggest that US military bases can provide opportunities for social and economic mobility that may not otherwise be present for minority groups in some host countries. We show that the relationships between minority populations and their attitudes towards US actors are complex. First, minority self-identification alone correlates with a higher likelihood of expressing a positive view of US actors. Second, once we adjust for demographic and attitudinal variables that correlate with minority status, like income and education, this relationship disappears in the population-level effects and minority self-identification correlates with less positive views of all three groups. Finally, we find that there is substantial local variability---adjusting for other individual-level factors the difference between minority and non-minority views of US actors depends on the national context, as well as individual experiences.} 
\\[-0.5cm] 
\rule{\linewidth}{0.10pt}

\vspace*{0.5cm}

%%%%NOTE: Just went through the interview with the Wiesbaden mayor's office liaison and there's a ton about ethnic minorities that we haven't used yet. Let's make sure and go through and add some of that to this chapter. CM. 


``Okinawan people have a strong feeling that they bear too much of the cost,'' said the retired Japanese ambassador, sitting in his home in Tokyo \cite{tokyoone20200427}. In a polite, diplomatic fashion that is typical of his profession, and that on our third year of fieldwork we were starting to become familiar with, he further explained this sentiment to us, ``I have no intention of criticizing the US forces, but many crimes have happened because of the US soldiers.'' As we will discuss in more detail in Chapter \ref{cha:protest}, he noted how, among the general population, a single crime can undo years of public diplomacy efforts. He referenced the example of the 1995 rape of an Okinawan schoolgirl by three American service members. He noted that even though the rape ``has nothing to do with the US security situation,''  it still has a negative impact on Okinawans' perceptions of the US military. 

The US provides external security for Japan, yet 70 percent of US bases are in Okinawa, taking up 15 percent of the island's territory (Okinawa is only 0.6 percent of Japan's total territory) \cite{JPTimes2020,tokyoone20200427}. The sentiment among many Okinawans is that if US forces offer security to all of Japan, then other parts of the country should carry a more proportional share of the burden. In this respect, the experience of Okinawans is not much different from that of other minority populations around the world. The asymmetry between diffuse, collective goods and the small sector of the population bearing the burden for their provision is common in overseas basing for a variety reasons, including geography, logistics, and national policy.

Okinawans have long had grievances about discrimination by the Japanese government. The location of US bases in their region, along with the problems they bring, is a crucial point of contention between many in the region and the government. Okinawans are a minority ethnic group whose home island was annexed by Japan in 1879.\footnote{Though suppressed by the Japanese, the Ryukyuan languages (one of which is Okinawan) are actually distinct from Japanese, and those looking to preserve Okinawan identity are engaging in concerted efforts to teach the languages to the new generations to prevent their extinction \cite{Heinrich2004,Fifield2014,UNESCO2010}.} The topic of Okinawa's disproportionate burden remains relevant, with Okinawa's Governor Denny Tamaki noting in May of 2020, as annual anti-US military base protests had to be canceled due to the COVID-19 outbreak, ``I will completely devote myself to resolving issues [in Okinawa] including the heavy burden [of hosting US bases]'' \cite{JPTimes2020}.

Though the United States encouraged the Ryukyuan language and independence during American rule over the Ryukyu islands from 1945 to 1972, the US occupation resulted in increased Japanese nationalism in this area. It is of note that the Ryukyu islands saw bloody battles against American forces on their territory \cite{tokyoone20200427}. The territory that the Americans occupied was one in which the virtual totality of the population became displaced and impoverished \cite{Heinrich2004}. Thus, rather than Americans being viewed positively for their encouragement of Ryukyuan independence and culture, nationalized Japanese language and identity continued to expand and, after the occupation, tensions between Okinawa and Tokyo arose over the continued US military presence in Okinawa. Especially heinous offenses by service members against Okinawan civilians served to spark protests and turn public opinion against the US presence. In our sampling of reported criminal offenses by service members against civilians, 37.5\% of reported events occurred in Okinawa. These offenses include DUIs, drug-related incidents, manslaughter, kidnapping, and sexual assault.

In a stark contrast to the case of Okinawa, we also talked to US service members and locals at the RAF Lakenheath military base in Suffolk in the United Kingdom. In their glowing descriptions of each other, the English locals and the American service members cited a shared language and culture as part of what enables that positive connection. Over coffee at a series of discussions held at the base with American non-commissioned officers and two Britons representing the Ministry of Defense and the Royal Air Force, we heard a stream of fondly-recalled positive anecdotes: ``I live in government housing in the village. Local kids come through for trick or treating because they know they'll get American candy'' \cite{rafsix20190719}. ``Oddly enough, the locals like to celebrate the 4th of July with us. Any chance to drink'' \cite{rafeight20190719}. ``On the macro level, the base is well received in the local area. Why wouldn’t it be? It's been here so long'' \cite{rafthree20190719}. ``There is a sushi place by my house that I religiously eat at. One day they were getting their daily fresh fish run. I made a joke about whether they had any extra, the owner went back inside and gave me a whole lobster tail, and three pounds of shrimp. He said it was to thank me for everything the US does'' \cite{raffive20190719}. The locals were no less effusive, noting the importance of the shared language and culture between the British and Americans, highlighting the stark contrast between the relationship with the local population in England and in Okinawa \cite{councilone20190718}.

Beyond the shared culture, interviewed service members considered England a desirable assignment in part because locals are friendly regardless of the service members' race, ethnicity, gender, or sexual orientation. During our discussions, several service members mentioned that in England, ``they treat everyone exactly the same'' \cite{raffive20190719}.  There was a clear consensus that the type of discrimination that service members faced when deployed to other locations was uncommon in England. But what about \textit{local} minority groups?  Could the positive perceptions we observed in England be related to the fact that US service members interact mostly with members of the majority ethnic group, as opposed to Okinawa, where many of the locals interacting with the US military belong to an ethnic minority? Would ethnic minorities in England perceive a US military presence in their area differently from majority groups? Given that most of the locations that the US deploys to also have significant minority populations, this becomes a key question to explore in understanding relations between the US military and local host populations. 

%Add this back in somewhere for book version.
%A lesbian service member noted that she walks around holding her wife's hand and does not get treated differently at all, observing that the British are actually more accepting of being gay than the Americans are. She contrasted this with her experience being stationed in South Korea, where she noted that locals often misgendered her, calling her ``sir'' and changing her name to a male name \cite{rafone20190719}. Another service member recalled more overt racism in Eastern Europe, recounting that going to dinner with African-American friends was ``difficult'' \cite{rafeight20190719}.

%I'm not sure about this paragraph, since it seems to be about some negative local attitudes toward American minorities, not necessarily minority attitudes toward the American presence. I'm not sure the local attitudes described here have any real relationship to the minority populations within the country either, especially in the SK or Eastern Europe examples. Might be good to save for something more about American experiences across contexts rather than minority vs. majority on the local end. - AS

It is of note that we conducted the Lakenheath interviews in the aftermath of Britain's vote to leave the European Union in 2016 (``Brexit''), as the country negotiated the terms of Britain's exit, and just days before Boris Johnson's election as Prime Minister in July 2019.\footnote{At this point in the process, people widely assumed that Johnson would become the next Prime Minister and the country was just awaiting the formal institutional processes to formalize the decision.} Voters in Great Britain voted to exit in large part because of concerns about immigration from the European continent (especially Central and Eastern Europe) and changes to British identity \cite{Goodwin2016,Goodwin2017}. The large influx of refugees into Europe in 2015 exacerbated these concerns as people sought asylum from, among other places, the conflicts in Syria, Iraq, and Afghanistan (and broadly from Africa, the Middle East and South Asia) \cite{Chan2015}. Research on the Brexit vote found that areas experiencing the highest increases in immigration in the years prior to the vote were most likely to vote to leave. This was particularly true in areas that had been previously ethnically homogenous and experienced sudden demographic changes due to immigration. Britons' fear of how these sudden changes would alter British identity affected their likelihood of voting ``Leave'' \cite{Goodwin2017}.

We confirmed some of these observations when talking to British locals in both London and Suffolk. A Lakenheath Council Member told us that he had voted ``Leave'' in the referendum, though he was starting to change his mind on it as he saw the economic impact Brexit was having on Britain. The town of Lakenheath voted for ``Leave'' (18,160 votes in favor of leaving, 65 percent of the vote, with 72.5 percent turnout). When we asked what it was that had driven the high turnout and ``Leave'' vote, he responded, without hesitation, that it was immigration. He recounted how in Brandon, a town about 6 miles away from Lakenheath, immigrants (especially Eastern European and Portuguese immigrants) were more likely to frequent the local libraries than the English. He noted, ``wanting to avoid racial overtones; it throws people because it's a change in culture. I'm in the supermarket and I can't hear English being spoken.'' He was quick to clarify that ``it's not bad that [demographic change] happens, it's that it happens too fast,'' summarizing the sentiment as, ``it made people feel like England wasn't England anymore'' \cite{councilone20190718}.

The Council Member contrasted interactions with immigrants and refugees to ones with the American military personnel, noting that when it comes to the Americans, ``because our cultures are similar, it is not as big of an issue. The fact that our cultures are very similar helps. It would be different without that.'' He also highlighted the importance of the common language in building positive relations between locals and the US military base. Yet, when we asked him about relations between the base and minority populations in the area, he just said that ``It hasn't come up'' \cite{councilone20190718}.\footnote{We note that despite many active efforts to reach out and interview civil society leaders for minority groups in the area, we received no response. The lack of response is perhaps telling in and of itself, as we identified ourselves in all of our correspondence and calls as researchers based at US universities studying perceptions of the US military abroad.}

The interview subjects at RAF Lakenheath were equally ambiguous when we asked them about relations between the base and minority populations in the region. We found that very few of them had interacted with members of local minority communities, and that they mostly held positive but vague views of them. For example, when we asked about their interactions with Portuguese or Eastern European locals, only one out of six service members volunteered an anecdote, noting that her ``lash lady'' (meaning the technician who applies fake eyelashes) was Portuguese and was ``very nice,'' and that the ``night life is diverse.'' She recalled having friendly conversations with immigrants while out at nightclubs and bars \cite{raffive20190719}.\footnote{It is of note that this individual was the youngest in the group of service members we interviewed at Lakenheath, describing herself as ``mid-20s'' and self-identified as an ethnic minority in the United States (``I'm half Filipina'').} 
%One of the Britons interviewed at the base noted that the immigrants mostly do agricultural work in the area and told us about how in his village once a year they have a yard sale and ``usually the rubbish gets bought by the Eastern Europeans'' \cite{rafthree20190719}.
%NOTE: I went back and forth on that last quote about the rubbish. It's very telling of attitudes towards minorities, but the commander is very easily identifiable and it makes him seem kind of shitty. I'm leaning towards scrapping it to not burn any bridges.CM.
%NOTE: I agree. I'd like to keep it but I can't think of a way to refer to it without referencing him or the specific thing he said, since it's pretty specific. - AS

People living in Japan and in England had very different experiences in their interactions with American service members, but there was a clear common thread across both sets of interviews: The experiences that ethnic minorities had in dealing with the US military were vastly different from those that dominant ethnic groups held. Generally, minorities seemed to be having more negative, or at least less positive, interactions. However, the few comments from American service members about local minorities were largely positive. This highlights the importance of not homogenizing local experiences with the US military and acknowledging that an individual's perception may be significantly influenced by whether they see themselves as being part of a state's dominant ethnic group or belonging to a minority. 

%We don't actually test anything related to the ethnicities of service members, so taking this out. 
%Given that some positive comments about relationships with local minority populations also came from service members who considered themselves minorities, the presence of minority groups within populations of US personnel on installations may also be an important factor in determining local minority views. 

\subsection*{Expectations}
 
Ethnic and minority group relations related to US military deployments abroad have been a concern since the US first started deploying its troops abroad as a major power. The United States often portrays itself as an ally of minority groups abroad. For example, through its interventions in the Balkans, in different parts of Africa, and its support of the Kurds in Iraq, American military power has been brought to bear to protect minority groups from hostile and repressive governments. Though these interventions have sometimes been long-term and costly, they have helped some groups realize ambitions for an independent national state; for others, it has led to the cessation of brutal oppression by majority-led governments. Other times, the United States has used its non-military power to press governments around the world for an increased respect for human rights, particularly in the treatment of minority populations. In fact, states who host US military forces have been found to see an increased respect for human rights \cite{bell2017}.

However, the United States itself has also had problems with violating the rights of ethnic minorities within the United States and has struggled with systematic discrimination against ethnic minorities \cite{Williamson2018}. Beyond its borders, scholars, activists, and journalists have accused the United States of exporting attitudes of racial inferiority, stereotypes, and hierarchies into foreign states. In particular, South Korea and Saudi Arabia, two states that have hosted large numbers of US military forces, demonstrate the proliferation of US racial attitudes \cite{Moon1997,Vitalis2007}. Furthermore, after conducting long wars in Iraq and Afghanistan, and recently curtailing immigration and refugee flows from predominantly Islamic countries, Muslims, whether in majority- or minority-Muslim contexts, may not see the United States as an ally against oppression. 
%The history of Diego Garcia, as covered in Chapter 1, gives further credence to the idea that the United States would rather not deal with human populations at all if it can be avoided entirely.
 
These two trends and narratives leave us with a puzzle. Do minority populations in states that host the US military see it as a friend or a foe? There are several mechanisms by which either of these two narratives could be true. Despite US efforts, long-term deployments in peaceful countries could create direct and indirect paths toward minority discontent with the United States. Directly, Status of Forces Agreements between host states and the United States place some of the basing burden on minority groups not well represented by host-state governments. Okinawa, which in a 2019 referendum voted 72\% against the construction of a new US military base on the Okinawan town of Henoko (a project that involved various environmental concerns) is an example of this dynamic \cite{tokyoone20200427,McCurry2019}.  

%changing language to be more hypothetical and still open to adjudication

Indirectly, the very act of basing also sends credible signals about US commitment to the host state and could empower it to maintain status quo policies towards disenfranchised groups. As an example of this, in Bahrain, where the United States has a significant basing presence by housing both the United States Naval Forces Central Command (NAVCENT) and the 5th fleet, the US seemed to turn a blind eye to Shia-led protests during the Arab Spring, which the Sunni government put down violently \cite{McDaniel2013}. While the Shia population is not a numerical minority in Bahrain, they are the non-dominant ethnic group. These combined direct and indirect mechanisms could compel minority groups to see the United States in less positive terms than the dominant group does. Contrarily, the United States spent decades in Iraq protecting Kurdish populations from Saddam Hussein's central government. The US also attempted to restrain Turkish cross-border incursions into Kurdish areas. Overall, the United States went to great lengths to shield this minority population from adversarial governments and groups throughout the region. 

This chapter thus focuses on studying the differences between the views of individuals who self-identify as ethnic minorities in the 14 countries in our survey sample from the rest of the population in order to adjudicate between these two potential lines of thinking. This chapter goes further than previous ones by contextualizing the contact hypotheses from chapter \ref{cha:meth}. Specifically, we expect that contact will be conditionally different for minority groups relative to the general population. Given the different relationships between minority and majority populations and their governments, and the potentially systematically distinct views of the role the United States plays in both the host country and the international system, we expect that those self-identifying as minorities will have consistently different attitudes toward the US military across host nations. 
%Expand a bit on the Kuwait stuff later.   

%\footnote{We address the case of Kuwait in more detail in our research design, as this is also a case in which the dominant ethnic group, ethnic Kuwaitis, is a numeric minority. Ethnic Kuwaitis, the dominant ethnic group, make up 30.4\% of the population, as compared to 40.3\% Asians, many of whom are immigrant labourers. This makes inference of our variables related to being in a ``minority'' challenging. All other cases consist of countries in which the ethnic majority is also the dominant group, necessitating that Kuwait be considered separately.} 


% We expect, conditionally, that contact with minority members of the population will moderate the positive effects of contact and have a distinctly more negative effect than traditional contact with majority groups.  %change amplify/moderate and negative/positive.

%[ADD a summary of the findings when we have them]
%This chapter's findings thus help to add nuance to policy decisions regarding military engagement with communities around military installations. Briefly, across the countries in our survey sample we find that there is a general pattern of self-identified minority group members holding less positive views of the US than non-minority individuals. However, when we look at the countries individually we find that context matters, and the relationships between minority self-identification and views of the US vary across countries. In some cases minority groups appear to hold more favorable views of the US, while in other cases minority groups hold less positive views of those same actors. In still other cases, we find that that is a stronger tendency towards neutral views among members of minority groups.

%Taken together our quantitative and qualitative findings demonstrate the importance of context-specific factors related to individual host societies, and highlight the importance of reaching out to minority groups who will often perceive the United States as supporting a system that discriminates against or represses them. It also shows that generally positive perceptions of the US military in the host country may obscure systematic differences across different racial, ethnic, or religious groups. Because minority groups tend to hold less political power in their home countries, it is up to the US and its military to create positive relations with these communities and in some cases intercede on their behalf when dealing with host governments. Particularly in cases where the US messaging system toward a host country is run at the national level, through embassy staff, or through the national host government, information systems related to the US presence may not be adequately reaching minority populations. This may both reduce the effect of messaging about the purpose and opportunities associated with the US presence and also limit the potential for fruitful interpersonal connections and economic relationships with minority populations that exist with the rest of the host country.

%To test our hypothesis, we conducted a survey in 14 different countries with approximately 1,000 respondents in each country. Using a multilevel fixed-effects logistic regression, we find that minority population support for US troops to be less than majoritarian populations. However, this effect does not hold when we ask about minority perceptions about US people or the government. Such findings for research on United States Foreign Policy, Human Rights and Democracy promotion, and for policymaking directly. US decision-making in the security realm affects US objectives in other areas and may undermine efforts towards liberal institution building. 

\section*{Ethnic Minorities and the US Military}

Research on the relationship between the US military and local minority populations is relatively thin, especially with regards to quantitative, large-sample research. The work that has been done in this area largely consists of episodes in which minority populations have played a key role in American combat operations somewhere in the world, theories on how minority populations can use third-party support to accomplish their goals, or case study research on specific episodes in US-host state relations. The specific work on the US military's treatment of minorities in the context of basing focuses on individual cases, such as those previously mentioned in South Korea and Saudi Arabia. 

Katherine McCaffrey has studied this issue on the island of Vieques, Puerto Rico, where until recently the US military maintained a bombing range \cite{Mccaffrey2002}. McCaffrey states in her work that ``bases are frequently established on the political margins of national territory, on lands occupied by ethnic or cultural minorities or otherwise disadvantaged populations'' (p. 9-10). As examples, she points to the US military bases in the Philippines that were located on land reserved for indigenous populations, along with Okinawa. She describes how the local minority populations often had to comb through military trash to survive after their governments evicted them from their land. While these are certainly powerful examples, to our knowledge there are no systematic and cross-national studies that test whether the dynamics illustrated by these examples are consistent and widespread across contexts. Taking it a step further, we ask the question of whether these examples form a representative sample of how minority populations view an American military presence, or whether these stark cases of mistreatment are the exception to the rule of generally positive relationships. 

Counter to these negative precedents, the United States also has a long history of intervening in civil wars and on the side of minority populations around the world. Particularly since World War II and the realization of the horrors of the Holocaust, there has been a growing consensus within the American foreign policy community that the United States has the ability and, therefore, the responsibility to protect populations around the world from extreme abuses by their governments. This idea culminated in the ``Responsibility to Protect'' (R2P) doctrine, endorsed by the United States and United Nations in 2005, after American interventions in the Balkans, along with the establishment of no-fly zones to protect Kurdish populations in Northern Iraq and the protection of rebel groups from government forces in Libya. While those episodes were largely seen as successfully protecting minority groups, the failure to intervene in Rwanda and the resulting genocide also stood as a strong counterexample of what could happen in the absence of American military involvement. 

If we extrapolate from these interactions that take place in the context of conflict, we might conclude that minority ethnic groups will be likely to have positive perceptions of a peacetime US military presence in their home country.  An initial expectation might be that minority groups are likely to view the US positively because the US military presence will make the host government less likely to repress them. If the US actively promotes human rights abroad, and is active in naming and shaming human rights violators and pressuring them to change their repressive behaviors, then minority groups, which majority governments often try to repressed, should be glad to have the US military present to act as a restraining influence on the host government. After all, previous work finds a positive correlation between a US military presence and respect for physical integrity rights in the host country \cite{bell2017}. The presence of US forces could act as an implicit threat against defection from the US hierarchical system, which endorses respect for human rights. As occurred in Panama in 1989, these troops can be used against the host government if it is seen as acting against the wishes of the United States. This type of implicit threat can keep host state behavior within the bounds of what is considered acceptable by norms established by the United States, as the leader of a global hierarchical system \cite{Towns2012}. 

%The United States can be a positive influence on central governments in their treatment of minorities.

Yet, this dynamic may not play out in practice in countries in which the United States has a large military presence. The decrease in human rights violations that accompanies a US military presence only occurs in states that are not considered strategically important to the United States \cite{bell2017}. In fact, \citeasnoun{StraversElKurd2018} find that a US military presence in autocratic states that are also strategically important correlates with increased autocratization. A good example of this is in the aforementioned case of Bahrain during the Arab Spring protests, when the presence of American military forces did not, in fact, restrain the government from violating the human rights of Shia protesters. With these theories and types of cases in mind, minorities may not only have their everyday lives disrupted by the US military presence, but may be made to feel less secure through the US support of a government that engages in discrimination against them. Through the need to stay in the good graces of a government led by the more powerful majority in order to maintain the basing apparatus, the United States often turns a blind eye to discrimination in host countries and in some cases contributes to it. This process may then result in a systematically more negative view of the US military presence among minority populations. 

First, we argue that the US military presence will be more disruptive for minority members' everyday lives than the rest of the population. While the host country gains economic, security, and political benefits from the US military presence, it can also face domestic political costs for it. In particular, military installations that are located close to major cities and population centers are most likely to encounter opposition. These installations are more likely to serve as a constant reminder to the population of the host's subordinate role relative to the United States, and of the United States' ``imperialism'' \cite{cooley2008}. As we note in Chapter \ref{cha:protest}, cities are also more likely to facilitate anti-base activists' coordination capabilities, with more access to transnational anti-basing organizations and demographic groups that are predisposed to protest, such as students. This is in addition to the fact that coordination is easier in cities, which helps reduce collective action problems involved in protest. Thus, host states will often choose to locate American bases in less contentious, more remote locations \cite{cooley2008}.

The case of the US Thule Air Base in northern Greenland, established in 1951 and expanded to add missile defense in 2004, is an example of this type of dynamic. Thule's strategic location close to the Arctic is of course the major part of its appeal to the Americans as a site for missile defense, yet Greenland's position within the Danish government also makes it a politically advantageous location from Denmark's perspective. Greenland was a Danish colony until 1953, when it became a Danish county, with representation in the Danish Parliament. Though Greenland has expanded its self-rule in 2008 and 2009, there are still many aspects of Greenlandic policy that remain under Danish control \cite{Dragsdahl2005}. 

Ethnically, Greenland's population of 57,000 is 89\% Greenlandic Inughuit (Inuit). This contrasts with Denmark's population of which 83\% are ethnically Danish of Nordic European ancestry. Thule Air Base has had a disproportionately negative effect on the Inughuit people's livelihood, while benefiting the general Danish population. The Air Base was established in 1951 in the town of Thule and, when the military added anti-aircraft guns in 1953, the Danish government forcibly relocated the Inughuit people of Thule to the nearby town of Qaanaq \cite{Spiermann2004}. In 2004, the United States added missile defense systems to the base, increasing concerns by the Inughuit population of being targeted by nuclear weapons. This added to existing grievances of the base harming the environment through toxic waste and affecting the hunting fields traditionally used by the Inughuit \cite{Dragsdahl2005}.

In May of 2004, the Inughuit group Hingitaq 53 (meaning ``The Expelled of 53'') brought a case to the European Court of Human Rights asking for the authority to return to their original home in Thule. While the court did not grant them the ability to return, it did order the Danish Prime Minister to compensate the Thule Tribe \cite{Spiermann2004}. Meanwhile, Denmark, knowing the strategic importance of Thule Air Base, has been able to use the base as a bargaining chip with the United States. As a NATO member, Denmark often uses the base as evidence of its contribution to the alliance \cite{Dragsdahl2005}. Thus, the benefits from the base distribute to the Danish population generally, while the costs are borne disproportionately by an ethnic minority that lacks political power. This dynamic has continued up to the present day with the President of the United States publicly proposing that the United States purchase Greenland. Senator Tom Cotton, a close administration ally, discussed the idea with the Danish ambassador to the United States, as well \cite{hart2019,wu2019}. While the Danish government's response was swift and dismissive, this series of events took place without the United States ever involving the local population in the decision-making process, which highlights the degree to which minority populations can often be sidelined. 

%%This sounds more like actually emphasizing Greenlandic autonomy, which is not really the point we're making, so I'm commenting it out for now.
%Even the Prime Minister of Denmark insinuated as much about the American approach, saying ``Greenland is not Danish, Greenland is Greenlandic \cite{jorgensen2019}.'' One of the two Greenlandic representatives in the Danish Parliament supported this sentiment by commenting that ``Greenland is not a commodity which can just be sold'' \cite{thelocal}. 

%and ``I'd be concerned about the type of society we'd have if Greenland becomes American rather than Danish." 

This example highlights the role of host governments in basing access and its relationship to minority populations. When host governments negotiate with the United States over base access, they are doing so on two-levels; directly with the United States government, but also needing to gain enough support from their selectorate at home \cite{Putnam1988,demesquita2005,cooley2008}. They thus need to not only obtain benefits from the United States, but also secure the support of key political players at home. Leaders can do this by maximizing the net benefits received by the leader's winning coalitions (maximizing their gains while minimizing their costs) \cite{Mesquitaetal2005}. In determining where to place US military facilities, a common solution is to have them in remote locations that are often homes to politically powerless ethnic minorities. This allows the dominant populations in the country to receive the benefits of a US military presence without experiencing the full weight of its negative consequences.

Okinawa, exemplifies another asymmetry between the collective benefit of national security versus the target costs of hosting bases in territory with a concentrated minority group. The movement against bases in Okinawa started in the Cold War and negative interactions between troops and civilians build support for protests and national action against the heavy Marine presence in the region. In classic Olsonian collective action problems, concentrating the costs onto a smaller section of the population makes collective action against the presence of bases more likely than a strategy of diffusing bases nationally and, thereby, diffusing the costs of hosting bases in Japan \cite{Olson1965}. However, as we will discuss in further detail in the chapter on protest, this also potentially limits how cross-cutting protest demographics are, which reduces the likelihood of their long-term success \cite{Yeo2011}.

Second, the US has a troubled history with domestic ethnic relations and government respect for rights among minority groups. While domestic issues do not always translate into international behavior, the United States has exported its domestic turmoil to other countries it bases in. US President Harry Truman formally integrated the armed forces in 1948, yet this did not prevent soldiers' ideas about segregation from being perpetuated in other countries.\footnote{The various branches up through World War II had differing policies on racial segregation. Truman's executive order 9981 uniformly removed segregation across the branches.}  Moon \citeyear{Moon1997} goes into detail about how US racial beliefs conditioned how South Korean business establishments treated white and Black soldiers differently in the post-Korean War era. Businesses would exclusively serve white soldiers, play music more specifically targeting white clientele, and create conditions that were generally hostile to Black soldiers. White soldiers would reinforce these conditions both in how they patronized establishments as well as in high-profile cases of soldier behavior amplifying such messages. In one particular case, a white soldier murdered a sex worker after he saw her working with both white and Black clientele \cite{Moon1997}. 

The US propagation of racial attitudes has a direct negative impact on US operations and the de facto segregation encouraged in the 1950s and 1960s required decades of dedicated work to begin to reverse. These attitudes were not simply confined to how US personnel and host populations related to each other. Instead, they became pervasive in host societies and oftentimes altered majority relations with minority populations, providing a template for how racial and ethnic discrimination can be conducted. As such, individual soldier attitudes, even if they do not reflect government policy (but especially when they do), can do further damage to minority perceptions about the United States' mission in their country by impacting not only their perceptions of US and US military attitudes toward minorities, but also affecting their place within their own societies. 

Third, by virtue of the military presence in a host state, the US government and the US military have a preexisting, institutional, and strategic relationship with the host state. Given the basing relationship, the United States and the central government in host states are already on friendly terms. The presence of US forces also signals that the location or country is strategically valuable to the United States and that maintaining basing access will likely be a priority of US policy \cite{StraversElKurd2018}. These dynamics show that the starting point in any tension between substate actors and the central government will see bias toward the central government from the United States. While minority populations repressed by the host government could certainly draw US support under certain circumstances, it is key to note that the United States is not a neutral party. 

When the US places troops in a foreign country, it is making an implicit (or explicit) commitment to the security of that country. Whether US troops are placed there to directly defend the host country or serve as a trip-wire mechanism that would trigger a US intervention if the host country were to be attacked and US troops threatened, troops stationed consensually abroad protect the host country from external threats \cite{Schelling1966}.  Though in most cases US troops are not there to protect the host country from internal threats, the very fact that US troops are providing security from external threats means that the host country can redirect its own resources from protecting against external threats to protecting against internal threats \cite{machainandmorgan2013,allenetal2016,allenetal2017}.

In addition, a US military presence often comes with tangible material benefits for the government. For instance, a basing relationship with the United States routinely brings increased bilateral military training exercises that increase the coercive capability of the central government \cite{Ruby2010}. The United States also tends to provide more access to foreign military sales for countries that host US military forces, along with sharing existing technology that allows for the growth of a domestic arms industry. These kinds of training, arms transfer, and technology benefits also empower the central government in relation to the rest of the country and potentially make challenges from substate actors easier to manage. 

If the host country views minority groups as an internal threat (something that is common among states with significant minority populations), then it is likely to use repression as a way to protect itself from the perceived internal threat \cite{Regan2005,Jakobsen2009,Brathwaite2014,Hendrix2019}.  If the central government has more resources available to it (because it is now spending less on external security or because the US is directly providing it with resources), then it can devote more of those resources to repression.  A basing relationship with the United States can then have the effect of materially enhancing the power of the central government in relation to substate actors and therefore make challenges to the central government less likely. Thus, a US military presence is increasing the resources available for the repression of minorities.  This will naturally condition minority groups to be less supportive of a US military presence that is not only legitimizing the central government, but also allowing for the expansion of its repressive power.

Furthermore, the mere presence of US military forces on host territory provides legitimacy to the central government from the most powerful country in the international system. Lending legitimacy to central governments can have profound effects that further empower the central government by backing the idea that the government is the rightful power in the country and dissuade challenges to its authority.  For example, the US placing military forces in Spain through the 1953 Madrid Pact legitimized General Francisco Franco's rule in Spain at a time when the international community had generally ostracized him \cite{cooley2008}. The presence of US forces also gives a sense that there is a significant external security concern for the country, which may be necessary to defend against before challenging the central government directly. This is particularly important in cases where a minority group feels that a central government is mistreating them. Other groups within the country and even within the minority group itself will see the legitimacy and authority conferred on the host government via the presence of US military forces and be less likely to challenge the central government's authority. 

By enhancing the power of the central government, the US basing apparatus may further centralize power in host states. The United States deals mainly with the central governments in host state relationships, not regional or local governments. For example, when determining where within a country to send humanitarian deployments, the US consults first with the central government, which presents a series of options \cite{Flynn2018}. By dealing directly with it, the US not only enhances the power of the central government in relation to minority groups, but it also enhances the power of the central government in relation to regional or local governments. Decentralization reduces ethnic conflict and secessionism by increasing opportunities for minority groups to participate in government \cite{Brancati2006}. By empowering the central government both materially and in its legitimacy, the presence of US forces may inadvertently centralize power in a way that disempowers minority groups from effectively engaging in politics. 

Even in democratic countries, minority groups may see themselves at odds with the majority groups that control the government and view a US military presence as supporting the majority population rather than the whole population equally. They may not always be wrong to assume this. While the United States may not be supportive of repression and discrimination against ethnic minorities, US military deployments' efforts to build goodwill are most often focused on those members of society whose support is needed to maintain the presence. This usually is the ethnic majority, which controls the government and to whom the government answers.

Many of these dynamics came to light when we visited the Clay Kaserne Army Post in Wiesbaden Germany, where we heard a lot about outreach efforts targeted at local government officials. The base's Government Relations Officer noted that in less than a year since she had taken that position, she had already met with every mayor of each town surrounding the base. The mayors all get permanent passes that allow them to drive onto the base without an escort. The Government Relations Officer also offers tours of the base for local council members, as well as collaborating with the police and fire departments in neighboring towns. She noted the importance of cultivating relationships with local governments and making the relationship feel less transactional for them. A gregarious and extroverted individual, she knew that she was well-suited for the job: ``You have to stroke and smooch them,'' she said to us in a jovial tone, ``and I am a good smoocher'' \cite{kaserneone20190725}.   

Given the very different views towards ethnic minorities and majorities that we had seen in Lakenheath, we were particularly eager to hear about the base's relations with and outreach efforts towards refugee and immigrant communities in Germany, which had dramatically increased in size since the country began accepting large numbers of refugees from the wars in Syria and Iraq. The topic actually came up on its own when we asked the Government Relations Officer about locals' views of economic benefits from the base. She told us that while there had been a trend of younger German generations (who had not lived through the Marshall Plan) becoming increasingly frustrated with American service members for not learning German, the trend was somewhat reversing with the recent influx of refugees. She noted that Germany was becoming more like the United States and experiencing ``crime, rape, and pedophilia'' \cite{kaserneone20190725}. 


Research on in-group/out-group dynamics shows that crises and the introductions of an outside threat can redefine in-groups and out-groups \cite{Coser1998,Levy1989,Simmel2010}. This dynamic was clearly at play in Germany. The Government Relations Officer put it plainly, ``Before the refugees, Americans were seen as outsiders.'' As an illustration, she noted that, as refugee populations have increased, ``You don't see as much `US Go Home,' but instead `Country X go home' '' (with ``Country X'' referring to the home countries of refugees) \cite{kaserneone20190725}. There is thus a redefinition of the outgroup occurring with the native Germans viewing the Americans as part of their in-group and perceiving refugees and immigrants as outgroup members.

From the American perspective, this dynamic is also at play. While the Americans are unlikely to feel threatened by the influx of refugees in the way that the Germans are, they also know that policy concessions will come from those Germans who control government institutions, not the politically powerless refugees or other ethnic minorities. We specifically asked if the base did any outreach directed at the civil society groups of different ethnic populations, similarly to how it reaches out to the mayors and council members of surrounding towns. The government relations noted that such groups ``probably exist, but I don't meet with them.'' Regarding the Turkish community, she noted, ``They are not my priority to reach out to. Our primary target is the city and mayors'' \cite{kaserneone20190725}.

As far as the refugees go, she anecdotally confirmed that they are less likely to feel positively towards the United States than other Germans: ``The refugees that come from the countries that we are at war with, they don't like us.'' Interactions between refugees US service members are thus more likely to be negative ones. She gave the example of young people going to clubs, getting drunk, and fights breaking out between US service members and ``a foreign national'' (meaning a non-German) after people from countries that the US is in conflict with say things like ``You killed my parents, you bombed my town.'' The alcohol consumption and the Americans being ``somewhat arrogant'' and patriotic exacerbates the confrontations. 

Previously we have shown that direct and indirect contact with the US military in a non-combat deployment setting can lead to both a positive and a negative effect on perceptions of US actors (with the positive effect generally being larger than the negative one).  It is thus of interest to us to understand the determinants of the direction of the effect. As noted by \citeasnoun{cooley2008}, a US military presence in a host country can mean different things to different actors within society. In this case, we believe that an ethnic minority identification will be one of the factors that make it more likely to observe the negative, instead of positive, effect of interactions on perceptions of US actors for the theoretical reasons described above, along with supporting testimony from local populations and US officials. 

Given both direct and indirect mechanisms, we draw the following hypothesis:

\begin{hyp}
Minority populations in states that host US military forces will be more likely to have negative perceptions of US military forces, the United States government, and the United States people. 
\end{hyp}

% I don't think we need this. If we can reject the null in a positive direction, we can then also reject the first hypothesis. - AS
%The start of this section noted that there is an outstanding belief that the United States is a force for minority empowerment and protection globally. As such, we allow for the opposing hypotheses as well. Theoretically, we are less convinced by the arguments supporting it, but include it for completeness of the theoretical record:

%\begin{hyp}
%	Minority populations in states that host US military forces will be more likely to have positive perceptions of US military forces, the United States government, and the United States people.
%\end{hyp}

Combining our research in chapter \ref{cha:meth} about the role of contact, interactions with minority members of the community is likely to have a uniquely different effect on perceptions alone. As such, while we include contact in the models, we also include an interactive effect between minority respondents that report contact. Given that that negative contact with troops generates movements against the US presence in places like Okinawa, we expect that this type of contact will amplify the negative view that minority populations have of the US presence and make it even more negative.

\begin{hyp}
	Minority populations in states that host US military forces will be more likely to have more negative perceptions of US military forces, the United States government, and the United States people as a result of interacting with the US military. 
\end{hyp}

%Same thing here. I don't think we need the reverse hypothesis. Can deduce it from what we find. - AS
% However, both the positive view of the United States and the social theory we express in contact, it is possible that contact has a mitigating effect on people's perception. As such, we test whether contact reduces the expected negative perception. 

%\begin{hyp}
%	Minority populations in states that host US military forces will be more likely to have less negative perceptions of US military forces, the United States government, and the United States people as a result of interacting with the US military. 
%\end{hyp}

%NOTE: Do we think this effect will carry over to perceptions of the US government and US people? Or is it just about the US military being there and disrupting their lives? Maybe the direct effect of having the military disrupting your day to day life affects your views of the military but not the other two, whereas the indirect effect of having the US support the oppressive state also affects views of the US govt and people? Is there a way we can think of to distinguish between the two when testing? Maybe look at the base locations? If you have a base in your subnational unit you're more likely to face everyday disruptions?

 %add additional hypotheses here.




















\section*{Research Methodology}
%\ref{cha:theory} and \ref{cha:meth}
Our research design follows the basic approach we discuss in chapters 2 and 3, but here we shift the focus to thinking more specifically about how minority populations within each country view various US actors. We begin by estimating a series of multilevel Bayesian categorical logit models with varying intercepts by country, which allow us to consider the possibility of different dynamics in each country. The analysis in this section proceeds in three stages. First, we estimate a set of models using a single predictor variable---minority self-identification. Minority status is potentially determined by factors unrelated to other individual-level characteristics that we include in our data (meaning that minority status is \textit{exogenous} to other individual characteristics). Accordingly, it can be informative to begin by looking simply at basic differences between individuals who self-identify as belonging to some racial, ethnic, or religious minority, and those who do not. 

Second, because minority status likely correlates with a number of other individual-level and group-level traits, we estimate another set of models where we adjust for some of these other considerations. For example, members of minority groups may have less access to education, lower incomes, or may be geographically concentrated in areas that are more vulnerable to the negative externalities of US basing, like noise, crime, or pollution. The aim here is to better understand how the conditions of self-identified minorities differ from other individuals with respect to attitudes towards US basing and troops, and what additional information an individual's minority status conveys once we've adjusted for these other considerations. 

Third, we estimate these same fuller model specifications again, but this time allow the effect of the minority self-identification variable to vary across countries (by allowing its coefficient to vary across countries). This will provide us with some insight into whether the differences between self-identified minorities and non-minorities are similar, or if they vary, across countries. As we discuss above, it is possible that relations between US service members and minority populations are dependent upon various contextual factors. In countries where minority populations disproportionately experience the negative effects of US deployments, we might expect to see more negative attitudes among minority groups towards the US military presence in their country. However, if some countries are home to minority groups who have greater equality in terms of income, education, and geographic mobility, we might expect more positive attitudes among those minority populations, or even no discernible difference from the rest of the population with respect to their views of US military personnel and other US actors. 

\subsection*{Outcome Variable}

As we outline in Chapter \ref{cha:meth}, we have three outcome variables of interest: Individuals' views of the United States military, the United States government, and the United States people. We collapse the survey responses down into four categories to estimate our models: 1) Positive, 2) Negative, 3) Neutral, and 4) Don't know/Decline to answer. Our primary focus here is on how minority self-identification affects individual attitudes about the US military presence  in each country. However, we are also interested in how these relationships compare across different groups, and so we run two additional sets of models using views of the US government and US people as outcome variables. Using these additional outcomes will help us to determine how general the effects of the key predictor variables are and will help us to better understand the contours of attitudes towards the US in general. 

\subsection*{Predictor Variables}

Our main predictor variable is the respondent's self-identification as a member of a minority group within the referent country. Our survey asked the question, ``Do you identify as a racial, ethnic, or religious minority?'' Respondents could answer ``Yes,'' ``No,'' or ``Don't Know/Decline to Answer.'' Out of our full sample of 41,545 respondents, 22\% percent responded ``Yes'' to identifying as a minority.  

We emphasize two vital details here. First, readers should note that this includes racial, ethnic, and religious groups. Accordingly, this question casts a broad net when considering an individual's status as a minority group. Second, this represents the respondent's self-assessment of their status within a given country. The implications of this are important in this context---the rates of minority self-identification in our survey data may differ from various official categorization schemes as embodied in national census counts. For example, in our survey data 16.2\% of Japanese respondents self-identify as a minority of the type mentioned above. At first glance this seems at odds with common stereotypes of Japanese homogeneity. However, some authors have discussed the idea that such ideas are often based on oversimplifications that mask variation in group identities \cite{Johnson2019}. While a fuller discussion of this topic is beyond the scope of this book, readers should keep this point in mind as they proceed through the rest of the chapter. 

% Check this. We probably want to add more details here about demographics as we get them.
Table \ref{tab:minoritypercent} shows the percentage of individuals who self-identified as belonging to a minority group in each country across the three years of our survey. Most countries in our sample tend to cluster around a 10--20\% minority self-identification rate. Germany is considerably lower across all three years of our survey. In contrast, the Philippines and Kuwait are consistently among the highest. The Philippines consistently returns 70--84\% minority self-identification rates. Kuwait exhibits by far the most variation in our sample, with a low of 30\% in 2018 and a high of 98\% in 2020. We expect this is the result of a relatively large population of migrant workers in Kuwait and a bias in the survey firm's ability to reach more affluent Arab Kuwaiti citizens as respondents. Notably, this also highlights one of the benefits to using the multilevel modeling framework---outliers like Kuwait are less likely to exert considerable influence on the general results, particularly when we allow effects to vary across countries. That said, we are still sensitive to the possibility that the data collection process for Kuwait is particularly skewed, and so we run a series of robustness checks to further assess its influence on the results.


\input{../Tables/Chapter-Minority/table-minority-percent.tex}

%%%NOTE check the sentence below, it said no before, but i think it needed to say yes 

In general, we expect that individuals that respond ``yes'' to the question of whether they identify as a minority will be more likely to have a negative view of the United States military. Beyond that, given the theoretical interest and empirical support for ``contact'' thus far, we expect there to be an interactive effect between contact and self-reported minority status. We include the contact variable (see chapters \ref{cha:theory} and \ref{cha:meth}) as a constituent variable as well as a multiplicative interaction term. Jointly, these variables will amplify or dampen the effects of the constituent variables on the dependent variable (views on various actors). Likewise, we maintain the previous control variables discussed in previous chapters. 

%add more if we add anything specific. 




  







\section*{Results}

First, we discuss the results for the population-level effects from our most basic model that includes only the minority self-identification variable, shown in Table \ref{tab:minoritybase} in the Appendix. These basic models give use a first look at how minority compare to non-minority populations without adjusting for other factors. To ease interpretation of the results throughout this chapter, we include figures showing the predicted coefficient values. Figure \ref{fig:minoritycoefbase} shows the results of this basic model. This plots the predicted coefficient of minority self-identification on attitudes towards each of the three groups. Each coefficient tells us, given the data we have, how strongly a variable correlates with the outcome, and whether that correlation is positive or negative. For each coefficient our model then generates a range of possible values that are more or less likely. Following a typical bell curve, values closer to the middle are more likely, while values at the edges are less likely The three panels each represent the effects associated with the three outcome variables, and each row represents the effect on the possible responses (e.g. a positive response, a negative response, or a don't know/decline response). Each estimate that we depict shows the median point estimate and the 90\% and 95\% high posterior density credible intervals from the posterior distribution\footnote{The posterior predictive distribution refers to the probability of an individual giving a particular response after we have taken other relevant information into account. See Chapter \ref{cha:meth} for a more detailed discussion of this statistical method.}. 

In these basic models we see that, compared to non-minority individuals, those identifying as belonging to a minority group tend to have a higher probability of expressing a positive view of all three US groups. Alternatively, self-identified minorities have a lower probability of expressing a negative attitude of US troops in the host country, and also the US government. However, there is some evidence that they also have a higher likelihood of expressing a negative view of the US people---81\% of the probability mass of this coefficient's distribution falls above 0 (a zero coefficient is essentially like a 50/50 coin flip, in terms of probability). Finally, minority respondents tend to have a lower probability of responding with ``Don't know/Decline to answer'' compared to non-minorities. This holds except for the question on views of the US people in general.

\begin{figure}[t]
	\includegraphics[scale=0.8]{../Figures/Chapter-Minority/fig-coefplot-base.png}
	\caption{Coefficient estimates for minority self-identification and views of US actors. Credible intervals show the 50\%, 80\%, and 95\% highest posterior density intervals around the median point estimate. }
	\label{fig:minoritycoefbase}
\end{figure}


As we note above, these results present an initial look at exploring how minority groups view US military deployments and other US actors and groups. Minority status often correlates with a range of other important characteristics that are also likely to influence views and personal experiences. Once we adjust for these other factors, we may find that minority experience itself yields a different picture of individuals' attitudes. The models presented in Table \ref{tab:minorityfull} in the appendix replicates the first model we estimate, but this time we include several of these other individual-level and group-level variables. As we do above, Figure \ref{fig:minoritycoeffull} plots the coefficient for the minority self-identification variable along with a few other demographic variables to ease interpretation and comparisons across models. 


\begin{figure}[t]
	\includegraphics[scale=0.8]{../Figures/Chapter-Minority/fig-coefplot-full.controls.png}
	\caption{Coefficient estimates for minority self-identification and views of US actors. Models include various individual demographic and attitudinal variables. Credible intervals show the 50\%, 80\%, and 95\% highest posterior density intervals around the median point estimate.}
	\label{fig:minoritycoeffull}
\end{figure}



There are some important changes when we include these other variables in our models. The bottom row in Figure \ref{fig:minoritycoeffull} shows the coefficients on the minority variable, with the different colors representing the three outcomes (e.g. positive, negative, or don't know). Across all three models we see that after adjusting for these other variables individuals who self-identify as belonging to a minority group are less likely to report a positive view of US military personnel deployed to their country, the US government, and the US people in general. They are also less likely to report a \textit{negative} view of the US government. Overall the results suggest that minority populations are less likely than non-minorities to see US actors in a positive light, but this does not necessarily translate into more negative attitudes. Across all three reference groups it appears that the lower likelihood of positive views is more offset by an increase in the propensity to have neutral attitudes. 


Comparing the minority variable coefficients with coefficients for the other variables is informative. First, looking at the age variables we can see that older age cohorts tend to have a higher likelihood of expressing a positive view of all three US groups as compared to younger age cohorts, and a lower probability of responding with ``I don't know''. Older age cohorts also appear to be less likely to express a negative view of US actors as compared to younger groups.  The reference category here is the 18--24 age cohort, meaning that we should evaluate all of the the coefficients with respect to that baseline. However, we can still compare the coefficient values of the other groups but need to be cautious where there is significant overlap between the posterior distributions. 

Individuals with more education appear to be more likely to express both positive and negative views of US actors. However, there is no clear indication that more education correlates with larger differences in negative versus positive views. As we discuss in the Chapter \ref{cha:theory}, this likely reflects a similar process whereby individuals with more education simply have more informed opinions, one way or the other. 

%fix references to the appendix

Regarding gender, individuals identifying as female tend to be less likely to express positive views as compared to those who identify as male. Female respondents also appear more likely than male respondents to give a ``Don't know/Decline to Answer'' response when asked their views of these three groups.

Finally, when comparing across income quintiles we find some interesting patterns. First, the baseline category here is the lowest income grouping---the 0--20\% quintile. Compared to this group, there is some indication that higher income groups tend to have slightly less favorable views of US military personnel, but these differences are very small. We find a similar pattern when looking at views of the US government, where only the highest income group shows a slightly positive coefficient value. When looking at views of the US people, groups above the 40\% income group appear to have slightly more positive probability of expressing positive views as compared to the bottom two income categories. We find somewhat similar patterns when looking at negative views, though we do see larger differences in the probability of expressing negative views among the highest income groups when looking at views of the US government. Perhaps the most notable feature here is that income appears to correlate quite strongly with a lower probability of respondents saying ``Don't Know/Decline to Answer'' when asked about any of the three US groups. Only 2.7\% of the highest income groups across all 14 countries replied ``Don't Know/Decline to Answer'' when asked about the US military presence in their country. On the other hand, 7.7\% of respondents in the lowest income group gave the same response. Like education, income appears to correlate more strongly with more informed opinions, as opposed to clearly lining up with more positive or more negative attitudes. 

Next, we discuss the results of a model that is identical to the one we developed in Table \ref{tab:minorityfull}, but this time we allow the coefficient on the minority variable to vary across countries. We show the results from this model in Appendix Table \ref{tab:minorityvarying} and we plot the minority variable coefficient in Figure \ref{fig:minoritycoefvarying}. This figure shows that there is considerable variation in the coefficient on the minority variable across countries. Put differently, the differences between minority and non-minority populations with respect to their views of US actors, is highly context dependent. 

Beginning with the model predicting views of US troops, minority self-identification correlates with a higher probability of a positive response in Turkey, Portugal, and Italy. The posterior distributions for the minority coefficient for Spain and South Korea are both largely positive, with a 93\% and 92\% chance of a positive coefficient given the data. Alternatively, we see minority self-identification correlates with a higher probability of a \textit{negative} response in the United Kingdom, the Netherlands, and Belgium, with slightly more mixed evidence in South Korea, the Philippines, and Australia. In the case of South Korea there's an 83\% chance of a positive effect given the data.

%fix references to the appendix

In the model predicting attitudes towards the US government we find that minority self-identification leads to an increase in the probability of a positive response in several countries, including Turkey, Spain, South Korea, and Belgium. We see slightly more limited evidence of this effect in Italy and Germany as well. There is also evidence that minority self-identification leads to a \textit{decrease} in the probability of a respondent expressing a \textit{negative view} of the US government. That is, individuals are more likely to report that they do not know, decline to answer, or a positive view. This is clearest in Spain, Portugal, the Netherlands, and Australia, with largely negative (but slightly more mixed) coefficients in Belgium, Italy, the Netherlands, and Turkey. 

Looking at the models of attitudes towards US people, minority status correlates with a higher probability of a positive response in Turkey and Portugal, with some evidence of a smaller effect in South Korea, Belgium, and Germany. We see some evidence of a higher probability of a negative response across multiple countries, but fairly sizeable portions of the posterior distributions overlap with 0 in many of these cases, indicating slightly more uncertainty over the direction of the effect on negative responses. The clearest cases of a higher probability of negative views among minority communities are in Belgium, Germany, Japan, Kuwait, the Netherlands, the Philippines, South Korea, and the United Kingdom. These countries all yield posterior distributions for the minority coefficient that are greater than or equal to $70\%$.  Further, we again see some evidence of minority self-identification correlating with a lower probability of positive responses, as in the United Kingdom, Spain, the Philippines, the Netherlands, Kuwait, and Australia.

\begin{figure}[t]
	\includegraphics[scale=0.8]{../Figures/Chapter-Minority/fig-coefplot-varying.png}
	\caption{Coefficient estimates for minority self-identification and views of US actors across countries. Models include various individual demographic and attitudinal variables. Credible intervals show the 50\%, 80\%, and 95\% highest posterior density intervals around the median point estimate.}
	\label{fig:minoritycoefvarying}
\end{figure}


Taken together these models highlight a few key points. First, the way that minority communities view various US actors as compared to non-minority communities varies across countries and can be quite a bit more complicated than our basic expectations suggest. This cross-national variation is interesting not just because minority groups in some countries have more favorable attitudes of US actors than minority groups in other countries, but because minority self-identification does not appear to cause a clear linear increase or decrease in views of the United States. In some cases, like in Portugal, we see clearer evidence of minority self-identification leading to more positive views and less negative views than non-minority respondents. In the United Kingdom, we see the exact opposite pattern, with minority respondents more likely to express negative views of US military personnel and less likely to express positive views. However, in South Korea we see some evidence that minority status correlates with increases in \textit{both} positive and negative views, suggesting that minority respondents have stronger opinions about US military personnel and the US people, and fewer individuals expressing neutral views, as compared to non-minority individuals. In still other cases, like Poland, we find evidence that minority respondents are less likely to express positive views of US actors as compared to non-minority respondents, but minority status does not appear to affect negative views in the same way. 


\input{../Tables/Chapter-Minority/table-varying-coefficient-posterior-percent.tex}


Thus far we have only examined the effect of the minority self-identification variable alone. However, we also expect that differences between minority and non-minority groups may also be conditional upon the level of interpersonal contact individuals report. Specifically, we believe there is reason to expect that minority groups will be more likely to hold negative attitudes towards US groups if they report more interactions with US personnel. As we discuss above, it is often the case that minority groups are exposed to more of the negative externalities associated with US basing and military deployments, and so the nature of their interpersonal contacts may also be different, on average, as compared to other groups. 


\begin{sidewaysfigure}[h!p]
	\centering\includegraphics[scale=0.48]{../Figures/Chapter-Minority/figure-posterior-predictive-varying.png}
	\caption{Posterior predictive check for varying effect models. Darkened dots with credible intervals show the mean predicted count of the outcome categories based on 1,000 simulations from the model. Light blue vars show the actual count of each outcome category as observed in the data. Better fitting models should produce simulated values that are closer to the actual observed data.}
	\label{fig:minorityvaryingppcheck}
\end{sidewaysfigure}



We present the results of these models in Figure \ref{fig:minorityinteractionvaryingeffect}.\footnote{We generate these comparisons by holding other variables constant at specific categories and values. Age is set to ``35--45 years'', Income to the ``41$^{st}$--60$^{th}$'' percentile, Gender to ``Female'', Personal Benefits to ``No'', Experience with Crime to ``No'', the Amount of American Influence in the referent state to ``Some'', the Quality of American Influence to ``Neither positive nor negative'', and the remaining variables---education, ideology, the count of US bases, GDP, population, and troop deployment size---to their mean values.} For the sake of simplicity we present only a figure showing the primary comparisons of interest for the positive and negative outcome equations. Here we seek to illustrate two primary comparisons: 1) how minority respondents reporting interpersonal contact with US personnel compare to minority respondents who do not report such contacts, and 2) how minority respondents compare to non-minority respondents among those who report having interpersonal contacts with US military personnel. We do this by generating simulated posterior predictions of the probability of a positive or negative response for each of these groups and then comparing the posterior distributions across different groups. These models are the most flexible with respect to allowing the coefficients to vary across countries and factor groupings, and therefore allow for the most variation in inter-group comparisons. Accordingly, we focus on what we think are some of the more notable points, but cannot claim to provide an exhaustive overview of all possible comparisons.

Most generally, it is worth observing that the combination of interpersonal contact and minority self-identification can, and often does, generate substantively large changes in the predicted probabilities across all three models. We see the most variation in the US troops model, where personal contact with US military personnel among self-identified minority populations tends to produce larger positive changes in the predicted probability of a positive response. In contrast, when contact is held constant at a ``Yes'' value, toggling minority self-identification on tends to produce more mixed changes, with several countries seeing some reduction in the predicted probability of a positive response, but a few cases, like Turkey, Spain, Portugal, and Italy yielding increases in the probability of a positive response. 

Looking at the negative responses we find that there is a mixture in how minority respondents compare to non-minority respondents when both groups report interpersonal contact with US personnel. In some countries like Turkey, Spain, and Italy we find that minority respondents tend to be less likely to give a negative response, whereas in other countries, like Poland, the Philippines, and to a lesser extent the Netherlands and Belgium, we find that minority respondents are slightly more likely to give a negative response. When we focus on minority respondents and compare those reporting contact versus those not reporting contact, we find that the contact group has a lower probability of giving a negative response in several countries, including the Untied Kingdom, Turkey, Spain, and, Belgium. We find smaller differences in Italy and Australia where the comparisons of the posterior distributions do not uniformly show the contact group to be less likely to give a negative response.

\begin{figure}[t!]
	\centering\includegraphics[scale=0.7]{../Figures/Chapter-Minority/fig-contact-minority-interaction.png}
	\caption{Figure shows the change in predicted probability of positive and negative responses according to changes in the personal contact and minority self-identification variables. These changes were derived from varying effects models identical to those shown in Table \ref{tab:minorityvarying}, except we now include an interaction term between minority self-identification in the population-level equation, and allow the effects of these three terms to vary across countries. 50\%, 80\%, and 95\% credible intervals shown around the predicted values.}
	\label{fig:minorityinteractionvaryingeffect}
\end{figure}

When we look at the US Government models we see more consistent increases in the probability of a positive response when we toggle both variables ``on'' while holding the other constant in the ``on'' position across most countries in our data. Poland, the Philippines, and Kuwait are the notable exceptions here. As we have discussed elsewhere, however, Kuwait and the Philippines in particular are significant outliers with respect to the size of the minority populations in our survey data. While this over-representation is not necessarily a problem in and of itself, the extreme differences we observe in the case of Kuwait may reflect the more unique status of the large migrant worker communities rather than ``domestic'' minority communities.

We find slightly more variation when we look at the negative response comparisons. In general we find that, among those reporting interpersonal contacts with US personnel, minority respondents are often less likely to give a negative response as compared to non-minority respondents. These differences are most evident in the United Kingdom, Spain, Portugal, the Netherlands, Germany, and Belgium. Among minority respondents only, we find that those reporting interpersonal contacts with US service personnel are less likely to give a negative response in the United Kingdom, Portugal, and Belgium, with more limited evidence of similar differences in countries like Japan and Germany.

Finally, when we look at the ``US people'' models we find more a more consistent pattern of minority respondents with personal contact reporting a higher probability of a positive response in most countries. We find these differences are more muted in Poland, Japan, and Italy. In general, the differences between minority respondents who report contact and those who do not is largest in Australia, and Belgium, with Turkey, Portugal, and Kuwait all showing changes of approximately 10--15 percentage points in the predicted probability of a positive response. And of those individuals reporting having had personal contact with US military personnel, we find that minority respondents tend to have a lower probability of expressing a positive view of the US people in most countries. The notable exceptions here are Turkey, Portugal, Germany, and Belgium where the differences between these groups are more muted.

The comparisons are decidedly different when we look at the negative response models. To facilitate comparisons across response equations and models we held the range of X-axis values constant across panels in the figure. That is, we are using the same scale for the negative models for comparison even though this compresses the perceivable range of the values for these models.  We do indeed find that there are differences between minority and non-minority respondents who report contacts, as well as among minority respondents who do and do not report interpersonal contacts, but the magnitude of these differences is considerably smaller in this model and equation. For example, we find that minority respondents are generally less likely to give a negative response in Turkey and Portugal, with some more limited differences in Belgium, and to a lesser extent Germany. Among minority respondents, we also find that those reporting interpersonal contacts are less likely to give a negative response in Turkey, Portugal, Belgium, and Australia, and again find more limited differences in the United Kingdom, the Netherlands, Spain, Kuwait, and Germany. Importantly, while we do observe differences, the magnitude of these differences as compared to other models and outcomes is much smaller. In general the differences we see here are in the 3--6 percentage point range. 

Regarding broader model performance, the varying coefficient models generally perform well. Figure \ref{fig:minorityvaryingppcheck} shows the results of posterior predictive checks on the three sets of varying coefficient models shown in Appendix Table \ref{tab:minorityvarying}, broken down by the country groups in each model. These checks function by using the models to generate a number of simulated ``data sets'' for the outcome variable. For example, we use the model to generate $\sim$ 38,000 simulated response values for each respondent in the data, and then we do this iteratively 1,000 times. We can compare relative frequency of these simulated values to the actual frequency of the outcome variable categories as observed in the actual data. In this plot the dark dots represent the mean of 1,000 different simulations and 95\% credible intervals. The lighter blue bars represent the actual count of each category observed in the data for a given country. Overall the models perform well and the simulated counts fall very close to the observed counts for all levels of the outcome across all countries.




\subsection*{A look at Japan}

Japan is well-known for the hotly contested presence of US military facilities, particularly in areas like Okinawa. However, US military facilities exist all throughout the country, and so the relationship between minority groups and US facilities may vary depending on geographic context. For example, while many people associate minority-military interactions with Okinawa it is often the case that there is overlap between minority communities and US military facilities in general. That is, aggregating to the regional or province level and looking for relationships between the location of US facilities and regions with high minority populations can be misleading. 

Figure \ref{fig:japanminoritymap} shows the breakdown of minority respondents across regions within Japan (shaded green), and the distribution of US military bases (blue dots).\footnote{Note that this map aggregates Okinawa into the Kyushu region. In the analysis that follows we break Okinawa out into a separate region given its historical importance. Unfortunately we only have data parsing Okinawa out in this way for two of the three years of our survey.} As the map shows the largest proportion of self-identified minority group members in our survey are located in the Kanto Region of Japan, which includes Tokyo as well as some key US military installations. While our survey is not designed to be geographically representative there is still an important point to be made here---while there is often a correlation between geographic region and minority status, these two variables do not always map neatly on to one another. Even in regions that are not normally synonymous with particular minority groups there may still be large minority populations, and those populations may have very different interactions with US military personnel than individuals who do not belong to a minority group. For example, occupying space close to bases, or spaces that are particularly exposed to the negative externalities of bases.


\begin{figure}[t]
	\centering\includegraphics[trim = 1.75cm 1cm 1cm 1cm, scale=0.85]{../Figures/Chapter-Minority/fig-map-japan-min-location.png}
	\caption{The distribution of Japanese respondents who self-identified as belonging to a minority group and the location of US military bases. The shading intensity represents the share of all minority respondents present in each region.}
	\label{fig:japanminoritymap}
\end{figure}


To examine how minority compare to non-minority groups in Japan we estimate a model identical to the one shown in Appendix Table \ref{tab:minorityvarying}, but this time include only Japan. As in the full cross-national version of this model, we include several predictor variables that may be associated with minority status, and we allow the minority self-identification variable to vary across regions. Figure \ref{fig:coefplot-japan} shows the coefficients for the minority self-identification variables for the Positive, Negative, and Don't know/Decline to answer equations for the three models and reference groups.

Across all three models, we find little difference between minority and non-minority respondents with respect to the probability of a respondent expressing a positive view of US actors. The predicted coefficient value is extremely close to zero and a substantial amount of the posterior distribution falls on either side of 0. Only in the Kanto region is there some limited evidence of minority respondents having a slightly higher probability of expressing a positive view of the US people, specifically. 

We find slightly more evidence that minority respondents differ from non-minority respondents when we look at the negative outcome equations in each model. We also see some evidence that the differences between minority and non-minority respondents are partly conditional upon the region in which respondents reside. Most of the median point predictions for the posterior distributions fall below 0, and in some cases fairly large proportions of those distributions fall below 0 as well. In Kanto, for example, over 96\% of the posterior distribution falls below 0, indicating that minority respondents there are slightly less likely to express a \textit{negative} view of US troops as compared to non-minority respondents, once we adjust for various other individual-level and environmental characteristics. Similarly, approximately 90\% and 92\% of the posterior falls below 0 in the Chuugoku and Hokkaido regions, respectively. Alternatively, in the Kinki region we find that approximately 94\% of the posterior falls \textit{above} 0, indicating that minority populations there are slightly more likely to express a negative view of US military personnel than non-minority groups.


\begin{figure}[t]
	\centering\includegraphics[scale=0.75]{../Figures/Chapter-Minority/fig-coefplot-2-japan.png}
	\caption{Varying coefficient estimates for minority self-identification and views of US actors in Japan. Coefficients vary across regions. 50\%, 80\%, and 95\% credible intervals shown around point estimate. Credible intervals are calculated using the highest density posterior interval.}
	\label{fig:coefplot-japan}
\end{figure}

Shifting to views of the US government the results look very similar to the model predicting views of US military personnel with respect to the differences between minority and non-minority groups in giving positive responses. For all regions the median posterior prediction is close to zero, with substantial portions of the distribution falling on either size of 0. Ultimately this indicates that there is little difference between minority and non-minority groups with respect to positive responses. Shifting to negative views, we get a slightly more homogeneous pattern wherein minority groups generally appear \textit{more} likely than non-minorities to give a negative response when asked their views of the US government. In Kinki we again find the clearest evidence that minority groups tend to be more likely to express a negative view than non-minorities, with approximately 95\% of the posterior distribution falling above 0. However, in Chubu, Kyushu, Shikoku, and Tohoku we find 75\% or more of the posterior falling above 0 in each case.

Finally, looking at the model predicting attitudes of US people we find slightly stronger evidence of a general positive trend, wherein minority respondents are more likely to express a favorable view than non-minority respondents. This is clearest in the Kanto region, where approximately 90\% of the posterior distribution for the coefficient falls above 0. In nearly every other region we find 70\% or more of the posterior distribution falling above 0, with only Kinki and Tohoku having values in the 60\% range. Interestingly, we also find a more consistent pattern indicating that minority respondents are also more likely to express a negative view of the US people. The posterior distribution on the minority coefficient is positive across every region, but in Chubu, Chugoku, and Tohoku we find 90\% or more of the posterior falling above 0. We find slightly more muted patterns in Hokkaido, Kanto, Kinki, Okinawa, and Shikoku, where approximately 70\%--83\% of the posterior falls above 0.

Overall, it is somewhat surprising that we do not find more consistent relationships between minority self-identification and anti-US attitudes, as this is what many historical narratives and anti-base activists would suggest we should expect to find. Some of the large variation (credible intervals) may be due to the relatively small size of the minority sample---with only about 500 Japanese respondents identifying as belonging to a minority group and many regions not having large minority respondent rates, we must allow for the possibility that some of variation we see here is mostly noise rather than genuine variability in effects. One advantage of the multilevel modeling approach is that it allows us to draw information from the general population to help inform estimates in cases where we have fewer observations for particular groups. While this is certainly an advantage compared to other statistical approaches, like using binary fixed effects to capture differences in group-level intercepts, we are cautious given that we expect the differences between minority and non-minority respondents to vary across groups. The shrinkage induced by our approach here may also tend to compress some of the between-group variation we expect to see where we lack more observations.
\section*{Conclusions}

When conducting interviews, we spoke with an anti-base activist in Berlin. During the conversation, he made it clear that minority populations, particularly immigrants and refugees, were largely an afterthought in the dynamics of basing: ``They are not playing a key role in the base discussion. The Germans and the US have security concerns related to hiring immigrants and refugees to work on the base. They are afraid to hire immigrants. They have strong security checks and controls. They are afraid to hire someone from al-Qaeda'' \cite{berlinone20190723}. At the beginning of the chapter, we discussed the idea that many minority populations may see US military installations positively through the potential economic opportunity that they provide. However, as both this anti-base activist and the Government Relations Officer at the US installation in Wiesbaden attest, there was little consideration given to local minority populations at all, let alone any concerted efforts undertaken to reach out to such communities. 

%To the extent that they consider minority populations, it is to sideline them from the types of employment otherwise available to local people on the base.  - This statement seems strong, do we have evidence of active sidelining for employment?

The data that we collected from countries around the world offers a glimpse into how these issues present themselves in public opinion. In our most general models, minority groups are less likely to report positive views of US actors than majority groups. When we allow the effect of minority self-identification to vary across countries, however, we find that the effect of minority status on views of the US is more mixed. In many countries, minority groups are less likely to report positive views of US actors than non-minority respondents, but often this simply translates into more neutral views, while in other cases there is a clearer increase in actual negative sentiment. This complexity holds true when we look specifically at the effect of interpersonal contact, where in some countries, self-identified minorities who report interpersonal contact with US forces have more positive views and in others they have more negative views. 


%\ref{cha:meth}
What we have demonstrated in this chapter is that it is important for decision makers to see host nation populations in more complexity. For example, contrary to many popular assumptions, \citeasnoun{Johnson2019} describes the relationships between the Okinawan population and the US military personnel as a deep, multilayered subject, characterized by both affection and nostalgia, but also resentment and anger. The people in host nations do not have uniform experiences of the US military, and they do not have uniform views. As discussed in Chapter \ref{cha:meth}, views can depend on whether individuals have had interpersonal contact with members of the US military, and in this chapter, we have uncovered another complicating factor --- minority status. Minority communities in some cases may be on the receiving end of more negative externalities than majority communities. In such cases, the negative relationship between minorities and the US military is direct. In more indirect relationships like the one described in Germany, minority populations are given very little consideration in the dynamics of military basing. In such circumstances, a more circuitous causal mechanism is likely at work, in which the American relationship with the host central government empowers both the government and the dominant ethnic group. By intertwining itself with the ethnic-majority central government, the United States becomes conflated with groups contributing to minority discrimination and reducing pathways to minority representation through the centralization of power. 

Much as we would expect from public opinion polling in the United States, an individual's identity matters a great deal in how they view the world. This is because an individual's experiences will influence their views. The same is true in other states and researches should consider it in any analysis of how the United States interacts with societies in the international arena. While the US military may see a high degree of support from majority groups in some countries, not all populations have uniform experiences and views of the American military presence. Unique experiences with the US military, the host government, and the majority population in the host country will influence individuals' views. From a policy perspective, this issue is vital  to understand and apply to the US military's relations with local populations in host countries around the world. Even in societies that see high levels of favorable views toward the US military, negative views among a cohesive minority population can cause any number of issues --- from security concerns to land use issues, and threaten the stability of political support for a US presence. In previous chapters, we argue that outreach to surrounding communities is a way in which the US military can improve relations with surrounding communities. Extending these outreach efforts to minority groups, and specifically tailoring to their wants and needs, would be a way to build better relationships with minority populations and promote better relations between them and central governments.  

%outreach recommendation might depend on what the contact variable looks like for minorities. - AS
These results are a call to those engaging in policymaking surrounding US military bases and their relationship to ethnic minorities to more deeply understand the communities that will be engaged through the base-making and base-sustaining processes. Many national governments have vastly different relationships with minority populations within their borders than do others, and many ethnic minority groups within the same country can relate very differently with both their national governments and an outside military force. While some will see an American presence as a sign of potential economic prosperity, others may see it as an unwelcome intrusion. Policymakers must deeply engage in the history and preferences of the individual countries and groups involved in basing decisions to understand whether and how to engage in outreach efforts, so that potential basing relationships between the United States and minority populations can be healthy and productive.
\end{comment}








