\chapter{Crime and Perception \label{cha:crimes}}
\doublespacing




%\section*{Introduction}
\vspace*{-0.5cm}
\rule{\linewidth}{0.10pt} \\[-1cm]
{\footnotesize\paragraph{Summary:}  The previous chapters have examined positive contact, economic benefits, and the role of minority groups in basing and perceptions. We take a deeper look at a negative interaction between service members and host-state civilians by looking at how national media reports, personal victimization, and criminal victimization within a respondent's network influence respondents' views of the US military and other actors. Those that live near a well-publicized criminal event, experience criminal victimization, or report victimization in their social network are both more likely to have an informed view of the US military presence and more likely to have a negative perception of crime. However, these effects are (compare to contact/economic benefits). Given this comparison, in the competition for consent, it seems the good outweighs the bad absent media reporting.} 
\\[-0.5cm] 
\rule{\linewidth}{0.10pt}

\vspace*{0.5cm}

%anecdote lead needed, something about general perceptions or people's views being tied to something. Carla or Andy will have the best idea as to what fits since they were in all four sets of interviews.


During the early morning of a Wednesday on April 17th in 2019, local police arrested an airman from Kadena Air Base in Okinawa for suspicion of drunk driving. The 26-year old ran into a motorcycle and injured two civilians; a police-issued breathalyzer calculated his blood-alcohol level ``was nearly twice Japan's legal limit of 0.03 percent.''\cite{Burke2019} Although the act of driving under the influence and injuring host-state civilians is undoubtedly something that would make the local news, it was not the only case of misconduct in Okinawa that April. An additional drunk driving incident happened a week later. More staggeringly, a Hospital Corpsman 3rd Class in the Navy murdered a 44-year old Okinawan woman by stabbing her in the neck and then committing suicide by surgically cutting his groin.\cite{Simkins2109} The clustering of events provided further fuel to Okinawa's anti-basing movement. The movement argues that the presence of US service members and their history of heinous and routine crimes had rocked the prefecture for decades before the events in 2019. Moreover, Okinawa has a long history of US military scandals that shocked people's conscience in both the region and the country as a whole.

Service members commit crimes against local populations that range from the mundane to the heinous. When these crimes occur, some subset of them will gain local, regional, national, or international attention. As such stories trickle back to the United States, it is not clear if the United States sees these as isolated events, another cost of providing global security, or a pattern of behavior that the military has been incapable of curtailing. Even overseas, it is not clear if the behavior is systemic, sensationalized, or somewhere in between. It is feasible that the crime rate among service members may be at or below the crime rate in the nations that host them; the relatively lower rate of the criminality of the military relative to other host populations is something US military leaders have used as a defense.\cite{Gillem2007} Still, the very fact that it is a foreign military member committing crimes against a host population makes each incident more noteworthy and more likely to garner media and local attention. Our research found some evidence that military deployments correlate with increases in property-related crimes but less evidence on other changes in the rate of crime within society.\cite{Allen2011} In understanding how military crimes play a role in this book, the incidence rate is less of the focus than the perception those incidents create. However, the more crimes that occur, the more opportunity for negative views generated by those crimes to proliferate. 

In the previous chapters of the book, we discussed the role that contact has on producing both positive and negative perceptions of the United States. We have discussed interpersonal contact creating positive interactions and how economic benefits can reinforce positive support for the US military presence. As such, much of our discussion has focused on the positive aspects of interpersonal contact in breaking down stereotypes and encouraging familiarity. While the chapter on minority presence indicates fewer positive trends in perceptions of the United States, this chapter delves into a directly negative effect: crime. In this chapter, we develop our expectations as to where we see service member crime events make national media venues and the role of crime in affecting perceptions among the 42,000 people we surveyed. 

In understanding competition for host-state consent to base, crime plays an insidious role for basing powers that may be hard to curtail. A country that deploys tens of thousands or hundreds of thousands of people into other nations will have some criminal infractions within its deployed populations. It would be difficult for any military or institution to perfectly select members of its community that have no capacity or willingness to commit a crime. Doing so would rely on outdated notions of why crime occurs. There are complex determinants of what causes crime, and no social institution has successfully eliminated all crime within its membership. In considering reducing or limiting criminal offending within its ranks, the military can certainly invest in structural, incentive, and training changes. The US military has engaged in efforts to curtail crime against communities surrounding bases by having ``courtesy patrols'' of US military officers in civilian clothes that patrol local entertainment venues to keep US personnel in line\cite{koreaone20210926}. Such a tactic is used at domestic bases as well, in efforts to maintain good relationships with local communities and stave off the same destructive behavior at home\cite{Wood8272010}. 

However, crimes still occur, and these enforcement mechanisms often seem to be put in place only after a crime has been committed against the local population. For instance, the US military on Okinawa instituted new courtesy patrols after two US sailors were accused of raping a local woman\cite{McCurry10172012}. These patrols are not a police force themselves, and their authority largely extends as far as enforcing curfew\cite{DVIDS11212012}. So, a bigger picture decision for a basing power is whether to decrease the opportunities for crime by isolating their own military population from the local communities or increase the monitoring of, incentives against, and punishments for criminal behavior. The United States has increasingly chosen the first option for its overseas bases over the last two decades. By choosing isolation, the United States attempts to internalize criminal behavior to its population and decreases the likelihood of a negative story causing basing ramifications. 

Additionally, building self-contained military cities requires mostly large fixed costs (land, buildings, etc.), while the option to monitor and police behavior requires long-term, enduring costs to maintain. Isolation is cheaper and easier than integration and monitoring. While the Navy has a history of maintaining an insular presence, the rest of the military started following a similar path by the mid-2000s.\cite{Gillem2007} The trade-off with this strategy is that the non-personnel-related costs from deployment will remain and the positive contact and economic flows from personnel will evaporate as a result of the strategy of making US bases `American cities as a fortres.' As we've seen elsewhere here, while isolation may limit the bad interactions, it may also limit the good ones and create an increased sense of suspicion among local populations that could further exacerbate tension between US personnel and locals. This dynamic could create space for conspiratorial counter-narratives about the nature of the US presence, either organically or such narratives pushed by competitor basing powers. This could leave the politics of basing worse off than if crimes occurred in a basing community, if they came against the backdrop of a robust series of positive social relations between local populations and US basing personnel.

To understand our data, we first explore our current understanding of the relationship between US military deployments and crime within host states. We then use this to build our expectations of and how crime affects the perceptions of host-state civilians. Next, we use these hypotheses to return to our model of perceptions to estimate how crime influences people's positive and negative views of the United States. Finally, we conclude by evaluating how crime influences the domain of competitive consent.


\section*{Research and Expectations}

We begin our theoretical development of the impact of crime on perceptions by host-state civilians by reviewing the existing body of work on service member crime. The presence of crime among foreign-deployed military communities might be one of the better-studied aspects of overseas deployments. The documentation of the crimes through local police and internal military investigations allows for a closer examination of the record than other aspects of overseas deployments. These records, when public, also allow for ease of media reporting and scrutiny that other social harms may be harder to track and document. Beyond the nature of crime reporting, case study research on particular countries or deployment communities has been key in how basing changes the social fabric of established towns and cities.

The foundational work that systematized understanding and conceptualizing crime in the armed forces is Clifton D. Bryant's \textit{Khaki-collar Crime}.\cite{Bryant1979} This sociological and criminological examination argues that understanding criminal deviance requires additional assumptions than civilian crime as it takes place in a context that has additional, layered social institutions and responsibilities within it.\footnote{To be clear, deviance in the sociological and criminological research is not a normative judgment but is instead a label used to those that violate social norms. Bryant goes to lengths to argue that crime is a socially constructed event. Social scientists miss studying several kinds of behavior that would be criminal by most accounts except that it has legal sanctions. Going beyond this observation, he argues that crime, as we conceptualize it, is a natural partner to the social institutions that we construct.} In discussing these added layers, Bryant adds several additional causal factors into why military crime occurs, including the nature of the military population (typically young, male, lower- or lower-middle-class, and low skilled when he is writing in the 1970s), the stress of military existence, the conditioning aimed at reducing individual choices, over-regulation through a dearth of formal rules, informal norms, responsibilities, and resources make infractions inevitable. Military culture and the totality of the institution in a service member's life, normalization of abuse of subordinates throughout training, a culture of violence that cultivates violent responses, combat offering opportunities to cloak violence against superiors, over-bureaucratization, military socialization, official tolerance for some criminal infractions, and subcultures encouraging deviance from official military training all play a role in creating military crime. In getting a handle on the multifaceted ways the military creates space for crime to occur and how service members can commit a crime, Bryant categorizes crime into three types: crime against property, crime against people, and crimes against performance. The first two are more well-known, while the final category deals with crimes that hinder one's own duties or duties of others.

For those three types, there are three supra-categories where crime occurs. Crimes can occur within one's service (intraoccupational), against those outside of one's service (extraoccupational), or against those that are in another service (interoccupational). Extraoccupational crimes warrant subdivision across three different areas as the unique nature of military service affords three kinds of interactions in which crime can occur: against American civilians, against foreign-friendly civilians, and against enemy civilians.  The types, details, examples, and causes of these crimes are vast within his work. Since we are primarily interested in how crime affects the perceptions of host-state civilians, the primary category this chapter examines is extraoccupational crimes against foreign-friendly civilians. This focus has a few caveats. First, there is quite a bit of overlap between the factors related to crimes against American civilians and foreign civilians in Bryant's analysis of many crimes. Still, there are additional opportunities in interactions with foreign civilians, often owing to slippage in the degree to which individuals will be held accountable by the host government or the US military. Second, Bryant lists alcohol and drug use as an intraoccupational crime due to the use of the drug. Still, we certainly see the purchasing and sale of illegal substances, which is true of general participation in the black market, as an extraoccupational crime in addition to its intraoccupational nature.\footnote{Notably, much of the research, surveys, and analysis of offending is either on intraoccupational offenses or crime by veterans. There is robust research on substance abuse \cite[see:][]{bray2010}, sexual harassment and assault \cite[see:][]{bostock2007,stander2016}, intimate partner violence \cite[see:][]{sparrow2020}, and offending among veterans \cite[see:][]{moorhead2021}.}

The third caveat in understanding his work for our purposes is that much of his study of crimes against foreign civilians is in the context of a US military deployed during wartime. Our study is exclusive to deployments in non-war zones. This distinction creates a qualitative difference in interpreting the opportunity, meaning, and punishment of crimes against civilians. As Bryant illustrates overly well, service members can hide their crimes during a conflict. Primarily drawing upon examples from the Vietnam conflict, there are several examples where service members victimize South Vietnamese civilians in brutal ways and claim before, during, or after the person was a member of the Vietcong. Many of these acts would be war crimes even if the victim were enemy combatants, but the designation as an enemy excuses the behavior in the perpetrator's account. Beyond the cover conflict gives to criminal actions, the salience of the conflict makes prosecution of crimes less likely than if the same crime happened during a peacetime deployment. These dual factors make the impact, reporting, and punishment of crime fundamentally different from the contexts we examine within this chapter.  

Several important works in this area help us understand the prevalence and role of service member extraoccupational crime globally. Bucher builds upon these foundations to argue that ``General Strain Theory'' is useful in understanding military offending.\cite{bucher2011} A combination of straining factors (such as the inability to achieve personal goals, negative experiences, inability to meet family needs, etc.) can lead individuals to pursue crime when they lack access to adequate alternative coping mechanisms. The study finds a relationship between measures of strain and drug use, violence, and theft.

Moon's book is a close examination of how both the United States and South Korea institutionalize, manage, and react to the culture of sex work and prostitution within South Korea.\cite{Moon1997} Historically, the United States has maintained a fine line between seeing prostitution as an illegal activity in most jurisdictions it operates in while also tolerating or encouraging its promotion as a seemingly necessary outlet for deployed personnel. This dual-track of both officially being against prostitution while, at times, all but condoning the existence of prostitution has led to inconsistencies in how the military approaches the issue. The US military in South Korea has gone through several periods of tacit encouragement and complete rejection by the US military.\cite{Baker2004} Enloe argues that the US military historically treats prostitution as both a resource (for personnel) and as a consistent threat due to the spread of venereal disease---these views mimic Great Britain's approach to similar issues within its empire.\cite{Enloe2000,Gillem2007} In South Korea, towns go through a period where illegal sex work flourishes despite local and military restrictions. Eventually, when the US military becomes more interested in reshaping its domestic image, it engages in a cleanup campaign with the South Korean military to curtail prostitution in the camptowns surrounding the nearby bases. At other times, the military largely ignores the issue because ``you don't wash your laundry in public''\cite{amb20180713}.

Having a tolerated black market for prostitution leads to a proliferation of additional types of crimes. Fundamentally, without legal sanctification of a business that flourishes, the state has abdicated the responsibility for enforcing property rights. By the state criminalizing particular aspects of the business, associated crimes become easier as the victims are afraid to go to the police. For example, if a armed forces member decides to assault a sex worker while being a client of their services, the sex worker is unlikely to report the crime to authorities. The victim's occupation makes it possible to face legal sanction without any recourse for their victimization. Beyond this, private actors need to enforce property rights in place of the state without legal protection of property rights. This leads to increases in violence to protect people and goods and to enforce contracts. For example, if a client refuses to pay a sex worker, the sex worker either loses out on that income or turns to another person to threaten or commit violence to recoup their lost revenue. This issue is prevalent in nearly any market where the state abdicates the role of protecting property rights, whether it is sex work, drugs, farms in New World colonies, or just heavily taxed goods.\cite{Resignato2000,Fleenor2003,Reynolds2010,Vandusky2011}

In the specific example of South Korea, the black market for sex work created several negative externalities. Moon details how the social stigma around sex work often made it impossible for women to reintegrate into Korean society and, for many, the end goal was to marry an American soldier in the hopes of leaving Korea altogether.\cite{Moon1997} While marriage and leaving South Korea were prospects, the danger of violence and murder was also present. The occupation was not lucrative for many women either as bar owners that controlled the lives of many of these workers would keep them in a debt bondage system that made escape financially impossible.\cite{Moon1997,Gillem2007} The lack of state regulation allows social norms to regulate the market in Korea partially. Sex workers with predominately white clients became the targets for abuse, alienation, and murder if they were found working for black soldiers.\cite{Moon1997} Enloe mentions a similar situation arising in Vietnam, and Gillem discusses how contemporary South Korean bar owners seek only to serve Americans in camptowns and dissuade local Koreans from patronizing their establishments.\cite{Enloe2000,Gillem2007} Moon demonstrates that even during the Camptown cleanup campaigns in the 1970s, there was an effort to reduce racial discrimination among sex workers and enforcement and punishment of the anti-discrimination regulations fell on the women. More recently and despite these cleanup campaigns, NGOs have worked to raise awareness about women's sexual labor in US basing districts \cite{Cooley2008}. Local bars and nightclubs are said to still often feature so-called ``juicy girls'' that ``bordered on prostitution and sex trafficking\cite{koreaone20210926}.''

Given the proliferation of other grey and black market services that arise around a tolerated black market, it follows that permissive conditions for one type of crime will spawn additional crime networks to support the tolerated market and provide cover for other kinds of crime to proliferate. This building of networks is not unique to prostitution. Still, we see similar effects when significant issues with drug use lead to a contagion of other crimes, including drug trafficking, sexual assault, and robbery. Drugs have been an endemic problem within the armed forces in different periods, though the military has successfully combating drug use in some high profile deployments, though marijuana and steroid use remained high in the 2010s.\cite{Nelson1987,ballweg1991,Baker2004,bucher2012} 

In many US deployments, the US brings with it millions of dollars of capital-intensive goods, supplies, food, and other resources that are ripe for service members to pilfer and sell on secondary markets.\cite{Bryant1979,Nelson1987} Not only do service members act as a demand in the marketplace for illicit goods, but by having access to robust stores of goods, they can act as suppliers for local demand. During the Cold War and post-Cold War eras, US deployments have several cases where millions of dollars of equipment or, more concern, thousands of firearms have disappeared from the military and found their way onto the black market. (citations) The influx of cash and the proliferation of grey and black market activities because of the blind eye turned to certain areas produces overall trends, where US military deployments correlate with increases in aggregate property crime rates \cite{allenandflynn2013}.

%In an early project, we found that US military deployments correlate with increases in aggregate property crime rates \cite{allenandflynn2013}.

Trying to quantify the number of crimes service members do is a difficult task. Notably, base commanders have historically pointed out that service members commit crime at lower rates than the country's general population.\cite{Gillem2007} Early research within our team tried to assess whether the presence of troops correlates with higher levels of criminal activity within a country; that is, does the demand for illicit goods create a proliferation of other crimes within society. Our research found some connection with property-related crimes but was generally limited in its overall assessment.\cite{allenandflynn2013} Much like we pointed out about other research in Chapter \ref{cha:meth}, this is another case of ecological inference issues that make conclusions about individual behaviors difficult to draw. It is possible that the military deployments are not causally related and that other factors create the relationship we found in that case. There are threats to inferences about individual behavior and some questions about the spuriousness of the relationship.\cite{King2004} 

Recent work by Efrat uses primary source documents from the US army to catalog 361,487 offenses from 1954-1970, with 74\% of those crimes taking place in NATO countries.\cite{efrat2021a} As he notes, these crimes are likely under-reported as crimes statistics are already an under-reported figure for a variety of reasons; the perpetrators being US military members may increase that under-reporting further since victims may fear retribution or believe that reporting the crime will not result in meaningful justice \cite{allenandflynn2013}. In a separate study, Efrat also uses judicial records to find that most crimes do get referred to the military for prosecution.\cite{efrat2021b} Cooley also finds evidence that service members committed as many as 50,000 crimes in South Korea alone between 1967 and 2003 by tracking the number of SOFA incidents and criminal jurisdiction and waiver cases that occurred in the ROK \cite{Cooley2008}. This number also does not consider crimes committed by dependents and other civilians present due to the US basing arrangements. 





\subsection*{Theorizing crime reporting and perceptions}

While we have reviewed the major understandings of crime within this chapter, our goal is to look at the effects of those crimes. Setting up our argument in this chapter is a short affair as much of our discussion extends from chapters \ref{cha:theory} and \ref{cha:meth}. First, we are interested in whether highly reported crimes are more likely to influence the views of our respondents. These events are most likely to galvanize public backlash and understand whether the crimes media outlets decide to cover have an independent effect on individuals' perceptions. Using a new data set on widely reported crimes by service members, we expect people nearer to places with reported crimes to be more likely to have negative views of the US actors we ask about.

\begin{hyp}
	Individuals living near the reported location of a crime committed by a US service member will be more likely to express negative views of the American presence/government/people.  %kind of wordy, could rewrite.
\end{hyp}

Naturally, there might be a presumption that countries with more service member crimes are more likely to have overall more negative assessments of the US military, but this does not seem to bear out in the data. Gillem reports: 

\begin{quote}
	More telling from a socio-spatial perspective is the variation in crime rates within the military community. In the same three-year period that South Korea experienced 1,246 criminal acts by US soldiers, in Okinawa, Nawa, Japan, soldiers committed 198 criminal acts. The annualized per capita difference is quite instructive. In South Korea, there were 11.2 crimes per 1,000 soldiers. In Okinawa, there were 2.4 crimes per 1,000 US soldiers. Why is the disparity so striking? I suggest that it is largely the result of policies related to housing and land use.\cite[p.48]{Gillem2007}
\end{quote}
% isn't crimes per soldier the wrong metric here? Crimes per host nation person seems better. I'm not sure what year is being reported here but South Korea's population is 50 times larger than Okinawa's so 6 times the crime is going to have much lower impact on South Korean society than Okinawan society. I added some in the paragraph below on this.
We expect to see more reports of Japanese crime rates than Korean crime rates despite the disparity in offenses between the two countries. Given a higher reporting for a lower rate of occurrence, those near those crimes are more likely to report negative views. In addition, Gillem's measure of crimes per US soldier may not be the most instructive measure of impact since the crime rate per US soldier may matter much less than the crime rate compared to the population of the host community. South Korea's much larger population means that the media will likely ignore some level of crime in a country with a larger population. However, in Okinawa, which has a much smaller population, even a lower number of crimes is likely to impact society and receive greater news coverage disproportionately. 

While some may argue that the proper comparison is South Korea to Japan as a whole and not Okinawa in particular, Okinawa's isolated geography, distinct history, and ethnic identity mean that crimes committed in Okinawa are likely to be treated as a highly important local issue. The same may not be the case of a crime committed in Tokyo or Seoul. Given these varying dynamics that likely play a key role in how societies react to crimes committed by US military members, we expect distance from a crime and news coverage of the crime to matter more than Gillem's measure of crimes per US soldier. For most people not directly impacted by crimes to themselves or their social network, their perception of crime will be highly mediated by the media. Therefore, a critical determining factor is not only whether someone has been directly impacted by crimes to themselves or their social network, but whether the media chooses to highlight stories featuring crimes by US soldiers \cite{Song2004}.

When it comes to media coverage of crimes committed by members of the US military, prior work in this area indicates that the wider context likely matters a great deal. Whether the basing arrangement is highly politicized helps determine the degree of coverage that crimes receive, even when crime rates by members of the US military appear to be at a low ebb \cite[p.123]{Cooley2008}. Recent events at overseas basing locations certainly lends anecdotal weight to that conclusion, as well. The 2012 case of two US sailors raping a local woman in Okinawa came at a time of heightened basing protests over the deployment of Osprey aircraft\cite{McCurry10172012, USAToday312013}. US military officials object to the media's coverage of crimes, claiming that the media overly focuses on negative stories and distort the facts in an effort to paint the US presence in a negative light. Such complaints miss the idea that the news stories about crime are a reflection of a wider public discontent over the basing arrangement, rather than necessarily a reflection of the views on crime itself. US officials can point to lowered crime rates or additional restrictions on US service members, but the coverage will continue because the fundamental political issues related to the basing arrangement remain.

Returning to questions within our survey data, we clearly expect that respondents who report criminal victimization will be more likely to have negative views of the United States military presence, government, and people. 

\begin{hyp}
	Individuals who report being criminally victimized by a member of the US military will be more likely to express negative views of the American presence/government/people. 
\end{hyp}

As with Chapter \ref{cha:meth}, we expect this to extend to social networks as well. Hearing reports of friends or family members becoming victims of crimes carried out by US military service members should correlate with more negative perceptions of the US military and other actors. 

\begin{hyp}
	Individuals who report criminal victimization within their social network by a member of the US military will be more likely to express negative views of the American presence/government/people. 
\end{hyp}

So far our discussion has all been about how exposure to a negative stimulus (crime) is related to negative perceptions of the group that the perpetrator came from, in this case the US military. Yet the impact of these crimes will not always be uniform. As we have discussed throughout the book, having a context of friendly (and even monetarily advantageous) relations with US servicemembers sets individuals up for the negative impact of crime to be smaller. 

For example, if an individual hears about a crime committed by US service member but has always had positive interactions with members of the US military, they will be less likely to have that incident negatively influence their views of US actors. In contrast, when they have had negative interactions, or no other interactions to form strong prior opinions on, then this new piece of information will be more likely to sway them towards a negative perception. Political psychology tells us that when new information does not match our beliefs about a particular actor, it takes more  (and less ambiguous) of it to alter perceptions. Thus, someone who positively views US servicemembers would likely not be swayed by sporadic reports of crime. In contrast, someone who has had no contact with Americans and then has their first exposure to them happen in the form of crime will have their theory of what Americans are like shaped by that interaction, and it will be difficult to then change that perception through further positive interactions.\cite{Jervis1968} 

Because, as we have shown in Chapter \ref{cha:meth} most interactions betwen US servicemembers and host country civilians tend to be somewhat shallow and positive, we expect that personal contact will have a mitigating effect on crime's negative effect. The same will hold true for financial benefits received from the US military, which are positive by definition. We thus derive the next hypothesis:

\begin{hyp}
	Individuals living near the reported location of a crime committed by a US service member will be less likely to express negative views of the American presence/government/people if they also have had personal contact with or financial benefits from the US military.  %kind of wordy, could rewrite.
\end{hyp}

While existing positive views may be less likely to be changed by hearing of a crime committed by US military personnel, being the victim of a crime is a much more traumatic experience that can lead to emotional responses like anger, as we explain in further detail in \ref{cha:protest}. Yet even in this case there may be variation in the effect that this action can have on the realtionship between individuals and the US military presence. Personally experiencing a crime is certainly more likely to push individuals towards having negative views of the US; it may be that who the particular crime is attributed to depends on preexisting views of the US. 

Attribution theory notes that individuals attribute their own negative actions to situational factors (being in a bad situation), while they attribute the negative actions of others to their disposition (being a bad person).\cite{Jones1987} When it comes to outgroups, those who hold prejudiced or stereotyped views of the outgroup are more likely to attribute their negative behavior to negative dispositions that are related to stereotypes about the group.\cite{Hewstone1985,Heradstveit1996} Thus, in this case, individuals who hold negative stereotypes of Americans and then experience crime are more likely to attribute the negative experience to Americans as a whole being the type of people who engage in crime. In contrast, those who do not hold these types of stereotypes will be more likely to attribute even a personal experience of crime to a particular individual, or to circumstances, but this perception will not be generalized to the American military as a whole. As we discussed in Chapter \ref{cha:meth}, personal contact decreases bias and stereotypes, and so we thus expect that in the case of individuals who experience crimes perpetrated by members of the US military, those who have had contact with the US military will be less likely to generalize the negative criminal experience into a negative view of the US military as a whole:

\begin{hyp}
	Individuals who report being criminally victimized by a member of the US military will be less likely to express negative views of the American presence/government/people if they have had personal contact with US service members. 
\end{hyp}

Finally, we argue that the type of crime experienced matters, with those crimes the involve violence being more likely to lead to negative views of the US military as a whole than property-related crimes because...[link with anger stuff from protest chapter][Also might be a good spot to talk about the outcome of the crime, or maybe that's something we mention in the conclusion, since we won't be able to test it, but putting this here so that I don't forget to add it]


With these expectations, we turn to develop the data and models to test our hypotheses.


\section*{Estimating the Effect of Crime and Crime Reporting}

%some introduction here, follow minority RD section to build up a lead in. MF is probably the key person to write this.

\subsection*{Outcome Variable}
%copied wholesale from the minority chapter, could use revisions
As we outline in Chapter \ref{cha:meth}, we have three outcome variables of interest: Individuals' views of the United States military, the United States government, and the United States people. We collapse the survey responses into four categories to estimate our models: 1) Positive, 2) Negative, 3) Neutral, and 4) Don't know/Decline to answer. Our primary focus here is on how crime experiences affects individual attitudes about the US military presence in each country. However, we are also interested in how these relationships compare across different groups, so we run two additional sets of models using views of the US government and US people as outcome variables. Using these additional outcomes will help us determine how general the effects of the key predictor variables are and will help us better understand the contours of attitudes towards the US in general. 

\subsection*{Predictor Variables}

We are interested in how exposure to crime by the media, personally, or through social networks affects people's views of the three actors we have used throughout this book. To do so, we employ the same models that we use in Chapter \ref{cha:meth}, but we add a few relevant controls for crime-related experiences. First, turning to the survey, we ask two important questions of respondents. First, we ask respondents, ``Have you personally been the victim of a crime committed by a member of the US military?'' As with our contact questions, they can answer yes, no, or don't know/decline to answer. We ask them about their social network as well when we ask, ``Do you know someone who has been the victim of a crime committed by a member of the US military?'' Their set of responses to these questions are identical to the previous question.

We expect responses to these questions to have a similar effect as responses about contact and economic benefits. People who respond yes to experiencing or knowing about crime are more likely to have informed opinions on the US military than those that say they don't know or decline to answer. Having that experience or hearing stories of other people's experiences will correlate with people having more and stronger thoughts about the military. Furthermore, we expect that people who have exposure to crime are more likely to report negative perceptions of the US military. 

Beyond our survey, we have collected additional data to help refine our expectations regarding reporting and views of the military. We are interested in criminal offenses committed in peacetime deployments by United States service members. Using a database of news reports written in English, we used a series of search terms to construct a list of crimes officially reported by various national and global news agencies from 1988-2020.\footnote{Primarily, we looked for strings of words that were proximate to each other, including United States (and its abbreviation), soldier or service member, and crime (or various kinds of disaggregated crime types). Then we examined every article that included these terms and examined if an article was a case of a service member committing a crime during a peacetime deployment or a false-positive. We ignored the false positives and coded information about the date, city, year, type of crime, service branch of the offender, source of the story, and any other important information about the event.} Importantly, we know that this list is not a comprehensive list of all crimes committed by military personnel nor all crimes reported in local or national news media.\footnote{The English speaking papers included \textit{Japan Times}, \textit{Stars \& Stripes}, \textit{Malaysia Global News}, \textit{Xinhua}, wires like \textit{Agence France Presse}, and other national news sites.} Notably, as Efrat points out in recent work, there are hundreds of thousands of crimes occurring over the first few decades of the Cold War for just army deployments, so our list of 32 reported crimes is woefully incomplete as a comprehensive listing.\cite{efrat2021a} However, what the list does represent is a list of cases that have bubbled up from that latent background of ongoing crimes and results in substantial news coverage. 

We have geocoded the city center where any crime has occurred and use that in conjunction with respondent's self-reported city locations to see how far away they are from a major crime event. We expect that those closer to such news events are more likely to report negative perceptions of the military (and the other actors by proxy). This variable inclusion is mostly a correlative exercise to examine if a relationship exists at all. We expect a bidirectional relationship in that negative events correlate with negative views and negative views make it more likely that the news will cover significant events in those areas. In other words, preexisting negative perceptions make news coverage of negative events more likely, which in turn drive even more negative views.



\section*{The Effect of Contact and Economic Reliance}

%%interaction between contact & crime and economic benefits & crime

\section*{Conclusions}

%reframethis based on our results
The results of our models may not be too surprising. Writing in 2006, Gillem points out that the media focus on crime and other negative externalities seems to have less long-term effect on people's attitudes than the physical presence of the base itself.\cite{Gillem2007} The presence and massive land use of the base are also fundamental factors of opposition for people in Okinawa than the litany of social harms caused by the base. However, as examined throughout this book, perceptions of the US military are multidimensional and bring to bear a variety of inputs. Local populations incorporate crimes committed by American military personnel into their wider view of the US military presence, its wisdom, and their other personal experiences with American personnel. 

While we do not have data on public opinion related to how cases are handled in the aftermath of a crime, anecdotal accounts indicate a situation in which there is not much the US military can do to lessen public outcry, especially in the presence of significant media coverage. Many instances of harsh justice against US service members who committed a crime did not positively impact public opinion. One former ambassador to a country that hosts US military personnel stated it directly, ``no matter how strictly we deal with them, it will never be enough.'' \cite{amb20180713} In that particular instance, the US service member received a 50-year sentence after being tried in the United States. Panamanian journalists went to the United States to cover the case, and a JAG officer was brought to Panama to explain the case to the local population on television. In the end, ``it doesn't matter when it comes to public opinion,'' he said. While this is one example, others from around the world indicate a similar story, with high-profile and lengthy sentences for US service members not being met with a public response that is less negative.

Thus, two main issues likely help determine the effect of crimes on overall public perception. First, it appears critical to prevent crimes from occurring in the first place through monitoring and policing of service members deployed abroad. Of course, crimes against individuals and their social networks are likely to have a highly negative effect on the views of those directly affected, as shown here. Still, it also shows the degree to which more widespread positive interactions can create a buffer to this type of sharp public response among the wider society. In the absence of personal experience to the contrary, individuals are far more likely to believe that a criminal act by a US service member is representative of the whole presence rather than an anomaly. While isolation of US personnel from the community would appeal to cautious base commanders, in the event of a crime against a member of the local population, there is little they could do to limit the fallout without a robust set of social connections to the community. In the presence of dense social networks between the US presence and the community, locals are more likely to attribute the criminal behavior to individuals responsible rather than the wider presence.

Second, it shows the degree to which crimes committed against local civilians can create focal points for discontent in places that are already disapproving of the American military presence. News coverage is more likely to capitalize on crimes that paint American personnel in a negative light when the local community already perceives the presence to be a malign influence. Thus, crimes are likely to have the most dramatic effects on public perception where there is a reservoir of discontent. That has certainly been the case in South Korea and Okinawa in particular moments of heightened coverage over crimes committed against the local population, which occur in the backdrop of larger conversations in these host societies about the wisdom of hosting American forces.

Addressing these larger concerns about the general relationship between the United States military presence and the host community, about the size of the US presence, or about land use or environmental destruction can likely play a large role in determining the effect of crimes on overall public perception. Without addressing these more fundamental issues, crimes gainst local civilians can see anti-basing sentiment grow beyond individual groups and create a wider, cross-cutting movement against the US presence, which will be discussed in the next chapter.
\begin{comment}





%\section*{Introduction}
\vspace*{-0.5cm}
\rule{\linewidth}{0.10pt} \\[-1cm]
{\footnotesize\paragraph{Summary:}  The previous chapters have examined positive contact, economic benefits, and the role of minority groups in basing and perceptions. We take a deeper look at a negative interaction between service members and host-state civilians by looking at how national media reports, personal victimization, and criminal victimization within a respondent's network affect respondents' views of the US military and other actors. Those that live near a well-publicized criminal event, experience criminal victimization, or report victimization in their social network are both more likely to have an informed view of the US military presence and more likely to have a negative perception of crime. However, these effects are (compare to contact/economic benefits). Given this comparison, in the competition for consent, it seems the good outweighs the bad absent media reporting.} 
\\[-0.5cm] 
\rule{\linewidth}{0.10pt}

\vspace*{0.5cm}

%anecdote lead needed, something about general perceptions or people's views being tied to something. Carla or Andy will have the best idea as to what fits since they were in all four sets of interviews.


During the early morning of a Wednesday on April 17th in 2019, local police arrested an airman from Kadena Air Base in Okinawa for suspicion of drunk driving. The 26-year old ran into a motorcycle and injured two civilians; a police-issued breathalyzer calculated his blood-alcohol level ``was nearly twice Japan's legal limit of 0.03 percent'' \cite{Burke2019}. Although, the act of driving under the influence and injuring host-state civilians is certainly something that would make the local papers, it was not the only case of misconduct in Okinawa that April. An additional drunk driving incident happened a week later and, more staggeringly, a Hospital Corpsman 3rd Class in the Navy murdered a 44-year old Okinawan woman by stabbing her in the neck and then committed suicide by surgically cutting his groin \cite{Simkins2109}. The clustering of events provided unneeded further evidence for Okinawa's anti-basing movement that the presence of US service members; a history of heinous and routine crimes by US personnel, had rocked prefecture for decades before the events in 2019. Moreover, Okinawa has a long history of US military scandals that rocked the region's conscience and the country.

As reports of service members committing mundane to heinous infractions in the regions that host them, some subset of these behaviors will gain local, regional, national, or international attention. As such stories trickle back to the United States, it is not clear if the United States sees these as isolated events, another cost of providing security globally, or a pattern of behavior that the military has been incapable of curtailing. Even overseas, it is not clear if the behavior is systemic, sensationalized or somewhere in between. It is feasible that the crime rate among service members may be at or below the crime rate in the populations they base in; the relatively lower rate of criminality of the military relative to other host population is something US military leaders have used as a defense \cite{Gillem2007}. Still, the very fact that it is a foreign military member committing crimes against a host population makes each incident more noteworthy and more likely to garner media and local attention. In our research, we have found some evidence that the presence of military deployments correlates with increases in property-related crimes, but less evidence on other changes in the rate of crime within a society \cite{Allen2011}. In understanding how military member crimes play a role in this book, the incidence rate is less of a concern than the perception those incidents create. Of course, the more crimes that occur, the more opportunity for negative views generated by those crimes to proliferate. %cite is wrong, fix when you have the internet. Create an appendix file to start offloading tables

In the previous chapters of the book, we have discussed the role that contact has on producing both positive and negative perceptions of the United States. However, most of our discussion has focused on the positive aspects of the role contact in breaking down stereotypes and encourage familiarity. We have discussed both contact creating positive interactions and how economic benefits can reinforce positive support for the US military presence. While the chapter on minority presence has indications of fewer positive trends for the United States, this chapters delves into a directly negative effect: crime. In this chapter, we develop our expectations as to both where we see service member crime events make national media venues as well as the role of crime in affecting perceptions among the 42,000 people we surveyed. 

In understanding competition for host-state consent to base, crime plays an insidious role for basing powers that may be hard to curtail. A country that deploys tens of thousands or hundreds of thousands of people into other nations will have some criminal infractions within its deployed populations. It would be difficult for any military or institution to perfectly select members of its community that have no capacity or willingness for committing a crime. Doing so would also rely on outdated notions of why crime even occurs. There are a complex set of determinants of what causes crime and no social institution has successfully eliminated all crime within its membership. In considering how to reduce or limit criminal offending within its ranks, the military can certainly invest in other structural, incentive, and training changes. A big picture decision for a basing power is whether they decrease the opportunities for crime by isolating their own population or increase the amount of monitoring of, incentives against, and punishments for criminal behavior.  The United States has increasingly chosen the first option in the last two decades. By choosing isolation, the United States is attempting to internalize criminal behavior to its own population and decreases the likelihood of a story causing basing ramifications.

Additionally, building self-contained military cities requires mostly large fixed costs (land, buildings, etc.), while the option to monitor and police behavior requires long-term, enduring costs to maintain. Isolation is cheaper and easier than integration and monitoring. While the Navy has a history of maintaining an insular presence, the rest of the military started following a similar path by the mid-2000s \cite{Gillem2007}. The trade-off with this strategy is that the non-personnel related costs from deployment will remain and the positive contact and economic flows from personnel will evaporate as a result of the American city with a fortress strategy. %return to the idea of how basing will remain unpopular, but without contact, there is no counter-narrative.

To understand our data, we first explore our current understanding of the relationship between US military deployments and crime within host states. We then use this to build our expectations of and how crime affects the perceptions of host-state civilians. Next, we use these hypotheses to return to our model of perceptions to estimate how crime influences people's positive and negative views of the United States. Finally, we conclude by evaluating how crime influences the domain of competitive consent.

\section*{Research and Expectations}

We begin our theoretical development of the impact of crime on perceptions by host-state civilians by reviewing the existing body of work on service member crime. The presence of crime among foreign-deployed military communities might be one of the better-studied aspects of overseas deployments. The documentation of the crimes through both local police investigation and internal military investigations allow for a close examination of the record than other aspects of overseas deployments. These records, when public, also allow for ease of media reporting and scrutiny that other social harms may be harder to track and document. Beyond the nature of the reporting of crime, case study research on particular countries or deployment communities has been key in how basing changes the social fabric of established towns and cities.

The foundational work that systematized understanding and conceptualizing crime in the armed forces is Clifton D. Bryant's \textit{Khaki-collar Crime} \citeyear{Bryant1979}. This sociological and criminological examination argues that understanding criminal deviance\footnote{To be clear, deviance in the sociological and criminological research is not a normative judgment, but is instead a label used to those that violate social norms. Bryant goes to lengths to argue that crime is a socially constructed event. Social scientists miss studying several kinds of behavior that would be criminal by most accounts except that it has legal sanction. Going beyond this observation, he argues that crime, as we conceptualize it, is a natural partner to the social institutions that we construct.} requires additional assumptions than civilian crime as it takes place in a context that has additional, layered social institutions and responsibilities within it. In discussing these added layers, Bryant adds several additional causal factors into why military crime occurs including the nature of the military population (typically young, male, lower- or lower-middle-class, and low skilled when he is writing in the 1970s), the stress of military existence, the conditioning aimed at reducing individual choices, over-regulation through a dearth of formal rules, informal norms, responsibilities, and resources making infractions inevitable, military culture, the totality of the institution in service member life, normalization of abuse of subordinates throughout training, a culture of violence that cultivates violent responses, combat offering opportunities to cloak violence against superiors, military culture, over-bureaucratization, military socialization, official tolerance for some criminal infractions, and subcultures encouraging deviance from official military training. In getting a handle on the multifaceted ways in which the military creates space for crime to occur and how service members can commit a crime, Bryant categorizes crime into three types: crime against property, crime against people, and crimes against performance. The first two are more well-known, while the final category deals with crimes that hinder one's own duties or duties of others.

For those three types, there are three supra-categories where crime occurs. Crimes can occur within one's service (intraoccupational), against those outside of one's service (extraoccupational), or against those that are in another service (interoccupational). Extraoccupational crimes warrant subdivision across three different areas as the unique nature of military service affords three kinds of interactions in which crime can occur: against American civilians, against foreign-friendly civilians, and against enemy civilians.  The types, details, examples, and causes of these crimes are vast within his work. Since we are primarily interested in how crime affects the perceptions of host-state civilians, the primary category this chapter examines is extraoccupational crimes against foreign-friendly civilians. This focus has a few caveats. First, there is quite a bit of overlap between the factors related to crimes against American civilians and foreign civilians in Bryant's analysis of many crimes. Still, there are additional opportunities in interactions with foreign civilians. Second, Bryant lists alcohol and drug use as an intraoccupational crime due to the use of the drug, but we certainly see the purchasing and sale of illegal substances, which is true of general participation in the black market, as an extraoccupational crime in addition to its intraoccupational nature.\footnote{Notably, much of the research, surveys, and analysis of offending is either on intraoccupational offenses or crime by veterans. There is robust research on substance abuse \cite{bray2010}, sexual harassment and assault \cite{bostock2007,stander2016}, intimate partner violence \cite{sparrow2020}, and offending among veterans \cite{moorhead2021}.}

The third caveat in understanding his work for our purposes is that much of his study of crimes against foreign civilians is in the context of a US military deployed during wartime. Our study is exclusive to deployments in non-war zones. This distinction creates a qualitative difference in interpreting the opportunity, meaning, and punishment of crimes against civilians. As Bryant illustrates overly well, service members can hide their crimes during a conflict. Primarily drawing upon examples from the Vietnam conflict, there are several examples where service members victimize South Vietnamese civilians in brutal ways and claim before, during, or after the fact that the person was a member of the Vietcong. Many of these acts would be war crimes even if the victim were enemy combatants, but the designation as an enemy excuses the behavior in the perpetrator's account. Beyond the cover conflict gives to criminal actions, the salience of the conflict makes prosecution of crimes less likely than if the same crime happened during a peacetime deployment. These dual factors make the impact, reporting, and punishment of crime fundamentally different than in the contexts we examine within this chapter.  

Several important works in this area help us understand the prevalence and role of service member extraoccupational crime globally. \citeasnoun{bucher2011} builds upon these foundations to argue that ``General Strain Theory'' is a useful theory in understanding military offending. A combination of straining factors (such as the inability to achieve personal goals, negative experiences, inability to meet family needs, etc.) can lead individuals to pursue crime when they lack access to adequate alternative coping mechanisms. The study finds a relationship between measures of strain and drug use, violence, and theft.

\possessivecite{Moon1997} book is a close examination of how both the United States and South Korea institutionalize, manage, and react to the culture of sex work and prostitution within South Korea. Historically, the United States has maintained a fine line between seeing prostitution as an illegal activity in most jurisdictions it operates in, while also tolerating or encouraging its promotion as a seemingly necessary outlet for deployed personnel. This dual-track of both officially being against prostitution while, at times, all but condoning the existence of prostitution has led to inconsistencies in how the military approaches the issue. The US military in South Korea has gone through several periods of tacit encouragement, and complete rejection by the US military \cite{Baker2004}. \citeasnoun{Enloe2000} argues that the US military historically treats prostitution as both a resource (for personnel) and as a consistent threat due to the spread of venereal disease---these views mimic Great Britain's approach to similar issues within its empire \cite{Gillem2007}. In South Korea, towns go through a period where illegal sex work flourishes despite local and military restrictions. Eventually, when the US military becomes more interested in reshaping its domestic image, it engages in a cleanup campaign with the South Korean military to curtail prostitution in the camptowns surrounding the nearby bases.

Having a tolerated black market for prostitution leads to a whole proliferation of additional crimes. Fundamentally, without legal sanctification of a business that flourishes, the state has abdicated the responsibility for enforcing property rights. By the state criminalizing particular aspects of the business, associated crimes become easier to commit as the victims are afraid to go to the police. For example, if a member of the armed forces decides to assault a sex worker while being a client of their services, the sex worker is unlikely to report the crime to authorities. The victim's occupation makes it such that they may face legal sanction without any recourse for their victimization. Beyond this, private actors need to enforce property rights in place of the state without legal protection of property rights. This leads to increases in violence to protect people, goods, and enforce contracts. For example, if a client refuses to pay a sex worker, the sex worker either loses out on that income or turns to another person to threaten or commit violence to recoup their lost revenue. This issue is prevalent in nearly any market where the state abdicates the role of protecting property rights, whether it is sex work, drugs, farms in New World colonies, or just heavily taxed goods \cite{Resignato2000,Fleenor2003,Reynolds2010,Vandusky2011}. 

In the specific example of South Korea, the black market for sex work created several negative externalities. \citeasnoun{Moon1997} details how the social stigma around sex work often made it impossible for women to reintegrate into Korean society and, for many, the end goal was to marry an American soldier in the hopes of leaving Korea altogether. While marriage and leaving South Korea were prospects, the danger of violence and murder was also present. The occupation was not lucrative for many women either as bar owners that controlled the lives of many of these workers would keep them in a debt bondage system that made escape financially impossible \cite{Moon1997,Gillem2007}. The lack of state regulation allows social norms to regulate the market in Korea partially. Sex workers that had predominately white clients became the targets for abuse, alienation, and murder if they were found working for black soldiers as well \cite{Moon1997}. \citeasnoun{Enloe2000} mentions a similar situation arising in Vietnam, and \citeasnoun{Gillem2007} discusses how contemporary South Korean bar owners seek only to serve Americans in camptowns and dissuade local Koreans from patronizing their establishments. Moon demonstrates that even during the Camptown cleanup campaigns in the 1970s, there was an effort to reduce racial discrimination among sex workers and enforcement and punishment of the anti-discrimination regulations fell on the women. 

Given the proliferation of other grey and black market services that arise around a tolerated black market, it follows that permissive conditions for one type of crime will spawn additional crime networks to support the tolerated market and provide cover for other kinds of crime to proliferate. This building of networks is not unique to prostitution, but we see similar effects when deployments significant issues with drug use leading to a contagion of other crimes, including drug trafficking, sexual assault, and robbery. Drugs have been an endemic problem within the service in different periods, though the military has been successful at combating drug use in some high profile deployments, though marijuana and steroid use remained high in the 2010s \cite{Nelson1987,ballweg1991,Baker2004,bucher2012}. 

In many US deployments, the US brings with it millions of dollars of capital-intensive goods, supplies, food, and other resources that are ripe for service members to pilfer and sell on secondary markets \cite{Bryant1979,Nelson1987}. Not only do service members act as a demand in the marketplace for illicit goods, but by having access to robust stores of goods, they can act as suppliers for local demand. US deployments during the Cold War and post-Cold War eras have several cases where millions of dollars of equipment or, more concern, thousands of firearms have disappeared from the military and found their way onto the black market. (citations)

%In an early project, we found that US military deployments correlate with increases in aggregate property crime rates \cite{allenandflynn2013}.

Trying to quantify the number of crime service members do is a difficult task. Notably, base commanders have historically pointed out that service members commit crime at lower rates than the general population of the country \cite{Gillem2007}. Early research within our team tried to assess whether the presence of troops correlates with higher levels of criminal activity within a country; that is, does the demand for illicit goods create a proliferation of other crimes within society. Our research found some connection with property-related crimes but was generally limited in its overall assessment \cite{allenandflynn2013}. Much like we pointed out about other research in Chapter \ref{cha:meth}, this is another issue of ecological inference issues that makes inferences about individual behaviors difficult. It is possible that the military deployments are not causally related here, but other factors create the relationship we found as there are threats to inferences about individual behavior and some questions about the spuriousness of the relationship \cite{King2004}. Recent work by \citeasnoun{efrat2021a} uses primary source documents from the US army to catalog 361,487 offenses from 1954-1970 with 74\% of those crimes taking place in NATO countries. As he notes, these crimes are likely under-reported as crimes statistics are already an under-reported figure for a variety of reasons; the perpetrators being US military members may increase that under-reporting further since victims may fear retribution or believe that reporting the crime will not result in meaningful justice \cite{allenandflynn2013}. In a separate study, \citeasnoun{efrat2021b} uses judicial records to find that most crimes do get referred to the military for prosecution. 





\subsection*{Theorizing crime reporting and perceptions}

While we have reviewed the major understandings of crime within this chapter, our goal is to look at the effects of those crimes. Setting up our argument in this chapter is a short affair as much of our discussion extends from chapters \ref{cha:theory} and \ref{cha:meth} First, we are interested in whether highly reported crimes are more likely to influence the views of our respondents. These are the events that are most likely to galvanize public backlash and understanding whether the crimes media outlets decide to portray have an independent effect on individuals' perceptions. Using a new data set on widely reported crimes by service members, we expect that people who are nearer to places with reported crimes are more likely to have negative views of the US actors we ask about.

\begin{hyp}
	Individuals living near a location where a news organization reported that a US service member committed a crime at will be more likely to express negative views of the American presence/government/people.  %kind of wordy, could rewrite.
\end{hyp}

Naturally, there might be a presumption that countries with more service member crimes are more likely to have overall more negative assessments of the US military, but this does not seem to bear out in the data. \citeasnoun[p.48]{Gillem2007} reports: 

\begin{quote}
	More telling from a socio-spatial perspective is the variation in crime rates within the military community. In the same three-year period that South Korea experienced 1,246 criminal acts by US soldiers, in Okinawa, Nawa, Japan, soldiers committed 198 criminal acts. The annualized per capita difference is quite instructive. In South Korea, there were 11.2 crimes per 1,000 soldiers. In Okinawa, there were 2.4 crimes per 1,000 US soldiers. Why is the disparity so striking? I suggest that it is largely the result of policies related to housing and land use.
\end{quote}

We expect to see more reports of Japanese crime rates than Korean crime rates despite the disparity in offenses between the two countries. Given a higher reporting for a smaller rate of occurrence, those near those crimes are more likely to report negative views.

Returning to questions within our survey data, we have a clear expectation that respondents that report criminal victimization personally will be more likely to have negative views of the United States military presence, government, and people. 

\begin{hyp}
	Individuals who report being criminally victimized by a member of the US military will be more likely to express negative views of the American presence/government/people. 
\end{hyp}

As with Chapter \ref{cha:meth}, we expect this to extend to social networks as well. Hearing reports of friends or family members becoming victims of crimes carried out by US military service members should correlate with more negative perceptions of the US military and other actors. 

\begin{hyp}
	Individuals who report criminal victimization within their social network by a member of the US military will be more likely to express negative views of the American presence/government/people. 
\end{hyp}

With these expectations, we turn to developing the data and models to test our hypotheses.



\section*{Estimating the Effect of Crime and Crime Reporting}

%some introduction here, follow minority RD section to build up a lead in. MF is probably the key person to write this.

\subsection*{Outcome Variable}
%copied wholesale from the minority chapter, could use revisions
As we outline in Chapter \ref{cha:meth}, we have three outcome variables of interest: Individuals' views of the United States military, the United States government, and the United States people. We collapse the survey responses down into four categories to estimate our models: 1) Positive, 2) Negative, 3) Neutral, and 4) Don't know/Decline to answer. Our primary focus here is on how crime experiences affects individual attitudes about the US military presence  in each country. However, we are also interested in how these relationships compare across different groups, and so we run two additional sets of models using views of the US government and US people as outcome variables. Using these additional outcomes will help us to determine how general the effects of the key predictor variables are and will help us to better understand the contours of attitudes towards the US in general. 

\subsection*{Predictor Variables}

We are interested in the effect of exposure to crime by the media, personally, or through social networks affects people's views of the three actors we have used throughout this book. To do so, we employ the same models that we use in Chapter \ref{cha:meth}, but we add a few relevant controls for crime-related experiences. First, turning to the survey, we ask two important questions of respondents. First, we ask respondents, ``Have you personally been the victim of a crime committed by a member of the US military?'' As with our contact questions, they can answer yes, no, or don't know/decline to answer. We ask them about their social network as well when we ask, ``Do you know someone who has been the victim of a crime committed by a member of the US military?'' Their set of responses to these questions are identical to the previous question.

We expect responses to these questions to have a similar effect that responses to the questions about contact and economic benefits did. People who respond yes to experiencing or knowing about crime are more likely to have informed opinions on the US military than those that say they don't know or decline to answer. Having that experience or hearing stories of other people's experiences will correlate with people having more and stronger thoughts about the military. It is feasible that people who say no are more likely to have informed opinions and are willing to offer a response to that question, think about the military, and maybe more willing to answer our previous questions about their views. Additionally, we expect that people who have exposure to crime are more likely to report negative perceptions of the US military. 

Beyond our survey, we have collected some additional data to help refine our expectations as to reporting and views of the military. We are interested in criminal offenses committed in peacetime deployments by United States service members. Using a database of news reports written in English, we used a series of search terms to construct a list of crimes officially reported by various national and global news agencies from 1988-2020.\footnote{Primarily, we looked for strings of words that were proximate to each other, including United States (and its abbreviation), soldier or service member, and crime (or various kinds of disaggregated crime types). Then we examined every article that included these terms and examined if an article was a case of a service member committing a crime during a peacetime deployment or a false-positive. We ignored the false positives and proceeded to code information about the date, city, year, type of crime, service branch of the offender, source of the story, and any other important information about the event.} Importantly, we know that this list is not a comprehensive list of all crimes committed by military personnel nor all crimes reported in local or national news media.\footnote{The English speaking papers included \textit{Japan Times}, \textit{Stars \& Stripes}, \textit{Malaysia Global News}, \textit{Xinhua}, wires like \textit{Agence France Presse}, and other national news sites.} Notably, as \citeasnoun{efrat2021a} points out, there are hundreds of thousands of crimes occurring over the first few decades of the Cold War for just army deployments, so our list of 32 reported crimes is woefully incomplete as a comprehensive listing. However, what the list does represent is a list of cases that have bubbled up from that latent background of ongoing crimes and results in substantial news coverage. 

We have geocoded the city center where any crime has occurred and use that in conjunction with respondent's self-reported city locations to see how far away they are from a major crime event. We expect that those closer to such news events are more likely to report negative perceptions of the military (and the other actors by proxy). This variable inclusion is mostly a correlative exercise to examine if a relationship exists at all. We expect a bidirectional relationship in that negative events correlate with negative views and negative views, making it more likely that the news will cover significant events in those areas. 




\section*{The Effect of Contact and Economic Reliance}

\section*{Conclusions}

%reframethis based on our results
The results of our models may not be too surprising. Writing in 2006, \citeasnoun{Gillem2007} points out that the media focus on crime and other negative externalities seems to have less long term effect on people's attitudes than the physical presence of the base itself. The presence and massive land use of the base are more fundamental factors of opposition for people in Okinawa than the litany of social harms caused by the base.

\end{comment}
