\section*{Research and Expectations}

We begin our theoretical development of the impact of crime on perceptions by host-state civilians by reviewing the existing body of work on service member crime. The presence of crime among foreign-deployed military communities might be one of the better-studied aspects of overseas deployments. The documentation of the crimes through both local police investigation and internal military investigations allow for a close examination of the record than other aspects of overseas deployments. These records, when public, also allow for ease of media reporting and scrutiny that other social harms may be harder to track and document. Beyond the nature of the reporting of crime, case study research on particular countries or deployment communities has been key in how basing changes the social fabric of established towns and cities.

The foundational work that systematized understanding and conceptualizing crime in the armed forces is Clifton D. Bryant's \textit{Khaki-collar Crime} \citeyear{Bryant1979}. This sociological and criminological examination argues that understanding criminal deviance\footnote{To be clear, deviance in the sociological and criminological research is not a normative judgment, but is instead a label used to those that violate social norms. Bryant goes to lengths to argue that crime is a socially constructed event. Social scientists miss studying several kinds of behavior that would be criminal by most accounts except that it has legal sanction. Going beyond this observation, he argues that crime, as we conceptualize it, is a natural partner to the social institutions that we construct.} requires additional assumptions than civilian crime as it takes place in a context that has additional, layered social institutions and responsibilities within it. In discussing these added layers, Bryant adds several additional causal factors into why military crime occurs including the nature of the military population (typically young, male, lower- or lower-middle-class, and low skilled when he is writing in the 1970s), the stress of military existence, the conditioning aimed at reducing individual choices, over-regulation through a dearth of formal rules, informal norms, responsibilities, and resources making infractions inevitable, military culture, the totality of the institution in service member life, normalization of abuse of subordinates throughout training, a culture of violence that cultivates violent responses, combat offering opportunities to cloak violence against superiors, military culture, over-bureaucratization, military socialization, official tolerance for some criminal infractions, and subcultures encouraging deviance from official military training. In getting a handle on the multifaceted ways in which the military creates space for crime to occur and how service members can commit a crime, Bryant categorizes crime into three types: crime against property, crime against people, and crimes against performance. The first two are more well-known, while the final category deals with crimes that hinder one's own duties or duties of others.

For those three types, there are three supra-categories where crime occurs. Crimes can occur within one's service (intraoccupational), against those outside of one's service (extraoccupational), or against those that are in another service (interoccupational). Extraoccupational crimes warrant subdivision across three different areas as the unique nature of military service affords three kinds of interactions in which crime can occur: against American civilians, against foreign-friendly civilians, and against enemy civilians.  The types, details, examples, and causes of these crimes are vast within his work. Since we are primarily interested in how crime affects the perceptions of host-state civilians, the primary category this chapter examines is extraoccupational crimes against foreign-friendly civilians. This focus has a few caveats. First, there is quite a bit of overlap between the factors related to crimes against American civilians and foreign civilians in Bryant's analysis of many crimes. Still, there are additional opportunities in interactions with foreign civilians. Second, Bryant lists alcohol and drug use as an intraoccupational crime due to the use of the drug, but we certainly see the purchasing and sale of illegal substances, which is true of general participation in the black market, as an extraoccupational crime in addition to its intraoccupational nature.\footnote{Notably, much of the research, surveys, and analysis of offending is either on intraoccupational offenses or crime by veterans. There is robust research on substance abuse \cite{bray2010}, sexual harassment and assault \cite{bostock2007,stander2016}, intimate partner violence \cite{sparrow2020}, and offending among veterans \cite{moorhead2021}.}

The third caveat in understanding his work for our purposes is that much of his study of crimes against foreign civilians is in the context of a US military deployed during wartime. Our study is exclusive to deployments in non-war zones. This distinction creates a qualitative difference in interpreting the opportunity, meaning, and punishment of crimes against civilians. As Bryant illustrates overly well, service members can hide their crimes during a conflict. Primarily drawing upon examples from the Vietnam conflict, there are several examples where service members victimize South Vietnamese civilians in brutal ways and claim before, during, or after the fact that the person was a member of the Vietcong. Many of these acts would be war crimes even if the victim were enemy combatants, but the designation as an enemy excuses the behavior in the perpetrator's account. Beyond the cover conflict gives to criminal actions, the salience of the conflict makes prosecution of crimes less likely than if the same crime happened during a peacetime deployment. These dual factors make the impact, reporting, and punishment of crime fundamentally different than in the contexts we examine within this chapter.  

Several important works in this area help us understand the prevalence and role of service member extraoccupational crime globally. \citeasnoun{bucher2011} builds upon these foundations to argue that ``General Strain Theory'' is a useful theory in understanding military offending. A combination of straining factors (such as the inability to achieve personal goals, negative experiences, inability to meet family needs, etc.) can lead individuals to pursue crime when they lack access to adequate alternative coping mechanisms. The study finds a relationship between measures of strain and drug use, violence, and theft.

\possessivecite{Moon1997} book is a close examination of how both the United States and South Korea institutionalize, manage, and react to the culture of sex work and prostitution within South Korea. Historically, the United States has maintained a fine line between seeing prostitution as an illegal activity in most jurisdictions it operates in, while also tolerating or encouraging its promotion as a seemingly necessary outlet for deployed personnel. This dual-track of both officially being against prostitution while, at times, all but condoning the existence of prostitution has led to inconsistencies in how the military approaches the issue. The US military in South Korea has gone through several periods of tacit encouragement, and complete rejection by the US military \cite{Baker2004}. \citeasnoun{Enloe2000} argues that the US military historically treats prostitution as both a resource (for personnel) and as a consistent threat due to the spread of venereal disease---these views mimic Great Britain's approach to similar issues within its empire \cite{Gillem2007}. In South Korea, towns go through a period where illegal sex work flourishes despite local and military restrictions. Eventually, when the US military becomes more interested in reshaping its domestic image, it engages in a cleanup campaign with the South Korean military to curtail prostitution in the camptowns surrounding the nearby bases.

Having a tolerated black market for prostitution leads to a whole proliferation of additional crimes. Fundamentally, without legal sanctification of a business that flourishes, the state has abdicated the responsibility for enforcing property rights. By the state criminalizing particular aspects of the business, associated crimes become easier to commit as the victims are afraid to go to the police. For example, if a member of the armed forces decides to assault a sex worker while being a client of their services, the sex worker is unlikely to report the crime to authorities. The victim's occupation makes it such that they may face legal sanction without any recourse for their victimization. Beyond this, private actors need to enforce property rights in place of the state without legal protection of property rights. This leads to increases in violence to protect people, goods, and enforce contracts. For example, if a client refuses to pay a sex worker, the sex worker either loses out on that income or turns to another person to threaten or commit violence to recoup their lost revenue. This issue is prevalent in nearly any market where the state abdicates the role of protecting property rights, whether it is sex work, drugs, farms in New World colonies, or just heavily taxed goods \cite{Resignato2000,Fleenor2003,Reynolds2010,Vandusky2011}. 

In the specific example of South Korea, the black market for sex work created several negative externalities. \citeasnoun{Moon1997} details how the social stigma around sex work often made it impossible for women to reintegrate into Korean society and, for many, the end goal was to marry an American soldier in the hopes of leaving Korea altogether. While marriage and leaving South Korea were prospects, the danger of violence and murder was also present. The occupation was not lucrative for many women either as bar owners that controlled the lives of many of these workers would keep them in a debt bondage system that made escape financially impossible \cite{Moon1997,Gillem2007}. The lack of state regulation allows social norms to regulate the market in Korea partially. Sex workers that had predominately white clients became the targets for abuse, alienation, and murder if they were found working for black soldiers as well \cite{Moon1997}. \citeasnoun{Enloe2000} mentions a similar situation arising in Vietnam, and \citeasnoun{Gillem2007} discusses how contemporary South Korean bar owners seek only to serve Americans in camptowns and dissuade local Koreans from patronizing their establishments. Moon demonstrates that even during the Camptown cleanup campaigns in the 1970s, there was an effort to reduce racial discrimination among sex workers and enforcement and punishment of the anti-discrimination regulations fell on the women. 

Given the proliferation of other grey and black market services that arise around a tolerated black market, it follows that permissive conditions for one type of crime will spawn additional crime networks to support the tolerated market and provide cover for other kinds of crime to proliferate. This building of networks is not unique to prostitution, but we see similar effects when deployments significant issues with drug use leading to a contagion of other crimes, including drug trafficking, sexual assault, and robbery. Drugs have been an endemic problem within the service in different periods, though the military has been successful at combating drug use in some high profile deployments, though marijuana and steroid use remained high in the 2010s \cite{Nelson1987,ballweg1991,Baker2004,bucher2012}. 

In many US deployments, the US brings with it millions of dollars of capital-intensive goods, supplies, food, and other resources that are ripe for service members to pilfer and sell on secondary markets \cite{Bryant1979,Nelson1987}. Not only do service members act as a demand in the marketplace for illicit goods, but by having access to robust stores of goods, they can act as suppliers for local demand. US deployments during the Cold War and post-Cold War eras have several cases where millions of dollars of equipment or, more concern, thousands of firearms have disappeared from the military and found their way onto the black market. (citations)

%In an early project, we found that US military deployments correlate with increases in aggregate property crime rates \cite{allenandflynn2013}.

Trying to quantify the number of crime service members do is a difficult task. Notably, base commanders have historically pointed out that service members commit crime at lower rates than the general population of the country \cite{Gillem2007}. Early research within our team tried to assess whether the presence of troops correlates with higher levels of criminal activity within a country; that is, does the demand for illicit goods create a proliferation of other crimes within society. Our research found some connection with property-related crimes but was generally limited in its overall assessment \cite{allenandflynn2013}. Much like we pointed out about other research in Chapter \ref{cha:meth}, this is another issue of ecological inference issues that makes inferences about individual behaviors difficult. It is possible that the military deployments are not causally related here, but other factors create the relationship we found as there are threats to inferences about individual behavior and some questions about the spuriousness of the relationship \cite{King2004}. Recent work by \citeasnoun{efrat2021a} uses primary source documents from the US army to catalog 361,487 offenses from 1954-1970 with 74\% of those crimes taking place in NATO countries. As he notes, these crimes are likely under-reported as crimes statistics are already an under-reported figure for a variety of reasons; the perpetrators being US military members may increase that under-reporting further since victims may fear retribution or believe that reporting the crime will not result in meaningful justice \cite{allenandflynn2013}. In a separate study, \citeasnoun{efrat2021b} uses judicial records to find that most crimes do get referred to the military for prosecution. 





\subsection*{Theorizing crime reporting and perceptions}

While we have reviewed the major understandings of crime within this chapter, our goal is to look at the effects of those crimes. Setting up our argument in this chapter is a short affair as much of our discussion extends from chapters \ref{cha:theory} and \ref{cha:meth} First, we are interested in whether highly reported crimes are more likely to influence the views of our respondents. These are the events that are most likely to galvanize public backlash and understanding whether the crimes media outlets decide to portray have an independent effect on individuals' perceptions. Using a new data set on widely reported crimes by service members, we expect that people who are nearer to places with reported crimes are more likely to have negative views of the US actors we ask about.

\begin{hyp}
	Individuals living near a location where a news organization reported that a US service member committed a crime at will be more likely to express negative views of the American presence/government/people.  %kind of wordy, could rewrite.
\end{hyp}

Naturally, there might be a presumption that countries with more service member crimes are more likely to have overall more negative assessments of the US military, but this does not seem to bear out in the data. \citeasnoun[p.48]{Gillem2007} reports: 

\begin{quote}
	More telling from a socio-spatial perspective is the variation in crime rates within the military community. In the same three-year period that South Korea experienced 1,246 criminal acts by US soldiers, in Okinawa, Nawa, Japan, soldiers committed 198 criminal acts. The annualized per capita difference is quite instructive. In South Korea, there were 11.2 crimes per 1,000 soldiers. In Okinawa, there were 2.4 crimes per 1,000 US soldiers. Why is the disparity so striking? I suggest that it is largely the result of policies related to housing and land use.
\end{quote}

We expect to see more reports of Japanese crime rates than Korean crime rates despite the disparity in offenses between the two countries. Given a higher reporting for a smaller rate of occurrence, those near those crimes are more likely to report negative views.

Returning to questions within our survey data, we have a clear expectation that respondents that report criminal victimization personally will be more likely to have negative views of the United States military presence, government, and people. 

\begin{hyp}
	Individuals who report being criminally victimized by a member of the US military will be more likely to express negative views of the American presence/government/people. 
\end{hyp}

As with Chapter \ref{cha:meth}, we expect this to extend to social networks as well. Hearing reports of friends or family members becoming victims of crimes carried out by US military service members should correlate with more negative perceptions of the US military and other actors. 

\begin{hyp}
	Individuals who report criminal victimization within their social network by a member of the US military will be more likely to express negative views of the American presence/government/people. 
\end{hyp}

With these expectations, we turn to developing the data and models to test our hypotheses.

