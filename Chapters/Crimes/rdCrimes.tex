
\section*{Estimating the Effect of Crime and Crime Reporting}

%some introduction here, follow minority RD section to build up a lead in. MF is probably the key person to write this.

\subsection*{Outcome Variable}
%copied wholesale from the minority chapter, could use revisions
As we outline in Chapter \ref{cha:meth}, we have three outcome variables of interest: Individuals' views of the United States military, the United States government, and the United States people. We collapse the survey responses down into four categories to estimate our models: 1) Positive, 2) Negative, 3) Neutral, and 4) Don't know/Decline to answer. Our primary focus here is on how crime experiences affects individual attitudes about the US military presence  in each country. However, we are also interested in how these relationships compare across different groups, and so we run two additional sets of models using views of the US government and US people as outcome variables. Using these additional outcomes will help us to determine how general the effects of the key predictor variables are and will help us to better understand the contours of attitudes towards the US in general. 

\subsection*{Predictor Variables}

We are interested in the effect of exposure to crime by the media, personally, or through social networks affects people's views of the three actors we have used throughout this book. To do so, we employ the same models that we use in Chapter \ref{cha:meth}, but we add a few relevant controls for crime-related experiences. First, turning to the survey, we ask two important questions of respondents. First, we ask respondents, ``Have you personally been the victim of a crime committed by a member of the US military?'' As with our contact questions, they can answer yes, no, or don't know/decline to answer. We ask them about their social network as well when we ask, ``Do you know someone who has been the victim of a crime committed by a member of the US military?'' Their set of responses to these questions are identical to the previous question.

We expect responses to these questions to have a similar effect that responses to the questions about contact and economic benefits did. People who respond yes to experiencing or knowing about crime are more likely to have informed opinions on the US military than those that say they don't know or decline to answer. Having that experience or hearing stories of other people's experiences will correlate with people having more and stronger thoughts about the military. It is feasible that people who say no are more likely to have informed opinions and are willing to offer a response to that question, think about the military, and maybe more willing to answer our previous questions about their views. Additionally, we expect that people who have exposure to crime are more likely to report negative perceptions of the US military. 

Beyond our survey, we have collected some additional data to help refine our expectations as to reporting and views of the military. We are interested in criminal offenses committed in peacetime deployments by United States service members. Using a database of news reports written in English, we used a series of search terms to construct a list of crimes officially reported by various national and global news agencies from 1988-2020.\footnote{Primarily, we looked for strings of words that were proximate to each other, including United States (and its abbreviation), soldier or service member, and crime (or various kinds of disaggregated crime types). Then we examined every article that included these terms and examined if an article was a case of a service member committing a crime during a peacetime deployment or a false-positive. We ignored the false positives and proceeded to code information about the date, city, year, type of crime, service branch of the offender, source of the story, and any other important information about the event.} Importantly, we know that this list is not a comprehensive list of all crimes committed by military personnel nor all crimes reported in local or national news media.\footnote{The English speaking papers included \textit{Japan Times}, \textit{Stars \& Stripes}, \textit{Malaysia Global News}, \textit{Xinhua}, wires like \textit{Agence France Presse}, and other national news sites.} Notably, as \citeasnoun{efrat2021a} points out, there are hundreds of thousands of crimes occurring over the first few decades of the Cold War for just army deployments, so our list of 32 reported crimes is woefully incomplete as a comprehensive listing. However, what the list does represent is a list of cases that have bubbled up from that latent background of ongoing crimes and results in substantial news coverage. 

We have geocoded the city center where any crime has occurred and use that in conjunction with respondent's self-reported city locations to see how far away they are from a major crime event. We expect that those closer to such news events are more likely to report negative perceptions of the military (and the other actors by proxy). This variable inclusion is mostly a correlative exercise to examine if a relationship exists at all. We expect a bidirectional relationship in that negative events correlate with negative views and negative views, making it more likely that the news will cover significant events in those areas. 



