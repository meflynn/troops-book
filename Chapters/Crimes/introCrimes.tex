


%\section*{Introduction}
\vspace*{-0.5cm}
\rule{\linewidth}{0.10pt} \\[-1cm]
{\footnotesize\paragraph{Summary:}  The previous chapters have examined positive contact, economic benefits, and the role of minority groups in basing and perceptions. We take a deeper look at a negative interaction between service members and host-state civilians by looking at how national media reports, personal victimization, and criminal victimization within a respondent's network affect respondents' views of the US military and other actors. Those that live near a well-publicized criminal event, experience criminal victimization, or report victimization in their social network are both more likely to have an informed view of the US military presence and more likely to have a negative perception of crime. However, these effects are (compare to contact/economic benefits). Given this comparison, in the competition for consent, it seems the good outweighs the bad absent media reporting.} 
\\[-0.5cm] 
\rule{\linewidth}{0.10pt}

\vspace*{0.5cm}

%anecdote lead needed, something about general perceptions or people's views being tied to something. Carla or Andy will have the best idea as to what fits since they were in all four sets of interviews.


During the early morning of a Wednesday on April 17th in 2019, local police arrested an airman from Kadena Air Base in Okinawa for suspicion of drunk driving. The 26-year old ran into a motorcycle and injured two civilians; a police-issued breathalyzer calculated his blood-alcohol level ``was nearly twice Japan's legal limit of 0.03 percent'' \cite{Burke2019}. Although, the act of driving under the influence and injuring host-state civilians is certainly something that would make the local papers, it was not the only case of misconduct in Okinawa that April. An additional drunk driving incident happened a week later and, more staggeringly, a Hospital Corpsman 3rd Class in the Navy murdered a 44-year old Okinawan woman by stabbing her in the neck and then committed suicide by surgically cutting his groin \cite{Simkins2109}. The clustering of events provided unneeded further evidence for Okinawa's anti-basing movement that the presence of US service members; a history of heinous and routine crimes by US personnel, had rocked prefecture for decades before the events in 2019. Moreover, Okinawa has a long history of US military scandals that rocked the region's conscience and the country.

As reports of service members committing mundane to heinous infractions in the regions that host them, some subset of these behaviors will gain local, regional, national, or international attention. As such stories trickle back to the United States, it is not clear if the United States sees these as isolated events, another cost of providing security globally, or a pattern of behavior that the military has been incapable of curtailing. Even overseas, it is not clear if the behavior is systemic, sensationalized or somewhere in between. It is feasible that the crime rate among service members may be at or below the crime rate in the populations they base in; the relatively lower rate of criminality of the military relative to other host population is something US military leaders have used as a defense \cite{Gillem2007}. Still, the very fact that it is a foreign military member committing crimes against a host population makes each incident more noteworthy and more likely to garner media and local attention. In our research, we have found some evidence that the presence of military deployments correlates with increases in property-related crimes, but less evidence on other changes in the rate of crime within a society \cite{Allen2011}. In understanding how military member crimes play a role in this book, the incidence rate is less of a concern than the perception those incidents create. Of course, the more crimes that occur, the more opportunity for negative views generated by those crimes to proliferate. %cite is wrong, fix when you have the internet. Create an appendix file to start offloading tables

In the previous chapters of the book, we have discussed the role that contact has on producing both positive and negative perceptions of the United States. However, most of our discussion has focused on the positive aspects of the role contact in breaking down stereotypes and encourage familiarity. We have discussed both contact creating positive interactions and how economic benefits can reinforce positive support for the US military presence. While the chapter on minority presence has indications of fewer positive trends for the United States, this chapters delves into a directly negative effect: crime. In this chapter, we develop our expectations as to both where we see service member crime events make national media venues as well as the role of crime in affecting perceptions among the 42,000 people we surveyed. 

In understanding competition for host-state consent to base, crime plays an insidious role for basing powers that may be hard to curtail. A country that deploys tens of thousands or hundreds of thousands of people into other nations will have some criminal infractions within its deployed populations. It would be difficult for any military or institution to perfectly select members of its community that have no capacity or willingness for committing a crime. Doing so would also rely on outdated notions of why crime even occurs. There are a complex set of determinants of what causes crime and no social institution has successfully eliminated all crime within its membership. In considering how to reduce or limit criminal offending within its ranks, the military can certainly invest in other structural, incentive, and training changes. A big picture decision for a basing power is whether they decrease the opportunities for crime by isolating their own population or increase the amount of monitoring of, incentives against, and punishments for criminal behavior.  The United States has increasingly chosen the first option in the last two decades. By choosing isolation, the United States is attempting to internalize criminal behavior to its own population and decreases the likelihood of a story causing basing ramifications.

Additionally, building self-contained military cities requires mostly large fixed costs (land, buildings, etc.), while the option to monitor and police behavior requires long-term, enduring costs to maintain. Isolation is cheaper and easier than integration and monitoring. While the Navy has a history of maintaining an insular presence, the rest of the military started following a similar path by the mid-2000s \cite{Gillem2007}. The trade-off with this strategy is that the non-personnel related costs from deployment will remain and the positive contact and economic flows from personnel will evaporate as a result of the American city with a fortress strategy. %return to the idea of how basing will remain unpopular, but without contact, there is no counter-narrative.

To understand our data, we first explore our current understanding of the relationship between US military deployments and crime within host states. We then use this to build our expectations of and how crime affects the perceptions of host-state civilians. Next, we use these hypotheses to return to our model of perceptions to estimate how crime influences people's positive and negative views of the United States. Finally, we conclude by evaluating how crime influences the domain of competitive consent.
