\chapter{Conclusion: The Pursuit of Influence in the Time of Competitive Consent \label{cha:conclusion}}

When we interviewed\index{interview} an American Government\index{government} Relations\index{Government Relations Officer}  Officer at Clay Kaserne Army Post\index{Clay Kaserne Army Post} in Wiesbaden\index{Germany!Wiesbaden}, Germany\index{Germany}, one of her observations was that she was somewhat of an ``endangered species.'' Hers was a civilian position at the Army\index{Army, US} post that entailed building a line of communication between the base and local government\index{government} officials and communities. She would give tours of the base to the mayors\index{politicians} of neighboring towns, send out newsletters about the base to surrounding communities and newsletters about local events to the service members stationed at the base, receive noise\index{noise pollution} complaints at a specialized e-mail address, and many more activities. Even though she had gone ``above and beyond'' in building positive relations between the base and its community, she noted that her base commander\index{base commander} had to fight for her position to exist. She said that this was because these positions do not really exist anymore, as ``people don't see the value'' of them.\cite{kaserneone20190725}

In this book, we have presented a variety of findings of the positive and negative effects of the US military presence abroad. In most of the chapters, we have discussed our analysis and left the decision to apply these findings to policy up to the reader. Yet, having worked on this project for years, one prescription we are willing to get behind is protecting these self-described ``endangered species,'' as the Government\index{government} Relations Officer.\index{Government Relations Officer} 

Our work is sometimes erroneously interpreted as us advocating for the more frequent use of the US military to carry out diplomatic roles\index{diplomats}. Though we study how the US military engages in public diplomacy\index{public diplomacy} and produces positive perceptions of US actors in some interactions, the US already has a professional diplomatic corps\index{diplomats} in the State Department\index{State Department}. For example, one of our interview\index{interview} subjects in Panama\index{Panama} noted how the military could deliver humanitarian\index{humanitarian} aid to communities abroad but is limited by not having the language and cultural training that the State Department\index{State Department} does.\cite{embone20180712}

What we do suggest is that when the US military is deployed abroad, there be more active engagement with host communities. Over and over again, we heard in our interviews\index{interview} across regions that secrecy tends to be interpreted negatively by host state civilians and that they are likely to assume the worst about US military facilities. There is uncertainty regarding the extent to which the United States will maintain its global hegemony\index{hegemony} in the next few decades. In the following section, we discuss some potential scenarios for the future of US basing and the domain of competitive consent\index{Domain of Competitive Consent}. All of them include some amount of foreign US deployments. Thus, if the US will continue to maintain a significant military presence abroad, preserving the jobs of those individuals who work to build relations with local governments\index{government} and communities will be a worthy investment. This will be particularly important in the future when the US has to operate in an increasingly competitive environment in the pursuit of influence abroad\index{American influence}. 

This chapter will summarize and discuss the book's main findings, project future trends, and establish research paths for subsequent scholarship. We begin by surveying the present state of basing and what our research tells us about how US decisions shape those trends. The status quo of basing and deployment policy then establishes distinct directions in foreign policy and international relations\index{international relations}. We thus forecast potential scenarios that are likely to play out over the next three decades. Next, we discuss the different global political developments that may shape basing and attitudes towards deployments in the future using our research. Given these scenarios, we discuss the clear policy recommendations stemming from our work. Finally, there is always more science to do. We conclude the chapter with future research directions. We illustrate pathways to answer questions about bases and deployments, their impacts, and whether and how they will endure. 

\section*{The United States and Basing in 2022}

The United States is at a basing crossroads in 2022. The last few presidential administrations have also thought they were at a crossroads, but their view of the paths differs from what we present. It is of note that none of the previous administrations in the past seventy years have been willing to give up the US global military presence and the power projection ability that comes with it. Instead, the past few administrations have sought to satisfy the competing goals of drawing down the US foreign military presence while maintaining the US security hegemony\index{hegemony}. Beyond the domestic incentives to trim the budget, in the last decade, US executive leadership has looked towards slimming down the US presence for several other reasons: a response to the increased vulnerability of US personnel to possible terror\index{terrorism} attacks, to decrease the likelihood of interactions producing catalytic moments against US deployments, and a continued shift towards capital-intensive military forces.\cite{Woody2021} These forces have led decision-makers towards reducing the footprint of US bases overseas in the hope of maintaining the enduring deployments of the current US basing network. 

This strategy is a misdiagnosis of the core issue for the US basing network's present and future maintenance. As we discuss in Chapters \ref{cha:theory} and \ref{cha:meth}, the interaction between service members and host-state civilian populations is a multifaceted event that can occur via a whole set of negative and positive pathways. While negative interactions catalyze anti-basing movements, several positive pathways extend from interactions with military personnel and the economic benefits\index{economic effect} of a US presence. These interactions are a passive, normalizing force that builds support for the United States and its mission within a country. Additionally, there are qualitative\index{qualitative} accounts that support how these pathways can have mollifying effects on anti-basing opposition. First, as our interviews\index{interview} with activists\index{activists} in Berlin\index{Germany!Berlin}\index{Germany} demonstrated, recruitment against basing is often more challenging near a base than in metropoles that are more distant from the base and its inhabitants. Second, individuals surrounding a base come to economically\index{economic effect} rely upon the presence of troops for local service sector jobs, retail, and real estate. For example, as Gillem discusses, bar owners in South Korea\index{South Korea} actively stifle opposition surrounding a base as their business benefits greatly from American deployments\index{mobilization}.\cite[][To be clear, Gillem mentions this civilian support not as a positive outcome but as another facet of the oppressive nature of bases. Still, it is an example of varying preferences towards deployments and bases from the part of civilians.]{Gillem2007}

Thus, a significant consequence of pursuing capital-intensive deployments that draw down personnel is that many negative costs will remain while some benefits are lost. The costly nature of noise pollution\index{noise pollution} will stay as the military will continue to station aircraft for all military branches, and drones\index{Remotely Piloted Aircraft} remain a noise nuisance.\cite{Schaffer2021} Though perhaps less so, base-related traffic will continue to be an issue around military installations. Environmental consequences of the capital-intensive technologies will continue and possibly surpass the pollution produced by larger human workforces. While particular personnel events have appeared catalytic, people's actions are not the sole cause of opposition to the military presence. The underlying externalities of bases will continue to mobilize\index{mobilization} dissent, even with reduced personnel. Removing a significant percentage of military personnel from a base is unlikely to remove the resistance to deployments. The United States' current overhaul of the military basing structure will not solve the primary problems.


In addition, the US may be losing some of its advantages in building soft power\index{soft power} through basing. For example, the hiring of local labor and being perceived as fair in business dealings, and the effect this can have on support for the US military presence becomes particularly relevant as the United States moves into an era in which it is likely to be competing for access with China\index{China}\index{Domain of Competitive Consent}. The belief that Chinese construction\index{China}\index{construction} firms overseas are insular and only hire Chinese laborers for local projects is prevalent among Americans yet somewhat outdated. During his presidency, Barack Obama\index{Obama, Brack} warned African\index{Africa} nations to make sure Chinese construction\index{China}\index{construction} projects use local labor.\cite{economist2014}  For many in the US, the view is that Chinese overseas projects are characterized by a lack of local labor usage, especially in regards to its major push through the Belt and Road Initiative (BRI)\index{Belt and Road Initiative}.\footnote{In addition, there is concern about the large amount of African\index{Africa} debt that China\index{China} holds and how it could lead to Chinese control of African\index{Africa} states' assets and can threaten national sovereignty.\cite{Gavin2021}} Therefore, the United States will be able to maintain its advantage in providing economic incentives\index{economic effect} to local populations and building foreign support. This would produce an optimistic finding for the evolving domain of competitive consent\index{Domain of Competitive Consent} for the United States. It would maintain at least one unique and active way of generating positive perceptions despite creating more insular bases overseas. However, the most recent data does not support this optimism.  When evaluating different firms, research shows that between 70-90\% of Chinese\index{China} hiring in Africa\index{Africa} are local laborers.\cite{Chiyemura2021} 

Another advantage that the United States has traditionally enjoyed is that China\index{China} has receives heavy criticism for treating local laborers in Africa\index{Africa} and other regions. Yet, even that point is one that the Chinese government\index{government}\index{China} has addressed by establishing labor standards in the mineral sector that are on par with the OECD countries' guidelines.\cite{Buhmann2017} Some scholars argue that at least in the case of Africa\index{Africa}, not only have some individuals' standards of living improved from what they were previously as a result of the Chinese\index{China} presence, but also that Western powers have also violated Africans'\index{Africa} human rights\index{human rights} throughout their interactions with African\index{Africa} states. As such, the United States does not have a unique edge in filtering out economic\index{economic effect} and developmental benefits to local communities even though local bases may focus on hiring locally. 

The problem of reduced US advantages becomes more acute when we return to the evolving domain of competitive consent\index{Domain of Competitive Consent}. While the United States is not actively competing with other basing powers for access rights to various countries in 2022, that domain is just beginning. Having to share operational space with an expected competitor in Djibouti\index{Djibouti} fundamentally changes and shapes what the US can do in Eastern Africa\index{Africa}. Not only does the US need to consider strategic movements in both air and space domains when conducting routine operations in Djibouti\index{Djibouti}, but off-base behavior by personnel may be more complicated. The Chinese\index{China} People's Liberation Army\index{People's Liberation Army} support base is a thirty-minute drive from Camp Lemonnier, as both bases are near the capital of Djibouti\index{Djibouti} City. Altercations between service members and host-state civilians can undoubtedly become an international incident, but what if there were to be an altercation between Chinese and US troops?\cite{Bryant1979} The prospect of the consequences of such an encounter may give more reason to make sure personnel in Camp Lemonnier\index{Djibouti!Camp Lemonnier} stay within the confines of the base and do not venture outside of its installations when off-duty. 

The American military's insularity in Djibouti\index{Djibouti} has already put it at a disadvantage in the soft-power\index{soft power} realm. The Covid-19\index{COVID-19} pandemic, of course, affected interactions between deployed military, but even before it was known among locals that the French\index{France} military was allowed off base to exercise, patronized local businesses\index{economic effect}, and socialized with locals much more than the Americans. If the US continues along this path, it will cede any positive soft power\index{soft power} building opportunities to China\index{China}. While the French, a former colonial power\index{colonialism}, have a francophone\index{French} advantage in Djibouti\index{Djibouti}, locals interviewed by reporters\index{journalists} have expressed an active desire for American military personnel to contribute to the local economy by frequenting local restaurants or engaging in community outreach of the sort that is done in other countries (building schools, etc.).\cite{Jacobs2017} While the Chinese\index{China} have also lagged behind the French in interactions with locals, the trend of increasing soft power\index{soft power} efforts by China\index{China} may lead them to increase their off-base outreach. Thus, a potential situation where Camp Lemonnier\index{Djibouti!Camp Lemmonier} remains off-limits to most Djibouti\index{Djibouti}an civilians, American personnel are deployed to Djibouti\index{Djibouti} but confined to the base, and Chinese troops frequent local establishments\index{economic effect} and live among civilians could be a future reality.  Already, the United States has been losing the soft-power\index{soft power} game economically\index{economic effect} in Djibouti\index{Djibouti} as Chinese connections have helped build economic infrastructure locally\index{economic effect}. In contrast, the United States contribution is mostly limited to the rent it pays for its base.\cite{Bearak2019} Of the three basing states in Djibouti\index{Djibouti}, the United States has less history than the French\index{France} and fewer ties than the Chinese; if a popular movement against basing were to rise and demand action by the government\index{government}, the United States would likely be the first to lose access.

These dynamics lead us to believe that in 2022, the United States has not fully understood the increasing influence of foreign domestic consent\index{Domain of Competitive Consent}. Throughout the book, we have identified both cases in which negative interactions with US service members have decreased support for the US military presence and lost opportunities for positive interactions to build up support for them. All of them show how support and opposition affect the cost of maintaining the US military presence abroad.  

%Starting with the George W. Bush Administration, discussing plans to redeploy troops from legacy deployments in Germany, South Korea\index{South Korea}, and Japan faced support and opposition.\cite{Sang-Hun2006,Sang-Hun2018,AssociatedPress2020} Especially in Germany and South Korea\index{South Korea}, sustained demand for a continued presence and a discussion of considering withdrawal has provoked backlash at both elite and domestic levels. Japan and South Korea\index{South Korea} pay billions of dollars in burden-sharing to maintain troops within a country as the governments of these countries see a need for a continued US presence.\cite{Flynn2019} They are willing to pay to maintain those troops; however, this demand was borne through conquest and war (for Japan and South Korea\index{South Korea}, respectively) and seven decades of US-host state ties. The US investment in all three countries to help build them up post-conflict looks more akin to what China\index{China} is doing in Djibouti\index{Djibouti} than what the United States is doing in Djibouti\index{Djibouti}.

Chapter \ref{cha:min} addresses how US military deployments affect minority groups\index{minority}. Overall, our results show that the United States does not do as well with them as with majoritarian groups. Our qualitative interviews\index{interview} reinforce that the US has not focused on making in-roads to minority groups\index{minority} within host countries. In the same way that non-White service members report different experiences depending on the country, they are deployed to (for example, one service member noted that people in England\index{England} were less discriminatory than in the US South where he had grown up, while another stated that for Black service members, going out to dinner in Eastern European\index{Eastern Europe}\index{Europe} states was ``difficult'' \cite{rafeight20190719,rafseven20190719}), ethnic minorities\index{minority!ethnic} in host countries have varying experiences with the US military.

Given the United States' rhetoric of promoting human rights\index{human rights} globally, more active engagement with ethnic minorities\index{minority!ethnic} in host countries is an area where the US could improve the experiences of minorities\index{minority} that have faced discrimination. Through our surveys\index{survey}, we find that while minority\index{minority} groups do not express some of the most positive views of the US military that ethnic majorities do, they also do not hold extremely negative views (or at least are unwilling to express those views when surveyed\index{survey}). Instead, minority\index{minority} groups tend to be more neutral towards the US military presence. This implies for the US that there may be a latent group of stakeholders who could be open to supporting the United States (and channel that support to politicians\index{politicians}) if the United States gave them a reason to do so. As of now, the United States is leaving potential support on the table.\footnote{We note that in some particular cases in which the US military has disproportionately burdened ethnic minority groups\index{minority}\index{minority!ethnic}, such as the Okinawans\index{Japan!Okinawa} in Japan\index{Japan}, the patch to establishing positive relations is a much more difficult one.}



In contrast to outreach, crime\index{crime} committed by US service members is a straightforward way for the US to lose support for its military abroad. In chapter \ref{cha:crimes}, we focused on the role that crime\index{crime} by service members plays in harming perceptions of the United States. These adverse events will almost always receive extensive media coverage in host countries. In addition, our interviews\index{interview}  showed that as long as the host country's judicial\index{judicial} system does not try US service members who commit crimes\index{crime} against host country nationals, no amount of judicial\index{judicial} consequences will be satisfactory to activists\index{activists} \cite{amb20180713}. Given the political impossibility of allowing US service members to be tried by foreign courts, US diplomats\index{diplomats} will have to continue damage control efforts \cite{Freeman2021}. Crime\index{crime}s and crime\index{crime} coverage correlate with negative perceptions of the United States military, government\index{government}, and people. An overly simple solution often implemented by the United States is to simply not allow service members to leave their bases or to give them the fewest incentives to do so by fulfilling all their needs on base. Yet we argue that this solution is one that ``throws the baby out with the bathwater.'' Yes, some amount of crime\index{crime} will be inevitable when human beings interact with each other, but there are effective ways to prevent it and diminish its harmful effects when it does occur. Our analysis finds that even considering crime\index{crime}, contact\index{contact} with US service members and economic benefits\index{economic effect} still influence individual views of the US. Thus creating isolation stations in bases to segment American populations from host-state populations trades off the negative effect for two positive results while keeping other negative externalities flowing into local communities. Eliminating crime\index{crime} is not a realistic possibility but building better institutions to reduce its prevalence is feasible.
%%Add something here about SOFAs and racism from Bianca's work? 

Finally, we turn to mobilization\index{mobilization} in \ref{cha:protest} and discuss the causal pathways that lead to protest\index{protest} and removal. Similar to its reaction to crime\index{crime} by service members, as of 2022, the United States military has responded to anti-basing protests\index{protest} by isolating itself to avoid further provocation. However, this strategy is unlikely to change the nature of anti-base activism. As we note in Chapter \ref{cha:protest} many protest\index{protest} movements are not based in communities neighboring military installations but in the capital cities. These movements also have transnational elements. They are not going to stop protesting\index{protest} the use of drones\index{Remotely Piloted Aircraft} at Ramstein Air Base\index{Germany!Ramstein Air Base}\index{Germany} because US service members have stopped shopping at the local supermarket. Many protest\index{protest} movements fail in limiting the US military presence. However, anti-base opposition that fails at one point in time may regroup and grow stronger at a later point in time to continue the campaign\index{mobilization}. A policy of disengagement allows anti-base groups to continue to try their luck and neglects to build the conditions for pro-base groups to create a successful counter-narrative. 

In the realm of foreign military presence, the US is in a tenable but diminishing position. The status quo could be disturbed by a series of possible events like increasingly mobilized\index{mobilization} anti-basing opposition, China\index{China} expanding its foreign military presence beyond Djibouti\index{Djibouti}, rising waves of nationalist\index{nationalism} and anti-imperialism in states that host US bases, the European Union\index{Europe}\index{European Union} deciding to construct its military alliance in a way that excludes the United States, or Russia\index{Russia} pushing to expand its sphere of influence. In each of these situations, the US will be fully competing for the consent\index{Domain of Competitive Consent} of the governed. Even subnational protests\index{protest} could unravel their position if anti-basing movements become transnational pressure points against the US.\cite{Cooley2020} Forecasting these scenarios out additionally, we turn to the evolution of the domain over the next thirty years. 



\section*{The United States and Basing in 2055}

To contemplate how basing trends in 2022 translate into three decades in the future, we consider three different scenarios in global politics: First, a scenario where the US can maintain its relative global position. Second, a future where the United States shares a new bipolarity\index{bipolarity} with the People's Republic of China\index{China}. Third, we conclude with a final possibility of a multipolar\index{multipolarity} system with China\index{China} and Russia\index{Russia} rivaling the United States. 

\textit{Scenario 1: The US maintains primacy}

We begin with the scenario closest to the US's present situation and therefore requires the least amount of change required for the American global strategy and its necessary subcomponent of basing. This is a case in which the seemingly inevitable ascent of China\index{China} does not occur. In this potential world, challenges to the US basing system will come from local opposition and regional powers rather than from a rival superpower.

In the next thirty years, the United States may keep pace with the PRC. While the US edge would erode to some degree, China\index{China} would not become a global contender for world power.\footnote{Doshi (2021) outlines a strategy of blunting, building, and asymmetry that the United States may pursue to curtail China's\index{China}'s ambitions.} Many indicators point to this possibility being more than just American wishful thinking. For example, the PRC's\index{China} growth in economic and military power may slow down or collapse entirely. Some critics of China\index{China}'s growth point to artificial projects (such as the development of ``ghost cities'' where the government\index{government} builds entire towns but no one is actively living in them) and inflated GDP numbers to suggest a mismatch between reported and actual growth in China\index{China}.\cite{Yu2014,Roy2020} If this is the case, current projections may be overestimating China\index{China}'s future growth.

Even if China\index{China} does continue its projected economic growth and surpasses the United States in leadership, it is entirely possible that its future relations with the US will not be conflictual. If the United States' project of ``congagement'', where it economically engages China\index{China} while militarily containing the country, is relatively successful, the PRC would not emerge as a rival but as a global supporter of the current economic and international institutions.\cite{Khalilzad1999} In this scenario, China\index{China} becomes a US mentee and the heir-apparent to the current global system. This transition would be more akin to when the United States passed Great Britain\index{Great Britain} without significant conflict. For power transition theory, a satisfied rising power that seeks to maintain the status quo of the international order is less likely to provoke conflict with the current dominant state.\cite[][We do note the very different context of the transition in global leadership between the UK and the US, which happens after two disastrous wars in which the potential competitors are allies and not adversaries. Additionally, there is not a scholarly consensus on whether such a transition is even feasible.]{Organski1980,Efird2003,layne2018} 


Further, even if China\index{China} continues to grow, it may never fully don the mantle of global leadership that scholars argue is requisite for power to manifest in hegemony\index{hegemony}.\cite{kindleberger1973} This could occur either through inability or unwillingness. Failure to lead would result from a failure of China\index{China}'s soft-\index{soft power} and hard-power projects to build enough support for China\index{China} and its ideology. Unwillingness is also a real possibility, as scholars have argued that many projections of China\index{China}'s rise view it from a Western power transition\index{power transition} lens. In contrast, China\index{China}'s priorities lie in maintaining control over its sphere of influence in Asia\index{Asia} and the Pacific but are much less ambitious when it comes to political influence over the rest of the world \cite{Chan2017}.  


%Note, mention something about predictions from the 1980s about Japan overtaking the US in power that didn't follow through. A lot of this may be based on Blade Runner and how its predictions about the future did not age well, but I'm sure there is actual literature about this.CM. 

While an indefinite American hegemony\index{hegemony} seems unlikely as of the writing of this manuscript, it is true that in the past reports of the end of American dominance have been vastly exaggerated. For example, the conclusion of the Vietnam\index{Vietnam} war, oil shocks, and stagflation all seemed to indicate that the United States was in decline and was on the way to exit its great power status by the end of the 1980s.\cite{keohane1984,bergesen1985} The 1970s and 1980s were also fraught with concerns about the decline of the United States compared to the economic and political rise of Germany\index{Germany} and Japan\index{Japan} as potential powerhouses (a cursory look at the science fiction of the era shows a future of Japanese\index{Japan} cultural dominance with Hollywood movies such as \textit{Blade Runner} and \textit{Back to the Future II}); of grave concern was whether either state would seek also to become a military power.\cite{maull1989} Pundits and scholars in the 1990s and 2000s saw India\index{India} and China\index{China} as the next two possible global contenders for great power status.\cite{gupta2006} Since those early concerns, India\index{India} has faded away in much of the discourse about coming great powers, with only a handful of research considering the possibility of India\index{India} becoming a great-power state in the near future.\cite{pant2007,carranza2017,narlikar2019} Japan\index{Japan}  and Germany\index{Germany} have remained economic powerhouses, but scholars do not consider either as a military rival for hegemony\index{hegemony}. 


In terms of thinking about basing in the 2050s, however, we note that even if the United States retains relative power projection primacy, it is still in a weaker position than it was in the early 1990s. While it would not be dealing with a countervailing political force, it would need to navigate regional powers in Europe\index{Europe}, Central Asia\index{Asia}\index{Central Asia}, and East Asia\index{Asia}\index{East Asias} as a (perhaps) more consolidated European Union\index{European Union}, Russia\index{Russia}, China\index{China}, and India\index{India} wield influence with their neighbors and allies. Even if only one or two of these states (or a collection of states) offer regional pushback to US aims, it will change the political and military environment for the United States. This is a key point that Cooley and Nexon label as a ``challenge from below'' where international primacy can falter from actors other than great powers.\cite{Cooley2020}

%\footnote{Of course, there are parallels to 19th century and British hegemony, but the texture of the United States position and the mechanisms of it follow different routes than the 1816-1916 period.}

Thus, even without a rival global challenger, the United States would face significant constraints. The domain of competitive consent\index{Domain of Competitive Consent} will remain salient when the United States needs to compete with regional powers and hegemons instead. It does not take a global power to set up a network of bases. For example, despite no longer being a superpower, Russia\index{Russia} maintains a collection of foreign deployed bases in 2022. 

As regional powers look to have a more substantial say in military affairs in a region or seek access to airspace or ports, they will inevitably reach out to their neighbors for access rights and basing arrangements. Some subset of likely partners for these regional powers will be countries where the United States traditionally has held bases. If the budding regional power is adversarial towards the United States, or the new security agreement would replace the United States' security provision for that state, then the US is forced into a position where it has to demonstrate that its provision of security is superior to that of the competitor.\cite[][Morrow argues that strategic security alliances between strong and weak states offer a trade-off for both countries. The more vulnerable state gets to outsource its costly security provision to a stronger state, and the stronger state gets some control over the state's foreign policy. It trades security for autonomy. Under those conditions, additional security may be complementary, but major power security provision is more likely to be a substitutable good. In these cases, the United States could lose out to ideologically aligned allies, such as how the United States has supplanted the United Kingdom in several of its former bases.]{morrow1991} Ultimately, if a government\index{government} needs to respond to civilian pressures due to a legacy of harm inflicted by a US presence, the United States will lose its position.\cite{Mcmanus2017}

In this scenario, the United States may have to compromise on its exclusive access in several states. Even currently, Djibouti\index{Djibouti} hosts three basing states (the United States, France\index{France}, and China\index{China}) and other forces on occasion (such as the British\index{British} military personnel at Camp Lemonnier\index{Djibouti!Camp Lemmonier}). This shared basing arrangement may increasingly become the norm as regional states wish to curry favor or be under the umbrella of their regional hegemon\index{hegemony}. The US will increasingly have little say in these arrangements, as US requests for exclusivity may result in its exclusion. If a central Asian\index{Asia}\index{Central Asia} state must choose between hosting the PRC or the US, the US can lose out to the more proximate neighbor. Leaving a territory certainly limits operational freedom for the United States. Alternatively, the US may have to sacrifice autonomy for access. Shared living spaces impose further restrictions as the United States may have to negotiate airspace and port rights or limit its exercises and operations due to the risk of conflicting with other forces. While this may seem farfetched, even in cases where the US is the exclusive foreign power, there can be competition over operating areas. A US Army\index{Army, US} operations officer deployed to Camp Hovey\index{South Korea!Camp Hovey} in Korea noted to us that something that created a lot of ``anxiety'' for him was the fact that because the Korean Army\index{South Korea} had priority over training areas, they could just show up and take over a training area the US Army\index{Army, US} was using, and the Americans could lose the space halfway through a training exercise \cite{koreaone20210926}. This type of scenario would only become more common if the US shared space with other deployed foreign powers.  

If the United States wants to remain an active player in several regions it needs to be able to project power, even if it is at a reduced capacity. It can do so by maintaining key allies in each region or keeping access to important seas and ports. Importantly, as the Department of Defense\index{Department of Defense} and the Biden administration\index{Biden administration} change the US basing posture from massive legacy deployments to more minor, flexible bases with smaller footprints, this may reduce some associated harms. However, it diminishes two aspects of basing for the United States. First, drones\index{Remotely Piloted Aircraft} and missiles are not substitutes for ground forces. While drones\index{Remotely Piloted Aircraft} are effective at quick interventions or targeted assassinations, they serve as a less credible assurance tool to allies.\cite{cypher2016,mehta2019} The threat to respond to aggression towards an ally will be less credible without American forces in a region. Second, the associated pathways of support US troops have, which we have reviewed in detail in this volume, disappear if bases become drone\index{Remotely Piloted Aircraft} operating centers instead of military outposts. To wit, as we report about Ramstein\index{Germany!Ramstein Air Base}, drones\index{Remotely Piloted Aircraft} are already unpopular among allies and have no capacity for building positive connections with local communities. They do not integrate into local communities, go shopping at grocery stores, patronize nail salons, or coach little league soccer. Members of the military do.

In the 2055 scenario where the United States maintains its global position but must contend with regional powers, it will continue to face a situation where competitive consent\index{Domain of Competitive Consent} is the underlying force enabling and disabling its basing presence. Where the competition occurs may be more scattershot as there is no ideological rival seeking to convert and subvert states to their preferred system of government\index{government} or economics. However, the US still loses ground as regional security arrangements, which may be less likely to mobilize\index{mobilization} popular dissent and create political costs for leaders, become preferable. Even when there are no natural alternatives to the United States, it has lost bases to popular discontent (such as the air base in Manta\index{Ecuador!Manta}, Ecuador\index{Ecuador}). Thus, in this setting, the US will not be fighting an ideological war against a single basing rival but rather tailoring its approach to each basing host in a way that makes the American military presence more palatable when placed in comparison with regional hegemons\index{hegemony}. 


%cite above claims %MA: I am not sure what needs to be cited here as we are partly riffing


\textit{Scenario 2: Bipolarity\index{bipolarity} with a rise of China and/or US decline}

A rising China\index{China} that becomes a global competitor is a more commonly forecasted scenario for the future international political situation; however, there is an outstanding debate over how conflictual or cooperative bipolarity\index{bipolarity} will become.\cite{maher2018,ross1999} For some, 2055 might be too soon to see China\index{China} ascending to global power. Within its planning, China\index{China} hopes to achieve both a period of ``national rejuvenation''\index{China!national rejuvenation} and become the most powerful country in the world by 2050.\cite{Doshi2021} The Correlates of War\index{Correlates of War} project has a metric that it uses as a proxy of power. The Concentration in National Capabilities index (CINC) measures six power sources, aggregates them for every country in the inter-state system, and then produces a share for each country in the international system dating back to 1816.\cite{Singer1972,Singer1987} The United States eclipsed Great Britain\index{Great Britain} in 1897. Still, most scholars of international affairs do not put the United States as a global power until sometime between 1919 and 1945, suggesting a 40-50 year lag between the indicator and when the United States began acting like a world power. According to this index, the PRC passed the United States in 1995; if China\index{China} followed a similar trajectory as the United States in assuming global power status, then 2055 would be a conservative estimate. Charting the rise of China\index{China} as akin to the United States would be an oversimplification of great power ascendancy, but it is at least an interesting comparison to draw. 

The bipolar\index{bipolarity} competition between the United States and China\index{China} will likely present the United States with the biggest challenge for its current network of bases. While China\index{China} has thus far only publicly looked at sharing access rights in Djibouti\index{Djibouti} and artificial platforms in the South China Sea\index{South China Sea}, it has several basing and basing-like projects ongoing at the time of this writing.\cite{Doshi2021} China\index{China} will likely expand its basing network as part of the drive to grow its power. Since World War II\index{World War II}, the United States military position has rested largely on its ability to project its might in multiple regions simultaneously. While only a fraction of the United States military is in any given region, the presence of its forces affords it the implication of additional force should tension escalate. If China\index{China} seeks to become a global contender, it will need force projection capabilities in land, sea, and air.\footnote{While China may soon create and sustain smaller outposts to support drone operations in other countries, such behaviors are less about force projection and more about internal stability, likely targeting terrorist threats as the United States has.}

After Djibouti, China\index{China} will likely continue expanding its influence\index{China!influence} to other African\index{Africa} states. China\index{China}'s soft-power enticements and military hard power are guarantees to bring more countries into its network. In response, the United States is likely to escalate its position in Africa\index{Africa}. It had already stationed more troops on the continent in 2020 than it had a decade prior and is likely doing so in anticipation of further Chinese military investments in the region. China\index{China} will likely seek to expand in its far neighborhood as well.\cite{OSD2020} It will further transcend from a regional power into a global player if it can garner agreements in the Middle East\index{Middle East}, South Asia\index{Asia}\index{South Asia}, or Africa\index{Africa}.\cite{yung2014,cabestan2020} Its current holdings and attempts at gaining ground in Asia\index{Asia}, Central Asia\index{Asia}\index{Central Asia}, Africa\index{Africa}, and even Europe\index{Europe} demonstrate that this future is immediate.\cite{Doshi2021} Ultimately, if the United States could not stop development in a country like Cuba\index{Cuba} or Venezuela\index{Venezuela}, a base in Latin America\index{Latin America} would signal the erosion of the United States' ``backyard'' that it has maintained almost exclusively since it established the Monroe Doctrine\index{Monroe Doctrine}. 

On a region-by-region basis, the United States will be on alert for potential opportunities for China\index{China}, which are risks to the US position. An advantage China\index{China} will likely have over the US is its ability to place basing sites in autocratic\index{autocracy} countries more easily. Unlike China\index{China}, the United States may face domestic opposition to basing in these regimes, despite being accepted during the Cold War\index{Cold War}. By 2031, the post-Cold War era will have lasted longer than the Cold War\index{Cold War} itself. The limitations and the permissiveness of the Soviet-US\index{Soviet Union} bipolarity\index{bipolarity} will be an increasingly distant memory for the general public and, with a few more decades, the leadership of most countries. If the public constrains the US leadership from securing sites in illiberal regimes, then the relative diplomatic\index{diplomats} cost for China\index{China} to do the same will be lower.


If the post-Cold War trend continues and the US becomes more selective about not placing bases in autocratic\index{autocracy} states, it will face increased basing costs. As in any bargaining situation, having fewer outside options places a bargainer in a worse position. If the US is unwilling to deploy forces to autocratic\index{autocracy} states, it will have fewer credible alternatives. In that case, host-states can effectively increase the cost of basing unilaterally, and the US will face either paying more or losing regional access. Further, suppose the US maintains its current policy of insular bases and deployments. In that case, the costs of those bases will continue to accumulate with only limited benefits flowing directly to regional populations. Without benefits to offset costs, host-state governments\index{government} are more likely to face discontent from the populations. As hosting the US becomes more costly to host leaders, these governments\index{government} will be more likely to pass on these costs to the US government\index{government}. This may take the form of reduced burden-sharing from the host or increased rents for bases when renewing leases. As the cost of deployments increases, the global position of the US becomes less tenable. 




These conditions set up a potentially vicious cycle for the United States. The US chooses to pay more, scale down bases, withdraw from essential sites, create better basing islands, or combine all of those. While following this strategy, drone\index{Remotely Piloted Aircraft} usage will remain unpopular, noise pollution\index{noise pollution} will continue to plague local communities, traffic\index{traffic} will congest local streets, environmental consequences\index{environment} to bases will still spread, and people will increasingly become suspicious of what the United States is ``really doing'' behind its fortified walls. With reduced positive efforts to passively win over the hearts and minds of partner-state civilian populations, then the negative externalities continue to flow unabated or, perhaps, are managed to some degree. In this situation, some states may opt not to have a US presence within their country. Some countries may fill the void left by the United States with the militaries of regional allies, and others will increase their security cooperation with China\index{China}. When the US military loses access to a country, the remaining listening posts, fuel depots, drone\index{Remotely Piloted Aircraft} command centers, and similar operations offer a reduced capacity for US power projection. % might be redundant with a previous point?



The United States' position will also depend largely on how tenable its position in the Middle East\index{Middle East} and Central Asia\index{Asia}\index{Central Asia} remains. Through the 2000s, there was a shift from bases in Saudi Arabia\index{Saudi Arabia} to new deployments in Afghanistan\index{Afghanistan}, Iraq\index{Iraq}, and maintaining the US position in Kuwait\index{Kuwait}.\cite{Engelhardt2009} Since 2016 the United States has shuttered the permanent bases constructed in Iraq\index{Iraq} and Afghanistan\index{Afghanistan} as it sought to exit its security arrangements in both countries. 

The US retains a presence in other MENA countries\index{Middle East}\index{North Africa}\index{Africa}, but in thirty years will Bahrain\index{Bahrain} and Kuwait\index{Kuwait}  remain palatable deployments for the US public. Will either country face democratization\index{democracy}\index{democratic transition} prospects in the near future? Either pressure (from US citizens opposed to bases in autocratic\index{autocracy} states or host-country civilians with an increased say in their government\index{government} opposed to US bases in their country) severely limits US influence to a few sites near the region. Even NATO\index{North Atlantic Treaty Organization} ally Turkey\index{Turkey}, expected to reliably support the US, voted against the US using its bases in the Iraq\index{Iraq} War\index{Iraq War} of 2003.\footnote{Tensions have varied in the last decade between the US and Turkey, regarding the fight against the Islamic State, the status of Kurdish forces, and Turkish procurement of Russian weapon systems.} Alternatively, the US can increase its deployments or negotiate new bases with Saudi Arabia\index{Saudi Arabia}, Qatar\index{Qatar} or become increasingly reliant on Diego Garcia\index{Diego Garcia} in the Indian Ocean\index{Indian Ocean}. In contrast to the US, the Chinese leadership will have fewer qualms about strategic alliances with autocratic\index{autocracy} countries. Expansion to wield influence over the Gulf of Guinea\index{Gulf of Guinea} or Southern Africa\index{Africa} is already on the People's Liberation Army's\index{China}\index{People's Liberation Army} agenda.\cite{cabestan2020} Continued strategic cooperation between China\index{China} and Iran\index{Iran} can also expand the PLA's\index{People's Liberation Army} access.\cite{Cordesman2021} 

In essence, the 2010s pinnacle of 800 overseas assets held by the United States will decline significantly in this scenario. While China\index{China} is unlikely to surpass the United States as global hegemon by 2055, Chinese access to various regions will expand dramatically as it strengthens ties in Africa\index{Africa}, Central Asia\index{Asia}\index{Central Asia}, and East Asia\index{Asia}. Thus, by 2055, in an era of bipolar\index{bipolarity} competition, a new concern for the US will be about how soon China\index{China} will surpass it in basing sites. 

\textit{Scenario 3: Multipolarity\index{multipolarity} with Russia\index{Russia} and China\index{China}'s relative rise}

The final scenario we discuss is a relative US decline where Russia\index{Russia} and China\index{China} gain significant economic, diplomatic\index{diplomats}, and military power in their respective spheres of influence. While the rise of middling powers (such as the BRIC countries of Brazil, Russia\index{Russia}, India\index{India}, and China\index{China}) would create a multipolar\index{multipolarity} world, China\index{China} and Russia\index{Russia} are most likely to become great powers and compete with the United States on basing.\cite{cooper2013,Mattis2018,Cooley2020} Russian\index{Russia} deployments in mostly Central Asian\index{Central Asia}\index{Asia} countries may expand to include more Middle Eastern\index{Middle East} positions beyond Syria\index{Syria}. Additionally, Russia\index{Russia} may seek additional influence in sub-Saharan Africa\index{Africa}\index{sub-Saharan Africa}. China\index{China}, in this scenario, extends beyond Djibouti\index{Djibouti} as well but can only make more modest gains by 2055 than in the previous scenario of Chinese primacy\index{hegemony}. 
%%this may be a good place to cite some of the papers from the security studies special issue
%MA: Let's wait to see if we get those through the process by the end of hte year


While having two competitors for global politics may be more daunting for US access, it does afford the US some opportunities. For Neorealist thinkers, tripolarity\index{multipoliarity} is an unstable configuration as balancing is hard to achieve among an unequal set of states; this does not make China\index{China} and Russia\index{Russia} natural allies.\cite{Waltz1979} As the last few decades of the Cold War\index{Cold War} demonstrated, the United States, at times, can effectively encourage Sino-Russian\index{China}\index{Russia} rivalries.\cite{goh2005} Given that bilateral relations between Russia\index{Russia} and China\index{China} have been contingent on US relations with both countries in the post-Cold War era, there is room for the US to again sow distrust between the two.\cite{ferdinand2007} 

Basing is a realm where the US may effectively balance China\index{China} and Russia\index{Russia} against each other in a multipolar\index{multipolarity} system. For example, the Russia\index{Russia}n use of the Cam Ranh\index{Vietnam!Cam Ranh} base in Vietnam\index{Vietnam} has been a minor affair in the last decade, but if Russia\index{Russia} and China\index{China} seek to expand their facilities, Vietnam\index{Vietnam} may become a source of tension between the budding powers.\cite{yen2021} Vietnam\index{Vietnam}'s rivalry with China\index{China} may make it more likely to oppose expansionary operations by its largest trading partner. Still, China\index{China} may be willing to ease its position over disputed access to the South China\index{China} Seas under a new military cooperation agreement. Alternatively, Russia\index{Russia} will seek to reactivate its port and air base in Vietnam\index{Vietnam} to regain access to this part of the globe. A move by either country in this direction will likely put China\index{China} and Russia\index{Russia} at odds.




Vietnam\index{Vietnam} is not the only territory where China\index{China} and Russia\index{Russia} may contest over security expansion. The portfolios of possible basing partners in Central Asia\index{Asia}\index{Central Asia}, the Middle East\index{Middle East}, and Africa\index{Africa} are similar. Failure or success in one theater may spill over to diplomatic\index{diplomats} contests in other regions as both powers set up their networks. In some cases, a joint basing arrangement might facilitate cooperation or acceptance, though deployments to the same country increase the likelihood of conflict between those basing states. If competitive consent\index{Domain of Competitive Consent} becomes a contest for states seeking authoritarian partners, the US position may become stronger than it would under a bipolar\index{bipolarity} competition with China\index{China}. However, it is also feasible that joint competition pushes China\index{China} and Russia\index{Russia} to seek out basing agreements with states that the US already deploys to or will deploy to in the coming decades. These competitive spillovers jeopardize the US position. While competition can place a host state in an advantageous position, it increases the cost local elites' (or their constituents') support for a basing state. The United States and Russia\index{Russia} learned this lesson in 2008 when Kyrgystan\index{Kyrgystan} forced the two countries into a bidding war for basing rights. Russia\index{Russia} paid \$300 million in aid as a result (of a promised \$2.1 billion), and the US rent more than tripled for its existing base.\cite{Cooley2020}

In sum, multipolarity\index{multipolarity} offers more opportunities for the United States than bipolarity\index{bipolarity}. However, with the addition of competition between two of the basing powers, there exists a possibility that those two powers ally to the detriment of the United States. Under these conditions, the United States' success requires smart diplomacy\index{diplomats} and strategy that is a gamble, not a given. Whether the United States leadership in 2055 will be up to the task of a global game of dual-containment is yet to be determined. 

Several other possibilities exist beyond the three scenarios we highlight here; however, our goal is to focus on the immediate, most likely outcomes as we write in 2022. For example, some forecasts may consider how a political and military rise of India or a more consolidated European Union\index{Europe}\index{European Union} (with a common defense policy) might alter the need for US bases if allies are also increasing their power projection along with rivals. As friendly states open up the possibility of substituting for US forces, offshore balancing may become a more appealing option.\cite{Nieman2020} While we believe these scenarios to be less likely to occur in the next thirty years, this possibility, and the unique set of principal-agent problems\index{principal-agent problem} it creates, would be worth exploring in works more dedicated to such a task.



\section*{Policy Recommendations}

Regardless of the state of the world the US finds itself operating under in 30 years, there will be aspects of the interactions between US service members and host communities that will remain malleable, and which interested actors may influence. This book has so far mostly limited itself to explaining these interactions. Yet, the findings from this book can also be applied in practice by those that have control over bases, basing policy, and how service members interact with local communities. 

When thinking about the optimal strategy for different actors interested in basing, it is important to note that they are not operating in a vacuum. What this means is that the best strategy will depend on, and need to be a response to, the actions of others. It is essential to understand what different actors, both opposed to and in favor of US military deployments, have to learn from our research and how they can, and already do, incorporate lessons from our research to accomplish their respective goals. Some of their interactions will be ``zero-sum'' in the sense that one's gain will have to come at the expense of the other (for example, this is true when one actor prefers the US military presence to stay and the other for it to leave). Other interactions allow for achieving common goals, such as decreasing crimes\index{crime} committed by service members against the local population.

Thus, rather than give direct recommendations to those advocating for or against US military bases we focus on explaining what leads to positive relations between the US military presence and local communities. From the perspective of US policymakers, these recommendations may be used to foster positive relations and thus preserve the basing presence and enhance American soft power\index{soft power}. Local government\index{government} officials responsive to host country individuals should share the incentive of maintaining positive civil-military relations between the American military and host country populations, as this will create fewer grievances among their constituents. Finally, even those activists\index{activists} who oppose the US military presence may have some interest in diminishing the harm faced by the host country publics. While some of their demands (such as the removal of a foreign military presence on their homeland) could only be met through a complete exit of the United States' bases, several grievances, such as suspicions about US secrecy indicating malicious activities, can indeed be addressed through improved relations with the US military. 




\subsection*{Positive and Negative Contact}

A key conclusion from chapter \ref{cha:meth} was that contact\index{contact} between US service members and host country communities could lead to either more positive or more negative views of the US and its military. We argue that which direction the relationship takes depends on the nature of the contact\index{contact} experience. This is relevant because both the US and host country governments\index{government} have a significant amount of control over the shape that these interactions take. While US military officials stationed abroad cannot control unpopular US foreign policy that leads to discontent with the US (such as invading Iraq\index{Iraq}\index{Iraq War} or the use of drones for targetted strikes in Central Asia\index{Remotely Piloted Aircraft}\index{Central Asia}), they can indeed provide a conducive setting to more positive interactions between US service members and surrounding areas communities. 

For example, as noted in Chapters \ref{cha:crimes} and \ref{cha:protest}, people who have stories of victimization\index{crime} within their social network are more likely to have negative views of the military, and people who have experienced crimes\index{crime} committed by US service members are also more likely to join an anti-base protest\index{protest}.  While our results focus primarily on crime\index{crime}, it would be reasonable to extrapolate that other externalities also convey knowledge of harm to those that listen. If enough stories proliferate, it becomes a matter of when the military presence might victimize someone instead of whether it will. Crime\index{crime} victimization is not a lottery anyone wants to win and greater odds of becoming a victim may help mobilize\index{mobilization} support to avoid that outcome. 

In this case, the US military taking actions to prevent crime\index{crime} against local communities (such as providing rides to service members who go out to bars to avoid drunk\index{crime!DUI} driving) is an action that benefits both the US military presence and those who are against it. Yet, when there is human interaction, it is unavoidable that some amount of crime\index{crime} will occur. One aspect that is important to local populations is that perpetrators are held responsible for their actions if and when it does. Yet, as noted by a former ambassador\index{diplomats}, in some cases, no punishment will be considered harsh enough if US service members are tried in the US and not in the host country. As mentioned a few times, headline-grabbing behavior will not disappear even under the best social institutions. This approach may require revisiting Status of Forces Agreements\index{Status of Forces Agreements} to allow civilian trial\index{judicial} over crimes\index{crime}s that are the most damaging to the social fabric of overseas communities and active engagement in restorative attempts of justice for when a member or an institution fails the larger military.\cite{efrat2021a,Freeman2021} This last point will be particularly important given that, as Freeman find, the US has historically been less likely to allow civilian jurisdiction over its troops in non-European states. The implied racism behind this action has not gone unnoticed by host states. While painful, a good faith effort will go longer to having local groups see the military as an institution for harm and purely seeking to protect its own.\cite{Freeman2021} Yet, as noted by Freeman, in some cases the US has legitimate reasons to be concerned that US service members would not receive a fair trial in a host country, such that turning over jurisdiction over crimes\index{crime} to host countries is not a panacea against the problem of service member crime\index{crime}. \cite{Freeman2021} 

Instead, in many cases, the US has responded to negative incidents by turning inward and restricting service members' ability to leave the bases. The military's push to isolated fortresses away from the public shows that civilian voices matter and that the US military has responded in a global matter. However, given what we have written throughout this chapter and this book, this move makes the US military's job of maintaining communities of support for its mission harder. In that capacity, it is clear that closing off bases to civilians and limiting military personnel activities off base can have long-term consequences for the US presence abroad. Historically, host-state civilians were far more likely to meet an American GI than any American diplomat\index{diplomats} or politician\index{politicians}. The military personnel overseas served as American diplomats\index{diplomats} that both directly enacted the US mission overseas and became passive and active advocates for the broader things the United States sought to secure globally. 

This aspect of service overseas ought to be seen as a feature, not a bug, of deployments if the United States seeks to maintain its dominance in overseas basing networks. It is harder to see the effects of benign, routine interactions as they exist as a passive force that draws in local communities to support the US presence. The explosive scandals involving service member misbehavior garner far more immediate reactions, coverage, and opposition that seem to undermine the US mission, but only examining one outcome of this equation likely misinforms current policy. Additionally, we found an ongoing theme of civilians and journalists\index{journalists} being concerned about US covert behavior since civilians have less access to bases than in the 1990s and earlier. Shutting down avenues that have historically built the trust of the US military while also having security measures decrease access is creating the conditions where locals experience negative externalities and can believe the worst about what happens inside and outside of military bases.

Our proposal is not unrestricted access for civilians or unrestrained overseas behavior for military personnel. Instead, it recognizes that individuals have generated value by living in local communities for the last 70 years. In recognition of this utility and how it translates to local, regional, and national support for the United States to maintain bases and for national governments\index{government} to engage in costly burden-sharing for those troops, the decision of how to deal with troop-driven conflicts off base requires reassessment. The fortress of solitude with self-sustaining fast food, housing, malls, grocery stores, bars, and everything else a service member needs seems like a quick and easy fix for the bad press that happens out of bases. However, suppose the United States invested in better systems of training, maintenance, and monitoring of US personnel, a more costly endeavor. In that case, one of the most valuable assets of overseas deployments can sustain itself. 

It should be clear that the US presence will continue to face some opposition in nearly all countries it deploys to. Universal acceptance is impossible where the very presence of a foreign military creates both winners and losers. Additionally, the US military is a guest, sometimes invited and sometimes uninvited, in many countries. For some segments of the population in each country, the guest has overstayed its welcome. The US military should avoid creating conditions where a critical mass of politically influential activists\index{activists} can convince host-state legislators to restrict and evict the United States by using passive and active routes to create support for its condition. 

Presently, Djibouti\index{Djibouti} is an experiment\index{experiments} as to how different militaries within the same location build support for their presence and, based on the ground game and economic support that underlies existing deployments, the United States is losing out.\cite{Bearak2019} This particular case represents an opportunity where the United States can see what is not working, experiment\index{experiments} with different local strategies, and evaluate how service members build support in an increasingly competitive consent\index{Domain of Competitive Consent} environment. 

\subsection*{Protest Activity}

Protests\index{protest} against the US military are almost always seen as a negative by US officials---they not only signal discontent with the US military and serve to rally even more support against the US military. In contrast, from the perspective of activist\index{activists}s, protest\index{protest} and mobilization\index{mobilization} can indeed be an effective way to bring attention to the harmful effects of military bases and to make the US address these effects, as we have discussed in chapter \ref{cha:protest}. The common ground here is that both the US military and activists\index{activists} would prefer that protests\index{protest} be prevented by decreasing the types of interactions that would mobilize\index{mobilization} people against the US military.

%\cite[While a small proportion of accelerationist activists\index{activists} will have a preference for using increased harms of bases to drive the US out faster, we think this is a dangerous prospect for several vital reasons and do not think it will lead to the success that proponents of accelerationism usually advocate. First, when given a choice between helping people under duress and allowing their harms to continue, the ethical choice is clearly to help those people despite the longer-term goals. Second, it is feasible that the threshold of change is either more extreme than accelerationists expect or does not exist as more supporters rally to defend a more damaging status quo. Third, the expectation of a political rebound that is equal to and opposite from the direction of the existing harm does not appear to occur in reality. Finally, we have seen moderation as a response to extremism in several critical moments in political history, and expecting that suffering will bring redemption is far from guaranteed. Worse, if a government\index{government} is indifferent to the costs of basing, then the accelerated harms of basing is needless suffering.][]{laurence2017} 

Protests\index{protest} can serve as a separating equilibrium that signals to US and host country officials what domestic populations care most about and should be prioritized. This is because by incurring the costs associated with protests\index{protest} and mobilizations\index{mobilization}, protestors\index{protest} are credibly signaling their resolve. Anti-base activists\index{activists} mobilize\index{mobilization} people to their side by pointing to the harms of bases within their region. However, mobilization\index{mobilization} research suggests that generalized harms are not enough to get people to take costly actions \cite{Olson1965,lichbach1993}. Even something as simple as voting or protesting\index{protest} requires time and resources to accomplish\cite{downs1957}. In some states, participating in either of those may risk violence, imprisonment, or even death. Individuals willing to risk those costs are thus sending a clear message of the issue's salience. % (cite some stuff here) --added a few, not sure if enough MA

%Selective incentives for participation are a tried and true method of gaining supporters for your cause, and the cause against basing is not different in that regard. Selective incentives are individualized rewards for people who can range from simple social symbols to monetary payments.\cite{Olson1965} Scholars still understand the role social media has in uniquely overcoming collective action problems as it is clear that it enables action in novel ways.\cite{saebo2020} For our understanding of activist dynamics, the act of posting pictures from a protest or a picture displaying a selective incentive received from donating to or participating in a movement has two effects. First, it allows the person to post content online and advertise their dedication to a cause they care about. Second, it is an advertisement for the group. Consumers of social media content see the action the poster participates in, their online support for their activism, and the direct reward for doing so (online status and material stickers or pins).

Though social media\index{social media} has to some degree supplanted protest\index{protest} and mobilization\index{mobilization} as a dominant form of activism, it also made activism ``cheap.'' It is very easy for a social media\index{social media} influencer to post a ``Black Lives Matter''\index{Black Lives Matter} sticker on their profile or share a post about a particular cause. This has thus decreased the social capital obtainable through such gestures. Instead, showing proof of having been at a particular event is a separating equilibrium that sifts out those ``activists\index{activists}'' unwilling to be present at an actual protest\index{protest} event. By giving protests\index{protest} themes or particular hashtags, activists\index{activists} can drive people out into the streets and beyond the more basic social media\index{social media} support. The Black Lives Matter\index{Black Lives Matter} 2020 protests\index{protest} in the United States showed the mobilizing\index{mobilization} effect that social pressure to post a photo of oneself (masked, of course) at an actual protest\index{protest} had. While switching profile pictures over to BLM logos was seen as trite ``virtue signaling,'' the costly signal of attending a physical demonstration, and the perceived risks that it entailed gave those who posted protest\index{protest} photos extra credibility.

Much in the same way that Black Lives Matter\index{Black Lives Matter} protests\index{protest} signaled to US government\index{government} officials that a broad majority of the US population wanted to address systemic racism and police brutality, protests\index{protest} against US military bases can be seen as a way to understand what the key issues taht drive discontent among host country populations are. The fact that protestors\index{protest} actually have to pay some cost (even if it is just the opportunity cost of attending a march) to protest\index{protest} makes these types of mobilization\index{mobilization}, in a way, a more effective way of communicating the population's preferences to the US government\index{government} officials.

We note that we are not advocating for US foreign policy to be guided by protestors\index{protest}. We are, though, arguing that policymakers can draw useful information from protests\index{protest} and that they should not just be ignored as a nuisance that will invariably accompany the US military presence. As noted by the US Marine Colonel we interviewed and discussed in \ref{cha:crimes}, protests\index{protest} that are a reaction to local incidents (as opposed to global politics) are more ``emotional'' \cite{marinecolonel20211011}. Understanding the types of events and incidents that motivate these types of protests\index{protest}, and focusing on prevention, can result in a win-win for both sides.  
%something about how social media\index{social media} has made sharing a message a ``cheap'' signal. showing that you were actually a protest, a selfie opportunity, matters. Cite some people who do social media\index{social media} and protest stuff. side note, if Amanda says yes let's def get her on the protest chapter. 

%Likewise, people seem keen on the idea of status and promotion. Opportunities within an organization to achieve a job or title confer monetary or social benefits from participating in the organization. These opportunities are often in short supply due to the resource intensity of having paid positions within an organization, but giving out titles for certain levels of commitment can be cheap and effective. Additionally, having roles for people to fulfill is a probabilistic selective incentive where people may over-contribute to compete with others to achieve that position in some organizations. Once awarded, however, that incentive evaporates unless previous commitment has generated enough volunteer buy-in that they remain committed to the organization or there is some possibility of turnover and the position becomes available again. With limited resources, rotation of paid jobs may be a strategy to concern.

%Selective incentives apply to a wide variety of social, political, and economic organizations. The discussion thus far is generalizable to a large class of phenomena, and it is essential to remember that multiple groups are playing the same group. Activism is a competitive game where there are numerous organizations around a specific topic and various causes that all request potential activists' finite attention and resources. In addition, the government\index{government}  and the United States may play the opposing role if it expects that an anti-basing movement is gaining traction to its detriment. It can provide selective incentives to support the base through jobs, celebrations, and other activities that effectively encourage people to support the presence of the base to continue to benefit from the goods it provides. Actors likewise can provide selective disincentives for activism and coercion by the state through arrests or violence, which many states have taken in the past to discourage activism.

%%also add something about how governments may not actually care enough about individuals to get rid of the base, and then the extra harm will have been done for no reason.  

%Relatedly, there is likely to be a discussion of the ultimate aim of any anti-base movement as it seeks allies across the political spectrum to build support. Research suggests that the most enduring and successful movements form a broad base of support instead of those that pigeonhole themselves to a particular segment of ideological thought.\cite{Yeo2011} In that aim, the envisioned alternative by the movement likely matters. If the campaign is nationalist and anti-imperialist, then a focus on domestic military institutions and self-reliance may be more successful at building a left-right coalition against basing. Being a country under active threat from another power may make this appeal more successful as well. Anti-militarism movements that seek to reduce military activities may have a more challenging time building a broad movement. A targeted movement that seeks its advocacy in the wrongs committed by the US without a more general goal may be co-opted by domestic militarism advocates or those seeking a more robust military relationship with other states that may prove as harmful or worse than the United States, such as a former colonial power, a regional organization, or a budding regional power. %is the yeo cite correct here, or am I thinking of someone else? Cross-check with the protest chapter.

%Turning to the other side of the equation, we engage those that wish bases to endure. 

%\subsection*{Recommendations for Basing Advocates} % this section needs to be stronger/more fleshed out.

%To start this section, we will say something counter-intuitive to the audience reading our book: Read the previous section. While this may seem obvious, there may be a natural inclination to skip the last section as it does not directly speak to those supporting bases. However, as we began this section on advice, to understand what one needs to maintain their support, one should understand those seeking to diminish or end it. We could write much of the advice in this section to understand what base opponents desire and their tools to achieve those ends. Understanding opposition is vital in understanding how to meet it, compromise with it, and find a path forward that allows you to achieve your policy and security objectives. 

%Given this direction, we are writing this section from the perspective of someone who might influence US basing policy and not pro-base activists as the latter can do much of what we suggest above but for the opposing viewpoint. As pro-basing policymakers or decision-makers that control local policy (such as a base commander), broader decision-making across multiple bases (within a service or the Department of Defense), or from a civilian perspective that influences legislation or executive choices in basing abroad, there are some straightforward ideas that we have iterated on throughout this manuscript. 

\subsection*{Host-state governments}

Much of the book focuses on the relationship between the US military and host-state civilians as we argue that there is an increasing role that citizens play in determining which state, if any, bases within their country's borders. However, the host-state government is an active party to the presence of US bases. Historically, base negotiations were an US foreign policy elite to host-state government elite negotiation and, in the contemporary period, the host-state government likely has the ultimate say in whether bases can deploy or remain in a country. Host-state governments are in a more difficult role as they balance their own preferences for security arrangements, their political survival from the demands of their citizens, and demands from the United States that may be difficult to meet. 

Generally, in a country that hosts United States, the foreign policy elite consensus is to have the foreign military presence within their country. The demand for the provision of security by the United States could come from security threats, a desire to outsource security provision to free up budget constraints, or to deal with internal issues (such as drug interdiction\index{crime!drug-related}, terrorism\index{terrorism}, or other sources of instability). When the US deploys abroad, the amount it pays is not uniform across cases but can range from it solely paying for most everything related to the operation to the host state paying for the operation. These burden sharing agreements can depend on the need of either actor for the deployment to happen. Discontent or protests\index{protest} within the host state can serve as leverage for the host state to demand more (or to play less) for a deployment situation. Using changes in the burden of providing security can facilitate a government to distribute public goods to areas most harmed by deployments if the government values maintaining the US presence.

While domestic discontent can be leverage for a host state to gain more from a security relationship, US politics can push in the opposite direction. During its tenure, the Trump administration demanded allies within NATO\index{North Atlantic Treaty Organization} as well as Japan\index{Japan} and South Korea\index{South Korea} to increase their burden sharing. Active discontent in South Korea\index{South Korea} and Japan\index{Japan} and new pressure from the United States made it difficult for countries to both appease domestic audiences while meeting the demands of the United States. Ultimately, the Biden administration\index{Biden administration} reversed Trump's\index{Trump administration} course on this demand and facilitated more favorable burden sharing rates and justified it domestically by citing increased competition with China\index{China} and Russia\index{Russia}.

Most host states will seek to maintain a stable domestic environment and encourage positive relations between a foreign military presence and the civilians that live near such deployments. To achieve such ends, then our recommendations mirror those that we make in the previous sections. Increasing avenues for cultural exchange\index{culture}, contact\index{contact}, and economic interactions\index{economic effect} will likely both increase opportunities for inter-community friction but also build passive support for the presence. Negotiating SOFAs to give some local control over criminalized interactions may allow for the building of trust in otherwise tense situations. However, being able to get jurisdiction over inter-community crimes\index{crime} is not something most countries have been able to successful negotiate as the United States has every incentive to protect its service members from criminal prosecution. There also may be some trade off between having judicial sovereignty over foreign troop interactions and how willing the United States would be in allowing its members to associate freely within a country. Trading judicial jurisdiction for integration between the US military and host-state civilians likely trades legitimacy for some level of economic dispersion\index{economic effect} and building support for the national security mission of the state.


\section*{Research Directions}

Our research is exploratory as much as it is testing our expectations about deployments. We still have several remaining questions. The United States established its first overseas base in Cuba\index{Cuba} in 1898; however, the concept of placing the US military among people who it does not directly represent is much older, as the United States established bases in indigenous territory during westward expansion. Other empires, prior and contemporary to the United States also used externally deployed bases dating back to the first Western democracies\index{democracy} and republics in Greece\index{Greece} and Italy\index{Italy}. However, much of the hitherto work on bases and deployments studied them from the perspective of great power competition and power projection. When individual interactions were studied, it was often from the sociology\index{sociology} or criminology\index{criminology} perspective, as opposed to the microcosm of international interactions occurring at the local level of deployment sites. There is thus much room to research how these microfoundations of US foreign policy\index{foreign policy} ripple through global communities.

Within the research we report here, there are more questions our survey\index{survey} answers that we invite others to follow up on when studying the role of the US deployments overseas. All of our data remains publicly available for other scholars to pick through.\footnote{All data, reports, and links to published research are available at \url{http://ma-allen.com/military-deployments/}.} Our conclusions inspire a host of follow-up research projects. For example, our surveys\index{survey} included questions about respondents' views about their country. These can be used to understand better how an individual's view of their government\index{government} can influence views of the US. Likewise, we ask people if the presence of American military forces affects the quality of democracy\index{democracy}, respect for human rights\index{human rights}, public safety, or security from foreign threats in three separate questions. 

Our survey\index{survey} results can also be used to understand how individuals' views may differ from reality and what causes them to over or underestimate the US military presence in their country. For example, We ask respondents to estimate the number of US troops in their country. Responses vary from very accurate to wildly large sums (and in some cases, individuals refuse to estimate if they are unsure of the true number, a response that is interesting in and of itself). We also asked respondents how long they think the American military should stay in their country. These questions offer additional pathways to assess information, quality of information, and the scope of the respondents' perspectives within our sample. 

Given increased polarization in the US and efforts by Republican\index{Republican Party} and Democratic\index{Democratic Party} politicians\index{politicians} to distinguish themselves from the opposition, we also ask respondents about their views of the Democratic\index{Democratic Party} and Republican\index{Republican Party} parties in the United States. We conducted our survey\index{survey} during three years of the Trump administration\index{Trump Administration} (2018-2020), which provides a glimpse into what the world thought of the US in this period. Future work can carry out further surveys\index{survey}, and better assess the ``Trump effect.''

%We also ask a few questions about respondents' views about their country. We include ideology in a measure here, but asking someone about their ``opinion of the relationship between the US military and the government\index{government}'' of your country may provoke a different kind of response than we measure here. Likewise, we ask people if the presence of American military forces affects the quality of democracy, respect for human rights, public safety, or security from foreign threats in three separate questions. In trying to get a better feel for the texture of people's information and ideas, we ask them different questions to estimate the number of US troops in their country and question how long they think the American military should stay in their country. These questions offer additional pathways to assess information, quality of information, and the scope of the respondents' perspectives within our sample. While these are just a sample of questions and not exhaustive of the material scholars can explore, we also wanted to highlight a pair of questions asking what people thought of the Democratic and Republican parties in the United States. Our survey only takes place during three years of the Trump administration, and the answers to these questions will reflect that. Still, it also may offer some insights as to how individual ideological beliefs and beliefs about the US presence may map onto left-right ideologies in the United States. 

As extensive as this project has been, we are just scratching the surface of exploring what competitive consent\index{Domain of Competitive Consent} looks like on a global scale. The domain is still at an early stage of its evolution and is contingent on international affairs in the coming decades. In the immediate future, we hope to explore further how the factors we identify in this book shape views of Chinese actors\index{China}. Our survey\index{survey} can only offer a single side of the equation. Still, it is essential to know whether US behavior in other countries builds support for or opposition to China\index{China} in the countries where it is the sole presence. The United States increasingly sees itself as locked in an emerging competitive struggle\index{Domain of Competitive Consent} with China\index{China}, and China\index{China} forecasts itself as overtaking the United States in the next few decades.\cite{Doshi2021} It is natural to wonder if third-party civilians internalize a trade-off between these two actors based on the positive and negative experiences with the US presence within their territory. If there is no trade-off, then each country is purely building its own support or opposition in isolation. If there is a trade-off, then each act matters doubly for competitive consent\index{Domain of Competitive Consent}. Additionally, trade-offs may only manifest in areas where either actor is proximate or both parties are actively engaged in local political, economic\index{economic effect}, or security matters. Understanding the scope of competing for consent\index{Domain of Competitive Consent} requires unveiling more about how people build their preferences in the international system.

Second, we hope to explore Djibouti\index{Djibouti} in a follow-up survey\index{survey}. The triple military presence of France\index{France}, the United States, and China\index{China} will lead to different sets of cross-cutting beliefs and interactions. Does building support for one country build support or opposition for another country? Does a terrible interaction with one basing country lower support for the other two basing countries? How are local commanders restricting service members locally and handling altercations with local civilians? How do local commanders deal with conflicts between their and other country's service members? Djibouti\index{Djibouti} will prove to be a microcosm of shared basing sites in the future and is a natural next step from this project. %expand a bit more

Beyond major power competition, the deployment of troops by other actors have effects that we do not chart here. The deployment of peacekeepers by another country, such as France\index{France} or the United Kingdom\index{United Kindom}, or by another international actor, such as the United Nations\index{United Nations} or the African Union\index{African Union}, likely creates a different set of outcomes than US deployments during peacetime. Actors deploy peacekeepers with some hope of a temporary presence though deployments can last for decades. The purpose of the peacekeepers varies by their mission, but often are for internal security between two or more factions instead of external security. As such, a peacekeeping presence may evoke disdain from one or more sides of a conflict that resents the peace. Understanding how such a force can build support for its mission among those who both want and do not want the presence may provide fascinating insights into both theories of contact\index{contact} and understanding military deployments generally. Additionally, multinational forces may provide unique, cross-cutting experiences as people interact with different service members from different militaries. %add some peacekeeping refs

The questions we explore here and the greater issue of maintaining and building support abroad is a wide-open field of inquiry. While we touch upon the roles of noise pollution\index{noise pollution} and traffic\index{traffic} in affecting people's perceptions of the deployments, we are less systematic in assessing how much these factors matter in building those perceptions. Future work can thus focus on the specific types of interactions that are occurring. Some scholars have studied the environmental\index{environment} impact of the Department of Defense's\index{Department of Defense} overseas facilities, but we know that it is a process that is shifting. For example, the DoD has recognized that climate change is a challenge for maintaining security in the future. While this recognition is essential, it lends itself to further research on when military bureaucracies recognize environmental issues\index{environment} as salient and are willing to change their behavior. Plenty of ecological consequences\index{environment} result from basing and how those issues manifest in support or opposition to basing matters. Likewise, will the domain of competitive consent\index{Domain of Competitive Consent} force the US to be more willing to address environmental grievances abroad\index{environment}? In sum, once scholars and policymakers recognize local stakeholders as influencing the US security arrangements abroad, a whole wealth of questions becomes pertinent to understanding deployments' longevity. Previous work on deployment-adjacent issues becomes pivotal to whether deployments exist. %(cite Moon's working paper here?)

While we addressed the issue of host country ethnic minorities'\index{minority} relationship to the US military in \ref{cha:min}, there is much room to study race and ethnicity's role in US military deployments abroad. Our interviews\index{interview} with US service members in Lakenheath\index{United Kingdom!Lakenheath}\index{United Kingdom}, England\index{England}, showed that many had had different experiences abroad due to their race and/or ethnicity\index{minority!ethnic}. Existing work explores the undercurrent of race during deployments, with Moon providing detailed accounts about how the United States had exported its norms about racial hierarchy\index{hierarchy} to Korea\index{South Korea} during the 1950s-1970s.\cite{Moon1997} The United States did not finish integrating the armed services until 1954, and the process thus played out in multiple host countries globally. Of course, racial issues within the military did not end with integration and continue to be a subject of scholarly attention. As noted above, Freeman notes that (White-majority) European\index{Europe} basing hosts are allowed more judicial\index{judicial} power over service members who commit crimes\index{crime} while deployed than non-European states, many of which are majority non-White. Yet Freeman also notes that the US has cause to believe that some of its service members would not be treated fairly in foreign courts due to discriminatory practices in host countries.\cite{Freeman2021} In addition, as the US military grows more diverse and incorporates more immigrants\index{immigrant}\index{immigration}, the relationship between US service members and host country civilians who share a race, ethnicity, or ancestry with them will be an important avenue for research. 

Another topic that came up in interviews\index{interview} with service members was the role of gender\index{gender}. The traditional view of male service members marrying local women and frequenting bars to find sex workers\index{sex workers}, though still true to some degree, is slowly being replaced by a new reality in which women serve alongside men and become a more significant part of the military. Thus, US service members engaging in behaviors traditionally associated with women, such as getting manicures and fake eyelashes, create new opportunities for interactions with local populations. The changing gender composition of the US military also creates conflict in societies where women do not have equal status to men. Similarly, as LGBTQ\index{LGBTQ} service members grow in numbers, the image of the American military family abroad will change, and with it its interactions with local communities. %[cite interviews here][I just added this section, we should beef it up a bit]


Last, there is a paucity of data on military deployments abroad. Each advance in better, more fine-grained data allows researchers to understand better the causal processes related to and resulting from deployments. The explosion of quantitative literature after Kane's 2004 publication provided an easy to import data set on the number of US troops in countries globally is indicative of how data-impoverished this very important subfield of research is.\cite{Kane2004} For researchers, it should be increasingly evident that producing new data related to this topic, and publishing it in top field journals within political science\index{political science} (not just international relations\index{international relations}) should be a discipline-wide aim. Not only do new data sets serve to produce foundational insights into the topic, but they create a wake of subsequent research. As such, the field needs better data on deployment-related issues concerning the environment\index{environment}, noise\index{noise pollution}, traffic\index{traffic}, crime\index{crime}, pollution\index{environment}, energy use, land use, military exercises, facility usage, the economic\index{economic effect} behavior of service members, the civilian logistic tail of the military, race, gender, dependents, military rank, vehicle deployments, naval patrols, social media\index{social media} usage, and dozens of other topics that have real effects on both people and the theories that these processes support. 



To conclude, we expect that competitive consent\index{Domain of Competitive Consent} will become an increasingly important factor in understanding how basing plays out in the next century. Basing has moved from negotiation among political elites to domestic constituents on both sides of a basing arrangement. Understanding these processes will move the scholars, activists\index{activists}, and policymakers forward in understanding what makes a base more likely to endure and what makes it more likely to be a temporary arrangement. As actors become increasingly invested in basing to grow their regional influence, the United States will come to a point where its current trajectory leads to it facing a basing site shortage if it does not seek to compete in winning the hearts and minds of civilians during peacetime. Understanding the positive and negative externalities of basing and, more importantly, how people perceive those externalities is a step towards understanding the latent and blatant manifestation of support for the current United States' global objectives and whether the current post-World War II\index{Cold War} configuration will endure, reshape to a limited capital-intensive deployment, or fade away in deference for other security arrangements.
