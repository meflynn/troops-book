\chapter{Conclusion and The Future \label{cha:conclusion}}

%anecdote that sums up our experience/leads into this chapter


Our direction in this chapter is to provide an overview of our research in this book, project trends in the future, and establish research paths for subsequent scholarship. We begin by summing up what we think the state of basing looks like in the present decade and what our research tells us about how US decisions shape those trends. However, we believe the status quo of basing and deployment policy establishes distinct directions in foreign policy and international relations. We forecast a few scenarios that are likely to play out in three decades. Next, using our research, we discuss the different global political developments that may shape basing and basing attitudes in the less than seven presidential administrations. Given these scenarios, we take a look at the clear policy recommendations stemming from our work. Finally, there is always more science to do. We conclude the chapter with future research directions that include things we still want to pursue within this work and some research avenues we are pursuing in subsequent work. We also illustrate pathways the subfield of basing should seek to continue to answer deep questions about basing, their impacts, and whether and how they will endure. 

\section*{The United States and Basing in 2022}

The United States is at a basing crossroads in 2022. The last few presidential administrations have also thought they were at a crossroads, but their view of the paths is different than what we present here. For the last decade, US executive leadership has looked towards slimming down the US presence for a few reasons \cite{Woody2021}. These include a response to the increased vulnerability of US personnel to possible terror attacks, to decrease the likelihood of interactions producing catalytic moments against US deployments, and a continued shift towards capital-intensive military forces. These forces have led decision-makers towards reducing the footprint of US bases overseas in the hope of maintaining enduring deployments of the current US basing network. 

This strategy, in our estimation, is a misdiagnosis of the core issue for both the present and the future. As we discuss in Chapters \ref{cha:theory} and \ref{cha:meth}, the interaction with host-state civilian populations is a multifaceted event that has a whole set of negative and positive pathways. While there are negative interactions that seem to be the catalyst for anti-basing movements, there are also several positive pathways extending from interactions with military personnel and the economic benefits of a US presence. These interactions are a passive, normalizing force that builds support for the United States and its mission within a country. Additionally, there are qualitative accounts that support how these pathways can have pacifying effects on anti-basing opposition. First, as our interview with activists in Berlin demonstrated, recruitment against basing is often more challenging to do near a base than in metropoles that do not have these interactions. Second, people surrounding a base come to economically rely upon the presence of soldiers for local service sector jobs, retail, and real estate. Likewise, as \citeasnoun{Gillem2007} discusses, bar owners in South Korea actively stifle opposition surrounding a base as it is terrible for their own business. To be clear, Gillem is not listing this as a positive outcome as it another facet of the oppressive nature of bases. Still, it is an civilians, as they collectively favor or oppose deployments and bases.

There is a belief that Chinese construction firms overseas are insular and only hire Chinese laborers for local projects. President Obama even warned African nations that they should make sure Chinese construction projects use local labor \cite{economist2014}. If the lack of using local labor were a persistent issue that typified Chinese overseas projects, especially in regards to its major push through the Belt and Road Initiative (BRI), then the United States may have a unique edge in providing economic incentives to local populations and building foreign support. This would produce an optimistic finding for the evolving domain of competitive consent for the United States as it would maintain at least one unique and active way of generating positive despite it creating more insular bases overseas.  However, the data does not support this optimism.  When evaluating different firms, research shows that between 70-90\% of Chinese hiring in Africa are local laborers \cite{Chiyemura2021}. As such, the United States does not have a unique edge in filtering out economics benefits to local communities even though local bases may focus on hiring local. 

The consequence of pursuing capital-intensive deployments that draw down the personnel is that many of the negative costs of deployments will remain. The costly nature of noise pollution will stay as the military will continue to station aircraft for all military branches, and drones remain a noise nuisance \cite{Schaffer2021}. Base-related traffic will continue to be an issue around bases, though, perhaps less so. Environmental consequences of the capital-intensive technologies will continue and possibly surpass the pollution required for maintaining military members. While particular personnel events have appeared catalytic, the underlying externalities of bases have caused opposition to their presence, not just the people. Removing a significant percentage of military personnel from a base is unlikely to remove the resistance to deployments. The United States' current reformation of the military basing structure will not solve the primary problems.


This problem becomes more acute when we return to the evolving domain of competitive consent. While the United States is not actively competing with other basing powers for access rights to various countries in 2022, that domain is just starting. Having to share operational space with an expected competitor in Djibouti fundamentally changes and shapes what the US is capable of doing in Eastern Africa. Not only does the US need to consider strategic movements in both air and space domains when conducting routine operations in Djibouti, but off-base behavior by personnel may be more complicated. The Chinese People's Liberation Army support base is a thirty-minute drive from Camp Lemonnier, as both bases are near the capital of Djibouti City. Altercations between service members and host-state civilians is undoubtedly an international incident, but what about an interoccupational altercation between Chinese and US troops \cite{Bryant1979}? This prospect may give more reason to make sure personnel in Camp Lemonnier stay within the confines of the base and do not venture outside of the camp when off-duty, but this would cede any positive soft-power building opportunities to China. A situation where Camp Lemmonnier remains off-limits to most Djiboutian civilians, American personnel shy away from any interactions while deployed in the area, and Chinese troops frequent local establishments, live among civilians. Their kids attend schools with locals builds far more trust for the PRC than it does America. Already, the United States has been losing the soft-power game economically in Djibouti as Chinese connections to the country have helped to build economic infrastructure locally. At the same time, the United States mostly pays rent for its base \cite{Bearak2019}. Of the three basing states in Djibouti, the United States has less history than the French and fewer ties than the Chinese; if a popular movement against basing were to rise and demand action by the government, the United States would likely be the first to lose access.

In 2022, the United States has not fully understood the increasing influence of foreign domestic consent. However, there are several obvious signs that we point to throughout the book. Starting with the George W. Bush Administration, discussing plans to redeploy troops from legacy deployments in Germany, South Korea, and Japan faced both support and opposition \cite{Sang-Hun2006,Sang-Hun2018,AssociatedPress2020}. Especially in Germany and South Korea, sustained demand for a continued presence and a discussion of considering withdrawal has provoked backlash at both elite and domestic levels. Japan and South Korea pay billions of dollars in burden-sharing to maintain troops within a country as the governments of these countries see a need for a continued US presence \cite{Flynn2019}. They are willing to pay to maintain those troops; however, this demand was borne through conquest and war (for Japan and South Korea, respectively) and seven decades of US-host state ties. The US investment in all three countries to help build them up post-conflict looks more akin to what China is doing in Djibouti than what the United States is doing in Djibouti.

%the two below paragraphs might need some more substance/thoughts.

In thinking about how bases affect minority groups in Chapter \ref{cha:min}, the United States appears not to do as well as it does with majoritarian groups. Our qualitative interviews reinforce the idea that the US has not focused on making in-roads to minority groups within other countries. Given the rhetoric of the United States in promoting human rights globally, this might be an area where the US could do better to engage historically excluded groups in other countries. We find that minority groups tend to be more neutral to the US presence than holding opposing views. It seems as if a latent group of stakeholders could find support for the United States (and channel that support to politicians) if the United States gave them a reason to do so. As of now, the United States is leaving potential support on the table. This relationship contrasts quite a bit to areas where it has actively provoked the ire of minority groups like the Okinawans in Japan.

In chapter \ref{cha:crimes}, we turn to the most well-known adverse event that influences perceptions of the United States by looking at the role of crime by service members. Crimes and crime coverage do correlate with negative perceptions of the United States military, government, and people. When controlling for crime, we find that the military presence's contact and economic benefits still do matter. However, the post-9/11 trend of creating isolation stations in bases to segment American populations from host-state populations trades off the negative effect for two positive results while keeping other negative externalities flowing into local communities. Thus, eliminating crime is not a possibility but building better institutions to reduce its prevalence is feasible. 

Finally, we turn to mobilization in \ref{cha:protest} and discuss the causal pathways that lead to protest and removal. In 2022, the United States seems to treat anti-basing events as something it can provoke and avoid through isolation. However, this strategy is unlikely to change the nature of anti-base activism, and the United States seems ready to roll the dice in whether protests mobilize in success or failure. It has experienced both, though, anti-base opposition that fails at one point in time may regroup and grow stronger at a later point in time to continue the campaign. Disengagement allows anti-base groups to continue to try their luck without building the conditions for pro-base groups to build a successful counter-narrative. Presently, the US is in a tenable but diminishing position. There exists a few threats to the status quo of mobilized anti-basing opposition including if China keeps expanding beyond Djibouti, waves of nationalist anti-imperialism take hold in states that host US bases, the European Union decides to construct its military alliance to exclude the United States, or Russia pushes its sphere of influence out further. In each of these situations the US will be fully competing for the consent of the governed. Even just subnational protests could unravel the United States position if anti-basing movements become transnational points of pressure against the US \cite{Cooley2020}. Forecasting these scenarios out additionally, we turn the evolution of the domain over the next thirty years. 


\section*{The United States and Basing in 2055}

To contemplate how basing trends in 2022 translate to three decades in the future, we consider three different scenarios in global politics. We first consider a scenario where the US can maintain its relative global position. Second, a future where a new bipolarity rises with the People's Republic of China. Third, we conclude with a final possibility of a multipolar system with China and Russia rivaling the United States. 

\textit{Scenario 1: The US maintains primacy}

We start with the scenario that has the least amount of change required for United States global strategy its necessary subcomponent of basing. There are a few scenarios where the seemingly inevitable ascent of China does not occur. First, it is possible that the United States keeps pace with the PRC and, while its edge erodes to some degree, China does not appear as a global contender for world power. \citeasnoun{Doshi2021} outlines a strategy of blunting, building, and asymmetry that the United States may pursue to curtail China's ambitions. Second, the PRC's growth in economic and military power may slow down or collapse entirely. Some critics of China's growth point to artificial projects (such as the development of ``ghost cities'' where the government builds entire towns but no one is actively living in them) and inflated GDP numbers to suggest a mismatch between reported and actual growth in China \cite{Yu2014,Roy2020}. Reconsidering the actual trajectory of China may lead to some backsliding from current projections. Third, the United States' project of ``congagement'', where it economically engages China while militarily containing the country, is relatively successful, and the PRC does not emerge as a rival, but as a supporter for current economic and international institutions globally \cite{Khalilzad1999}. In this scenario, China becomes the heir-apparent and mentee to the current global system. 

Fourth, building on the previous scenario, if China does surpass the United States in leadership, there is a transition that is more akin to when the United States passed Great Britain without conflict between those two powers. For power transition theory, a satisfied rising power that seeks to maintain the status quo of the international order is less likely to provoke conflict with the current dominant state \cite{Organski1980,Efird2003}.\footnote{It would be erroneous to suggest that this transition as peaceful as the transition in global leadership between the UK and the US happens after two disastrous wars, but those wars occur with the potential competitors as allies and not adversaries. Additionally, there is not a scholarly consensus on whether such a transition is even feasible \cite{layne2018}.} Fifth, even if China continues to grow, it is possible that it never can fully don the mantle of global leadership that scholars argue is requisite for power to manifest in hegemony \cite{kindleberger1973}. This inability to lead would result from a failure of its soft- and hard-power projects imbuing it with other states attaching itself to its support or ideology. 


These five scenarios all have different subjective likelihoods of being realized, with all seeming to be unlikely in the present writing of this manuscript; however, they are feasible. For example, the conclusion of the Vietnam war, oil shocks, and stagflation all seemed to make it clear that the United States was in decline and was on the way to exit its great power status. The 1970s and 1980s were also fraught with concerns about the decline of the United States compared to the economic and political rise of Germany and Japan as potential powerhouses; of grave concern was whether either state would seek to also become a military power \cite{maull1989}. Pundits and scholars in the 1990s and 2000s saw India and China as the next two possible global contenders for great power status \cite{gupta2006}. Since those early concerns, India has faded away in much of the discourse about coming great powers with only a handful of research considering the possibility of India becoming a great-power state in the near future \cite{pant2007,carranza2017,narlikar2019}. %cite hegemonic decline 1970s lit?

In terms of thinking about basing in the 2050s, however, we should consider these unlikely scenarios as the United States in a relatively unprecedented position globally.\footnote{Of course, there are parallels to 19th century and British hegemony, but the texture of the United States position and the mechanisms of it follow different routes than the 1816-1916 period.} In this situation, the United States is still in a weaker position than it was in the early 1990s as, while it is not dealing with a countervailing political force, it needs to navigate regional powers in Europe, Central Asia, and East Asia as a (perhaps) more consolidated European Union, Russia, China, and India wield influence with their neighbors and allies. Even if only one or two of these states (or a collection of states) offer regional push back to US aims, it does change the political and military environment for the United States. This is a key point that \citeasnoun{Cooley2020} label as a ``challenge from below'' where international primacy can falter from actors other than great powers.

There are a few factors to consider in understanding how the United States would face significant constraints without a rival global challenger. First, the domain of competitive consent will remain especially acute when the United States needs to compete with regional powers and hegemons instead of being the only real power in town. It does not take a global power to set up a network of bases. After all, Russia maintains a handful of foreign deployed bases in 2022. Second, as regional powers seek to have a more substantial say in military affairs in a region or have access to airspace or ports, they will inevitably reach out to their neighbors for access rights basing arrangements. Some subset of likely partners for these regional powers will be places where the United States traditionally has bases. If the budding regional power is adversarial with the United States, or the new security agreement would replace the United States' security provision for that state,\footnote{\citeasnoun{morrow1991} argues that strategic security alliances between strong and weak states offer a trade-off for both countries. The more vulnerable state gets to outsource its costly security provision to a stronger state, and the stronger state gets some control over the state's foreign policy. It trades security for autonomy. Under those conditions, additional security may be complementary, but major power security provision is more likely to be a substitutable good. In these cases, the United States could lose out to ideologically aligned allies, such as how the United States has supplanted Great Britain in several of its former bases.} then the US is forced into a position where it has to demonstrate that its provision of security is superior to that competitor. Ultimately, if a government needs to respond to civilian pressures due to a legacy of harm inflicted by a US presence, then the United States will lose its position.

Second, the United States may have to compromise on its exclusive access in several states. For example, Djibouti hosts three basing states (the United States, France, and China) and hosts other forces on occasion (Camp Lemonnier also has British military personnel). This shared basing arrangement may increasingly become the norm as regional states may wish to curry favor or be under the umbrella of their regional hegemon. The US will increasingly have little say in these arrangements as US requests for exclusivity may result in its exclusion. If a central Asian state must choose between hosting the PRC or the US, the US can lose out to the more proximate neighbor. Leaving a territory certainly limits operational freedom for the United States. Alternatively, the US may have to sacrifice autonomy for access.  Shared living spaces impose further restrictions as the United States may have to negotiate airspace and port rights or limit its exercises and operations due to the risk of conflicting with other forces. 

A reduced capacity in either situation limits the United States to be an active player in several regions. If it can maintain some allies in each region or have access to important seas and ports, it may still be able to project power at a reduced capacity. Importantly, as the Department of Defense and the Biden administration change the US basing posture from one of the massive legacy deployments to more minor, flexible bases with smaller footprints, this may reduce some associated harms. However, it diminishes two aspects of basing for the United States. First, drones and missiles are not substitutes for ground forces. While drones are effective at quick interventions or targeted assassinations, but serve as a less credible tool of assurance to allies \cite{cypher2016,mehta2019}. Depending on the kind of influence and position the United States wants to have in a region, the ability to credibly threaten invasion against or a standing army defense of will be less credible without such forces in a region. Second, the associated pathways of support such forces have, which we review in detail in this volume, disappear if bases become drone operating centers instead of military outposts. To wit, as we report about Ramstein, drones are already unpopular among allies and have no capacity for building positive connections with local communities. They do not integrate into local communities, go shopping at grocery stores, patronize nail salons, and coach little league soccer. Members of the military do.

In the 2055 scenario where the United States maintains its global position but must contend with regional powers, it will continue to face a situation where competitive consent is the underlying force enabling and disabling its basing presence. The places where it occurs may be more scattershot as there is no ideological rival seeking to convert and subvert states to their preferred system of government or economics. However, the US still loses ground as regional security arrangements become preferable over the mobilized protests against the United States. The U.S had lost bases to popular discontent when there were no natural alternatives to the United States. When choices become increasingly attractive, the US is likely to draw down its presence even further. 


%cite above claims

\textit{Scenario 2: Bipolarity with a rise of China or a decline of the US}

A rising China that becomes a global competitor is a more common scenario for those forecasting what the international political situation will look like in the future; however, there is an outstanding debate over how conflictual or cooperative that bipolarity will become \cite{maher2018,ross1999}. Though, for some, 2055 might be too soon of a period to see China ascending to global power. Within its own planning, China hopes to achieve both a period of ``national rejuvenation'' and become the most powerful country in the world by 2050 \cite{Doshi2021}. The Correlates of War project has a metric that it uses as a proxy of power. The Concentration in National Capabilities index (CINC) measures six sources of power, aggregates them for every country in the inter-state system, and then produces a share for each country in the international system dating back to 1816 \cite{Singer1972,Singer1987}. The United States eclipsed Great Britain in 1897. Still, most scholars of international affairs do not put the United States as a global power until sometime between 1919 and 1945, suggesting a 40-50 year lag between the indicator and when the United States began acting like a world power. According to this index, the PRC passed the United States in 1995; if China were to follow a similar trajectory as the United States in terms of assuming global power status, then 2055 would be a conservative estimate. Charting the rise of China as akin to the United States would be an oversimplification of great power ascendancy, but it is at least an interesting comparison to draw. 

The bipolar competition between the United States and China is likely to give the United States the biggest challenge for its current network of bases. While China has thus far only publicly looked at sharing access rights in Djibouti and artificial platforms in the South China Sea, it has several basing and basing-like projects ongoing at the time of this writing \cite{Doshi2021}. China will likely expand its basing network not purely due to power but as part of the drive to grow its power. Since World War II, the United States military position rests largely on its ability to project its might in multiple regions simultaneously. While only a fraction of the United States military is in any given region, the presence of its forces affords it the implication of additional force should tension escalate. If China seeks to become a global contender, it will likewise need force projection capabilities in land, sea, and air. While creating and sustaining smaller outposts to support drone operations in other countries might be something we observe, such behaviors are less about force projection and more internal stability. Such outposts likely target terrorist threats like the United States.

If China continues its path in Djibouti, then other regional African states are not far behind. China's use of soft-power enticements and military hard-power guarantees to bring more countries into its network. The United States is likely to escalate its position in Africa. It had already stationed more troops on the continent in 2020 than it had a decade prior and is likely doing so in anticipation of further Chinese military investments in the region. China will likely seek to expand in its far neighborhood as well \cite{OSD2020}. It will further transcend being a regional power into a global player if it can garner agreements in the Middle East, South Asia, or Africa \cite{yung2014,cabestan2020}. Its current holdings and attempts at gaining ground in Asia, Central Asia, Africa, and even Europe demonstrates that this future is immediate \cite{Doshi2021}. Ultimately, if the United States could not stop development in a country like Cuba or Venezuela, a base in Latin America would signal the erosion of the United States backyard that it has maintained almost exclusively since it established the Monroe Doctrine. 

On a region-by-region basis, the United States will be looking at opportunities for China and risks for its position. Autocratic countries may be more likely for Chinese basing sites as the United States may face domestic opposition to basing in acceptable regimes during the Cold War. By 2031, the post-Cold War era will have lasted longer than the Cold War itself. The limitations and the permissiveness of the Soviet-US bipolarity will be an increasingly distant memory for the general public and, with a few more decades, the leadership of most countries. If the public constrains the US leadership from securing sites in illiberal regimes, then the diplomatic cost for China to do the same will be lower.

Additionally, the cost of maintaining US bases will increase from such a limitation for a few reasons. First, suppose the US does not have credible alternatives. In that case, host-states can effectively increase the cost of basing unilaterally, and the US will face either paying more or losing regional access. Second, if the US maintains its current policy of isolated bases, the cost of those bases will have continued to accumulate with limited benefits flowing directly to regional populations. Such conditions mean that for a host-state government to make its population content with the US basing arrangement, it will have to contribute less to burden-sharing or demand more from the US when it renews leases. As the cost for deployments goes up, the global position of the US becomes less tenable. 

These conditions set up a potentially vicious cycle for the United States. The US chooses to pay more, scale down bases, withdraw from essential sites, create better basing islands, or combine all of those. While following this strategy, drone usage will remain unpopular, noise pollution will continue to plague local communities, traffic will congest local streets, environmental consequences to bases will still spread, and people will increasingly become suspicious of what the United States is ``really doing'' behind its fortified walls. With reduced positive efforts to passively win over the hearts and minds of allied, civilian populations, then the negative externalities continue to flow unabated or, perhaps, managed to some degree. Some states may opt not to have a US presence within their country. Still, other countries may fill it with regional allies of regional allies, and others will increase their security cooperation with China. When the Army loses access to a country, the remaining listening posts, fuel depots, drone command centers, and similar operations offer a reduced capacity for US power projection. % might be redundant with a previous point?

Part of the equation for the United States will be how tenable its position in the Middle East and Central Asia remains. Through the 2000s, there was a shift from bases in Saudi Arabia to new deployments in Afghanistan, Iraq, and maintaining its position in Kuwait \cite{Engelhardt2009}. Decisions over the last five years have shuttered the permanent bases constructed in Iraq and Afghanistan as the US sought to exit its security arrangements in both countries despite the pursuit of establishing democratic regimes in both countries. In thirty years, will Bahrain and Kuwait remain palatable deployments for the US public, and do either country face democratization prospects in the near future? Either pressure severely limits US influence to a few sites near the region. As discussed previously concerning the Iraq war in 2003, Turkey voted against the US using its bases. Tensions have varied in the last decade with the country regarding the fight against the Islamic State, the status of Kurdish forces, and Turkish procurement of Russian weapon systems. Alternatively, the US can increase its deployment or negotiate new bases with Saudi Arabia, Qatar or become increasingly reliant on Diego Garcia in the Indian Ocean. The Chinese leadership will continue to have fewer qualms about strategic alliances with autocratic countries. Expansion to wield influence over the Gulf of Guinea or Southern Africa is already on the PLA's agenda \cite{cabestan2020}. It is foreseeable that continued strategic cooperation between China and Iran leads to great access to the country's People's Liberation Army \cite{Cordesman2021}. 

In essence, the height of 800 overseas assets that the United States had in the 2010s will decline significantly in this area. While China is unlikely to pass the United States by 2055, its access to regions will expand dramatically as its ties in Africa, Central Asia, and East Asia become stronger. Thus, by 2055, in an era of bipolar competition, there will be concern about how soon China will surpass the US in basing sites. 

\textit{Scenario 3: Multipolarity with Russia and China's relative rise}

The final scenario we discuss is likely to follow a US relative decline where Russia and China boast significant economic, diplomatic, and military power in their spheres of influence. While many commentators have considered the rise of middling powers to create a multipolar world (especially the BRIC countries of Brazil, Russia, India, and China), it is China and Russia that are the most likely to become great powers and compete with the United States in basing \cite{cooper2013,Mattis2018,Cooley2020}. Russia deployments in mostly Central Asian countries may expand to include more Middle Eastern positions besides Syria. Additionally, Russia may seek additional influence in sub-Saharan Africa. China, in this scenario, extends beyond Djibouti as well but is only able to make more modest gains by 2055. While having two competitors for global politics may be more daunting for US access, it does afford some opportunities for the US For Neorealist thinkers, tripolarity is an unstable configuration as balancing is hard to achieve among an unequal set of states, but this does not make China and Russia natural allies \cite{Waltz1979}. As the last few decades of the Cold War demonstrated, the United States, at times, can effectively encourage Sino-Russian rivalries \cite{goh2005}. Bilateral relations between Russia and China have been contingent on US relations with both countries in the post-Cold War era \cite{ferdinand2007}. Basing is another realm where the US may be able to effectively balance both countries against each other in a multipolar system. For example, Russian use of the Cam Ranh base in Vietnam has been a minor affair in the last decade, but if Russia and China seek to expand their facilities, Vietnam may serve as a source of tension between the budding powers \cite{yen2021}. Vietnam's rivalry with China may make it more likely to oppose expansionary operations by its largest trading partner. Still, China may be willing to ease its side over disputed access to the South China Seas under a new military cooperation agreement. Alternatively, Russia will seek to reactivate the port and airbase to regain access to this part of the globe. A move by either country in this direction will likely put the countries at odds with each other.

Vietnam is not the only territory where they may contest over security expansion. The portfolios of possible basing partners in Central Asia, the Middle East, and Africa are similar. Failure or success in one theater may spill over to diplomatic contests in other regions as both powers set up their networks. In some cases, a joint basing arrangement might facilitate cooperation or acceptance, though deployments to the same country increase the likelihood of conflict between those basing states. If competitive consent becomes a contest for states seeking authoritarian partners, the US position may come out stronger than it would under a bipolar competition with China. However, it also feasible that joint competition pushes China and Russia to seek states that the US already deploys to or may seek deployment to in the coming decades. These competitive spillovers jeopardize the US position. While competition is healthy for a host state, it is costly for a basing state to win the support of local elites or domestic constituents of those elites. The United States and Russia learned this lesson in 2008 when Kyrgystan forced the two countries into a bidding war for basing rights. Russia paid \$300 million in aid as a result (of a promised \$2.1 billion promised) and the US' rent more than tripled for its existing base \cite{Cooley2020}.

In sum, multipolarity offers more opportunities for the United States than bipolarity. However, with the addition of competition between two of the basing powers, the possibility that those two powers ally to the detriment of the United States is a reasonable path. Under these conditions, the success of the United States requires smart diplomacy and strategy that is not a given but a gamble. Whether the United States leadership in 2055 is up to the task of a global game of dual-containment is not foreseeable until we are already living in that era. 

Beyond the three scenarios we highlight here, there are several other possibilities; however, our goal here is to focus on the immediate ones that seem to be the most likely as we write in 2022. For example, some forecasts may want to consider how a rise of India or a more consolidated European Union (with some military force) might alter the need for US bases if there are more aligned states fulfilling a similar role. Such a scenario may make previous calls for offshore balancing more appealing if other capable global powers can replace the US in a few key areas. Such scenarios strike us as more far-fetched and create a set of unique principal-agent problems that might be worth exploring in works more dedicated to such a task.

\section*{Policy Recommendations}

As we have discussed throughout the book, there are some clear policy recommendations for those that have control over bases, basing policy, and how service members interact with local communities. However, we take an indirect route to discuss these concepts a bit more explicitly. In the mathematical study of game theory, one of the powerful tools we use to understand optimal decisions when interacting with other people, organizations, or states is backward induction \cite{myerson2013}. Backward induction is a simple tool that requires a strategist to look down ``the game tree'' to see what people would decide at various terminal decision points and infer what decisions actors should make based on forecasting those subsequent decisions. Using this game-theoretic solution concept as a metaphor, it is essential to understand what anti-base activists have to learn from our research and how they can, and already do, incorporate lessons from our study to facilitate anti-base activism better. By understanding how activists mobilize people to their cause, policymakers can only create policy that can weather attempts at increasing domestic opposition to bases. 

To accomplish this task and fully understand the implications of our research, we first turn to activist strategies.

\subsection*{Recommendations for Activists}

Anti-base activists already mobilize people to their side by pointing to the harms of bases within their region. However, mobilization research suggests that having generalized harms is not enough to get people to take costly actions. Even something as simple as voting or protesting requires time and resources to accomplish. In some states, participating in either of those may risk violence, imprisonment, or even death. Selective incentives for participation are a tried and true method of gaining supporters for your cause, and the cause against basing is not different in that regard. Selective incentives are individualized rewards for people who can range from simple social symbols to monetary payments \cite{Olson1965}. Scholars are still understanding the role social media has in uniquely overcoming collective action problems as it is clear that it enables action in novel ways \cite{saebo2020}. For our understanding of activist dynamics, the act of posting pictures from a protest or a picture displaying a selective incentive received from donating to or participating in a movement has two effects. First, it gives the person the ability to post content online and advertise their dedication to a cause they care about. Second, it is an advertisement for the group. Consumers of social media content see the action the poster participates in, their online support for their activism, and the direct reward for doing so (online status and material stickers or pins).

Likewise, people seem keen on the idea of status and promotion. Opportunities within an organization to achieve a job or title confer monetary or social benefits from participating in the organization. These opportunities are often in short supply due to the resource intensity of having paid positions within an organization, but giving out titles for certain levels of commitment can be both cheap and effective. Additionally, having roles for people to fulfill is a probabilistic selective incentive where people may over-contribute to compete with others to achieve that position in some organizations. Once award, however, that incentive evaporates unless previous commitment has generated enough volunteer buy-in that they remain committed to the organization or there is some possibility of turnover and the position becomes available again. With limited resources, rotation of paid jobs may be a strategy to concern.

Selective incentives apply to a wide variety of social, political, and economic organizations. The discussion thus far is generalizable to a large class of phenomenon, and it is important to remember that multiple groups are playing the same group. Activism is a competitive game where there are numerous different organizations around a specific topic and various causes that all request potential activists' finite attention and resources. In addition, the government and the United States may play the opposing role if it expects that an anti-basing movement is gaining traction to its detriment. It can provide selective incentives to support the base through jobs, celebrations, and other activities that effectively encourage people to support the presence of the base to continue to benefit from the goods that it provides. Actors likewise can provide selective disincentives for activism and coercion by the state through arrests or violence, which many states have taken in the past to discourage activism.

Beyond the generalizable research on mobilization and activism, our research tells a bit more about basing in particular. First, it is clear from Chapter \ref{cha:crimes}'s focus on criminal offending that stories matter. People who have stories of victimization within their social network are more likely to have negative views of the military. They can only really know whether that exists if people are telling their stories. While focusing on high-profile cases in the media can mobilize people en masse to a cause, it is the personal connection that motivates our respondents. An organization may benefit by drawing those stories out. While our results focus primarily on crime, it would be reasonable to extrapolate that other externalities also convey knowledge of harm to those that listen. In that way, these stories act as a probabilistic selective disincentive, the opposite of what we describe above. If enough stories proliferate, it becomes a matter of when the military presence might victimize someone instead of when. Crime victimization is not a lottery anyone wants to win and may help mobilize support to avoid that outcome. Likewise, encourage official reporting of such incidents and giving people the tools and knowledge to know reporting matters even if such cases go to military courts instead of civilian courts. 

The military's push to isolated fortresses away from the public shows that civilian voices matter and that the US military has responded in a global matter. It is common for groups to see such public measures as defeat. A base moving to a remote area with little off-base personnel means that the base is still present, that there are still public consequences flowing from the base, and new towns will emerge around the base. However, given what we have written throughout this chapter and this book, this move makes the US military's job harder at maintaining communities of support for its mission. Such anti-basing struggles are likely to be an enduring struggle, and complete victory will only likely come after several successes in courts, through public opinion, and legislative bodies. 

One possible conclusion would be to advocate for a version of accelerationism \cite{laurence2017}. In the context of basing, accelerationism would advocate for the idea that increasing the harms of bases would strengthen opposition to them and cause a US withdrawal. However, we think this is a dangerous prospect for several vital reasons and do not think it will lead to the success that proponents of accelerationism usually advocate. First, when given a choice between helping people under duress and allowing their harms to continue, the ethical choice is clearly to help those people despite the longer-term goals. Second, it is feasible that the threshold of change is either more extreme than accelerationists expect or does not actually exist as more supporters rally to defend a more damaging status quo. Third, the expectation of a political rebound that is equal to and opposite from the direction of the existing harm does not appear to occur in reality. Finally, we have seen moderation as a response to extremism in several critical moments in political history, and expecting that suffering will bring redemption is far from guaranteed. Instead, advocacy movements should focus on small victories to build coalitions that can endure long struggles. With that advice, such organizations should be aware that concessions may undercut a campaign, and building for resistance is complex for many successful activist groups. 

Related, there is likely to be a discussion of the ultimate aim of any anti-base movement as it seeks allies across the political spectrum to build support. Research suggests that the most enduring and successful movements are those that form a broad base of supporters as opposed to those that pigeonhole themselves to a particular segment of ideological thought \cite{Yeo2011}. In that aim, the envisioned alternative by the movement likely matters. If the campaign is nationalist and anti-imperialist, then a focus on domestic military institutions and self-reliance may be more successful at building a left-right coalition against basing. Being a country under active threat from another power may make this appeal more successful as well. Anti-militarism movements that seek a reduction of military activities may have a more challenging time building a broad movement. A targeted movement that seeks its advocacy in the wrongs committed by the US without a more general goal may be co-opted by domestic militarism advocates or those seeking a more robust military relationship with other states that may prove as harmful or worse than the United States, such as a former colonial power, a regional organization, or a budding regional power. %is the yeo cite correct here, or am I thinking of someone else? Cross-check with the protest chapter.

Turning to the other side of the equation, we engage those that wish bases to endure. 

\subsection*{Recommendations for Basing Advocates} % this section needs to be stronger/more fleshed out.

To start this section, we will say something a bit counter-intuitive to the audience reading our book: Read the previous section. While this may seem obvious, there may be a natural inclination to skip the last section as it does not directly speak to those that support bases. However, as we began this section on advice, to understand what one needs to maintain their support, one should understand those seeking to diminish or end it. We could write much of the advice in this section to understand what base opponents desire and their tools to achieve those ends. Understanding opposition is vital in understanding how to meet it, compromise with it, and find a path forward that allows you to achieve your policy and security objectives. 

Given this direction, we are writing this section from the perspective of someone who might influence US basing policy and not pro-base activists as the latter can do much of what we suggest above but for the opposing viewpoint. As pro-basing policymakers or decision-makers that control local policy (such as a base commander), broader decision-making across multiple bases (within a service or the Department of Defense), or from a civilian perspective that influences legislation or executive choices in basing abroad, there are some straightforward ideas that we have iterated on throughout this manuscript. In that capacity, it is clear that closing off bases to civilians and limiting military personnel activities off base can have long-term consequences for the US presence abroad. Historically, host-state civilians were far more likely to meet an American GI than any American diplomat or politician. The military personnel overseas served as American diplomats that both directly enacted the US mission overseas and became passive and active advocates for the broader things America sought to secure globally. 

This aspect of service overseas ought to be seen as a feature, not a bug of deployments if the United States seeks to maintain its dominance in overseas basing networks. It is harder to see the effects of benign, routine interactions as they exist as a passive force that draws in local communities to support the US presence. The explosive scandals involving service member misbehavior garners far more immediate reactions, coverage, and opposition that seem to undermine the US mission, but only examining one outcome of this equation likely misinforms current policy. Additionally, we found an ongoing theme of civilians and journalists being concerned about US covert behavior since civilians have less access to bases than they did in the 1990s and early. Shutting down avenues that have historically built the trust of the US military while also having security measures decrease access is creating the conditions where locals experience negative externalities and can believe the worst about what happens inside and outside of military bases.

Our proposal is not unrestricted access for civilians or unrestrained overseas behavior for military personnel. Instead, it is to recognize that individual has generated value by living their lives in local communities for the last 70 years. In recognition of this utility and how it translates to local, regional, and national support for the United States to maintain bases and for national governments to engage in costly burden-sharing for those troops, then the decision of how to deal with troop-driven conflicts off of base requires reassessment. The fortress of solitude with self-sustaining fast food, housing, malls, grocery stores, bars, and everything else a service member need seems like a quick and easy fix for the bad press that happens out of bases. However, suppose the United States invested in better systems of training, maintenance, and monitoring of US personnel, a more costly endeavor. In that case, one of the most valuable assets of overseas deployments can sustain itself. As we have mentioned a few times, headline-grabbing behavior will not disappear even under the best social institutions. This approach may require revisiting Status of Forces Agreements to allow civilian trial over crimes that are the most damaging to the social fabric of overseas communities and active engagement in restorative attempts of justice for when a member or an institution fails the larger military. While painful, a good faith effort will go longer to having local groups see the military as an institution for harm and purely seeking to protect its own.

It should be clear that the US presence will continue to face opposition in nearly all countries it deploys. Universal acceptance is impossible where the very presence of a foreign military creates both winners and losers. Additionally, the US military is a guest, sometimes invited and sometimes uninvited, in many countries. For some segments of the population in each country, the guest has overstayed its welcome. Creating the conditions where a critical mass of politically influential activists can convince host-state legislators to restrict and evict the United States is what the military should seek to avoid by using passive and active routes to create support for its condition. 

Presently, Djibouti is an experiment as to how different militaries within the same location build support for their presence and, based on the ground game and economic support that underlies existing deployments, the United States is losing out \cite{Bearak2019}. This particular case represents an opportunity where the United States can see what is not working, experiment with different local strategies, and evaluate how service members build support in an increasingly competitive consent environment. 

\section*{Research Directions}

Our research is exploratory as much as it is hypothesis-testing our expectations about deployments. We still have several remaining questions that can fill the careers of social scientists. These processes support a global project that started in 1898 when the United States established its first overseas base. Though, the project as a whole is much older as the United States established bases in indigenous territory during westward expansion and empires during and prior to the United States used externally deployed bases back to the first Western democracies and republics in Greece and Italy. However, much of the hitherto work saw bases and deployments as a consequence of empire, foreign policy, hegemony, bipoarlity, or some other globally shaping condition and less as a microcosm of people and interactions worth their own study. There are notable exceptions, but these were the exceptions and not the norm of International Relations work when it came to understanding the role of security commitments by states. Scholars will continue to research how these microfoundations of US foreign policy ripple through global communities.

Within the research we report here, there are more questions our survey answer that we invite others to follow up on as they search for similar questions about the role of the US deployments overseas. Our original conception of this manuscript included an additional four chapters of content, much of that focusing on case study analysis of particular countries in profile, but this volume was already beyond an acceptable length. Several questions can serve junior and senior scholars' agenda within the survey itself as they search to under the complex dynamics we begin to explore here. To reiterate a point we make earlier in the book, all of our data remains publicly available for other scholars to pick through.\footnote{All data, reports, and links to published research are available at \url{http://ma-allen.com/military-deployments/}.} A few questions that remain unexplored here provide further depth and understanding about our findings here. For example, in asking about crime and crime victimization within a respondent's social network, we include questions that specify the kinds of crimes they or their network experienced. This deeper view may further disaggregate crime as to how it affects people's beliefs. Property-related crime, for example, maybe seen as a lower offense than assault or sexual assault and have a different long-term effect on respondent views.

We also ask a few questions about respondent's views about their country. We include ideology in a measure here, but asking someone about their ``opinion of the relationship between the US military and the government'' of your country may provoke a different kind of response than we measure here. Likewise, we ask people if the presence of American military forces affects the quality of democracy, respect for human rights, public safety, or security from foreign threats in three separate questions. In trying to get a better feel for the texture of people's information and ideas, we ask them different questions to estimate the number of US troops in their country and question how long they think the American military should stay in their country. These questions offer some additional pathways to assess information, quality of information, and scope of the respondents' perspectives within our sample. While these are just a sample of questions and not exhaustive of the material scholars can explore, we also wanted to highlight a pair of questions asking what people thought of the Democratic and Republican parties in the United States. Our survey only takes place during three years of the Trump administration, and the answers to these questions will reflect that, but it also may offer some insights as to how individual ideological beliefs and beliefs about the US presence may map onto left-right ideologies in the United States. 

Beyond the immediate offerings of our survey, we are just scratching the surface of exploring what competitive consent looks like on a global scale. Part of this surface-level dig is the domain being in an early stage of evolving and contingent on international affairs in the coming decades. However, we expect our research to take us to a few different areas understanding this effect. In the immediate future, we hope to explore further how the items we identify in this book shape views of the Chinese actors. Our survey can only offer a single side of the equation. Still, it is essential to know whether US behavior in other countries builds support for or opposition to China in the countries where it is the sole presence. The United States increasingly sees itself as locked in an emerging competitive struggle with China, and China forecasts itself as overtaking the United States in the next few decades \cite{Doshi2021}. It is natural to wonder if third-party civilians internalize a trade-off between these two actors based on the positive and negative experiences with the US presence within their territory. If there is no trade-off, then each country is purely building its own support or opposition in isolation. If there is a trade-off, then each act matters doubly for competitive consent. Additionally, trade-offs may only manifest in areas where either actor is proximate or both parties are actively engaged in local political, economic, or security matters. Understanding the scope of competing for consent requires unveiling more about how people build their preferences in the international system.

Second, we hope to explore Djibouti in a follow-up survey. The triple presence of France, the United States, and China deploying troops to the country will lead to different sets of cross-cutting beliefs and interactions. Does building support for one country build support or opposition for another country? Does a terrible interaction with one basing country lower support for the other two basing countries? How are local commanders restricting service members locally and handling altercations with local civilians? How do local commanders deal with conflicts between their and other country's service members? Djibouti will prove to be a microcosm of shared basing sites in the future and something we are eager to examine next. %expand a bit more

Beyond our expected forays, the questions we explore here and the greater issue of maintaining and building support abroad is a wide-open field of inquiry. While we touch upon the roles of noise pollution and traffic in affecting people's perceptions of the deployments, we are less systematic in assessing how much these factors matter in building those perceptions. However, it is crucial as it is one factor that interview subjects from multiple backgrounds brought up with us. Understanding the environmental impact of the Department of Defense's overseas facilities is a task some scholars have taken up, but we know that it is a process that is shifting. For example, the Department of Defense has recognized that global warming is a challenge for maintaining security in the future. While this recognition is essential, it lends itself to further research on when military bureaucracies recognize environmental issues as salient and willing to change behavior. There are plenty of ecological consequences that result from basing and how those issues manifest in support or opposition to basing matters. Likewise, will the domain of competitive consent force the US to be more willing to address environmental grievances abroad? In sum, once scholars and policymakers recognize local stakeholders as influencing the US security arrangements abroad, a whole wealth of questions becomes pertinent to understanding deployments' longevity. Previous work on deployment-adjacent issues becomes pivotal to whether deployments exist.

Our immediate future not withstanding, there is a wealth of avenues for scholars to explore regarding race. There has been a recent increased call to understand the role of race in international relations, international institutions, foreign policy, and the development of how we study international relations as a whole. There has been some work that explores the undercurrent of race during deployments with \citeasnoun{Moon1997} providing detailed accounts about how the United States had exported its norms about racial hierarchy to Korea during the 1950s-1970s. The United States did not finish integrating the armed services until 1954, and the outcome of those processes are dynamics that played out in multiple host countries globally. Of course, racial issues within the military did not end with integration and continues to be a subject of scholarly attention. In the present era of overseas deployments, understanding the cosmopolitan makeup of the US military, recruitment practices of the military, the interaction US forces have with local institutions overseas, and how both service members reshape racial relations abroad and how experiences abroad reshape service member attitudes are essential considerations to explore. These questions transcend international relations as a siloed discipline, but the answers to these questions will continue to inform the shape of IR as we understand them. Status of Forces Agreements vary by country and examining both the negotiation of those agreements and the outcomes of SOFAs reasonably offer essential areas of understanding race and service. %cite freeman/lake piece

Last, there is a paucity of data on the topic of military deployments abroad, and each advance in better, more fine-grained data allows researchers to understand better the causal processes related to and resulting from deployments. The explosion of quantitative literature after \citeasnoun{Kane2004} provided an easy to import data set on the number of US troops in countries globally is indicative of how data impoverished this very important subfield of research is to understand foreign policy and policy-making. For researchers, it should be increasingly evident that producing new data related to this topic finds home in top field journals within political science (not just international relations). Not only do new data sets serve to produce foundational insights into the topic, but it creates a wake of subsequent research that builds from the data to research their conclusions. As such, the field needs better data on deployment-related issues concerning the environment, noise, traffic, crime, pollution, energy use, land use, military exercises, facility usage, the economic behavior of service members, the civilian logistic tail of the military, race, gender, dependents, military rank, vehicle deployments, naval patrols, social media usage, and dozens of other topics that have real effects on both people and the theories that these processes support. 

To conclude, we expect that this topic of competitive consent will become an increasingly important factor in understanding how basing plays out in the next century. Basing has moved from negotiation among political elites to domestic constituents on both sides of a basing arrangement. Understanding these processes will move the scholars, activists, and policymakers forward in understanding what makes a base more likely to endure and what makes a base more likely to be a temporary arrangement. As actors become increasingly invested in basing to grow their regional influence, the United States will come to a point where its current trajectory leads to it facing a basing site shortage if it does not seek to compete in winning the hearts and minds of civilians during peacetime. Understanding the positive and negative externalities of basing and, more importantly, how people perceive those externalities is a step towards understanding the latent and blatant manifestation of support for the current United States' global objectives and whether the current post-World War II configuration will endure, reshape to a limited capital-intensive deployment, or fade away in deference for other security arrangements.
