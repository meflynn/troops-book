\chapter{The Domain of Competitive Consent \label{cha:theory}}

\vspace*{-0.85cm}
\rule{\linewidth}{0.10pt} \\[-1.25cm]
{\footnotesize\paragraph{Summary:} Traditional theories of international relations and foreign policy envision decision-making as an elite process. Historically, this is a sensible view, but the combination of global political democratization and the democratization of media has allowed citizens to influence their government more in more profound ways. The effects of basing lead citizens to have both positive and negative perceptions of of the US military presence, which may, in turn, take those views and lobby their government in favor or opposition to basing. The flow of consequences from perceptions to policy gives rise to a competitive domain that we call the ``Domain of Competitive Consent.'' As globalizing forces continue with the emergence of more global basing rivals, the impact basing has on civilians matter more for the longevity of deployments. 
\\[-0.5cm] 
\rule{\linewidth}{0.10pt}
	
	In the decade leading up to 2003, the United States maintained between 2,000-3,000 troops in Turkey, a key NATO ally. As the United States finalized its plans for the invasion of Iraq, it had requested the use of Turkey as a primary launching pad. The United States requested that the Turkish government allow 62,000 US troops to be deployed in and operate out of the country. On March 1st, just 19 days before the invasion, the Turkish parliament, to the surprise and shock of the US government (which had secured the support of then Prime Minister Recep Erdo\u{g}an), the Turkish parliament voted against allowing the US operation, by just three votes. While there is still uncertainty as to why exactly the vote failed, it is clear that overwhelming public opposition to the United States' plan convinced some Members of Parliament to reject the measure despite the attached \$6 billion in aid \cite{CSR2003,cooley2008,altinay2009}. While the Iraq war had unique international opposition, this example makes it evident that foreign leaders' support is sometimes not enough to allow for a foreign military deployment; public opinion in a host state can substantively alter or derail US foreign policy.

	Other such instances can be found around the world. Even in autocratic states, the population can exert a measure of control over the country's foreign policy. The only country that allowed the United States to refuel for Operation Nickel Grass during the 1973 Yom Kippur war was (then) autocratic Portugal \cite{sandars2000}.  After the Gulf War, the United States maintained a robust military presence in Saudi Arabia to enforce the southern no-fly zone in Iraq and act as a safeguard against further aggression from Saddam Hussein. Over time, this presence became a symbol to both Saudis and other Muslim-majority populations around the world of American domination and of Saudi ``subservience'' to the United States \cite{BBC2003}. The popular opposition to the US presence, without formal democratic avenues for expression, found its way into more extreme channels such as violence committed against US soldiers and Western populations, culminating in the bombing of the Khobar Towers complex that housed coalition forces.\footnote{While the attack on Khobar Towers was later attributed to Iran and Hezbollah, there are still conflicting accounts that attribute it to Al Qaeda \cite{NYT1996a, NYT1996b}. Critical here is that, at the time of the attack, the attack was attributed to Saudi militants. The political considerations surrounding the sensitivity and safety of the US presence and public support in Saudi Arabia itself became a serious issue for debate following the attack.} Even years after the US announced its withdrawal from Saudi Arabia, in a Gallup opinion poll the majority of those surveyed in Saudi Arabia stated that removing bases from the country would improve their opinion of the United States \cite{GallupSA2009}. These elements show that, regardless of whether it is expressed through legitimate or illegitimate channels, popular opposition to a US military presence can imperil the continuation of a basing relationship by exerting pressure on both the host government and the United States.

	

% AS - thinking about a better way to phrase the second sentence. "This" is a bit vague there, and it's also an assertion without backing.
	%MAde a few edits with this in mind. CM. 
	One of the foundational ideas behind our research is that the perceptions of host-state civilians matter. This was not always the case. Though grievances against foreign military presences have always existed, historically, basing powers often only had to convince the elites, or a segment of the elites, to gain basing access to a territory \cite{Gillem2007}. However, as democratization has expanded throughout the world, those members of the population most affected by a foreign military presence have made their voices heard (cite something here about increased democratization in the world). Even in non-democratic or partially democratic states the population has found a way to influence governments, oftentimes through the use of telecommunications techonology and social media and through boomerang effects brought about by reaching out to transnational allies \cite{Keck1999,Steinert2017}.
	
An increase in the population's influence over government will naturally affect US basing abroad. The population's increased influence can both expand and contract the available set of basing options for the United States. Host-state public support allows the United States operational freedom in a wide range of regions and domains when the public's consent filters up to decisions made by host-state governments. Alternatively, public opposition to the military presence can restrict that freedom, as governments become more responsive to their populations and either limit what the US military can do within the country or even terminate access. 

Given that the US and its allies had held a de facto monopoly on overseas basing since the end of World War II, competition over basing clients had not been a major concern.\footnote{Russia, the United States' Cold War rival, currently holds only a single foreign base outside of the former Soviet Union--the naval base in Tartus, Syria \cite{Nieman2020}.} This is changing with the opening of the People's Republic of China (PRC) establishment of its first permanent overseas military base in Djibouti in 2017, and plans in place to establish others\cite{Joyce2021,Kardon2021}. Just over the past decade, there are consistent rumors of PRC interest in establishing permanent bases in Pakistan, Myanmar, Cambodia, and others, as China seeks to extend its reach into the Indian Ocean especially. With the PRC as a potential competitor in the basing arena, the United States will be forced to be more proactive to preserve its current set of clients. 

Beyond competition with China, the United States must still consider the effect bases have on foreign civilians if it aims to maintain its global network of bases and military deployments. Past experience is a guide, as the US has had to find replacement hosts multiple times in the last 75 years, even in periods without a distinct competitor. The US has faced nationalist and anti-American challenges in multiple countries and has been successful in outlasting some, while less successful in others \cite{blaker1990,Baker2004}. Withdrawals and drawdowns in France, the Philippines, Spain, Saudi Arabia, Turkey, Uzbekistan, Panama, Afghanistan, and Iraq all point to unique conditions where the United States had to find basing alternatives or abandon particular regional strategies in the face of restrictions on or rejections of the United States military presence. 
	
	In this chapter, we build the theoretical foundations of the remaining substantive chapters. We begin by discussing the set of ways in which service members and bases can evoke both positive and negative perceptions of the United States military and its mission. This is the argument that will inform Chapter \ref{cha:meth}, in which we discuss the connection between contact, financial benefits, and perceptions of the US military (and which is expanded on in Chapter \ref{cha:min}).  We then establish the connection between these perceptions and both \textit{popular support for} and \textit{\textit{discontent with}} the US presence, which can then manifest into political action. This aspect of the theoretical argument thus informs Chapter \ref{cha:protest} on anti-base protests, as well as the exploration of the most negative interactions between US service members and host populations in Chapter \ref{cha:crime}, on crime by and against U.S. service members. Finally, we present an argument for how these perceptions and movements can lead to virtuous or vicious cycles that can enable or disable the United States military and diplomatic objectives regionally and globally. In our final chapter, Chapter \ref{cha:conclusions}, we explore the policy implications of these dynamics. While we remain agnostic as to the maintenance of the US military presence abroad, we lay out a path for the minimizing of harm towards host state populations and for the improvement of relations between US military bases and surrounding communities abroad.  
%Might edit this last sentence a bit after reading through the conclusion. CM.	
	
	
\section*{From Bases to Opinions}

% AS - might be something for the intro? It's a bit of a larger point about the project's motivation.
% On a prior research trip to Romania to examine the role of Congress in the extension of new deployments abroad, one of the authors jumped into a cab at the airport on the way to his hotel. As soon as the driver discovered that the author was an American, he raised the fact that a local civilian was killed by a drunk driver just a few days prior, and that the drunk driver was a member of the American military. Fully expecting the driver to then explain his overall opposition to the presence of the American military in Romania, the author was surprised when the next words from the cab driver were a glowing review of the American presence and an appreciation for the presence, given the Russian threat to the country. It was 2015, and Russia had only recently invaded Crimea and Eastern Ukraine, and much of the Romanian population was convinced that their country was next, so the presence of the American military was seen as necessary in the face of this threat. The complexity of the opinion was striking in its contrast to much of the received wisdom about the presence of the US military abroad, which gave a straightforward account of how such negative experience must invariably lead to negative perceptions of the US military and that host populations who have such experiences are highly likely to oppose the US presence. While only one person's anecdotal account, the recognition of both positive and negative within the evaluation began the process of looking in more detail at how people's opinions of the US military's presence abroad are formed and what factors are most impactful.

	In 2019 we interviewed the Wiesbaden (Germany) mayor's office liaison official in charge of coordinating relations between the city government and the US Army Garrison Wiesbaden. Despite a language barrier, he was eager to share with us funny stories of newly arrived American officers who were suprised to see their commanding officers at the Irish pub in town (``a meeting place for the Americans'') and of the base commander's 10-year-old daughter who enjoyed being able to say ``nasty words'' in German in front of her unaware, monolingual father. Moving beyond the levity, a point he highlighted was the significant amount of construction that accompanies the base, not just in terms of the initial establishment of the installation, but also with regards to expansions and updates. ``They build very large buildings,'' he said. He noted that even in 2018, a year in which there was a transition in base commanders, new housing units were built \cite{wiesbadenone20190725}. In the same conversation he discussed concerns about traffic and how the Americans have been careful to build new housing in locations that will not force the service members' commute through German towns in a way that would be disruptive. In terms of the impact that the Americans themselves have on the community, he noted that while ``there is a lot of friendship between the German people and the American people,'' the refusal of the American service members to learn German while stationed in Wiesbaden is a sore point for the local community (``In my ten years on the job, I have met maybe 10 [American service members] that know the German language'').

	
%FYI, I'm going through and replacing e.g. with for example, to improve the readability of the piece as a whole. 	
This conversation highlighted to us not only the benefits of a bilingual education for children, but also how the injection of a massive fixed asset and a foreign population into an existing community is necessarily going to disrupt its social flows. Creating a concrete fortress inevitably requires reshaping the local environment to facilitate the needed infrastructure to support a military presence. A deployment can come with incredible physical changes to the location, and both environmental (for example, clearing the land, reshaping habitats, and changing water resources) and infrastructure (for example, roads, power, and construction) externalities can result. As our conversation with the city official highlighted, these changes are not limited to new deployments; existing bases also require constant updating and renovations, and these updates often come when, from the outside, it can appear that the United States is downsizing its presence. 


As alluded to by the Wiesbaden city official, the presence of the military members themselves in a base and in the surrounding community will affect locals in both positive and negative ways. These can include increased economic activity, local jobs, higher traffic, additional noise pollution, increased opportunities for crime, interpersonal disputes, new neighbors, increased demand for housing, higher house prices, higher demands for gray or black-market goods, and numerous other effects that would not be present if not for a foreign military presence. No single individual will experience all the positive and negative effects of a foreign military presence in their community but will instead have a fractured experience of these externalities. Additionally, individuals will contextualize a base or military presence in more abstract terms such as national security and interest in the United States' mission beyond the direct and indirect externalities we discuss.\footnote{The definition of what constitutes a base, military presence, or sizable force is not something that scholars have agreed upon in their various works. \cite{Gillem2007}. \possessivecite{Harkavy2013} work on great power competition for bases argues that the dominant global power usually has 30-40 major bases in various countries globally. Other authors and activists have claimed that the United States has around 700-800 military bases due to the DoD's own accounting on what physical asset possessions it has overseas; \citeasnoun{Johnson2004} points out that within the DoD's list there are 17 large and 18 medium installations. Notably, that count includes component parts of a presence such as the main basing site, fuel depots, recreation areas (such as golf courses), listening posts, and other similar sites that do not qualify as a base. Though much of our language and discussion does include a focus on bases, in conducting our research, we are interested in the US military presence broadly conceived. We include countries that host a large number of troops but not a formal US base, as all US-used facilities are officially the host-state's possessions (such as Royal Air Force bases hosting the US Air Force in the UK or the Pine Gap joint facility in Australia). We are likewise interested in temporary deployments meant to achieve humanitarian or training objectives that have influences on local populations and affect how people perceive the United States.} 
	
	The presence of US military personnel presents a complex set of experiences that people can weigh in assessing their support or opposition to a foreign military presence. Each experience may or may not be meaningful to a person, and the sum salience of a foreign presence varies from individual to individual. The locals' experiences inevitably filter through individual beliefs about politics, religion, social welfare, community, and society. Going further, those filters are adjusted by demographic factors such as age, education, political identification, minority status, and religion to help determine whether someone will support, oppose, or be indifferent to the presence of foreign troops within their society. 
	
	There are some transparent demographic relationships we expect when people evaluate the presence of a base. Generally, older populations will be more in favor of the US presence if the deployment occurred in response to a crisis in that country's history; older generations are also more likely to have a memory of the national conversation surrounding the decision to allow US forces into their country in the first place. More educated individuals will generally be more likely to have an opinion on the presence than less educated individuals by virtue of their increased exposure to national defense debates and history. More conservative-leaning people are more likely to support their national military, and by extension the presence of a friendly foreign force often intended to train and equip the host-state forces. In contrast, more liberal-leaning individuals are more apt to be critical of a foreign presence, militarism in general, or perceive the presence to be a manifestation of US imperialism.
	
	Based on prominent theories of social and political psychology, we expect that some experiences will be more salient than others and will be larger determinants of individuals' beliefs. For example, if someone associates their livelihood with a foreign military presence, that presence will be more salient to that person. A landlord, construction worker, or retail business owner who derives much of their income from the US military will be more likely to support a base. Those who have interpersonal interactions with US service members are more likely to put a face to the otherwise abstract presence and associate those interactions (whether positive or negative) with the military presence overall\cite{Allen2020}. Someone who receives a vaccination from a US Army doctor in Peru will be more likely to see the US presence positively than someone who gets into a drunken bar fight with a US Marine in Okinawa \cite{Flynn2018}. These personal experiences are likely to heavily condition people's individual perceptions, even when they may seem to run counter to prevailing wisdom or the preponderant impact of a military presence in some objective sense \cite{Taber2006}.
	
	%need some social psych stuff above.
		%Come back to this. CM. 
	
	Interactions with US military personnel are, for most civilians, distinct from most of their day-to-day interactions. They are fundamentally an interaction with an armed state agent, specifically an armed agent of a foreign state. The behavior of a US military member is governed by multiple bureaucracies, including the local and national laws, the Status of Forces Agreement (SOFA), the military, and the laws of the United States. In some cases, this will constrain their behavior, while in others, it may seem more freeing if the SOFA allows the US military to prosecute local infractions and the military guidelines are seemingly more lenient than the local laws. The service member's behavior is also situated within unique cultural contexts distinct from the host-state's civilians. Their training, position within the military hierarchy, mission, guidelines, rules as to what they can do off-base, language, and other cultural norms govern their behavior that makes it stand out compared to other locals. Even in places where the United States is a long-term ally and viewed favorably, such as in the United States' special relationship with the United Kingdom, American service members stand out. Even the way Americans stand, sit, and dress can be just as telling as their spoken accent. 

%this was a bit fuzzy, wasn't sure if it implied US servicemembers carry arms or just the security apparatus in general.	
%While there is variation from state to state as to whether private individuals carry firearms, there is a guarantee that the person in charge of providing security for a state is either armed or has access to arms since they are part of the legitimate force-provision within or without the state. 	
	
A service member represents the US and is a symbol of the agreement a host state has with the United States. In fact, in many places globally, people are far more likely to interact with a member of the US military than any representative of the US diplomatic corps. The US military is pervasive in many places, numbers in the hundreds, thousands, or tens of thousands in peacetime deployments, and members often live off base and integrate themselves with local community members. Interactions between host-state civilians and US military personnel are notable, salient, and more common than interactions with other kinds of US representatives. The only more frequent encounter a host-state civilian may have with an American would be with a tourist that is not an official representative of the US or its foreign policy.
	
	Because of these factors, US service members often stand as de facto diplomats on behalf of their mission and the larger presence within a country. In this capacity, US personnel's interactions can do both harm and good. When a service member commits a crime in another country, they tarnish the view of the US military for those affected by the crime and for some portion of the people who learn about the incident either through word-of-mouth or through the local news. Someone who is victimized by a US military member, or who knows someone victimized by a US military member, is more likely to view the presence of the US as a net-negative. Likewise, views about environmental pollution, traffic, or noise pollution can contribute to negative perceptions. It is important that both the positive and negative interactions are highlighted, as they can both create intense experiences that alter public perceptions; previous work generally expects that proximity to military bases and contact with military personnel will produce negative perceptions of the US presence \cite{calder2007}. We do believe there is a causal pathway for this to occur but there is also a spuriousness to this expectation: people who have contact are more likely to experience the negative harms of basing, but that does not mean that contact is the direct cause of negative sentiment. Certainly, there are occasions when it is. Other times, it will be an artifact of other processes. We think this is less likely, however, with positive experiences, given the long social and political psychology literature on the effects of contact. 
	
	Alternatively, we expect that interpersonal interactions can also be a conduit to producing positive perceptions of the United States military. There are a few ways that this is possible. More directly, interactions that produce a positive result will likely influence individuals' views favorably. This may be true of a mission dedicated to humanitarian relief during a disaster or providing inoculations to rural and impoverished areas of a country. The people that directly benefit from the military will see it positively. The mission, and the resulting interactions, can be a public relations boon for the military, and the branches of the US military are aware of that. However, it is not purely about the mission either. Base commanders can, and do, get involved with public service missions for local communities near base, attend public meetings to help address grievances, and host joint celebratory events such as offering an American Independence Day celebration to encourage cultural exchange \cite{rafthree20190719}. \citeasnoun{Gillem2007} discusses how ``America Day'' in Okinawa and similar festivals in South Korea spread American culture and encourage locals to develop an affinity for Americans and the bases they live in despite previous grievances. 
	
	
	Beyond the intentional engagement of public relations, the day-to-day interactions of service members are a source of public diplomacy. When service members live off base, which they do in many of their deployments, they join a local community. They shop at local grocery stores, attend local bars, attend local religious services, and have relationships with locals. Their children attend local schools, they coach youth sports teams, get their hair cut, pay for manicures, mow their lawns on the weekend with their neighbors, and engage in thousands of behaviors that are part of community life. Each interaction builds social capital between the service member and members of the local community. As the local community members come to know and live with US service member, they become invested in their lives and, by proxy, the presence of the US forces. \citeasnoun{Putnam2001} mentions that a mutual head nod is an act of reciprocity that is both visible and measurable. Interactions build friendships, relationships, and community interdependence that bridge cultural divides \cite{Woolcock2000}. One of the local parishioners we spoke with in Lakenheath, England very openly confessed that he owed his very existence to the military presence as his parents were an American soldier and a local British civilian \cite{councilone20190718}. Notably, he spoke positively of the United States and identified as an American despite his parents no longer being together and him never having lived in the United States. Long-lasting US deployments have the opportunity to build a community of support, a virtuous cycle, as the community becomes intertwined with the deployments and sees the presence as fundamental to its existence and identity. Interestingly, quite a few US military personnel we met overseas expressed a desire to one day retire to the very community they were currently stationed in.
	
	Chapter \ref{cha:meth} puts these ideas to the test. We expect that contact with US military personnel produces both positive and negative perceptions of the United States military, government, and people. In thinking about this research and designing our survey, we expect people to have informed and uninformed views of the actors mentioned above. People's experiences filter through layers of different contexts, such as demographic background and ideology, which help to inform their views. Chapter \ref{cha:min} considers these filters a bit more deeply by examining those who specifically identify as a minority within their country. As a first cut, the number of people who respond that they do not know if they feel negatively or positively about the US military will decrease after having personal contact with members of the US military. We also ask if people rely on the US military economically for their livelihood and expect such reliance to positively associate with views of US actors. Finally, experiences are not the only way people update their views about the world. In addition to national media sources, people are likely to hear messages about the US military from their friends and family. We ask if people know of individuals within their social network that have had contact with or are economically reliant upon the US military. Those who respond positively to these questions will have more informed views of the US military. 
	
	The types of routine but deep interactions between members of the US military and local populations that we examine here have profound implications for our research and foreign policy in general. International relations scholars have been interested in soft power as a complement, supplement, or substitute for rigid (i.e. often military or economic) power for several decades, since Joseph Nye popularized the term \cite{Nye1990}. At the macro inter-state level, soft power is the ability to convince other actors that your goals are aligned and that it is in their interest to pursue the same things that you want. Traditionally, soft power extends from cultural, economic, or diplomatic processes and are a sort of passive power that, over time, brings other actors over to your side, so that they hold your interests as their own. Soft power is a form of cooptation compared to the coercive nature of hard power \cite{Nye2004}. For example, economic success may create soft power as other states mimic your strategy in the hope of finding similar success. Contrarily, economic sanctions, the withdraw of financial ties to change the behavior of another actor, are hard power and coerces the target. 
	
	While we will return to soft power a bit later in this chapter at the macro-level, in the context of US soldiers abroad, soft power works on the micro-level and becomes part of the microfoundations of ascent to US leadership both in the present and in the future. Humanitarian missions and routine interactions build a form of public diplomacy where both the visible good works of the military and cultural exchange with US service members can increase support for the United States \cite{cull2008cold}. Recent research has increasingly examined how the military, the ultimate tool of hard power, can also generate soft power \cite{atkinson2014military,Martinez2021} and these benign, daily micro-interactions between service members and civilians can do exactly that.
	
	In our argument, there is some level of psychological phenomena that we are positing for civilian actors. There are generally five competing, contemporary models of how individuals engage in social evaluation of different people and groups \cite{abele2021} and our task here does not afford us the ability to weigh into that debate directly. However, we expect that repeated experiences, interactions, and interdependencies (both positive and negative) with military members create feedback into host-state civilian evaluations of the US presence. Additionally, suppose people generalize their experiences from their personal situation to the broader US mission in their area, country, region, or globe. In that case, we also expect that host-state civilians likely make inferences about both the US government and US people based on these similar conditions. Notably, we expect that the transfer rate from one level of abstraction to another is not absolute; but instead, some fraction of people will generalize their experiences to other facets of the United States. Of course, this argument is not necessary to our larger argument about support for US foreign policy. Still, it does allow us to consider the full ramifications of just how far the military might advance or hinder soft power development for the United States.
	
	
\section*{From Opinions to Mobilization}
	
	The impressions, views, and perceptions we are exploring within this book are not the final step of the processes we care about here. Indeed, if we could fully isolate the causes of opinion formation by civilians in the host state we would have accomplished a momentous task. Still, we see foreign policy preferences as instrumental to defining the operational space for basing states in the decades to come. A key question, then, is, do opinions matter? Often, in social science research, when a study concludes favorably towards one hypothesis or another, and the results receive public comment on a blog post or a news story, it is not uncommon to see several comments that the results are obvious or common knowledge and, along with those comments, several other comments saying that that results are wrong and could not be further from the truth. Likewise, if we were to answer whether opinions on a foreign policy matter in the affirmative or negative, we would likely get a similar set of reactions, with many thinking that it is apparent that it does matter. Others, however, will say that it does not. Based on what we have written so far, it is clear that we believe it does matter and will increasingly matter going into the future. Given that there will be disagreement on the effect of public opinion on national and foreign policy, we detail in this and the subsequent section the causal paths between opinion formation and policy change.
	
	In the previous section, we have identified multiple pathways in which deployments and bases can generative positive and negative views. In this section, we will focus more on the negative view generation process as the case for mobilization is both more apparent and demonstrable. \possessivecite{calder2007} primary hypothesis regarding contact argues that proximity and interaction is a primary motivator for negative attitudes towards bases. Chapter \ref{cha:crimes} helps to build the basis for the linkage between negative interactions and views about host-state national security. Specifically, we drill down into those who report that the US military has criminally victimized either them or their social network---experiencing a personal violation, whether property theft, drug-related, assault, or sexual assault, is highly likely to make someone view it the presence of the military negatively. Additionally, direct personal or material harm may be enough to motivate people into political action against the US presence. Inevitably, through experiencing the negative externalities of bases, some subset of the population will be opposed to the US presence. For some, it may not even require a negative experience. The hosting of foreign military troops raises questions of nationalism, sovereignty, the threat of entanglement into unpopular conflicts, and a whole host of things that someone may ideologically oppose. Sometimes, if the presence of a base is tied to a specific political party and not a greater sense of national security, then political polarization will lead to discontent with that party's national security strategy and the bases may act as a proxy conflict for political debate. The mere existence of bases will have detractors. Additionally, every governmental decision will have winners and losers, and if the negative externalities of troop deployments are concentrated among particular populations, groups, or individuals, then those people are likely to oppose the existence of bases as well. 
	
	
	However, the existence of negative views is not a sufficient condition for political action. People who are either in favor of or opposed to a policy face collective action problems in mobilizing to change that policy \cite{Olson1965}. Taking political action is costly. Whether someone is voting, protesting, writing an article, lobbying a member of their legislature, boycotting a business, engaging in political violence, or taking any of the hundreds of politically meaningful actions, it requires resources and time to do so. Someone acting by themselves is extremely unlikely to create change by themselves. When facing the choice between spending time and money on taking action to have little effect versus doing something else with your time and money, many people choose to do something else instead. For \citeasnoun{Olson1965}, collective action arises when there are diffuse benefits or costs to some policy decision, and the costs to change policy overwhelm the private incentives for people to do anything about it. The problem is even more acute if individual action is anonymous. 
	
	Olson does identify a few solutions for those seeking to create change. First, if the affected population is small enough and participation is identifiable, people can use social institutions to encourage people to change. Second, if organizers can reward participation with selective incentives that help defray some of the private costs of action. If looking to provide selective incentives, a few different levers groups can encourage organization. Direct monetary payments may work for some groups. Advancement in rank or titles within an organization can motivate people to act on the collective benefits. Other groups may use other social institutions to highlight member's contributions. In the United States, those who vote receive an ``I voted'' sticker that people wear on their clothing or post on social media to signal to friends, family, and community members that they have contributed to the public good. The United States has adopted a similar strategy with COVID-19 vaccinations, where people receive stickers or take pictures next to signs to indicate that they too have been vaccinated. Third, and important for our discussion about basing, political and social entrepreneurs are pivotal to overcoming collective action problems. Such entrepreneurs may have an outsized stake in the outcome and be more willing to pay to organize collective action or see other individual benefits for managing a group to act collectively. Political entrepreneurs can take the diffuse harms and benefits of collective inaction and try to localize them into an effective movement for change.
	
	Game theoretically, collective action problems are a particular case of the prisoner's dilemma. It seems most rational for players to avoid cooperation and instead pursue their own rational self-interest. When it comes to high-stakes collective action that may put your life on the line to achieve change, such as protesting or rebelling in an autocratic state, then the incentive to stay home and not engage in political action is the safest option \cite{lichbach1993}. However, mobilization does happen, and we do see actors mobilize against the presence of US bases in Japan, South Korea, the Philippines, and other places \cite{lutz2009}. Following our understanding of collective action problems, we should expect to see political action against bases under a few conditions. First, if the negative externalities concentrate on a small section of the population, then that population will have a stronger incentive to change their local status quo. The heavy concentration of Navy and Marine bases in Okinawa means that the benefits of the bases in terms of security spread across the entirety of Japan, but the Okinawan population bears the brunt of the costs of having those bases \cite{akibayashi2009}. If some of the negative costs are relatively infrequent events, such as high-profile criminal activity by US service members, then those episodes start to look like a pattern in places where there are high concentrations of US forces. If the same bases were spread across the entire country, the likelihood of collective action would be lower.
	
	Second, if particular community members are more adversely impacted by bases in such a way that political entrepreneurship makes sense, then collective action against basing becomes more feasible. Those who bear the most harm have more incentive to overpay to provide a public good, organize politically, and incentivize others to do the same. While there is a large spectrum of political action people can take if they are interested in removing a base, media sources are most likely to report on either protests or when a politician adopts an anti-base policy or platform. When institutions allow for democratic action, savvy politicians that want to bolster their support may seek to win the influence of anti-base movements. If a movement increases in popularity, parties may adopt anti-basing planks and campaign on removing a US presence. Given that theories of asymmetric alliances argue that weaker states trade foreign policy autonomy for the security provision of a stronger state, such movements will likely invoke nationalism, dependence, and reclamation of sovereignty of the nation-state. Charles de Gaulle's eviction of NATO's headquarters from France and expelling all foreign troops in 1966-1976 came with a call for French autonomy in its defense.
	
	Beyond this, there is a simultaneous situation occurring for those that benefit from the presence of a base and the same considerations apply to that side of the equation. If the benefits of a base are diffuse and non-concentrated, then mobilization in favor of the base could be difficult. For example, if the primary benefit of a base is a nebulous concept of national security that begun to fade away at the end of the Cold War, more current generations may be skeptical of the added utility of a base. When we asked about generational divides, older respondents in Panama told us that the younger generation did not understand why the ``Americans were here'' \cite{journ20180713}. If there is a concentration of benefits, such as economic contracts or positive interactions, there is more room for mobilization of support for the United States. For example, South Korea has a history of pro-US, pro-base demonstrations in Seoul. As the US publicly debates drawdown or anti-basing sentiment seems to rise, pro-base supporters organize publicly to maintain the size of US forces on the peninsula. However, having mobilized supporters is not enough to guarantee that a base will endure. In the Philippines, President Aquino's pro-base rallies were insufficient to overturn the Filipino Senate's decision to allow a continued US presence \cite{Oberdorfer1991,simbulan2009}.
	
	Returning to our discussion of crime, there is some variability in how crime translates into catalyst for action against bases. The number of crimes committed by US personnel is much higher in South Korea than it is in Japan \cite{Gillem2007}, but crimes in Japan appear more likely to mobilize stronger action against bases. Like with any viral sensation, such as the \textit{Harry Potter} series versus other similar books that did not fare as well, there are likely multiple candidates for focal points and only a subset of possible interactions become such a catalyst. Additionally, the severity of the crime, cultural context, the reporting of the crime, and how authorities handle the perpetrator all play into whether an offense becomes a rallying point for action. \citeasnoun{kim2017} argues that a combination of media coverage and activist framing are the key determinants in whether a movement succeeds or fails.
	
	
	Chapter \ref{cha:protest} is where we explore the above processes in action. Using machine-coded data of political events, we identify all demonstrations against the United States and the United States military. Using this data, we identify the correlates of protests against these targets globally. In this research, we offer the first quantitative evidence that the increased presence of troops does increase the likelihood of protests against the US military and the United States. Additionally, employing our survey, we also create a useful predictive model in determining the probability that someone has attended a protest against the US military presence within their country. Our results show that demographics and attitudinal measures of an individual are insufficient, alone, in predicting the likelihood of protest behavior, but including variables assessing improves the model dramatically. 
	
	
	
	
\section*{From Mobilization to Bases}
	
	We have thus far connected the presence of bases with popular opinion and discussed how popular could mobilize into political action, but we have not fully addressed if this translates into outcomes. It is clear from the behavior of both the United States government and the governments of host states that they respond to public opinion. The United States, since the beginning of its intervention in European conflicts in World War I, sought to maintain positive views from local civilians. In 1918, the United States passed the Indemnity Act to allow citizens of France to seek compensation for harm by the United States military \cite{walerstein2009}.  While the act covered WWI, the United States made a stronger, more durable signal of the US attention to the costs of its security policy in establishing the Foreign Claims Act. Passed in 1942, the act compensates individuals for the loss of property, personal injury, or death caused by the non-combat activity of US armed forces; the law specifically states that the intent is to promote friendly relations with other states \cite{ForeignClaims1942}. This act, intended to decrease the burden of deployments of troops, has persisted as the United States' response to citizen harm in times of peace and conflict. Since 2003, The US has paid out tens of millions of dollars to claimants in Iraq and Afghanistan \cite{Currier2015}.
	
	\citeasnoun{calder2007} spends quite a bit of time dissecting the internal nature of state contests as it relates to bases in his careful comparative analysis of the politics surrounding US basing. Instead of treating countries as monolithic identities, he argues that the individuals within a society matter and game theory advances our understanding of the actors involved, their motivations, the institutions they work within, and the effectiveness of anti- and pro-basing movements. In dissecting domestic politics, Calder argues that local opinion and interest has been a historical force for shifting elite opinion away from basing. Institutional shifts, regime change, and technological change that empowers those harmed by bases are all events that enable anti-base advocates \cite{calder2007,cooley2008}. While a US-based example, Puerto Rico's activism against the naval presence in the island serves as an illuminating example as to how longterm opposition can manifest in dramatic shifts in military basing practices \cite{mccaffery2009}. The emergence of the internet enabled a small community surrounding the Vieques base in Puerto Rico to link up with transnational advocates, connect their grievances with the rest of Italy, and shift political incentives to encourage the Navy to withdraw from the facility \cite[p. 176-183]{calder2007}.

%%Move these two paragraphs to theory chapter??

There is a significant amount of work, particularly from a qualitative perspective, which has studied how it is that anti-US base protests can be successful in influencing host country policies. Work by Yeo \citeyear{Yeo2011} shows that successful anti-base mobilization efforts stem from two processes. First, the country must have a fractured elite security consensus. This fracture allows a protest movement ``air to breathe'' without the entirety of the country's elite establishment denying it. Since much of the public will take their cues from elites, many will avoid protests in the presence of an elite structure that stands united behind the idea that the host country shares security threats with the United States, that the country needs help from the US, and that American forces are present for these purposes. The opposite is also true: in the presence of prominent elites who question such tenets, individuals will be more likely to question the presence of American forces and join protest movements. 

Second, Yeo explains that successful protest movements include broad sections of the host state's society. This includes variation across ethnicity, gender, income, region, and more. Protest movements confined to single demographics are unlikely to succeed, as there is less incentive for host governments to respond to a fraction of the population. When host governments face broad coalitions of anti-base protesters, their political survival is at stake and it is difficult for them to build a counter-coalition. A broad-based movement also signals the salience of an issue by cutting across demographic lines that would normally remain isolated or even opposed to each other. While we do not purport to explain the success of protest movements, theories from \citeasnoun{Yeo2011} stress the importance of studying the determinants of protest. If certain factors make individuals more likely to engage in protest, and these protests can indeed be successful in changing host country policies on US  military bases, then protest movements can fundamentally weaken the United States' international position and its ability to maintain a global military presence.  

%%%	
	
%%Brought in stuff from the protest chapter instead of this	
	%\citeasnoun{Yeo2011} argues that understanding the role of protests requires an understanding of the internal dynamics, external relations, and the elite consensus within a society. While other theories of internaitonal relations would ignore the domestic component of opposition to bases, Yeo draws a causal path from the negative externalities of bases to changes in basing relationships. When people are harmed by the presence of a base and do not have a productive outlet for their grievances, protesting becomes a possible outlet to create change. If there is a weak security consensus among elites and protesters have the ability to influence elites through domestic institutions (such as those in a democracy), then a country may reject the US presence such as in the Philippines or Ecuador. However, if the security consensus is robust and aligns with the United States, then anti-base activist will have a more difficult time altering the status quo in places like Japan and South Korea. %maybe rework this, my notes are a bit old, so if someone knows Yeo's work better, update. Maybe throw in something about how the internet makes things more dynamic now than they did a decade ago when he wrote this.
	
	Host states also go out of their way to sell the benefits of hosting US troops to local populations while also minimizing the negative effects of basing. For example, in Latin America, as countries democratize and leaders face more political backlash from hosting US military bases that do not provide widespread benefits that include political opposition, the US has come to depend on informal agreements under which US military personnel deploy to existing military installations rather than building new American bases \cite{Bitar2016}. Building an American base comes with it cultural and political difficulties that politicians find easier to avoid entirely if possible. Threading a middle ground where the United States can still operating within a country without the visible presence of a new base can alleviate public opposition to local politicians. 
	
	Royal Air Force (RAF) Base Lakenheath, in Lakenheath, Suffolk in the United Kingdom, is a relevant example of the host country minimizing the exposure of the military presence.  In an interview, a British RAF Commander noted that the base is called ``RAF Lakenheath, not USAF Lakenheath,'' emphasizing that the base is a British one rented by the Americans, not technically an American base.  At the same time, he noted that there was only a single British officer at the base; everyone else is American \cite{rafthree20190719}.A public affairs officer for the base noted that many people in the immediate community know that it is essentially an American base, but that in the broader community, ``anything fifty miles, probably less than that, away'' people think it is a British base.  She noted that she gets calls from the general public about wanting RAF personnel to participate in local events because they believe there are RAF personnel there \cite{raftwo20190719}. The British Commander noted that a lot of people see RAF in the name of the base and ``may think there's a load of RAF around'' and maybe some Americans \cite{rafthree20190719}. Thus, even in a country that is highly favorable in its views towards the United States, and where US military personnel report very positive interactions with the local population, there appear to be no efforts made to correct these erroneous perceptions, as the base should remain associated with the local military.
	
	This tactic of a joint base that is under the banner of the host country but primarily used to assist US objectives is a common feature in places where domestic opposition may be particularly acute. The leaking of classified National Security Agency documents by Edward Snowden had several revelations that affected communities differently. In Australia, the leaked documents revealed to the public that Australia's Joint Defense Facility Pine Gap played a pivotal role in providing intelligence information to the United States military in conducting several different kinds of activities, including drone strikes \cite{Cronau2017}. Uncrewed aerial vehicle (UAV) strikes are largely controversial in several countries that the United States bases in, and the leaks created further scrutiny on the US presence and mission within Australia. The Australian ownership of Pine Gap suggested a more limited role of US forces and their behavior while at the facility. The perception, without any direct confirmation of that perception that Ramstein Air Base (a wholly American base) facilitated drone strikes in the Middle East and Central Asia was one of the primary motivators for protests against the base in Germany \cite{berlinone20190723}.
	
	While both basing states and hosting states seek to assuage civilian concerns about basing, there is additional evidence that domestic populations do influence foreign policy, and this should be a concern for both the United States and the countries that it chooses to base in. Existing work on democratization shows that effective mobilization by the population, particularly when it includes groups such as industrial workers, or the urban middle-class \cite{Dahlum2019}, can lead to regime change. We argue that state leaders have a preference for staying in power and that to do so, they require the support of some sector of the population, which \citeasnoun{BdM2005} term the ``winning coalition.'' The minimum winning coalition is the smallest subset of a selectorate (the people in a country who can select who governs the country), and leaders want to both keep it as small as possible so that they can more efficiently award benefits to that population while, simultaneously, making sure that the coalition is stable enough to continue to cement their position. In a dictatorship where the selectorate comprises of a small set of oligarchs or military generals, identifying and distributing goods to that group is relatively easy. Private goods are more efficient than public goods as you want to make sure those people are firmly in your camp and rely upon the benefits you give them. When the selectorate expands, so does the minimum winning coalition. At very large sizes of enfranchisement, public goods are more efficient at securing people's votes than providing private goods to every voter you want to appease. Of course, and particularly in states with large winning coalitions, such as democracies, it is not always simple for the leaders to interpret the preferences of the population. This is particularly given that in the case of most policy issues, there will be variation in preferences, even within the winning coalition of voters \cite{Lohmann1993}. Still, \citeasnoun{Lohmann1993} argues that all else being equal, leaders would still prefer to choose a policy outcome that is preferred by the majority of the population to ensure their survival as a leader. 
	
	Research shows that stronger and clearer signals of the winning coalition's preferences are more likely actually to influence policy \cite{Fassiotto2017}. Expressions of the public's preferences, such as public opinion and protest, have been found to matter to leaders, especially during key points in the political process, such as agenda-setting \cite{Baumgartner2015}. If both contact and economic benefits are related to stronger preferences over the US military presence, then it means that both of these factors are contributing to the host state public sending stronger and clearer signals to their leaders about their willingness to host a US military deployment. If negative interactions and harms are producing opposition from large coalitions within a society, then it sends a clear signal to the government that supporting a US presence may not be a sustainable policy preference. If more robust and more transparent signals are more likely to influence policy, then we can see how this reaction could affect the hierarchical relationship between the US and the host country. 
	
	% put in the stuff here about soft power?
	
	Though governments are faced with overwhelming amounts of information when making decisions, some of it becomes more salient than others. This is why activists and interest groups can have more influence than other actors within the political system; because they care more about an issue, they can spend more time and resources trying to influence the government (cite stuff here). It is true that when making policy decisions, governments will gather information themselves (through intelligence agencies, for example), but much of that information is provided to them by a variety of actors who want to transmit information to the government, such as activists, journalists, or interest groups \cite[p.15]{Baumgartner2015}. Governments, particularly more autocratic ones, will often try to suppress some of that information that does not fit their preferences or current policies, but they are still receiving the information. As \citeasnoun{Lohmann1993} notes, all else being equal, leaders would prefer to take actions that are preferred by the majority of the population. The ``all else is being equal'' is, of course, key in this point. We are not arguing that leaders who have a strong interest in maintaining a US military presence will be convinced to force US troops out because public opinion does not favor the military presence, but rather that negative perceptions of the US military are of concern to host country leaders, and that those leaders could potentially react by reducing the US military presence, or by requesting more policy concessions from the US in exchange for maintaining the US military presence, as a way to offset the political costs of an unpopular deployment. 
	
While negative views of the US by host country publics are bad in and of themselves, they are also problematic for US foreign policy. As individuals' negative views towards a US military presence deepen, they will be more likely to mobilize (be that through voting, social media posts, legal action, or, as we explore in this chapter, protest) against the military presence. As we have previously noted, acts of dissent, like protest, have the potential to impose costs on host country leadership. The government may in turn try to pass those costs on to the United States as a condition for continuing to host the troops. Popular opinion mobilization has removed United States forces from the Philippines and Spain and limited the functional operation of the United States Air Force in Turkey during the build up to the 2003 Iraq War \cite{cooley2008,Kakizaki2011}. Sustained opposition fed by grievances fundamentally weakens the US position and its ability to maintain its troop presence, or at least makes it more costly.
		
	%%here might also be a good place to discuss autocracies and why this stuff still matters for them too. 

%%this is a footnote from the protest chapter. Figure we can use it as a starting point to discuss non-democracies. 	
	We also note that protests have influenced even non-democratic regimes in leading to policy change. The Arab Spring protests were an example of a case in which non-violent protest was able to achieve concessions, if not outright regime change in most cases \cite{Chenoweth2013} [cite McManus and Yarhi Milo on autocracies here]
	
\subsection*{Competitive Consent}
	
	This chapter presents a series of logical steps that builds to our final argument. The United States deploys troops and bases abroad to have effective influence in regional and global affairs. Having the ability to rapidly respond to emerging events, assure allies, deter rivals, and shape the foreign policies of host states is fundamental to the Grand Strategy of the United States for the last nearly 80 years. These enablers of US foreign policy come with a host of negative and positive effects that are distributed asymmetrically across host state populations. People within the host state have pre-existing views that, when combined with their perceived costs and benefits from the troops, serve to shape their view about deployments. Sufficient positive experiences and support can mobilize actors to prefer policies, politicians, and parties that advocate for a continued US presence. Sufficient negative experiences and diffuse public benefits can mobilize people to prefer polities, politicians, and parties that advocate against the continued US presence. During the Cold War, the United States' basing portfolio contained several autocratic sites as it sought expedient access, autonomy, and used the vestiges of former colonial empires to build its basing network \cite{calder2007,Harkavy2013}. Since the US has increasingly pursued democratic states as hosting sites as non-democratic regimes have become increasingly untenable in US domestic politics and several former autocratic host states have democratized, it follows that public opinion will increasingly matter for the longevity of US basing strategies. Having fewer incidents like those in the Philippines, Spain, France, and Turkey will increase the operational environment for the United States and allow it to continue to compete with rising powers. 
	
	In periods of unparalleled hegemony, such as 1990-2010 after the collapse of the Soviet Union, public perceptions of US basing mattered less as there were fewer alternatives for states that sought to outsource some part of their security to the United States. Certainly, being removed from a basing location is costly in terms of material, time, and relocation. Bases, after all, are fixed assets that the US cannot simply redeploy but must construct anew when forced to a new location \cite{Allen2011}. Even relocating troops from one site to another can be a costly maneuver as it involved both the physical move as well as making sure that those personnel can receive adequate supplies; a move of tens of thousands of troops require building up extensive supply networks for those new locations. Despite the costs, in periods without competition, alternatives exist for the US to deploy to if it loses a basing site. Notably, the Turkish decision to reject US access for the 2003 Iraq war made operations more difficult and costly, the United States was still able to operate out of Kuwait and carrier deployments to the Persian Gulf, Red Sea, and the Mediterranean Sea. Even in this circumstance where there was an immediate need for a target site, the United States was able to find such an alternative despite being rejected by its first choice state. If, however, the United States had other rivals actively shopping for exclusive access for basing military personnel, such a shift may not have been feasible and will increasingly become less feasible in the future. 
	
	There are a few trends creating the conditions for limited US operationality in the upcoming decades. First, in the 2003 Iraq invasion, the United States preferred to use the territory of its NATO ally in the initial invasion, which underscores our argument that the United States decision to pursue democratic partners is a consistent feature of the post-Cold War era. Second, the United States is at the lowest percentage of troops overseas since the conclusion of World War II \cite{Huston1988,Warren2016,Allen2021a,Allen2021b}. Further cuts in the number of overseas troops will become increasingly painful as such cuts come at the cost of operational scope and flexibility as the United States will lack the forces to mirror its previous positions, reassurances, and deterrent capabilities. 
	
	%%this paragraph below might be a good place to also cite Brian and Renanah's China piece
	Third, the PRC is pursuing a network of overseas bases with its first deployment to Djibouti in 2017. The Department of Defense expects that China may expand to a whole subset of countries including Myanmar, Thailand, Singapore, Indonesia, Pakistan, Sri Lanka, United Arab Emirates, Kenya, Seychelles, Tanzania, Angola, and Tajikistan \cite{OSD2020}. China's own internal projections from the Naval Research Institute proposed new bases in ``the Bay of Bengal, Myanmar, Pakistan (Gwadar), Djibouti, Seychelles, Sri Lanka (Hambantota), and Tanzania (Dar es Salaam)'' \cite[p. 207]{Doshi2021}. Each of these additions will likely come at the expense of a potential basing site or country of operation for the United States, along with the degree of influence for the U.S. that may come along with it. Some countries are able to negotiate non-exclusivity with basing powers. Djibouti has French, American, and now Chinese bases on its territory while also hosting British, German, and Japanese personnel through use of their allied bases. However, Djibouti is likely to be a rarity in the coming decades as the competition between the US and China increases and concerns about logistical operations, security, and espionage makes cohabitation between the competing powers less tenable. While the PRC has been able to gain a foothold in Djibouti, the US will seek to maintain exclusivity with it security partners and the PRC will likely do the same in countries where it finds a receptive basing environment.
	
	The People's Republic of China has engaged in a sustained soft-power campaign for the past two decades where countries and civilians see it as a competing economic power and less of a military rival. When we interviewed a pair of Members of Parliament (MPs) in the United Kingdom, one theme was that there is a disconnect between the elite view of China and the people's view of China \cite{mpone20190717,mptwo20190717}. A marked difference between an American city and London is the number of Huawei advertisements throughout London, which are noticeably absent in New York and Los Angeles. One MP noted that public perception of Chinese influence happens from both China's economic investment as well as the diaspora community within the UK Chinese immigrants tend to be less involved politically than other immigrants within the country and are not seen as a politicized issue when compared to other immigrant communities \cite{mptwo20190717}. So, while political and security in the elites see a rising China as a threat to British security, the average citizen does not consider it a priority compared to other domestic and international issues. Further, these are dynamics that are occurring in one of the closest allies to the United States. Public perceptions are potentially even more greatly influenced by Chinese development aid and investment in places that are subject to much lower levels of countervailing pressure from the United States. This dynamic can be seen in the stable or increasingly positive views of China among Africans, who have seen some of the most concerted soft-power campaigns from the PRC. These positive views have persisted, despite controversies over China's authoritarianism, role in the COVID-19 pandemic, and increasingly assertive relationship with its debtors. These issues have thus far failed to filter down to the average person's view of China, where 59 percent of people viewed China favorably\cite{afrobarometer2020}. 
	
	This disconnect between the economic policy and the security policy of the PRC will fade as China becomes more active in security issues globally and as individuals have more interactions with PRC military forces as its basing network expands. However, in regard to competing basing networks, there will be path dependency between the decision to host a base of a major power and to continue to host that base. Reworking security arrangements is an expensive process and there needs to be a large public outcry that politicians care about for a country to remove bases from its country, a process we have been noting throughout this chapter, but it is has an important different meaning here: Once a country decides to host PRC forces to the exclusion of the United States, it is unlikely to change that decision for decades and, even then, only after egregious issues arise due to that presence. Every country that opts not to host US forces but chooses to host Chinese forces instead will lock out the United States from a hosting site for the foreseeable future. Likewise, those that expel US forces from their country due to the existing legacy of host-state civilian-US military issues are unlikely to invite the US back to its country for an extended period. The Philippines do offer an example where American forces have returned to the country, but not at the level that there were in the 1980s.
	
	%maybe a number up here.
	
	
	%need a paragraph on how public opinion still matters in autocracies
	
	
