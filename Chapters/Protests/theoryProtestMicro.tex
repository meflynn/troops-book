\subsection*{Micro-Behavior Theoretical Expectations}
%\ref{cha:meth}
Predicting protests at a  macro level suffers some shortcomings for drawing conclusions about the relationship between military deployments and the act of protest. Primarily, there is an ecological inference issue where assuming individual motivations and actions from higher levels of aggregate information could be erroneous \cite{King2004}. We cannot be certain that information at the national level, such as the number of troops in a country, actually affects the population we seek to study (those that choose to protest). While we can make reasonable inferences that connect these concepts, we would be relying upon conclusions that our data do not directly speak to. Consequently, we also analyze the microfoundations of protest, as there will likely be a difference in how individuals make the decision to participate in protest versus how protests are explained at the national level. To better understand individual attitudes toward protest, we again use our survey data (as detailed in Chapter \ref{cha:meth}). 

Before considering individuals' protest cost-benefit analysis, we first note that in many cases, participation in an anti-base protest involves coming into contact with members of the US military, as protests often occur at the site of the military installations themselves (or if this is not allowed, as close to them as possible). Thus, while we do not imply causation between these two variables, we do expect there to be a positive correlation between protest participation and contact with members of the US military.\footnote{To try to separate those individuals who come into contact with the US military because they participated in an anti-US military protest from those who have everyday contact with the US military in their communities we will also test a hypothesis on the relationship between protest behavior and living in a province that houses a US military installation.} Analogous to our first hypothesis about states that have a greater US military presence being more likely to experience protest, we derive our first individual-level hypothesis:

\begin{hyp}
	All else being equal, individuals who have personal or network contact with the US military are more likely to participate in an anti-base protest. 
\end{hyp}

Beyond exploring the role of contact, there are several other attributes that may associate with the likelihood of attending an anti-US presence protest. When analyzing protest at the individual level, we can focus on the factors that influence decisions to protest. While we do not suggest causal relationships for these variables, we can find the conditions that better facilitate the ability for individuals to overcome collective action problems and mobilize in protest. As previously mentioned, the decision to engage in protest is essentially a cost benefit analysis, where individuals consider the costs and benefits of participating in a protest, as well as the probability of the protest succeeding.
 
Given this micro focus, we can focus on ideas, feelings, or experiences that we can incorporate into the rational choice framework we are operating under \cite{Elster1999}. For example, emotions can help the individual make a decision under rationality by affecting preferences over outcomes and determining what is most salient to the individual at a given point in time \cite{deSousa1990,Pearlman2013}. Using this framework, we explore how anger can affect individual decisions to engage in protest. As we discussed in the macro-level theory section, research on mobilization shows that anger can make individuals more likely to mobilize, even when there are costs involved in doing so \cite{Gurr1968,Goodwin2009}. In turn, the motivating emotion can then transform into a collective emotion of solidarity that is shared among a group and keeps the protest movement going \cite{Goodwin2009}. Anger is of particular relevance to mobilization because as \citeasnoun[p. 388]{Pearlman2013} argues, anger is an ``emboldening emotion''. This means that anger influences individuals' cost-benefit analysis by making them more optimistic about their possibility of success, in this case of achieving their aim through protest. Anger is an emotion that drives individuals to action in order to right a perceived wrong \cite{Carver2009}. This is important, because the costs and benefits of each expected outcome have to be weighed by the possibility of each outcome occurring, in this case the probability of success in achieving the aim of removing (or modifying) the US military presence through protest. When individuals experience anger, they can be driven to mobilize, even if the probability of success, and therefore the utility of protesting, would otherwise be low.  

%Note: The Garcia & Young piece notes that violent crime is more likely to provoke anger, so if we want to distinguish across different kinds of crimes, we can always cite that as justification

Of course, emotions are particularly difficult to measure, as even assuming that individuals respond sincerely when asked about an emotion in a survey, levels of emotion such as anger vary in time. The anger that made an individual willing to mobilize in the past may no longer be present at the time of surveying. We therefore focus on particularly traumatic events that are highly likely to be correlated with anger; in this case being the victim of a crime. Previous research finds that individuals who are the victims of a crime, particularly ones who perceive themselves as innocent, are more likely to experience anger (and in turn to engage in mobilization against governments who have failed to protect them from crime) \cite{Garland2012,Garcia2019}. We assume that if an individual was the victim of a crime perpetrated by a US service member, then they would likely fault the US military, at some level, for its perpetration. We expect that those individuals will be more likely to participate in anti-US protests. 

Crime victimization may serve an added, direct route towards mobilization. Specifically, by being a victim of a crime, the survey respondent has experienced a specific, direct cost of the military presence. The negative externality of US troop deployments concentrates on the individual. Given their experience, they are more likely to see the troops as a net negative in their community and engaging in protest is one way for them to express their preferences over the presence of the troops. We believe that those individuals who have been the victim of a crime perpetrated by a US service member will be more likely to respond to mobilization efforts to oppose the United States' military presence. 

 

%\begin{hyp}
	%All else being equal, individuals who have been criminally victimized by the US military are more likely to participate in an anti-base protest. 
%\end{hyp}

As we have mentioned before, a significant part of the individual's cost benefit analysis used in determining whether to participate in a protest is calculating the probability of success; deciding to participate in a demonstration is a costly act (it requires time, transportation, and represents a host of opportunity costs) and engaging in protest with little likelihood of success would not be an attractive opportunity. Consequently, we expect that potential protesters care about the effect of their actions. Much of the probability of success of public demonstration depends on the potential protesters' capacities. 

In the context of theories of mobilization, a second strand of explanations focuses on the resources that individuals have available to them. While grievance-based explanations are common for civil wars, they are less satisfactory in explaining the likelihood of conflict for a few reasons. First, grievances often present collective action problems for mobilizers and the existence of a grievance is not sufficient for them to get people to mobilize around unless the direct cost to the individual is significant. Even then, free riding on the efforts of others tends to be a more tenable option than risking the individual's own resources or life in some cases \cite{lichbach1993}. Second, grievances are ubiquitous across societies as there are always issues that people want remedied. Social entrepreneurs that engage in mobilization can certainly capitalize on particular grievances by offering selective incentives, but this suggests that grievances by themselves are not sufficient.  Consequently, mobilization will only occur when groups have resources available to them that allow them to mobilize, provide selective incentives, or lower the costs of participation \cite{Olson1965,Tilly1973,Khawaja1994}. For example, in the case of the Arab Spring protests, many pointed to the availability of mobile phones and social media, resources which facilitated coordination, as facilitating mobilization itself \cite{Hussain2013}.\footnote{Similarly, during the Iranian Revolution, opportunities for social gathering also led to increased protest mobilization \cite{Rasler1996}}


We expect that situations in which resources are available for coordination will ease protest organization in the case of anti-US base protests. Of course, what types of resources an individual needs to mobilize will vary from person to person. One commonality across individuals, though, is that greater proximity to others and ease of communications should decrease the cost of mobilization. To better account for this, we consider whether individuals live in urban areas, which are likely to have greater access not only to other individuals (because of higher population densities), but also more reliable telecommunications networks and public transport, which should ease the ability to coordinate and travel to protest sites. 

In addition, those living in large urban areas (such as capital cities) are more likely to come into contact with transnational anti-basing activist organizations that will work to organize locals against a military presence \cite{Murdie2011,Murdie2015,Kiyani2020}. Prior work on protest and mobilization shows that having access to transnational activist networks, which already have in place the resources and infrastructure to facilitate protest, facilitates mobilization and leads to increased probability of protest. For example, that the German peace activist interviewed for this project was based in the capital city of Berlin, and he talked at length about the international connections to other activist groups he had. He noted that his organization had recently held an international conference against bases and war, with representatives from 40 different countries attending, and noting that he often gets invited to anti-base events in Japan \cite{berlinone20190723}.\footnote{He also highlighted the importance to mobilization of having an activism infrastructure in place, noting that the German peace movement has a very good peace infrastructure.} In addition, exposure to transnational activism may lead to individuals associating the American base with a broader network of American imperialism \cite{Immerwahr2019}.  

Further, urban areas are also more likely to house those groups that are more predisposed to protest. A common theme in our interviews, whether we were talking to US or host-country government officials, was that, similarly to how certain demographic groups are predisposed to feel more positively towards the US military, other demographics are more likely to protest. For example, there seemed to be agreement that students and labor unions were more likely to mobilize against the US military presence, a point echoed in the existing base protest literature \cite{Fitz2015,embone20180712,berlinone20190723,journ20180712,journ20180713,Allen2020}. Activists themselves seem to be aware of this because of failed recruitment efforts. Regarding the city of Wiesbaden, which is near the Clay Kaserne Army Garrison, the German peace activist noted, ``It is a horrible city to start a movement. Demographics are difficult. It is where the Russian Czars would visit. Lots of rich pensioners live there. Wiesbaden has a nice historical past, but the demographics are difficult for a peace movement'' \cite{berlinone20190723}.

%We are not using the base itself as a unit of analysis, so taking this out. CM. 
%In particular, we expect that deployments which are located in more heavily populated, urban areas will be more likely to draw protest because of the ease of protesting close to a US military installation. Of course, individuals do not have to protest at a base location in order to protest against a deployment, but we believe that the physical presence of troops is more likely to draw protesters than more remote deployments
%This presents us with the following hypothesis:

%cut urban hypothesis
%\begin{hyp}
%All else being equal, individuals living in urban areas are more likely to participate in an anti-base protest. 
%\end{hyp}

%\ref{cha:meth}
Related to this point, and as we discussed in the introduction to this chapter, we believe that individuals who are part of a community that interacts regularly with the US military will be less likely to protest the US military presence. In agreement with previous work, in Chapter \ref{cha:meth} we have found that even though contact with the US military is correlated with increased probabilities of both positive and negative views of the US military, the positive effect is stronger than the negative one \cite{Allen2020}. We have argued that an important reason for this is the fact that American personnel often become a part of local communities, creating both ``bridging'' and ``bonding'' opportunities between US personnel and host-state citizens \cite{Woolcock2000}.  This creates a sense of shared identity or experience that allows locals to overcome negative stereotypes about the US military. 

%Changing this up a bit to make sure we are differentiating from the contact hypothesis that we start the chapter with. This is about being part of a community that is close to a base.
Local residents are therefore more likely to support bases than more faraway residents who are more removed from the actual troops \cite{Fitz2015,Flynn2018}. Residents are more likely to feel a sense of affinity with the US forces. Over and over again during our interviews, regardless of which country we conducted them in, we heard reinforcing evidence for the idea that individuals whose communities were more proximate to US deployments were more likely to view the US military positively. This was true whether the interviewee viewed this relationship positively or not. 

%As noted in Chapter \ref{cha:meth} and in our previous work, this is an effect that is particularly strong in communities that are proximate to US bases or deployments, as there is also a network effect. Even if an individual has not had direct contact with a member of the US military others in their social network might. Individuals will be able to gather information about the US military from their social network \cite{Mcclurg2006,Huckfeldt2001}, having others in their social network interact with the US military leads to a tolerant environment  in which prejudice against the US military decreases and positive views of the US military increase \cite{Wright1997,Liebkind1999,Pettigrew2007}

For example, at the US embassy in Panama, a variety of individuals we met with referred to anti-US protesters as ``paid protesters'' and noted that most Panamanians felt positively towards the US because  of the long-term nature of the US presence and the ``strong cultural ties formed between American and Panamanian people'' \cite{embone20180712}. They specifically noted the high frequency of marriages between US service members and Panamanians. This point was confirmed by an interview with a former Panamanian President, who also noted that US Panamanian relations ``have been going very well'' \cite{embthree20180712,pres20180714}.\footnote{The President noted that intermarriage also created some problems, as Panamanians who married American service members often became stateless after renouncing their Panamanian citizenship but before acquiring American citizenship \cite{pres20180714}.} The same former President, when we asked him about more recent deployments that provide humanitarian assistance (such as Beyond the Horizon and New Horizons) said that they are viewed positively by the people who receive them, who perceive the economic benefits that they receive from them \cite{pres20180714}.

From the opposite perspective, a journalist in Panama City who noted that he had previously participated in anti-US marches made no secret of his suspicions over humanitarian outreach carried out by deployed US military personnel. As he talked to us at a mostly empty grilled meat restaurant he had asked us to meet at, he expressed concerns that the US had these programs in place to spy on Latin America; that the humanitarian help is just a disguise. He noted that the people who benefit from these programs (most of whom reside in rural areas) had ``low education levels and critical analysis skills'' and did not analyze the ``political implications'' of the help they received. Yet he acknowledged that they felt positively towards the deployments \cite{journ20180713}. A former Panamanian Cabinet Member, this one speaking to us at a much higher-end restaurant, was much less suspicious about the secrecy of US deployments. At the same time, she was also unenthusiastic about humanitarian-oriented deployments, noting that civilian personnel should deliver US aid. Yet she also noted that Panamanians who had regular interactions with deployed personnel had positive views of the US military. Her particular example involved sex workers, who she said had a preference for US military clients, as they were considered to be more physically fit and attractive \cite{journ20180712}. 

Given this, we expect that those individuals living in areas in which the US military is incorporated into society will be less likely to participate in anti-base protests. Even though we note in our initial hypothesis that contact with the US military makes protest more likely, by the simple logic that protest itself in many cases involves coming into contact with US service members, we also expect that individuals living in provinces that house a US military installation will be less likely to participate in an anti-base protest. 

%[NOTE: Add more here if needed. We can draw from interviews, etc. This is where we are linking directly to our theory from Ch. 2, so it should feel like it's a strong part of the chapter]

%\begin{hyp}
%All else being equal, individuals living in provinces that house a US military installation are less likely to participate in an anti-base protest. 
%\end{hyp}

%%%This might be too hard to do, not erasing it, but this seems like too hard of a road to go down**
%*****Note: we should have a hypothesis about the role of transnational activist groups.  A transnational activist presence should make protest more likely, but the transnational activists have to be perceived as sincere and credible, or otherwise they may draw backlash against them (Fitz-Henry 2015). 

%As state above, and as argued by \citeasnoun{Fitz2015}, the association of a military base with a broader idea of American imperialism, rather than with a local, bilateral and consensual agreement between the US military and the local government, is more likely to lead to opposition to the base.  Quasi-bases, which are ``foreign military bases that are not supported by a formal agreement'' are easier for governments to hide from potentially hostile opposition and civil society actors\cite{Bitar2016} (p. 50). For example, in Latin America, as countries democratize and leaders face more political backlash from hosting US military bases that do not provide widespread benefits that include political opposition, the US has come to depend on informal agreements under which US military personnel deploy to existing military installations rather than building new American bases \cite{Bitar2016}. 

%Thus, we expect that US military bases will experience less protest in cases in which they are housed within an existing military installation that also houses the host country's military.  As \citeasnoun{Fitz2015} notes, bases that are framed as forward operating locations rather than bases per se (which are viewed as more permanent and intrusive), will be more likely to draw negative attention and thus experience protests. 



 

%**NOTE: The USAF base in Manta is one example; I'm guessing that Lakenheath will also be a good example of this.  Hopefully we can insert a good anecdote about it here.  Bases in Germany and Japan are examples of the opposite.CM.**

%\begin{hyp}
%All else being equal, anti-US base protests are less likely to occur when US troops are housed alongside host country troops in an existing military installation. 
%\end{hyp}

%\begin{hyp}
%All else being equal, anti-US base protests are less likely to occur when US troops are present as part of a forward operating location rather than a permanent base. 
%\end{hyp}

%NOTE: add transition

While some of our interview subjects, particularly those who represented the US government and military, referred to US military spending and investment as contributing to pro-US views, we do not expect protest to correlate with pure measures of military expenditures in the country or in the area.\footnote{For example, an interview subject at the US embassy in Panama attributed the high level of acceptance for the US in Panama to long-term and widespread US investment in Panamanian infrastructure \cite{embthree20180712}.}  As we have argued in previous work and as other studies have found (see for example \citeasnoun{Fitz2015} on the case of the city of Manta in Ecuador), economic motivation and direct financial gains only go so far in creating a sense of acceptance \cite{Allen2020}.  In fact, excessive spending can even lead to negative externalities such as inflation in the area and locals being priced out of their homes \cite{Hohn2010,Fitz2015}.  We instead rely on an argument that focuses on negative and positive views of bases being what leads to protests, rather than purely financial motivations.
%\ref{cha:meth}
Though we do not expect aggregate military expenditures in an area to influence protest, we do believe that at the individual level, economic benefits received from the US military may be related to protest participation. In Chapter \ref{cha:meth} and in our previous work, we show that personal or network economic benefits from the US military can lead to more positive views of US actors and the US presence in the host country \cite{Allen2020}. As we have noted earlier in this chapter, we believe that those individuals who perceive the US military more positively will be less likely to participate in anti-US military protests. Unlike contact with members of the US military, economic benefits are not something that we expect to result from participation in an anti-base protest. 

%For this research, we are less concerned about the direction of causality in this case and derive our last micro-level hypothesis: 


%\begin{hyp}
	%All else being equal, individuals who receive, or whose network receives, economic benefits from the US military are less likely to participate in an anti-base protest. 
%\end{hyp}

%I figured we should go with one or the other, so probably economic benefits being negatively correlated to protest behavior? 

%\begin{hyp}
	%All else being equal, individuals who receive, or whose network receives, economic benefits from the US military are more likely to participate in an anti-base protest. 
%\end{hyp}



%aid-provision by the US military, in particular humanitarian work such as investing in infrastructure and other community engagement activities, is likely to lead to a lower incidence of protest.  Having the military engage in activities that provide needed projects to local communities can lead members of these communities to view the military  installation as a more fair exchange between them and the American military. Previous work shows that deployments that specifically engage in development-oriented work, such as building or renovating schools and hospitals and providing immunizations and medical care are related with more positive perceptions of the US military \cite{Flynn2018}. In this way, deployments that are present for a non-development-oriented mission may be able to avoid the negative backlash against them by providing projects such as infrastructure development or other forms of engagement with the local population.  