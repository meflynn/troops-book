
\subsection*{Micro-Behavior Models}

%At the micro level, we use survey data from surveys carried out in 14 different countries with a high level of US presence, with approximately 1,000 respondents per survey.\footnote{The countries covered in the survey were the same ones covered in previous chapters: Philippines, Poland, Australia, Portugal, Netherlands, Belgium, Turkey, Kuwait, Spain, United Kingdom, and Italy.} We conducted the surveys from early September until early November of 2018 and 2019. The specific question that referred to protest behavior asked: ``Have you ever attended a protest event against a US military base?'' The response options were the following: Never, one time, two times, three times, more than three times, and don't know/decline to answer.

In this section, we shift our attention back to our public opinion data to examine individual-level participation in protest events targeting the United States in general and the United States military more specifically. 


\subsubsection*{What Causes Individuals to Participate in Protest Activity?}

As noted earlier in this chapter, individuals carry out a cost-benefit analysis in determining whether to join an anti-base protest. To begin with, individuals must have some grievance that drives them to protest. In some cases, individuals will derive a benefit from participating in the protests itself and expressing their frustrations publicly. Of course, the protest is usually a means to an end, meaning that the protest is meant to change US policy or eject the US military from their country. The larger the probability of successfully achieving their desired outcome, the more likely individuals are to protest. Finally, the costlier protest participation is, either through the opportunity cost of participation or because of the consequences of it, such as repression by the government, the less likely individuals are to protest. 



\subsubsection*{Predicting Protest Activity}

The goal of this section  is twofold: First, we generally aim to predict individual involvement in protests against the United States military. Our survey questionnaire specifically asks individuals about their history of attending protest events. Using this information, we construct a series of predictive models to better understand the occurrence of protest activity among foreign publics. Unlike our country-level models of protest, these models do not explore the causal effects of various predictor variables and the coefficients for these variables generally do not have clear causal interpretations. Instead, we first focus here on constructing predictive models using available covariates and then assessing the model's overall performance in accurately predicting whether an individual will participate in a protest or not. 

The second goal will be to use this information to shed light on some of the individual-level characteristics that correlate with protest activity. Although we cannot offer more concrete causal interpretations given the structure of our survey, we can use the predictive models to generate useful descriptive comparisons across groups. This is effectively what we did in chapter \ref{cha:min} on minority views of the US military and other actors. In this case the coefficients are more usefully understood as providing information on differences in protest behavior between groups, once we adjust for various other predictors of interest.

\subsubsection*{Model Specification and Estimation}


Our survey asked individuals about their involvement in anti-US protest events. It asked them how many protests they had participated in. To simplify the prediction process we turn what is a count (of the number of protests attended) variable into a binary variable that simply indicates whether a respondent has previously attended an anti-US protest event or not ($1 = Yes; 0 = No$). 

We estimate multiple models to compare how well our models perform, beginning with a basic Bayesian multilevel logistic regression predicting protest participation, leaving the model unspecified except for the population-level intercepts, country-level error, and yearly error. In other words, this is a model that does not consider any individual characteristics, just tells us how likely an individual is to protest in a given year. This serves as our baseline model of protest involvement. Next, we estimate a similar model, but this time we include several individual-level demographic variables as reported by the respondents. This model includes only the individuals' characteristics and does consider their attitudes and views. Third, we estimate a fuller model including the individual-level demographic variables listed above, along with individual-level attitudes and experiences, which we ask about on a range of questions. Fourth, we estimate the previous model with demographic and attitudinal information again, but we add country-level variables including gross domestic product (GDP), population, the number of bases in the respondent's province, and the size of the US military deployment in that country.\footnote{Note that although we have data at the province level across all countries, we are often lacking in sufficiently large sample sizes at the province level to estimate reliable effects. Since we are grouping respondents according to country and year of the survey, the estimates for the base count variable are likely to be biased.} In other words, the fifth model allows for country-level characteristics to influence protest participation predictions as well. The fifth model builds on Model 4 by using all of the same predictor variables but adds varying coefficients for the age, gender, and ideology, and base count variables. Although the individual-level predictors are identical in models 3, 4, and 5, each model builds upon the previous in important ways. For example, by adding group-level variables in Model 4 we can account for systematic differences between the different countries included in our data. Similarly, what we are doing by ``varying coefficients'' in Model 5 is to allow for the possibility that different factors (such as, for example, gender) may influence protest behavior differently in different countries.  
This means that we are relaxing the assumption that there is a constant relationship between predictor factors and protest countries.\footnote{We add an additional layer of flexibility to the models by allowing the correlation between all of the varying or ``random'' to vary as well. This relaxes the assumption that the correlation between these effects is 0 and allows the model to estimate each instead.} Table \ref{tab:protestpredictionvariables} provides an overview of the specifications of these five models. The varying coefficients vary by country only (i.e. not by year). 

\input{../Tables/Chapter-Protests/model-protest-prediction-varlist.tex}

In using these types of models, it is important that we assess how good they are at predicting actual protest behavior. In order to do this, we divide our data into two samples: a training sample and a test one. This step helps to ensure that we are not overfitting our model to the particular data that we have collected. In other words, it ensures that the model is indeed a general one and not just one that works only with the data from the specific individuals we surveyed. This step also allows us to assess how well our model is able to predict protest when we test it out on ``new'' data that were not used to estimate the model (the ``test sample'' that we referred to earlier). To do this we group our opinion data according to the representative individual characteristics (which are age, income, and gender). We then randomly assign 80\% of the observations to the training data and the remaining 20\% to the test data. This ensures that both the training data and the test data are representative on these same variables. We use the training data to fit the models we present below. Once we fit the models, we use the test data to assess how accurate the models actually are at predicting protest.

In the pages that follow, we present the basic model-level results followed by diagnostic information to assess how including different categories of variables affects the model's overall performance. 