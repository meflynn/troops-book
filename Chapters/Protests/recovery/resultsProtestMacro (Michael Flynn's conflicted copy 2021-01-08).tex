\section*{Results}

In this section we review the results of our models. First, we review the models assessing the causal effects of U.S. military deployments on the number of observed protest events in countries across time. Next, we review the results of the models that focus on individual-level protest involvement as reported by the respondents in our cross-national survey data.

%%Note: I think we need some summary 

\subsection*{Macro-Behavior Results}

 Figure \ref{fig:atetroops} shows the estimated average treatment effects (ATE), and also the effect of the treatment history of troop deployments, for the two sets of models we estimate. The results of our marginal structural models indicate that U.S. military deployments do, on average, have a positive contemporaneous effect on both anti-U.S. protest events, and anti-U.S. military protest events more specifically. The ATE estimates for the anti-U.S. protest model appears in the bottom two panels and the estimates for the anti-U.S. military protest model appear in the top two panels. In each panel we display point estimates and credible intervals for the ATE, along with histograms showing the distribution of the ATE based on 10,000 simulations for each iteration of the model. As we note in the previous section, the fact that U.S. military deployments can increase or decrease sharply in a particular set of cases makes estimating the propensity scores and treatment weights difficult. To assess how sensitive our results are to the choice of truncation points for the weights we present the estimates for each of six separate truncation points. 
%%This is something for later, but we probably will want to have a couple of sentences explaining what the truncation points actually mean, for our less methodsy readers. CM.


\begin{figure}[t]
	\centering\includegraphics[scale=0.7]{../../Figures/Chapter-Protests/figure-ate-troops.pdf}
	\caption{Plots show the predicted average treatment effect of U.S. military deployments on protest events. Points represent the median point estimate and 50\%, 80\%, and 95\% highest density intervals. The histograms represent the distribution of the simulated draws from the model.}
	\label{fig:atetroops}
\end{figure}

Briefly, we find evidence of a positive average treatment effect (ATE) for the troop deployment variable in both models, indicating that countries that see an increase in the size of U.S. military deploymenys tend to see an increase in the likelihood of an anti-U.S. protest event, but also protests against the U.S. military, more specifically. Notably, the multiple iteratiosn of the model also show that the ATE estimate is indeed sensitive to the choice of truncation points, but the evidence generally indicates a positive effect. For example, in the case of the general anti-U.S. protests we find average treatment effect coefficient estimates ranging from approximately $-$0.06 to $+$1.24. However, the only negative value appears in the model run using the 10,000 IPTW truncation point, which is extremely large. The IPTW truncation point of 10 produces a mean IPTW score of just over 1, which is relatively close to the guidance given by \citeasnoun{ColeHernan2008}. Otherwise we find that the models using the IPTW truncation points of 10, 50, 500, and 1,000 all produce relatively similar positive estimates, and we generally find that as we increase the truncation point the error around the ATE estimates narrows and the coefficient decreases slightly. 

When we look at the models predicting anti-U.S. military protests we see that across all of the IPTW truncation points the ATE is estimated to be positive, ranging from 2.68 at the lowest truncation point to 3.85 at the highest truncation point. As in the previous model we find some evidence that the dispersion around the ATE estimates shrinks as the truncation point increases, but nowhere near as dramatically as in the more general protest model. It is also notable that the estimates are considerably larger for protests against the U.S. military as compared to the more general anti-U.S. protest model.  And while we do see some shift int he median ATE estimate across the various models, there is a considerable amount of stability among the ATE estimates for the four highest truncation points. This is due to the fact that there is far less variation in the anti-U.S. military protest variable as compared to the more general anti-U.S. protest variables, as can be seen in Figure \ref{fig:protesthistograms}. The maximum number of anti-U.S. military protests is 4 in a given country-year, while we observe multiple cases where there are 5 or more protest events in a given country-year. Recalling back to the research design, the structural weights are calculated on the basis of the treatment variable, and these essentially tell the model how many copies of a given observation to make when generating the pseudo data for the model. What this tells us is that, after a certain point, the introduction of additional observations from the pseudo data does not fundamentally change the relationships between the predictor and the outcome variable.

\begin{figure}[t]
	\centering\includegraphics[scale=0.7]{../../Figures/Chapter-Protests/fig-histogram-protests.png}
	\caption{Histograms showing the counts of anti-U.S. and anti-U.S. military protest events, 1990--2018.}
	\label{fig:protesthistograms}
\end{figure}


Notably, the treatment histories differ slightly in their expected effects. Larger deployment histories appear to cause a reduction in the expected number of anti-U.S. protest events in a given country-year. As with the contemporaneous effects, higher IPTW truncation points produce estimates with less dispersion, but the median point estimates themselves are fairly stable. Alternatively we find that evidence is more mixed when it comes to anti-U.S. military protest events, where we find some indication of a positive effect. At the lowest IPTW truncation point we find the coefficient value is quite close to 0 with about 50\% of the posterior for the ATE estimate falling above and below 0. For the other IPTW truncation points we find roughly an 87\% chance of a positive effect. Regarding the more general protest model, this negative effect of the treatment history makes some intuitive sense. Cases like Germany, South Korea, and Japan have long histories of hosting large U.S. military deployments. After a period of time these deployments may provoke less general hostility towards the United States. However, the possibility of a positive effect of treatment history on protests against the U.S. military also makes intuitive sense. In these same cases the presence of the military itself frequently remains highly contentious---even as individuals themselves frequently have close personal and professional relationships with U.S. service personnel in and around base communities. Even as individuals may come to be more familiar with Americans, reducing more general hostility, the presence of a long-term military facility suggests continued exposure to various negative externalities of the sort we have discussed, including pollution, crime, environmental degradation, and more. These stimuli are understandably likely to provoke a continued reaction among the host-nation's population. However, the results also suggest that this dynamic is more uncertain than the more general protest model.

%Add something like this to explain what the ATEs mean: In other words, an increase of in troop deployments would lead to an increase of .21 to .27 protests in a given country-year.

