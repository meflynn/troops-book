\section*{Conclusions}



%%NOTE: Add something about the "base in your area"' variable once we have that in the models.CM.
We opened this chapter with a conversation with a German peace activist, who recounted both the struggles and successes the German peace movement has faced in mobilizing the population against the U.S. military presence. Somewhat counterintuitively, he noted that while two-thirds of protesters mobilized by his organization are regional, ``the number of locals involved in the movement is growing, but not fast enough'' \cite{berlinone20190723}. While it may seem surprising that activist organizations would have a hard time recruiting from those communities that are closest to a U.S. military base, and therefore more likely to feel the negative effects of it, our analysis finds that some of these observations can be explained by both demographics and by individuals' experiences and perceptions.    

%discussing the successes and limitations of organizing anti-base activity in Germany. In Germany, anecdotally, opposition to bases is strongest in urban centers and not near the bases themselves. 
%Additionally, particular policies of the bases may drive organization while economic ties and good relations with servicemembers may decrease that opposition.

It is important to note that engaging in protest tends to be a higher-commitment form of political expression. Not everyone who dislikes the military presence will actually mobilize and protest against it.  The activist leader told us that he thinks there must be 500,000 people adversely affected by U.S. bases, yet the maximum turnout they can get at protests is 5,000 (``This gap is our challenge,'' he noted). While he expressed some bafflement at the fact that ``they are unhappy but they are not acting,'' he also seemed to have a strong understanding of why this is the case. The collective action problem is multifaceted as opposition groups want to increase their numbers, but the German and local governments and the United States prefer support for their position. Despite the 75-year history of U.S. bases in Germany, the German anti-base movement still faces challenges to overcome the collective action problem. Some of them stem from demographics, but others from the types of interactions that occur between locals and the U.S. military.

%Given the infancy of the anti-base movement, despite the 75 year history of U.S. bases in Germany, suggests that overcoming that organizing to protest requires a substantial effort of organizers. At least that is the case in Germany. 



%After surveying the literature, our expectations remained that those who have stronger incentives to protests were more likely to do so. Like previous literature, the conditions that favor reward participation in political action make protest more likely.

In this chapter, we looked beyond individuals' preferences and views on the U.S. military to see how those views manifest in political action. Considering both events-based and survey data, we identified the conditions that predict anti-U.S. military and anti-U.S. protests and protest behavior. We find evidence that there is indeed a causal relationship between the number of U.S. troops deployed to foreign country and the number of protests in that country against the U.S. and the U.S. military more specifically. While this may seem obvious, establishing causality is important because of the strong policy implications it carries. When deciding whether to deploy more troops abroad, the U.S. government should be aware that the increase in troops, even when there is already an existing deployment in place, will be likely to lead to increase in anti-base protests. In addition, this effect extends to more general anti-American protests, not just those focused on the military presence. Thus, a larger military presence can result in the expression of more anti-American sentiment.

Of course, reducing the size of its military deployments abroad is not always a realistic option for the United States government. We thus argue that the effect of the military presence on protest can be attenuated by a variety of other factors, many of them within the control of the U.S. government and military. In general, as we found in \ref{cha:meth}, contact and economic relationships can reduce the negative perceptions of the United States military. These interactions could produce basing situations that would be less likely to experience anti-U.S. protest as a reaction to a military deployment, and thus could create better hosting environments for the U.S. military. If the contact is negative, that can actually make people more likely to want to protest against the U.S. military, as we find in this chapter. 

%Though these are predictive, not causal models, they do provide information about the settings in which anti-U.S. protests are most likely to occur in. 

Within countries there is also variation in whether individuals will be more or less likely to want to protest against the U.S. military presence. Some of this propensity is related to demographic characteristics. People who are poorer and in democratic countries are more likely to self-report having participated in an anti-US military protest. In Chapter \ref{cha:min}, we discussed the unique place minority members in a host state have relative to bases. They are more likely to pay the costs of the base (through location, environmental consequences, and negative interactions) while reaping less of the benefits (national security, direct economic ties). We find continued support for this argument in that self-described minority status correlates with an increased likelihood of participating in protests.


Beyond demographics, people's experiences are also correlated with the probability that they will respond yes to the question about having participated in anti-U.S. military protests. The roles of contact, economic reliance, and respondent social network continued to play important roles as they did when we focused on perception of U.S. actors. While the models for the survey data were correlational and predictive, we did see that those with more contact (exposure) were more likely to protest a military base. Likewise, economic reliance correlated with increased protest attendance. Exceptionally negative experiences, such as being the victim of a crime committed by a member of the U.S. military, or knowing someone who had been victim of a criminal offense, also correlated with individuals being more likely to attend more anti-base protests. 

As the United States enters an era that requires more consent in maintaining its basing agreements and hopes to continue to received substantial burden sharing from allied states, the consent and support of the host country population will become increasingly important. Even though individual members of the population do not make hosting decisions themselves, their protest activity can indeed influence their governments' policies towards the United States. When protests are successful and influence public opinion, they will make the basing more costly for the military.

%loop back to the super argument



