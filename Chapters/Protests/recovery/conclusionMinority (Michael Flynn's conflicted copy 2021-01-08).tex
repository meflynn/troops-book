\section*{Conclusions}

In our interviews across Europe, we spoke with an anti-base activist in Berlin. During the conversation, he made it clear that minority populations, particularly immigrants and refugees, were largely an afterthought in the dynamics of basing: ``They are not playing a key role in the base discussion. The Germans and the U.S. have security concerns related to hiring immigrants and refugees to work on the base. They are afraid to hire immigrants. They have strong security checks and controls. They are afraid to hire someone from Al-Qaeda'' \cite{berlinone20190723}. At the beginning of the chapter, we discussed the idea that many minority populations may see U.S. military installations positively through the potential economic and social opportunities that they provide. However, as both this anti-base activist and the Government Relations Officer at the U.S. installation in Wiesbaden attest, there was little consideration given to local minority populations at all, let alone any concerted efforts undertaken to reach out to such communities. To the extent that minority populations are considered, it is to sideline them from the types of employment otherwise available to local people on the base. 

The data that we collected from countries around the world offers a glimpse into how these issues present themselves in public opinion. In our most general models minority groups are less likely to report positive views of U.S. actors than majority groups. When we allow the effect of minority self-identification to vary across countries, however, we find that the effect of minority status on views of the U.S. is more mixed. In many countries minority groups are less likely to report positive views of U.S. actors than non-minority respondents, but often this simply translates into more neutral views, while in other cases there is a clearer increase in actual negative sentiment.
%Something on the contact variable, as well - AS

What we have demonstrated in this chapter is that it is important for host nation populations to be seen in more complexity. In particular, we find that there is no clear answer to the question of whether or not minority populations view the U.S. more or less favorable, and the exact nature of this relationship is dependent upon national, and even sub-national, factors. For example, contrary to many popular assumptions, \citeasnoun{Johnson2019} describes the relationships between the Okinawan population and the U.S. military personnel as a deep, multilayered subject, characterized by both affection and nostalgia, but also resentment and anger. The people in host nations do not have uniform experiences of the U.S. military, and they do not have uniform views. As discussed in Chapter 4, views can depend on whether individuals have had interpersonal contact with members of the U.S. military, and in this chapter, we have uncovered another complicating factor---minority status. Minority communities in some cases may be on the receiving end of more negative externalities than majority communities, which may inflame tensions and exacerbate existing hostilities towards the U.S. and host government. In such cases, the negative relationship between minorities and the U.S. military is often direct, and the latter's responsibility for social, economic, or environmental ills facing minority communities is clear. In other cases there may be more indirect relationships, like the one described in Germany where minority populations are given very little consideration in the dynamics of military basing. In such circumstances, a more circuitous causal mechanism is likely at work, in which the American relationship with the host central government and majority empowers it. By intertwining itself with the majority central government, the United States becomes conflated with groups contributing to minority discrimination and reducing pathways to minority representation through the centralization of power. However, in other cases and circumstances we find that minority status can correlate with more positive views of the U.S. military, but also of other U.S. actors. As we discuss above, in these cases it may be the case that U.S. military bases provide minority populations with alternative avenues for social or economic advancement. 



Much as we would expect from public opinion polling in the United States, an individual's identity matters a great deal in how they view the world. This is due to the fact that an individual's experiences will influence their views. The same is true abroad and should be considered in any analysis of how the United States interacts with societies in the international arena. While the U.S. military may see a high degree of support from majority groups in some countries, not all populations have uniform experiences and views of the American military presence. Ultimately, understanding how these positive and negative forces compete is essential for better understanding the politics of U.S. basing and mass attitudes in the host country. Although minority communities are often marginalized in broader political discussions concerning U.S. basing, the fact that marginalized groups are often the most exposed to U.S. bases and their consequences also means that they are some of the most likely groups with whom U.S. personnel are likely to interact. Unique experiences with the U.S. military, the host government, and the majority population in the host country will influence individuals' view. From a policy perspective, this is vital issue to understand and apply to U.S. military relations with local populations in host countries around the world. Even in societies that see high levels of favorable views toward the U.S. military, negative views among a cohesive minority population can cause any number of issues---from security concerns to land use issues and the stability of political support for a U.S. presence. In previous chapters we argue that outreach to surrounding communities is a way in which the U.S. military can improve relations with surrounding communities. Extending these outreach efforts to minority groups, and specifically tailoring to their wants and needs, would be a way to build better relationships with minority populations and promote better relations between them and central governments.  

%outreach recommendation might depend on what the contact variable looks like for minorities. - AS