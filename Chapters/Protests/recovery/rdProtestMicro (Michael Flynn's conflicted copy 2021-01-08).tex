
\subsection*{Micro-Behavior Models}

%At the micro level, we use survey data from surveys carried out in 14 different countries with a high level of U.S. presence, with approximately 1,000 respondents per survey.\footnote{The countries covered in the survey were the same ones covered in previous chapters: Philippines, Poland, Australia, Portugal, Netherlands, Belgium, Turkey, Kuwait, Spain, United Kingdom, and Italy.} We conducted the surveys from early September until early November of 2018 and 2019. The specific question that referred to protest behavior asked: ``Have you ever attended a protest event against a US military base?'' The response options were the following: Never, one time, two times, three times, more than three times, and don't know/decline to answer.

In this section we shift our attention back to our public opinion data to examine individual-level participation in protest events targeting the United States in general, and the United States military more specifically. 



\subsubsection*{Predicting Protest Activity}

The goal of this section  is twofold: First, we generally aim to predict individual involvement in protests against the United States military. Our survey questionnaire specifically asks individuals about their history of attending protest events. Using this information we construct a series of predictive models to better understand the occurrence of protest activity among foreign publics. Unlike our country-level models of protest, these models do not explore the causal effects of various predictor variables and the coefficients for these variables generally do not have clear causal interpretations. Instead, we first focus here on constructing predictive models using available covariates and then assessing the model's overall performance in accurately predicting outcomes of interest. 

The second goal will be to use this information to shed light on some of the individual-level characteristics that correlate with protest activity. Although we cannot offer more concrete causal interpretations given the structure of our survey, we can use the predictive models to generate useful descriptive comparisons across groups. This is effectively what we did in the chapter on minority views of the U.S. military and other actors. In this case the coefficients are more usefully understood as providing information on differences between groups, once we adjust for various other predictors of interest.

\subsubsection*{Model Specification and Estimation}


Our survey asked individuals about the relative frequency of their involvement in anti-U.S. protest events. To simplify this process we turn this count variable into a binary variable that indicates whether a respondent has previously attended an anti-U.S. protest event or not ($1 = Yes; 0 = No$). 

We estimate multiple models to compare model performance, beginning with a basic Bayesian multilevel logistic regression predicting protest participation, leaving the model unspecified except for the population-level intercepts and country-level and yearly error. This serves as our baseline model of protest involvement. Next, we estimate a similar model, but this time we include several individual-level demographic variables as reported by the respondents. This model includes only individual characteristics, omitting variables on attitudes and views. Third, we estimate a fuller model including individual-level demographic variables listed above, along with individual-level attitudes and experiences on a range of questions. Fourth, we estimate the previous model with demographic and attitudinal information again, but add country-level variables including gross domestic product (GDP), population, the number of bases in the respondent's province, and the size of the U.S. military deployment in that country.\footnote{Note that although we have data at the province level across all countries we are often lacking in sufficiently large sample sizes at the province level to estimate reliable effects. Since we are grouping respondents according to country and year of the survey, the estimates for the base count variable are likely to be biased.} The fifth model builds on Model 4 by using all of the same predictor variables but adds varying coefficients for the age, gender, and ideology, and base count variables. Although the individual-level predictors are identical in models 3, 4, and 5, each model builds upon the previous in important ways. For example, the inclusion of group-level variables in model 4 can help us to account for systematic differences between the different countries included in our data. Similarly, the inclusion of varying coefficients adds a layer of complexity and flexibility to the final model, allowing the correlations between the predictors and the outcome to vary across country, thereby relaxing the assumption that there is a constant relationship between predictors and outcome across countries.\footnote{We add an additional layer of flexibility to the models by allowing the correlation between all of the varying or ``random'' to vary as well. This relaxes the assumption that the correlation between these effects is 0 and allows the model to estimate each instead.} Table \ref{tab:protestpredictionvariables} provides an overview of the specifications of these five models. The varying coefficients vary by country only (i.e. not by year). 

\input{../Tables/Chapter-Protests/model-protest-prediction-varlist.tex}

Since we are interested in assessing the predictive power and accuracy of these models we divide our data into a training and a test sample. This step helps to ensure that we are not overfitting our model to the particular data that we have collected, and also serves to assess the predictive performance of the model on ``new'' data that were not used to estimate the model. To do this we group our opinion data according to the representative individual characteristics (i.e. age, income, gender). Then we randomly assign 80\% of the observations to the training data and the remaining 20\% to the test data. This ensures that both the training data and the test data are representative on these same variables. We use the training data to fit the models we present below. Once we fit the models, we use the test data to assess the predictive accuracy and power of the models.

In the pages that follow we present the basic model-level results, followed by diagnostic information to assess how the inclusion of different categories of variables affects the model's overall performance. 