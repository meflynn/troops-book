
\section*{Research on Anti-US Protests}

To better understand protests against the US and its military facilities, we take two different tracks to disentangle the empirical relationships that appear to be either causal of or related to protest behavior. First, we examine protests through a macroscopic lens by considering the likelihood that a country experiences protests in a given year. This allows us to identify structural conditions that make protest more likely. Second, we return to our survey data and study the conditions that correlate with protest behavior at the individual level. What do our surveys say about individual behavior? Are there demographic, ideological, or geographic factors that are associated with an individual attending an anti-US protest? What predicts the likelihood of an individual engaging in protests against the United States or military bases? To better understand both sets of frames, macro and micro, we first turn to the established literature.


\subsection*{Macro-Behavior Theoretical Expectations}
 
%%Moved these two paragraphs to theory chapter

%There is a significant amount of work, particularly from a qualitative perspective, which has studied how it is that anti-US base protests can be successful in influencing host country policies. Work by Yeo \citeyear{Yeo2011} shows that successful anti-base mobilization efforts stem from two processes. First, the country must have a fractured elite security consensus. This fracture allows a protest movement ``air to breathe'' without the entirety of the country's elite establishment denying it. Since much of the public will take their cues from elites, many will avoid protests in the presence of an elite structure that stands united behind the idea that the host country shares security threats with the United States, that the country needs help from the US, and that American forces are present for these purposes. The opposite is also true: in the presence of prominent elites who question such tenets, individuals will be more likely to question the presence of American forces and join protest movements. 

%Second, Yeo explains that successful protest movements include broad sections of the host state's society. This includes variation across ethnicity, gender, income, region, and more. Protest movements confined to single demographics are unlikely to succeed, as there is less incentive for host governments to respond to a fraction of the population. When host governments face broad coalitions of anti-base protesters, their political survival is at stake and it is difficult for them to build a counter-coalition. A broad-based movement also signals the salience of an issue by cutting across demographic lines that would normally remain isolated or even opposed to each other. While we do not purport to explain the success of protest movements, theories from \citeasnoun{Yeo2011} stress the importance of studying the determinants of protest. If certain factors make individuals more likely to engage in protest, and these protests can indeed be successful in changing host country policies on US  military bases, then protest movements can fundamentally weaken the United States' international position and its ability to maintain a global military presence.  

%%%

%for the theory, situate it in the context of bad contact = dissent/opposition and fundamentally weakens the US' position/ability to stay

When explaining anti-US protests, we first discuss the determinants of protest and mobilization in general which are likely to also be drivers of anti-US protest.  We conceptualize protest as a form of collective action.  Like other forms of collective action, such as rebellion, protest can run into free-riding problems \cite{Olson1965,Lohmann1993}.  If others protest, a given individual can reap the benefits of the protest (policy change, for example) without incurring any of the costs (lost time, possible arrest, suffering violence, etc.).  Much like rebellion, or even non-violent actions such as voting, a single individual is unlikely to make a difference in the probability of a movement succeeding, but if enough individuals follow the same logic and attempt to free-ride off others' participation, the movement can fail \cite{lichbach1993}.  Therefore, we focus on the type of situations that make collective action, in this case anti-US protests, more likely to take place.  

One strand of explanations of collective action and mobilization focuses on relative deprivation; \citeasnoun{Gurr1968} provided much of the foundational work on relative deprivation in the early literature on the causes of civil war.  The idea behind this theory is that the societies that are most vulnerable to domestic political violence are not those that have the poorest people in the world, but those in which individuals are relatively worse off when compared to others within their own society. When individuals perceive that they are not receiving the benefits that they believe they deserve (as they can observe others in their society receiving) they will be more likely to experience anger. This anger becomes the basis for their mobilization into action.  In other words, if there is a dissonance between what people expect and what people receive (more colloquially, those that have and those that have-not), individuals will be more likely to overcome collective action problems and mobilize into action. 

In the case of anti-base protests, we should expect to see more anti-US protests in cases where the population expected to receive greater benefits from a US military presence but instead the negative externalities have outweighed the benefits. This situation will be even more dire when there is a separation between those that receive the benefits of military deployments (e.g. landlords, business owners, and people in urban centers) and those that bear the burden of the costs of deployment (e.g. marginalized peoples, rural communities, minority groups, etc.). We can thus think of the population's decision calculus in determining whether to take part in protests as a cost-benefit analysis where they weigh the benefits that they obtain from the US military presence against the costs associated with it. The greater the costs are compared to the benefits, the more likely we are to observe populations mobilizing in order to attempt to influence the host government to remove the US military. 

There is likely to be variation in the specific factors or behaviors members of the host population are likely to find objectionable. That said, we argue that the larger the US military presence is, all else being equal, the greater the \textit{opportunity} for locals to have grievances about the US military deployments. More specifically, larger deployments should lead to increases in many specific factors that appear likely to stoke anti-base sentiment, such as crime, environmental degradation, increased traffic, noise pollution, industrial accidents, and more. Accordingly, we begin by exploring the simple question or whether or not larger US military deployments cause an increase in protest activity. Simply put, assuming that there is some proportion of members of the US military who will engage in objectionable behavior, having a greater number of forces present increases the probability of negative interactions that can mobilize the population into protest \cite{allenandflynn2013}.

%Andy, I'm guessing you added this hypothesis below. Does the explanation above fit with what you were thinking? CM. 

 %more stuff on grievances

\begin{hyp}
	All else being equal, larger US military deployments in a country should lead to a higher likelihood of anti-US protests within that country.
\end{hyp}

This will be our primary motivating hypothesis for our initial set of country-level models. There are several other factors that will influence protests as well, and we include them in our models. Though we do not offer explicit hypotheses about these factors, it is important that we take them into account to adjust for their influence and accurately predict instances of protest. While we expect troop increases to lead to an increase in protest, the size of this effect will be moderated by the environment in which troops operate. This leads us to expect there to be variation in how likely different countries are to protest a US military presence.  

%There is, of course, variation in how likely populations are to protest a US military presence, regardless of the size of the deployment. 

When considering what makes countries more likely to experience a protest against the US military presence in a cost-benefit analysis context, we can think about the kinds of settings in which the benefits of the military presence accrue to all members of the population, not just a specific group (such as supporters of a specific government). The cases in which benefits accrue to (almost) all members of the population are less likely to experience protest, as even if costs associate with the military presence, they can be offset by the benefits provided \cite{Bitar2016}. 

We thus argue that when the military provides public goods the entire population will benefit from them.  When the goods provided by the military presence are private ones, the host state government will distribute these benefits to only its supporters \cite{demesquita2005}. In the setting of military deployments, we can think of the economic benefits that a military presence brings with it (such as contracts given to local contractors) as private goods and the security provided by the troops (such as deterring attacks against the host country) as public goods. We argue that in those states in which the military is providing security to the population we will observe fewer protests, as security, unlike economic benefits, tends to be a public good that everyone, not just supporters of the government, can benefit from. In theory any country that has a US military presence receives security, but the utility of that security to the population can vary. In particular, we argue that in countries experiencing internal or external threats, the general population will derive more utility from the US military presence. We thus expect that states that are facing higher levels of internal or external threats will be less likely to experience anti-US protests.



%This example doesn't really fit anymore, so i greyed it out for now
%In the case of the US military base in Manta, Ecuador, which the US gave up in 2009 after pressure from the Ecuadorean government and civil society, crime in the area surrounding the base actually increased after the departure of US troops \cite{Fitz2015}. 


% (NOTE: maybe?  I'm just kind of making stuff of now as I think through it, will come back through many times later and clean all of this up)

%NOTE: I changed this to internal and external threats; I think this makes more sense. We need to expand on the threat part, though. We can just poach from the bit in the original paper which says that countries that face higher external threat are more likely to have favorable opinions of the US military



%\begin{hyp}
%All else being equal, anti-US base protests are less likely to occur when there is an internal rebellion in the host country.
%\end{hyp}

The literature on the ``rally round the flag'' effect also supports this expectation. States facing either internal threats (like terrorism) or external ones (like aggression form a foreign rival), garner bumps in leader credibility and unification of opposition parties with the governing party \cite{Chowanietz2011}. Internally, public opinion shifts in favor of executives when a new crisis emerges \cite{lee1977,Norrander1993}.  Researchers have not deeply delved into whether support for alliances increases when countries find themselves under threat, but there is evidence host states tend to match US expenditures in their territory when they face a collective threat \cite{allenetal2016}. Regardless, if the argument is generalizable, support for the government in the face of a security threat should increase support for the state's security commitments as well.

%\begin{hyp}
%All else being equal, anti-US base protests are less likely to occur when there are high levels of external threat against the host country.
%\end{hyp}



Related to the idea of relative deprivation, when people are doing better economically than they expected to, they should be less likely to want to engage in mobilization. For example, a State Department Regional Analyst at the US embassy in Panama noted that locals felt positively towards the US because of the strong Panamanian economy and low levels of unemployment that they attributed in part to US investment in Panama \cite{embthree20180712}. In contrast, when people are experiencing economic hardship, they might become dissatisfied at the American military presence.\footnote{Though we note that at extreme levels of economic downturn, which would lead to extreme poverty, we may see less mobilization. As noted by an interview subject, ``When you have to fight for your day to day surviving [sic], you cannot be active [against bases]'' \cite{berlinone20190723}.}  This may be due to individuals believing that they, in particular, should benefit economically from the base but are not able to gain that sought-after benefit while seeing others enjoying their desired economic fortune.  If they are not receiving any such benefits, anger may manifest from dissatisfaction and protesting becomes an outlet for that dissatisfaction. We thus expect economic downturn in a country to correlate with more anti-US protest.

%%expand on this and cite more stuff here. CM.


%\begin{hyp}
%All else being equal, anti-US base protests are more likely to occur when there is economic downturn in the host country.
%\end{hyp}

%Work by \citeasnoun{Yeo2011} also shows that successful anti-base mobilization efforts are produced by two things. First, the country must have a fractured elite security consensus. This fracture allows a protest movement "air to breathe" without being completely denied by the entirety of the country's establishment. Since much of the public will take their cues from elites, many people will avoid protests in the presence of an elite structure that stands united behind the idea that the country shares security threats with the United States, that the country needs help from the US, and that American forces are present for these purposes. The opposite is also true, in the presence of prominent elites who question such tenets, individuals will be more likely to question the presence of American forces and join protest movements. 

%Second, Yeo explains that successful protest movements include broad sections of the host state's society. This includes variation across ethnicity, gender, income, region, and more. Protest movements that are confined to single demographics are unlikely to succeed, as there is less incentive for host governments to respond. When host governments are faced with broad coalitions of anti-base protesters, their political survival is at stake without being able to lean heavily on other demographics. It also signals the importance of the issue, in that the anti-base grievances are not confined to a single group. 

%While we do not purport to explain the success of protest movements, theories from \citeasnoun{Yeo2011} help us make predictions about the propensity of protests. Since successful protest movements are likely to be larger and last longer, and since successful protests stem from fractures in the security consensus and broad-based support within protests movements, we can make predictions about the likelihood of protests. 
%** NOTE: These hypotheses seem endogenous at best. Thoughts? - AS **

%\begin{hyp}
%Higher rates of negative opinions toward the US military presence will result in higher protest rates.
%\end{hyp}

%\begin{hyp}
%Higher rates of negative opinions toward the US military presence will result in larger protests.
%\end{hyp}

%\begin{hyp}
%More demographically cross-cutting levels of protest participation will be correlated with higher protest rates.
%\end{hyp}

%\begin{hyp}
%More demographically cross-cutting levels of protest participation will be correlated with larger protests.
%\end{hyp}

%**NOTE: We would have to use aggregate numbers from the survey instead of individual responses to test each of these hypotheses. It limits the sample to only surveyed countries and only for the years we have surveys. Not sure if that will be sufficient to find worthwhile results.**


%\ref{cha:theory}
Finally, we note that the decision calculus that populations engage in when deciding whether to protest a US military installation also involves the probability of the protest being successful. Even if there are many costs associated with a US military presence, if the population does not believe that it will be likely to achieve its aims (of removing or modifying the presence) through protest, it will be unlikely to mobilize. This relates to the broader point that we make in Chapter \ref{cha:theory} about why public opinion affects host-country governments and how the presence of US troops may mobilize their populations.

%this used to say that regime type is an indicator of protest success, but i changed it to just protest, since we don't actually measure protest success. CM. 
We argue that the host country's regime type will be a strong indicator of protest. As discussed by \citeasnoun{Murdie2015}, the relationship between protest and the openness of regime type is curvilinear.  In the most open regime types, it is easiest to organize a protest without fear of government repression.  At the same time, in these systems there are alternate, legitimate avenues through which grievances can be expressed, such as a functioning judicial system through which the political opposition can challenge the US military presence \cite{Bitar2016}. In contrast, in the most closed systems the protest is less likely to produce results and is also more likely to lead to repression by the government. We therefore expect there to be more anti-US protests in states with medium levels of openness.

This fits with observed trends in opposition to US military installations. \citeasnoun{Bitar2016} argue that the US has shifted its basing approach in Latin America to informal bases (rather than formal ones) because opposition to the US military presence has grown in these countries as they have democratized in recent years, but have yet to achieve full democracy status. \citeasnoun{cooley2008} also argues that political officials of anocratic states are more likely to politicize US military bases as a means of garnering mass levels of support. When a state's institutional structures are in flux, this type of politicization can activate latent nationalism within the public. With the base used as a point of friction within the electorate and as an example of the leader's nationalist bona fides, protest movements can erupt in support of nationalist candidates and in opposition to US basing, as they did in Uzbekistan, the Philippines, and Spain during anocratic periods. Without the institutional structures that regularize the basing relationship, this politization can make the basing contract uncertain \cite{stravers2018}. 

%This paragraph seems repetitive, so I cut it for now. 
%Democratic countries have institutional structures in place that incorporate the preferences of the population into government action, and the government can then act on these preferences within the normal course of government and the legal structure. Autocratic states also tend to have more developed institutional foundations than in anocratic states, though these structures often simply stifle the preferences of the public and do not allow for protests to occur and protest movements to develop. Anocracies lie between these two extremes, with insufficient preferences aggregation combined with often personalistic leadership styles along with sufficient openness to allow protests (oftentimes spurred by the government itself) to occur.

 %Even today, we see similar uses of US basing by leaders in Turkey and the Philippines during periods of democratic backsliding. However, despite numerous historical examples of this sort, there has been little systematic work done to examine whether such regimes actually produce more protest activity, though they have tested regime type's relationship to different aspects of basing . We thus derive our next hypothesis:  
 
%**Note - was this Bitar portion completed? I jumped to the next paragraph in case it wasn't.** It ws def not complete, rigth now it's all splashing ideas at the page, feel free to edit/add/etc CM.


%\begin{hyp}
%Anti-US base protests are more likely to occur in states that have medium levels of political openness. 
%\end{hyp}


Finally, when considering the cost-benefit analysis that determines whether host country populations engage in anti-base protests, we also consider the situations in which some of the costs of protest are sunk costs. An insight from our qualitative interviews is that embassy personnel tend to be careful about planning military exercises when there are other, unrelated protests occurring (or expected to occur) in the area.  They noted that if there was already a group of people mobilized and protesting, and they found a grievance against the US military, it was very easy for the protest to turn into an anti-base protest \cite{embone20180712,embthree20180712}.  Given that one of the major challenges of collective action is the act of mobilization, we expect that once individuals have been mobilized for a protest (once the initial costs of protest have been sunk), even if it is for a different cause, it becomes easier to organize a protest against US military installations. Thus, we expect countries that experience more protest in general to also be more likely to have anti-US and anti-base protests. 

We want to again emphasize that our current approach focuses primarily on identifying the causal effect of military deployments on protest at the country level. To do this we require a solid theoretical framework to help us develop a model capable of doing so. This set of expectations helps us form the necessary components to create a reasonable quantitative model for predicting protests by country. We will return to this topic later. We now turn to the relevant literature and theoretical expectations for our micro-behavioral model. 

%\begin{hyp}
%All else being equal, anti-US base protests are less likely to occur when there are there are high levels of other, unrelated protests occurring in the host country.
%\end{hyp}





%\begin{hyp}
%Anti-US base protests are less likely to occur when members of the US military are engaged in humanitarian and development work with local communities. 
%\end{hyp}

%**NOTE** This last hypothesis will be really hard to test, since it will be hard to know when the military is engaging the local community, but I'm leaving it in here for now just in case.  Might be something we poke into in the future (related to the work Mike A and I did with Ali). 





 
%Note sure we're actually going to get to this, so greying it out for now.CM. 

%It is important to distinguish between different types of protest.  We note that some protests are more fleeting, where they may last only a day and not mobilize further support or lead to significant change.  There are also protests that are part of a larger social movement that contain a civil-society component that provides them with better organization.  These are the types of protests that would be considered civil resistance \cite{Chenoweth2013}. In addition, protests that are connected to domestic or international organizations are able to draw more individuals to the protest through their connection with the organization and its network \cite{Bell2014,Murdie2011}.

%NOTE: Erica Chenoweth has a data project where they count crowds at protests, but I don't think they've published anything academic from it, just Monkey cage blog posts. I did cite her stuff on the ``3.5 percent rule''

%NOTE: We want to think about the distinction between violent and non-violent protest and look at what leads some protests to turn violent. The Chenoweth and Stephan book finds that non-violent protests are more effective than violent ones (can be more inclusive, can involve more kinds of people, do not need guns, can be spoken about more openly).


