


%\section*{Introduction}
\vspace*{-0.5cm}
\rule{\linewidth}{0.10pt} \\[-1cm]
{\footnotesize\paragraph{Summary:}  Previous chapters examined how military deployments affect beliefs and attitudes. This chapter turns to focusing on individuals' behavior. US military deployments have long been cited as causing negative externalities in host countries. These negative events may help to mobilize opposition to the US presence. Drawing on new country-level protest data and individual-level survey data our analyses yield a number of important findings. First, larger US troop deployments cause more frequent anti-US protest events. Second, our models of individual behavior correctly classify more than 90\% of survey respondents' involvement in anti-US protest activities. These models show that individuals' attitudes and experiences---not simple demographic traits---offer the strongest predictive power in determining who participates in anti-US protest events. Finally, crime victimization in particular is a very strong predictor of protest involvement.} 
\\[-0.5cm] 
\rule{\linewidth}{0.10pt}

\vspace*{0.5cm}

In the middle of an extraordinary July heat wave that was sweeping through continental Europe in 2019, while locals sought refuge from 40 degree Celsius temperatures in the few air-conditioned restaurants they could find, we sat in a small office in a nondescript building in Berlin's Mitte district. ``It's a long story,'' began the German peace activist, as he leaned back on his chair with two young interns looking on. 

In the interview that followed, this particular activist, whose organization focuses broadly on expanding the peace movement in Germany and specifically on protesting against the US Air Force's Ramstein Air Base, explained the history of the German protest movement. He discussed how the peace movement peaked in the 1980s and how, afterwards, the German population did not place as much importance on bases, leading to frustration by local activists. 2005 marked a turning point, as the activist launched a new appeal against Ramstein Air Base. Before we could ask, ``Why Ramstein?'' (the US, after all, has 87 military facilities in Germany \cite{DOD2018}), he volunteered the information. Ramstein has one specific characteristic that separates it from the others: It is the only air base in Germany from which military personnel operate remotely piloted aircraft (also known as Unmanned Aerial Vehicles, UAVs, or drones). As the activist noted, it is a key point to transmit the signals from Nevada and New Mexico for the drones to ``go elsewhere'' \cite{berlinone20190723}. 

This interview, like many others we conducted, illustrated how the presence of foreign military personnel is not itself sufficient to cause host-country citizen grievances against US overseas bases. That is, mobilizing against Ramstein is effective because of the remotely piloted aircraft use and not necessarily because of animosity towards the service members themselves. In the case of this activist's group, they did not want Germany to be morally complicit with the US military's drone strikes in Central Asia. The actions the military takes at Ramstein Air Base (or the belief about what actions take place there) and the resulting activist focus on the base for those actions suggest that the particular behavior at specific bases may motivate opposition to the presence of the US military. 

In 2019, opposition to the Ramstein's activities escalated from domestic protests to a set of legal challenges in the German court system \cite{Kloeckner2019,Reuters2019}. The specificity of the opposition implies variation in the number of protest or mobilization that a base can generate. Knowing what conditions encourage activists and others to protest the United States and its military is important to understanding opposition mobilization.  Protests against Ramstein center around both militarism and drone usage. Protesters at Henoko-Oura Bay in Japan (in Okinawa) protest against Ospreys, US occupation, offenses by US marines, and the environmental destruction caused by new base construction (especially its effects on the coral reef and the dugong population) \cite{Hibbett2019}. At Osan Air Base and Camp Humphreys in South Korea, protestors organized against racial injustice and for Black Lives Matter in the United States \cite{Sisk2020}. 
%I thought that this was a key point to highlight, that it's not just a simplistic "anti-base"' thing, that the actions that the bases take can actually affect the amount of protest that happens. This is kind of like the point that Cliff Morgan brought up at Peace Science, that we should focus on the stuff that the US can actually change. 

%%Relate vignette to quetions
A major aim of the German activist's peace organization is to disseminate information about the US military bases and their activities to the German (and international) public. Their expectation is that as German citizens become more aware of the negative effects of the US presence (such as air pollution from jet exhaust), and what Germans implicitly condone by hosting the US military, they will become more willing to mobilize against it. The group believes that their movement is growing. When we interviewed them, they told us that they had achieved record participation levels that they expected to continue to grow. The German activist recalled fondly how, during one year, protesters formed a human chain around the base and that, in the summer of 2019, the annual protest at Ramstein had 5,000 attendees and 50 different workshops and events.  

%%Set up puzzle
It is easy to see how a US military base could lead to negative reactions from locals and how larger bases can create more opportunities for negative interactions. Military jets are noisy and pollute the air through high emissions. Fuel waste that seeps into the soil can contaminate local drinking water supplies. US service members sometimes get into fights at local bars or drive drunk \cite{kasernetwo20190725}. An American government relations officer at a US base in Germany corroborated those last two points; when we drove into the base there was a wrecked car on exhibit near the entrance with a sign warning US service members to not drink and drive. In the UK, we heard consistent complaints that bases were noisy and worsened traffic. Given that many of a base's negative externalities occur in its immediate proximity, we might expect that it would be those located close to a military installation who would react the most negatively and be most active in opposing the base. But this is not always the case. In comparing the United Kingdom and Germany more broadly, it is clear that the amount of protest in each country, as well as the specific issues that trigger them, are different. 

This variation presents us with various questions. Why, despite having very similar histories with US military basing, are these two countries so different? At the macro-level, what makes Germany more or less likely to experience protests than the United Kingdom? Or, more broadly, what factors make anti-US protest events more likely in some countries and less likely in others? And at the micro-level, does a German have a higher or lower threshold of willingness to join a protest to express their dissatisfaction than a Briton does? What factors make some individuals more or less likely to participate in protests against the US military?

In our interview with the peace activist, we asked whether he would put us in contact with activists in local organizations found in other towns near military bases. In what we found to be a very honest and self-aware response, he noted that in many cases activists have struggled to build local peace movements because of the deep and long-lasting relationship between the US military and communities close to US bases. Locals simply are less willing to protest the US military. Some of it of course relates to the local population's economic dependence on the US military (``bakers, auto shops, and prostitution'' were the examples of businesses reliant on the US military given by the activist). Those who rent housing to US members benefit heavily from the presence. The housing allowance to service members appears to be common knowledge and tends be more generous than what most locals can afford. Local rental prices around military bases tend to match what soldiers can afford and not reflect local wages. This makes bases popular with property owners. Even beyond economics, the activist noted that the US military and local governments will hold common parties, public events, and ``friendship meetings'' to build community relations. ``It makes it not easy [to establish a local protest movement],'' he said. 

%%State chapter's research question

%switched "negatively"' to "`directly'' , since we just mentioned positive externalities too
To an outside observer, it may seem puzzling that those the US military presence affects most directly are also the ones least likely to protest it. Yet, Germany is not an isolated case. \citeasnoun{Fitz2015} finds that in the case of the US military base in Manta, Ecuador, the protest movement was based in the capital city of Quito, while locals in Manta reacted not against the base, but against the \textit{activists}. Locals may complain about inconveniences when pressed, but those concerns alone are not enough for them to mobilize against basing. Minor inconveniences may pale in comparison to the economic benefits of well-payed Americans patronizing local establishments.

%\ref{cha:meth}

Our argument in Chapter \ref{cha:meth} noted that both economic benefits (such as the military patronizing auto shops) and contact (like the ``friendship meetings'') can create more positive beliefs about the US military. It is not surprising that those closest to the base, those most likely to have contact and receive economic benefits, are also less likely to protest. Yet, it stands out that we also find a significant negative effect of contact on respondent beliefs about US actors. We have argued that while most interactions with the US military tend to be positive  (and somewhat casual), those negative interactions that do occur, particularly ones that harm the local individual in some way, can also create or reinforce negative perceptions, and mobilize individuals against the US military. This chapter extends our analysis further to not only understand another vector of negative perceptions, but also the conditions that can change negative perceptions into political action. Negative perceptions, or grievances generally, are not sufficient for anti-base or anti-US activism. This chapter's contribution, exploring the link between perceptions and actions, is fundamentally important to both academics as well as policymakers. This chapter explores both protest mobilization at the macro-level (``Which countries are more likely to experience protest?'') and the micro-level (``What makes an individual more likely to protest?'') to see how negative experiences and negative perceptions may, or may not, manifest in political behavior. 

%%State importance of research question
Even if the negative effect of contact with the military on perceptions is smaller than the positive one, it does not require a majority of members of a population to be opposed to a military base to effectively mobilize, particularly if those who feel positively about the base have only weakly positive feelings. Activists, those people willing to invest time and effort into organizing for a given cause, are usually in the minority in most societies \cite{Burstein2002}.  This is true of most anti-US base activists as well \cite{Fitz2015}.  At the same time, because activists, similar to lobbyists, care so much about a particular cause that others do not have as strong preferences over, they are willing to expend large amounts of effort and are more willing to incur costs than others would be.  Because of this, activists can obtain their preferred policy outcomes even when they are in the minority. As articulated in the ``3.5 percent rule'' developed by \citeasnoun{Chenoweth2011}, simply having 3.5 percent of a state's population involved in active, non-violent protest can be enough to lead to a successful outcome and achieve the protesters' aims. 


%%going to link this to the theory chapter. I think we can build up a pretty good argument based on the protest literature that mobilization and perceptions do matter to the regime. 
%\ref{cha:theory}
A major point that we made in Chapter 2 was that public opinion matters to regimes and that perceptions of the US military in host countries can indeed influence the stability of the hierarchical relationship between the host country and the United States. Research shows that protest and mobilization can influence political actors at the national level to alter policy actions, even when such actions are controversial, such as granting more rights to minority groups \cite{Gillion2013,Fassiotto2017}. In particular, stronger and clearer expressions of public opinion are more likely to influence policy \cite{Baumgartner2015,Fassiotto2017}. When governments make decisions on policy, they face an overwhelming amount of information, some of it gathered by the government itself (through intelligence agencies, for example). At the same time, journalists or activists can directly transmit some of that information to the government \cite[p. 15]{Baumgartner2015}. Protesting is a way in which the public broadcasts its preferences to the government, thus influencing their policy choices. 


Even though some have discounted the role that public opinion plays in leaders' policymaking decisions, existing work shows that leaders do care about public opinion, and are hesitant to engage in policy actions that go against it, in fear of potential political costs \cite{Tomz2018}. While there is variation in how responsive  different types of governments will be to their population's preferences, it is still true that protests are one way in which people communicate information to governments.\footnote{We also note that protests have influenced even non-democratic regimes in leading to policy change. The Arab Spring protests were an example of a case in which non-violent protest was able to achieve concessions, if not outright regime change in most cases \cite{Chenoweth2013}.} Anti-base protests can be an example of how even small groups of activists can obtain their desired outcome (removal of a base, for example) through organization and effective mobilization \cite{cooley2008}.  Even if anti-base protests are rare, it is important to understand their determinants, as these protests can have a large influence on US foreign policy.  This chapter asks the question of what situations are more likely to lead to local populations protesting a US military presence. Likewise, as shown by the case of Ramstein in Germany, this chapter highlights the idea that the United States and its military can take actions that increase or decrease the likelihood of mobilization against its presence. If we understand what causes protests to emerge, we may be able to understand what policy options reduce the likelihood of overseas opposition to peacetime deployments.  

%%Preview our argument (we probably need to add a bit more here on the specific argument once we nail it down a bit more)

Just as important as studying when anti-US base protests occur is studying cases in which they do not occur.  As noted by \citeasnoun{Fitz2015}, there are indeed cases in which communities in close proximity to US military installations do not only not protest, but actively mobilize in favor of the US base. South Korea provides examples of both pro- and anti-US military presence mobilizations that battle over the positive and negative effects of the bases. Particularly of interest are cases in which local communities are less likely to protest the US military presence than individuals in other parts of the country.  This fits with our argument that individuals who have more direct contact  with deployed personnel are more likely to support the US military presence, even in the face of other domestic opposition.  Alternatively (or in addition to this previous explanation), it is possible that the US is choosing to locate some of its military installations in areas whose geographic or demographic characteristics are less likely to mobilize opposition into protest; the US may base in more remote locations far away from the urban centers and the populations within them that are more likely to mobilize. 


%I don't think we're actually studying counter-protests (which would be interesting, but outside our scope, so cutting this out. CM. 
%Mobilization of any sort is not without costs, so it is especially interesting to also consider what makes people to mobilize against anti-US protests.

%%More macro implications

%%Moving this paragraph into main theory chapter
%While negative views of the US by host country publics are bad in and of themselves, they are also problematic for US foreign policy. As individuals' negative views towards a US military presence deepen, they will be more likely to mobilize (be that through voting, social media posts, legal action, or, as we explore in this chapter, protest) against the military presence. As we have previously noted, acts of dissent, like protest, have the potential to impose costs on host country leadership. The government may in turn try to pass those costs on to the United States as a condition for continuing to host the troops. Popular opinion mobilization has removed United States forces from the Philippines and Spain and limited the functional operation of the United States Air Force in Turkey during the build up to the 2003 Iraq War \cite{cooley2008,Kakizaki2011}. Sustained opposition fed by grievances fundamentally weakens the US position and its ability to maintain its troop presence, or at least makes it more costly.

We note the importance of studying specifically what the determinants of both \textit {anti-US} and \textit{anti-US base} protests are, both at the individual and national level. As we discuss in earlier chapters, we need to better understand the degree to which anti-military sentiment affects attitudes and behaviors vis-\'{a}-vis the United States more generally, and vice versa. While the relationship between US military deployments and host-state populations is complex, there are basic empirical and causal questions that still require systematic examination. In particular, we seek to answer the most general question of whether an increase in the number of deployed US military personnel increases the probability of seeing these types of protests. We have previously discussed the idea that deployments and interactions between host-state residents and US personnel can have mixed effects. But attempting to assess what the ``net'' effect might be remains a challenge. In the following sections we pursue multiple research goals. First, we look to see if there is a clear causal relationship between the size of a US military presence and protest events against the US and its military personnel. Second, we then examine the characteristics of both countries and individuals that make protest most likely. In general, we expect that a larger military presence will make protests more likely. We also expect that there are several individual-level characteristics that make participation in protests more or less likely, and that these factors may vary across countries and regions. 

%This seems to be less of a focus now, so I greyed it out. CM. 
%It is certainly the case that some areas of the world, or certain subnational areas within countries, are more prone to protest than others.  We are therefore interested in cases of citizens that choose not to protest US bases but do protest other policies, or alternatively, citizens who do not protest other policies but do choose to protest US bases.  Consequently, we study US military bases in a broader context of all protests in our temporal domain, regardless of the target of the protest. 

%%Outline the chapter

In what follows of this chapter we will discuss research on protest and mobilization in general and anti-US protests in particular. We then use a cost-benefit analysis framework to develop theoretical expectations on the determinants of protest at both the state and individual levels, focusing on the causal path that leads from military deployments to protest. Following, we set up our models for both our newly collected protest data and the survey data we have used throughout the book. We then use both sets of data to create two types of models. The first model examines the causal relationship between troop deployments and protests at the macro-level. The second model uses a predictive model to see how individual characteristics and relationships with the US military may inform us about people's likelihood to attend protest events. We conclude with policy implications focused on the actions that the US military can take to reduce anti-base protests as well as the actions that activist groups can take to broaden the appeal of their message to local populations. %surveys will be 2018-2020 eventually.
% We will also describe our newly-collected data, which includes both self-reported, individual-level data on protest participation drawn from surveys conducted in 2018, and events data on anti-US military protests. Not sure about this last point, but figured it was a way to make it sound like we're not just writing a manual for the US military. CM. 