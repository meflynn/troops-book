\documentclass[12pt]{article}
\usepackage{amsmath,amssymb,graphicx,paralist,setspace,amsthm,pdflscape,fancyhdr,fullpage,rotating,tikz,xcolor,float,varwidth,eurosym}
%\usepackage{times}
\usepackage{longtable}
\usepackage{sectsty}
\usepackage[compact]{titlesec}
\sectionfont{\centering\large}
\subsectionfont{\centering\normalsize}
\usepackage{subfigure}
%\usepackage{harvard}
\usepackage[]{har2nat}
\setcitestyle{aysep={}}
\setcitestyle{notesep={, }}
%\citationstyle{dcu}\usepackage{appendix}
\usepackage[space]{grffile}
\bibliographystyle{apsr}
\newtheorem{hypot}{Hypothesis}
\usepackage[urlcolor=blue,citecolor=blue,linkcolor=blue,linktocpage=true,backref=true]{hyperref}
\usepackage{setspace} % load setspace before footmisc
\usepackage{footmisc}
\renewcommand{\footnotelayout}{\doublespacing}
\setlength{\footnotesep}{\baselineskip}
\hypersetup{
	colorlinks=true,
	urlcolor=blue
}

\newtheorem{hyp}{Hypothesis}
\makeatletter
\newcounter{subhyp} 
\let\savedc@hyp\c@hyp
\newenvironment{subhyp}
 {%
  \setcounter{subhyp}{0}%
  \stepcounter{hyp}%
  \edef\saved@hyp{\thehyp}% Save the current value of hyp
  \let\c@hyp\c@subhyp     % Now hyp is subhyp
  \renewcommand{\thehyp}{\saved@hyp\alph{hyp}}%
 }
 {}
\newcommand{\normhyp}{%
  \let\c@hyp\savedc@hyp % revert to the old one
  \renewcommand\thehyp{\arabic{hyp}}%
} 


\usepackage[space]{grffile}
\graphicspath{{./}{./Data/}{./Tables/}{./Figures/}}

\newcommand{\possessivecite}[1]{\citeauthor{#1}'s \citeyear{#1}}


\begin{document}

\title{The Good, Bad, and Ugly American: Explaining Variation in Global Anti-US Military Base Protests\footnote{} \thanks{This material is based upon work supported by, or in part by, the U.S. Army Research Laboratory and the U.S. Army Research Office under grant number W911NF-18-1-0087. The authors would like to thank xxx for their valuable help in preparing this manuscript. All remaining errors are our own.}}
\author{Carla Martinez Machain\thanks{\textbf{Email:} \href{mailto:carlamm@ksu.edu}{\tt carlamm@ksu.edu}. \textbf{Contact:} Department of Political Science, 101D Calvin Hall, Kansas State University, Manhattan, KS 66506.} \\
Department of Political Science \\
Kansas State University \\
\and Andrew Stravers\thanks{\textbf{Email:} \href{mailto:stravers@utexas.edu}{\tt stravers@utexas.edu}. \textbf{Contact:} The University of Texas at Austin, 158 W 21st ST STOP A1800, Batts Hall 2.116, Austin, TX 78712-1704}\\
Department of Government\\
University of Texas--Austin \\
\and
Michael A. Allen\thanks{\textbf{Email:} \href{mailto:michaelaallen@boisestate.edu }{\tt michaelaallen@boisestate.edu }. \textbf{Contact:} Department of Political Science, Boise State University, 1910 University Drive, Boise, ID 83725.}\\
Department of Political Science\\
Boise State University \\
\and
Michael E. Flynn\thanks{\textbf{Email:} \href{mailto:meflynn@ksu.edu}{\tt meflynn@ksu.edu}. \textbf{Contact:} Department of Political Science, 019C Calvin Hall, Kansas State University, Manhattan, KS 66506.}\\
Department of Political Science\\
Kansas State University
}

\maketitle
\thispagestyle{empty}

%\begin{center}
% \textcolor{red}{WORKING PAPER. PLEASE DO NOT CITE}
%\end{center}

\clearpage

\begin{abstract}
What are the determinants of anti-military base protests abroad?  Are they different from anti-American protests in general?  How does the context of the deployment influence the level of protest against it in the host country?  Using newly-collected events data for the years 1990 to 2018 on protest events around the world we explain the circumstance that make anti-basing protests, as well as anti-U.S. protests, more likely to occur. We argue that anti-U.S. protests are more likely to occur when they are seen as more permanent installations that are independent of domestic military forces.  In addition, previous work finds that direct contact with military personnel leads to more positive perceptions of the U.S. military in host states. Thus, we expect to see more anti-base protests in locations that are more easily accessible (close to urban centers) to transnational activists, rather than only to locals. We thus expect to observe more protests against bases that are close to urban centers.  
\end{abstract}



\vfill

\titlespacing{\section}{0pt}{*2}{*0}
\titlespacing{\subsection}{0pt}{*2}{*0}
\titlespacing{\subsubsection}{0pt}{1cm}{*0}

\thispagestyle{empty}


\newpage
\setcounter{page}{1}
\doublespacing



%\section*{Introduction}
\vspace*{-0.5cm}
\rule{\linewidth}{0.10pt} \\[-1cm]
{\footnotesize\paragraph{Summary:}  Previous chapters examined how military deployments affect beliefs and attitudes. This chapter turns to focusing on individuals' behavior. US military deployments have long been cited as causing negative externalities in host countries. These negative events may help to mobilize opposition to the US presence. Drawing on new country-level protest data and individual-level survey data our analyses yield a number of important findings. First, larger US troop deployments cause more frequent anti-US protest events. Second, our models of individual behavior correctly classify more than 90\% of survey respondents' involvement in anti-US protest activities. These models show that individuals' attitudes and experiences---not simple demographic traits---offer the strongest predictive power in determining who participates in anti-US protest events. Finally, crime victimization in particular is a very strong predictor of protest involvement.} 
\\[-0.5cm] 
\rule{\linewidth}{0.10pt}

\vspace*{0.5cm}

In the middle of an extraordinary July heat wave that was sweeping through continental Europe in 2019, while locals sought refuge from 40 degree Celsius temperatures in the few air-conditioned restaurants they could find, we sat in a small office in a nondescript building in Berlin's Mitte district. ``It's a long story,'' began the German peace activist, as he leaned back on his chair with two young interns looking on. 

In the interview that followed, this particular activist, whose organization focuses broadly on expanding the peace movement in Germany and specifically on protesting against the US Air Force's Ramstein Air Base, explained the history of the German protest movement. He discussed how the peace movement peaked in the 1980s and how, afterwards, the German population did not place as much importance on bases, leading to frustration by local activists. 2005 marked a turning point, as the activist launched a new appeal against Ramstein Air Base. Before we could ask, ``Why Ramstein?'' (the US, after all, has 87 military facilities in Germany \cite{DOD2018}), he volunteered the information. Ramstein has one specific characteristic that separates it from the others: It is the only air base in Germany from which military personnel operate remotely piloted aircraft (also known as Unmanned Aerial Vehicles, UAVs, or drones). As the activist noted, it is a key point to transmit the signals from Nevada and New Mexico for the drones to ``go elsewhere'' \cite{berlinone20190723}. 

This interview, like many others we conducted, illustrated how the presence of foreign military personnel is not itself sufficient to cause host-country citizen grievances against US overseas bases. That is, mobilizing against Ramstein is effective because of the remotely piloted aircraft use and not necessarily because of animosity towards the service members themselves. In the case of this activist's group, they did not want Germany to be morally complicit with the US military's drone strikes in Central Asia. The actions the military takes at Ramstein Air Base (or the belief about what actions take place there) and the resulting activist focus on the base for those actions suggest that the particular behavior at specific bases may motivate opposition to the presence of the US military. 

In 2019, opposition to the Ramstein's activities escalated from domestic protests to a set of legal challenges in the German court system \cite{Kloeckner2019,Reuters2019}. The specificity of the opposition implies variation in the number of protest or mobilization that a base can generate. Knowing what conditions encourage activists and others to protest the United States and its military is important to understanding opposition mobilization.  Protests against Ramstein center around both militarism and drone usage. Protesters at Henoko-Oura Bay in Japan (in Okinawa) protest against Ospreys, US occupation, offenses by US marines, and the environmental destruction caused by new base construction (especially its effects on the coral reef and the dugong population) \cite{Hibbett2019}. At Osan Air Base and Camp Humphreys in South Korea, protestors organized against racial injustice and for Black Lives Matter in the United States \cite{Sisk2020}. 
%I thought that this was a key point to highlight, that it's not just a simplistic "anti-base"' thing, that the actions that the bases take can actually affect the amount of protest that happens. This is kind of like the point that Cliff Morgan brought up at Peace Science, that we should focus on the stuff that the US can actually change. 

%%Relate vignette to quetions
A major aim of the German activist's peace organization is to disseminate information about the US military bases and their activities to the German (and international) public. Their expectation is that as German citizens become more aware of the negative effects of the US presence (such as air pollution from jet exhaust), and what Germans implicitly condone by hosting the US military, they will become more willing to mobilize against it. The group believes that their movement is growing. When we interviewed them, they told us that they had achieved record participation levels that they expected to continue to grow. The German activist recalled fondly how, during one year, protesters formed a human chain around the base and that, in the summer of 2019, the annual protest at Ramstein had 5,000 attendees and 50 different workshops and events.  

%%Set up puzzle
It is easy to see how a US military base could lead to negative reactions from locals and how larger bases can create more opportunities for negative interactions. Military jets are noisy and pollute the air through high emissions. Fuel waste that seeps into the soil can contaminate local drinking water supplies. US service members sometimes get into fights at local bars or drive drunk \cite{kasernetwo20190725}. An American government relations officer at a US base in Germany corroborated those last two points; when we drove into the base there was a wrecked car on exhibit near the entrance with a sign warning US service members to not drink and drive. In the UK, we heard consistent complaints that bases were noisy and worsened traffic. Given that many of a base's negative externalities occur in its immediate proximity, we might expect that it would be those located close to a military installation who would react the most negatively and be most active in opposing the base. But this is not always the case. In comparing the United Kingdom and Germany more broadly, it is clear that the amount of protest in each country, as well as the specific issues that trigger them, are different. 

This variation presents us with various questions. Why, despite having very similar histories with US military basing, are these two countries so different? At the macro-level, what makes Germany more or less likely to experience protests than the United Kingdom? Or, more broadly, what factors make anti-US protest events more likely in some countries and less likely in others? And at the micro-level, does a German have a higher or lower threshold of willingness to join a protest to express their dissatisfaction than a Briton does? What factors make some individuals more or less likely to participate in protests against the US military?

In our interview with the peace activist, we asked whether he would put us in contact with activists in local organizations found in other towns near military bases. In what we found to be a very honest and self-aware response, he noted that in many cases activists have struggled to build local peace movements because of the deep and long-lasting relationship between the US military and communities close to US bases. Locals simply are less willing to protest the US military. Some of it of course relates to the local population's economic dependence on the US military (``bakers, auto shops, and prostitution'' were the examples of businesses reliant on the US military given by the activist). Those who rent housing to US members benefit heavily from the presence. The housing allowance to service members appears to be common knowledge and tends be more generous than what most locals can afford. Local rental prices around military bases tend to match what soldiers can afford and not reflect local wages. This makes bases popular with property owners. Even beyond economics, the activist noted that the US military and local governments will hold common parties, public events, and ``friendship meetings'' to build community relations. ``It makes it not easy [to establish a local protest movement],'' he said. 

%%State chapter's research question

%switched "negatively"' to "`directly'' , since we just mentioned positive externalities too
To an outside observer, it may seem puzzling that those the US military presence affects most directly are also the ones least likely to protest it. Yet, Germany is not an isolated case. \citeasnoun{Fitz2015} finds that in the case of the US military base in Manta, Ecuador, the protest movement was based in the capital city of Quito, while locals in Manta reacted not against the base, but against the \textit{activists}. Locals may complain about inconveniences when pressed, but those concerns alone are not enough for them to mobilize against basing. Minor inconveniences may pale in comparison to the economic benefits of well-payed Americans patronizing local establishments.

%\ref{cha:meth}

Our argument in Chapter \ref{cha:meth} noted that both economic benefits (such as the military patronizing auto shops) and contact (like the ``friendship meetings'') can create more positive beliefs about the US military. It is not surprising that those closest to the base, those most likely to have contact and receive economic benefits, are also less likely to protest. Yet, it stands out that we also find a significant negative effect of contact on respondent beliefs about US actors. We have argued that while most interactions with the US military tend to be positive  (and somewhat casual), those negative interactions that do occur, particularly ones that harm the local individual in some way, can also create or reinforce negative perceptions, and mobilize individuals against the US military. This chapter extends our analysis further to not only understand another vector of negative perceptions, but also the conditions that can change negative perceptions into political action. Negative perceptions, or grievances generally, are not sufficient for anti-base or anti-US activism. This chapter's contribution, exploring the link between perceptions and actions, is fundamentally important to both academics as well as policymakers. This chapter explores both protest mobilization at the macro-level (``Which countries are more likely to experience protest?'') and the micro-level (``What makes an individual more likely to protest?'') to see how negative experiences and negative perceptions may, or may not, manifest in political behavior. 

%%State importance of research question
Even if the negative effect of contact with the military on perceptions is smaller than the positive one, it does not require a majority of members of a population to be opposed to a military base to effectively mobilize, particularly if those who feel positively about the base have only weakly positive feelings. Activists, those people willing to invest time and effort into organizing for a given cause, are usually in the minority in most societies \cite{Burstein2002}.  This is true of most anti-US base activists as well \cite{Fitz2015}.  At the same time, because activists, similar to lobbyists, care so much about a particular cause that others do not have as strong preferences over, they are willing to expend large amounts of effort and are more willing to incur costs than others would be.  Because of this, activists can obtain their preferred policy outcomes even when they are in the minority. As articulated in the ``3.5 percent rule'' developed by \citeasnoun{Chenoweth2011}, simply having 3.5 percent of a state's population involved in active, non-violent protest can be enough to lead to a successful outcome and achieve the protesters' aims. 


%%going to link this to the theory chapter. I think we can build up a pretty good argument based on the protest literature that mobilization and perceptions do matter to the regime. 
%\ref{cha:theory}
A major point that we made in Chapter 2 was that public opinion matters to regimes and that perceptions of the US military in host countries can indeed influence the stability of the hierarchical relationship between the host country and the United States. Research shows that protest and mobilization can influence political actors at the national level to alter policy actions, even when such actions are controversial, such as granting more rights to minority groups \cite{Gillion2013,Fassiotto2017}. In particular, stronger and clearer expressions of public opinion are more likely to influence policy \cite{Baumgartner2015,Fassiotto2017}. When governments make decisions on policy, they face an overwhelming amount of information, some of it gathered by the government itself (through intelligence agencies, for example). At the same time, journalists or activists can directly transmit some of that information to the government \cite[p. 15]{Baumgartner2015}. Protesting is a way in which the public broadcasts its preferences to the government, thus influencing their policy choices. 


Even though some have discounted the role that public opinion plays in leaders' policymaking decisions, existing work shows that leaders do care about public opinion, and are hesitant to engage in policy actions that go against it, in fear of potential political costs \cite{Tomz2018}. While there is variation in how responsive  different types of governments will be to their population's preferences, it is still true that protests are one way in which people communicate information to governments.\footnote{We also note that protests have influenced even non-democratic regimes in leading to policy change. The Arab Spring protests were an example of a case in which non-violent protest was able to achieve concessions, if not outright regime change in most cases \cite{Chenoweth2013}.} Anti-base protests can be an example of how even small groups of activists can obtain their desired outcome (removal of a base, for example) through organization and effective mobilization \cite{cooley2008}.  Even if anti-base protests are rare, it is important to understand their determinants, as these protests can have a large influence on US foreign policy.  This chapter asks the question of what situations are more likely to lead to local populations protesting a US military presence. Likewise, as shown by the case of Ramstein in Germany, this chapter highlights the idea that the United States and its military can take actions that increase or decrease the likelihood of mobilization against its presence. If we understand what causes protests to emerge, we may be able to understand what policy options reduce the likelihood of overseas opposition to peacetime deployments.  

%%Preview our argument (we probably need to add a bit more here on the specific argument once we nail it down a bit more)

Just as important as studying when anti-US base protests occur is studying cases in which they do not occur.  As noted by \citeasnoun{Fitz2015}, there are indeed cases in which communities in close proximity to US military installations do not only not protest, but actively mobilize in favor of the US base. South Korea provides examples of both pro- and anti-US military presence mobilizations that battle over the positive and negative effects of the bases. Particularly of interest are cases in which local communities are less likely to protest the US military presence than individuals in other parts of the country.  This fits with our argument that individuals who have more direct contact  with deployed personnel are more likely to support the US military presence, even in the face of other domestic opposition.  Alternatively (or in addition to this previous explanation), it is possible that the US is choosing to locate some of its military installations in areas whose geographic or demographic characteristics are less likely to mobilize opposition into protest; the US may base in more remote locations far away from the urban centers and the populations within them that are more likely to mobilize. 


%I don't think we're actually studying counter-protests (which would be interesting, but outside our scope, so cutting this out. CM. 
%Mobilization of any sort is not without costs, so it is especially interesting to also consider what makes people to mobilize against anti-US protests.

%%More macro implications

%%Moving this paragraph into main theory chapter
%While negative views of the US by host country publics are bad in and of themselves, they are also problematic for US foreign policy. As individuals' negative views towards a US military presence deepen, they will be more likely to mobilize (be that through voting, social media posts, legal action, or, as we explore in this chapter, protest) against the military presence. As we have previously noted, acts of dissent, like protest, have the potential to impose costs on host country leadership. The government may in turn try to pass those costs on to the United States as a condition for continuing to host the troops. Popular opinion mobilization has removed United States forces from the Philippines and Spain and limited the functional operation of the United States Air Force in Turkey during the build up to the 2003 Iraq War \cite{cooley2008,Kakizaki2011}. Sustained opposition fed by grievances fundamentally weakens the US position and its ability to maintain its troop presence, or at least makes it more costly.

We note the importance of studying specifically what the determinants of both \textit {anti-US} and \textit{anti-US base} protests are, both at the individual and national level. As we discuss in earlier chapters, we need to better understand the degree to which anti-military sentiment affects attitudes and behaviors vis-\'{a}-vis the United States more generally, and vice versa. While the relationship between US military deployments and host-state populations is complex, there are basic empirical and causal questions that still require systematic examination. In particular, we seek to answer the most general question of whether an increase in the number of deployed US military personnel increases the probability of seeing these types of protests. We have previously discussed the idea that deployments and interactions between host-state residents and US personnel can have mixed effects. But attempting to assess what the ``net'' effect might be remains a challenge. In the following sections we pursue multiple research goals. First, we look to see if there is a clear causal relationship between the size of a US military presence and protest events against the US and its military personnel. Second, we then examine the characteristics of both countries and individuals that make protest most likely. In general, we expect that a larger military presence will make protests more likely. We also expect that there are several individual-level characteristics that make participation in protests more or less likely, and that these factors may vary across countries and regions. 

%This seems to be less of a focus now, so I greyed it out. CM. 
%It is certainly the case that some areas of the world, or certain subnational areas within countries, are more prone to protest than others.  We are therefore interested in cases of citizens that choose not to protest US bases but do protest other policies, or alternatively, citizens who do not protest other policies but do choose to protest US bases.  Consequently, we study US military bases in a broader context of all protests in our temporal domain, regardless of the target of the protest. 

%%Outline the chapter

In what follows of this chapter we will discuss research on protest and mobilization in general and anti-US protests in particular. We then use a cost-benefit analysis framework to develop theoretical expectations on the determinants of protest at both the state and individual levels, focusing on the causal path that leads from military deployments to protest. Following, we set up our models for both our newly collected protest data and the survey data we have used throughout the book. We then use both sets of data to create two types of models. The first model examines the causal relationship between troop deployments and protests at the macro-level. The second model uses a predictive model to see how individual characteristics and relationships with the US military may inform us about people's likelihood to attend protest events. We conclude with policy implications focused on the actions that the US military can take to reduce anti-base protests as well as the actions that activist groups can take to broaden the appeal of their message to local populations. %surveys will be 2018-2020 eventually.
% We will also describe our newly-collected data, which includes both self-reported, individual-level data on protest participation drawn from surveys conducted in 2018, and events data on anti-US military protests. Not sure about this last point, but figured it was a way to make it sound like we're not just writing a manual for the US military. CM. 
\input{theoryProtest}
\input{rdProtest}
\input{resultsProtest}
\section*{Conclusions}



%%NOTE: Add something about the "base in your area"' variable once we have that in the models.CM.
We opened this chapter with a conversation with a German peace activist who recounted both the struggles and successes the German peace movement has faced in mobilizing the population against the US military presence. Somewhat counter-intuitively, he noted that while two-thirds of protesters mobilized by his organization are regional, ``the number of locals involved in the movement is growing, but not fast enough'' \cite{berlinone20190723}. While it may seem surprising that activist organizations would have a hard time recruiting from those communities that are closest to a US military base and, therefore, more likely to feel the negative effects of it, our analysis finds that some of these observations can be explained by both demographics and by individuals' experiences and perceptions.    

%discussing the successes and limitations of organizing anti-base activity in Germany. In Germany, anecdotally, opposition to bases is strongest in urban centers and not near the bases themselves. 
%Additionally, particular policies of the bases may drive organization while economic ties and good relations with service members may decrease that opposition.

It is important to note that engaging in protest tends to be a higher-commitment form of political expression. Not everyone who dislikes the military presence will actually mobilize and protest against it.  The activist told us that he thinks there must be 500,000 people adversely affected by US bases, yet the maximum turnout they can get at protests is 5,000 (``This gap is our challenge,'' he noted) \cite{berlinone20190723}. While he expressed some bafflement at the fact that ``they are unhappy, but they are not acting,'' he also seemed to have a strong understanding of why this is the case. The collective action problem is difficult to overcome for opposition groups that want to increase their numbers while the German and local governments and the United States are simultaneously working to obtain support for their positions. Despite the 75-year history of US bases in Germany, the German anti-base movement still faces challenges to overcome the collective action problem. Some of them stem from demographics, but others from the types of interactions that occur between locals and the US military.

%Given the infancy of the anti-base movement, despite the 75 year history of US bases in Germany, suggests that overcoming that organizing to protest requires a substantial effort of organizers. At least that is the case in Germany. 



%After surveying the literature, our expectations remained that those who have stronger incentives to protests were more likely to do so. Like previous literature, the conditions that favor reward participation in political action make protest more likely.

In this chapter, we looked beyond individuals' preferences and views on the US military to see how those views manifest in political action. Considering both events-based and survey data, we identified the conditions that predict anti-US military and anti-US protests and protest participation. We find evidence that there is indeed a causal relationship between the number of US troops deployed to foreign country and the number of protests in that country against the US and the US military more specifically. While this may seem obvious, establishing causality is important because of the strong policy implications it carries. When deciding whether to deploy more troops abroad, the US government should be aware that the increase in troops, even when there is already an existing deployment in place, will be likely to lead to increase in anti-base protests. In addition, this effect extends to more general anti-American protests, not just those focused on the military presence. A larger military presence can result in the expression of more anti-American sentiment.

%\ref{cha:meth}
Of course, reducing the size of its military deployments abroad is not always a realistic option for the United States government. We argue that the effect of the military presence on protest can be attenuated by a variety of other factors, many of them within the control of the US government and military. In general, as we found in 3, contact and economic relationships can reduce the negative perceptions of the United States military. These interactions could produce basing situations that would be less likely to experience anti-US protest as a reaction to a military deployment, and thus could create better hosting environments for the US military. If the contact is negative, that can actually make people more likely to want to protest against the US military, as we find in this chapter. 

%Though these are predictive, not causal models, they do provide information about the settings in which anti-US protests are most likely to occur in. 

Within countries there is also variation in whether individuals will be more or less likely to want to protest against the US military presence. Some of this propensity is related to demographic characteristics. People who are poorer and in democratic countries are more likely to self-report having participated in an anti-US military protest. In Chapter \ref{cha:min}, we discussed the unique place minority members in a host state have relative to bases. They are more likely to pay the costs of the base (through location, environmental consequences, and negative interactions) while reaping less of the benefits (national security, direct economic ties). We find continued support for this argument in that self-described minority status correlates with an increased likelihood of participating in protests.


Beyond demographics, people's experiences are also correlated with the probability that they will respond yes to the question about having participated in anti-US military protests. The roles of contact, economic reliance, and respondent social network continued to play important roles as they did when we focused on perception of US actors. While the models for the survey data were correlational and predictive, we did see that those with more contact (exposure) were more likely to protest a military base. Likewise, economic reliance correlated with increased protest attendance. Exceptionally negative experiences, such as being the victim of a crime committed by a member of the US military, or knowing someone who had been victim of such a crime, also correlated with individuals being more likely to attend more anti-base protests. This is particularly important point as it relates to policy, since the US cannot change host country demographics, but it can influence the types of interactions that service members have with host country publics. It is thus cause for optimism to know that effective policy can affect many of the determinants of protest participation.

As the United States enters an era that requires more consent in maintaining its basing agreements and hopes to continue to received substantial burden sharing from allied states, the consent and support of the host country population will become increasingly important. Even though individual members of the population do not make hosting decisions themselves, their protest activity can indeed influence their governments' policies towards the United States. When protests are successful and influence public opinion, they will make the basing more costly for the military. 

%loop back to the supra argument






\newpage

%\bibliographystyle{apsr01}
\bibliography{one}

%\newpage
%\documentclass[12pt]{article}
\usepackage{amsmath,amssymb,paralist,setspace,amsthm,pdflscape,fancyhdr,fullpage,rotating,tikz,xcolor,float,varwidth,eurosym,graphicx}
%\usepackage{times}
\usepackage{longtable}
%\usepackage{underscore}
\usepackage{sectsty}
%\usepackage[compact]{titlesec}
\sectionfont{\flushleft\large}
\subsectionfont{\flushleft\normalsize}
\usepackage{subfigure}
%\usepackage{harvard}
\usepackage[]{har2nat}
\setcitestyle{aysep={}}
\setcitestyle{notesep={, }}
%\citationstyle{dcu}\usepackage{appendix}
\usepackage[space]{grffile}
\bibliographystyle{apsr}
\newtheorem{hypot}{Hypothesis}
\usepackage[urlcolor=blue,citecolor=blue,linkcolor=blue,linktocpage=true,backref=true]{hyperref}
\usepackage{setspace} % load setspace before footmisc
\usepackage{footmisc}
\renewcommand{\footnotelayout}{\doublespacing}
\setlength{\footnotesep}{\baselineskip}
\hypersetup{
	colorlinks=true,
	urlcolor=blue
}
\usepackage[space]{grffile}
\graphicspath{{./}{./Data/}}

\newcommand{\possessivecite}[1]{\citeauthor{#1}'s \citeyear{#1}}

%\titlespacing{\section}{0pt}{*0}{*0}
%\titlespacing{\subsection}{0pt}{*0}{*0}
%\titlespacing{\subsubsection}{0pt}{1cm}{*0}

\begin{document}


\appendix
\section*{Appendix} 
\setcounter{table}{0}
\renewcommand{\thetable}{A\arabic{table}}
\setcounter{figure}{0}
\renewcommand{\thefigure}{A\arabic{figure}}

This appendix contains additional information relevant to the primary manuscript, such as survey questions, a full table of our primary models, and additional information on how we conducted and executed our survey.\\  



\newpage

\tableofcontents
\listoftables
\listoffigures


\clearpage

\section{Variable codebook}

Below we provide a list of all of the variables included in our models as well as the coding schemes for each. 

\subsection{Country Information}
\newcounter{survey}
\noindent\textbf{\stepcounter{survey}\arabic{survey}  - Country Abbreviation (iso3c)} \\
Uses the Correlates of War Abbreviation. \\
\textbf{Values:}\\
\indent AUS - Australia \\
\indent BEL - Belgium\\
\indent GMY - Germany\\
\indent ITA - Italy\\
\indent JPN - Japan\\
\indent KUW - Kuwait\\
\indent NTH - Netherlands\\
\indent PHI - Philippines\\
\indent POL - Poland\\
\indent POR - Portugal\\
\indent ROK - South Korea\\
\indent SPN - Spain\\
\indent TUR - Turkey\\
\indent UKG - United Kingdom\\

\noindent\textbf{\stepcounter{survey}\arabic{survey}  - Language} \\
The language the respondent took the survey in.\\
\textbf{Values:}\\
\indent 1 - English\\
\indent 2 - Dutch\\
\indent 3 - French\\
\indent 4 - German\\
\indent 5 - Italian\\
\indent 6 - Japanese\\
\indent 7 - Arabic \\
\indent 8 - Tagalog\\
\indent 9 - Polish\\
\indent 10 - Portuguese\\ 
\indent 11 - Korean \\
\indent 12 - Spanish \\
\indent 13 - Turkish\\


\subsection{Dependent Variables}

Note that for the dependent variables listed below we recode the variables when we estimate our ordered logit models. Due to a bug present in Stata, ordered models run on dependent variables with a lowest category higher than ``1'' frequently return errors when running various post-estimation commands. We drop the ``Don't know/decline to answer'' responses and recode each other response by subtracting 1 to obtain a final coding scheme running 1--5. \\


\noindent\textbf{\stepcounter{survey}\arabic{survey}  - Question: US Military Presence (troops\_1)} \\
``In general, what is your opinion of the presence of American military forces in (respondent's country)?''\\
\textbf{Values:}\\
\indent 1 - Don’t know/decline to answer\\
\indent 2 - Very favorable \\
\indent 3 - Somewhat favorable\\
\indent 4 - Neutral\\
\indent 5 - Somewhat unfavorable\\
\indent 6 - Very unfavorable\\

\noindent\textbf{\stepcounter{survey}\arabic{survey}  - Question: American Government (american\_gov)} \\
``In general, what is your opinion of the American government?''\\
\textbf{Values:}\\
\indent 1 - Don’t know/decline to answer\\
\indent 2 - Very favorable \\
\indent 3 - Somewhat favorable\\
\indent 4 - Neutral\\
\indent 5 - Somewhat unfavorable\\
\indent 6 - Very unfavorable\\


\noindent\textbf{\stepcounter{survey}\arabic{survey}  - Question: American People (american\_people)} \\
``In general, what is your opinion of the American people?''\\
\textbf{Values:}\\
\indent 1 - Don’t know/decline to answer\\
\indent 2 - Very favorable \\
\indent 3 - Somewhat favorable\\
\indent 4 - Neutral\\
\indent 5 - Somewhat unfavorable\\
\indent 6 - Very unfavorable\\




\subsection{Independent Variables}

These variables represent either data coded automatically by our survey services or questions we asked the respondents. 


\noindent\textbf{\stepcounter{survey}\arabic{survey}  - Question: Direct Contact with US Military  (contact\_pers)} \\
``Have you personally had direct contact with a member of the American military in (respondent's country)?''\\
\textbf{Values:}\\
\indent 1 - Yes\\
\indent 2 - No\\
\indent 3 - Don’t know/Decline to answer\\


\noindent\textbf{\stepcounter{survey}\arabic{survey}  - Question: Family Contact with US Military  (contact\_nonpers)} \\
``Has a member of your family or close friend had direct contact with a member of the American military stationed in (respondent's country)?''\\
\textbf{Values:}\\
\indent 1 - Yes\\
\indent 2 - No\\
\indent 3 - Don’t know/Decline to answer\\

\noindent\textbf{\stepcounter{survey}\arabic{survey}  - Question: Economic benefit US Military  (benefit\_pers)} \\
``Have you personally received a direct economic benefit from the American military presence in (respondent's country)? Examples include employment by the US military, employment by a contractor that does business with the US military, or ownership/employment at a business that frequently serves US military personnel.''\\
\textbf{Values:}\\
\indent 1 - Yes\\
\indent 2 - No\\
\indent 3 - Don’t know/Decline to answer\\


\noindent\textbf{\stepcounter{survey}\arabic{survey}  - Question: Family Economic Benefit US Military  (benefit\_nonpers)} \\
``Has a member of your family or close friend received a direct economic benefit from the American military presence in (respondent's country)? Examples include employment by the US military, employment by a contractor that does business with the US military, or ownership/employment at a business that frequently serves US military personnel.''\\
\textbf{Values:}\\
\indent 1 - Yes\\
\indent 2 - No\\
\indent 3 - Don’t know/Decline to answer\\


 \noindent\textbf{\stepcounter{survey}\arabic{survey}  - Question: Gender} \\
What is your gender?\\
\textbf{Values:}\\
\indent 1 - Male \\
\indent 2 - Female\\
\indent 3 - Non-binary\\
\indent 4 - None of the above\\


\noindent\textbf{\stepcounter{survey}\arabic{survey}  - Question: Minority} \\
Do you identify as a racial, ethnic, or religious minority?\\
\textbf{Values:}\\
\indent 1 - Yes\\
\indent 2 - No\\
\indent 3 - Decline to Answer\\

\noindent\textbf{\stepcounter{survey}\arabic{survey}  - Question: Education} \\
How many years of formal education have you completed?\\
\textbf{Values:}\\
\textbf{Values:} 0-99999\\

\noindent Note: Given a number of outliers resulting from the self-coding process we used in our questions, we truncate the education variable at 25 years when we estimate our models to eliminate extreme outliers. This covers up to 9 years of graduate education. \\


\noindent\textbf{\stepcounter{survey}\arabic{survey}  - Question: Age} \\
What is your age?\\
\textbf{Values:}\\
\textbf{Values:} 0-99999\\


\noindent\textbf{\stepcounter{survey}\arabic{survey}  - Question: Income - Schmeidl (incomesm)} \\
What is your total household income during the past 12 months? \\
This had a range for each of the six countries, which is the following. All categories are combined as 1-6 in the data. We recommend combining scores 5 and 6 to represent the upper income bracket to match the quintile distributions from the Qualtrics survey.\\
\textbf{Values:}\\
\indent 1 - Bottom Bracket\\
\indent 2 - 2nd Bracket\\
\indent 3 - 3rd Bracket\\
\indent 4 - 4th Bracket\\
\indent 5 - 5th Bracket\\
\indent 6 - Top Bracket\\

United Kingdom:\\
\indent 1 - $<$\pounds20,000\\
\indent 2 - \pounds20,000 - $<$\pounds35,000\\
\indent 3 - \pounds35,000 - $<$\pounds50,000\\
\indent 4 - \pounds50,000 - $<$\pounds75,000\\
\indent 5 - \pounds75,000 - $<$\pounds100,000\\
\indent 6 - \pounds100,000 and more\\

Germany:\\
\indent 1 - <20.000\EUR{}\\
\indent 2 - 20.000\EUR{} - 29.999\EUR{}\\
\indent 3 - 30.000\EUR{} - 39.999\EUR{}\\
\indent 4 - 40.000\EUR{} - 49.999\EUR{}\\
\indent 5 - 50.000\EUR{} - 59.000\EUR{}\\
\indent 6 - 60.000\EUR{} +\\

Italy:\\
\indent 1 - <20.000\EUR{}
\indent 2 - 20.000\EUR{} - 29.999\EUR{}\\
\indent 3 - 30.000\EUR{} - 39.999\EUR{}\\
\indent 4 - 40.000\EUR{} - 49.999\EUR{}\\
\indent 5 - 50.000\EUR{} - 59.000\EUR{}\\
\indent 6 - 60.000\EUR{} +\\

Kuwait:\\
\indent 1 - Less than 3000 KWD\\
\indent 2 - 3 000 - less than 6 000 KWD\\
\indent 3 - 6 000 - less than 12 000 KWD\\
\indent 4 - 12 000 - less than 18 000 KWD\\
\indent 5 - 18 000 - less than 24000 KWD\\
\indent 6 - More than 24000 KWD\\

Japan:\\
\indent 1 - Less than 2 million yen\\
\indent 2 - 2 million yen - less than 4 million yen\\
\indent 3 - 4 million yen - less than 7 million yen\\
\indent 4 - 7 million yen - less than 10 million yen\\
\indent 5 - 10 million yen - less than 15 million yen\\
\indent 6 - More than 15 million yen\\

South Korea\\
\indent 1 - Less than 25 million KRW\\
\indent 2 - 25 million - less than 35 million KRW\\
\indent 3 - 35 million - less than 45 million KRW\\
\indent 4 - 45 million - less than 60 million KRW\\
\indent 5 - 60 million - less than 80 million KRW\\
\indent 6 - More than 80 million KRW\\


\noindent\textbf{\stepcounter{survey}\arabic{survey}  - Question: Religion} \\
What is your religion, if any?\\
\textbf{Values:}\\
\indent 1 - Christianity (Protestant) \\
\indent 2 - Catholicism\\
\indent 3 - Islam\\
\indent 4 - Agnostic/Atheist\\
\indent 5 - Hinduism\\
\indent 6 - Buddhism\\
\indent 7 - Shinto\\
\indent 8 - Judaism\\
\indent 9 - Mormonism\\
\indent 10 - Local religion\\
\indent 11 - Other: \\
\indent 12 - Decline to Answer\\


\noindent\textbf{\stepcounter{survey}\arabic{survey}  - Question: Religion Other (religionother)} \\
Open text responses for the Other bracket option in the previous question.\\
\noindent\textbf{Value Range:} Free-form text.\\




\noindent\textbf{\stepcounter{survey}\arabic{survey}  - Question: Political Views (ideology)} \\
``People often talk about political issues and views in terms of a “left” and “right” spectrum. Using the following scale, where would you place yourself in terms of political views?''\\
\textbf{Values:}\\
\indent 1 - 1 - LEFT\\
\indent 2 - 2\\
\indent 3 - 3\\
\indent 4 - 4\\
\indent 5 - 5\\
\indent 6 - 6\\
\indent 7 - 7\\
\indent 8 - 8\\
\indent 9 - 9\\
\indent 10 - 10 - RIGHT\\


\noindent\textbf{\stepcounter{survey}\arabic{survey}  - Question: Favor Democracy (demgov)} \\
``In general, how important is it to you that you live under a democratic government?''\\
\textbf{Values:}\\
\indent 1 - Very important\\
\indent 2 - Somewhat important\\
\indent 3 - Neutral\\
\indent 4 - Not important\\
\indent 5 - Don’t know/decline to answer\\



\noindent\textbf{\stepcounter{survey}\arabic{survey}  - Question: US Influence (american\_inf\_1)} \\
`` In your opinion, how much influence does the United States have in (respondent's country)?''\\
\textbf{Values:}\\
\indent 1 - A lot\\
\indent 2 - Some\\
\indent 3 - A little\\
\indent 4 - None\\
\indent 5 - Don’t know/Decline to answer\\


\noindent\textbf{\stepcounter{survey}\arabic{survey}  - Question: US Influence Effect (american\_inf\_2)} \\
``In your opinion, the influence that the United States has in (respondent's country) is\ldots''\\
\textbf{Values:}\\
\indent 1 - Very positive\\
\indent 2 - Positive\\
\indent 3 - Neither Positive nor Negative\\
\indent 4 - Negative\\
\indent 5 - Very Negative\\
\indent 6 - Don’t know/Decline to answer\\


\clearpage
\pagestyle{empty}

\section{Supplementary Tables and Figures}



\begin{table}[th]
\caption{Correlation matrix of key variables} 
\label{tab:corrmatrix}
\scalebox{.68}{
\begin{tabular}{l c c c c c c c}\hline
Variable Name & Military Personnel & U.S. Government & U.S. People & Contact & Network Contact & Benefit & Network Benefit \\ \hline
Military Personnel&1.000&0.500&0.424&-0.156&-0.139&-0.166&-0.159\\
U.S. Government&0.500&1.000&0.480&-0.104&-0.105&-0.175&-0.163\\
U.S. People&0.424&0.480&1.000&-0.130&-0.116&-0.129&-0.117\\
Contact&-0.156&-0.104&-0.130&1.000&0.475&0.397&0.364\\
Network Contact&-0.139&-0.105&-0.116&0.475&1.000&0.392&0.440\\
Benefit&-0.166&-0.175&-0.129&0.397&0.392&1.000&0.534\\
Network Benefit&-0.159&-0.163&-0.117&0.364&0.440&0.534&1.000\\ \hline\hline
\end{tabular}} \\ 
\end{table}

\subsection{Correlation among variables of interest:}

Table \ref{tab:corrmatrix} shows the correlation coefficients for the three dependent variables and the four primary independent variables of interest. Importantly, the highest correlation coefficients we find are between the benefit and network benefit variables (0.53), the military personnel and U.S. people variables (0.42), the contact and network contact variables (0.48), and between attitudes towards the U.S. government and U.S. military personnel (0.50). Most of the other variables yield lower correlation coefficients. 

Though some of these variables are clearly related, the relatively modest scores here help, as well as the results from the primary manuscript, give us greater confidence that these questions are tapping into distinct attitudes and experiences. 

\clearpage


\begin{figure}[th]
 \centering  \scalebox{.8}{\includegraphics{../Figures/figure-map-survey-coverage.pdf}}
  \caption{Map of survey coverage. Country shading represents the firm conducting the survey in that particular country.}
 \label{fig:surveycoverage}
\end{figure}

\begin{table}[h]
     \begin{center}
          \caption{Survey coverage table}
          \label{tab:surveytable}
          \input{../Tables/survey-info-table.tex}
     \end{center}
\end{table}

\section{Survey Coverage}

Figure \ref{fig:surveycoverage} and Table \ref{tab:surveytable} show the countries included in our analysis and help to illustrate the geographic coverage of our survey. Our initial rules for inclusion were based on a count of the countries that had an average of $\geq100$ U.S. military personnel per year deployed within their borders since 1990. This yielded a fairly large initial sample of 34 countries. From this list we further identified the countries that had average annual deployment levels $\geq$ 10,000 U.S. military personnel (the United Kingdom, Germany, Italy, South Korea, and Japan). We added Kuwait to this list as the value fell just below the 10,000 threshold. From there we proceeded to include other countries where the U.S. military presence was likely to be large enough to elicit a reaction from the public, thereby providing the variation we need the conduct our survey. We also focused on countries where the U.S. military had a historically notable presence, or cases that were of contemporary relevance. 

For example, the average value for the Philippines is only 1,042 but the the long historical presence of U.S. military facilities in the Philippines, as well as that country's status as a former colony, makes it an attractive case. Alternatively,  Poland has an average score of 28 U.S. personnel since 1990, but with recent Russian aggression and the increase in U.S. personnel deployed to Poland, this is a case that is of great contemporary relevance for our analysis as it will help us to look at attitudes towards U.S. military personnel in a country that does not have a long-term history of hosting such deployments. 

Other cases, like Belgium, Spain, the Netherlands, and Portugal allow us to assess variation in attitudes among countries that share a relatively similar geographic and political history, as well as countries who all belong to NATO. Further, countries like Portugal provide us with an opportunity to explore how variation in the type of U.S. military personnel affects attitudes, as it receives mostly Navy personnel as compared to the Army-heavy deployments in countries Germany. 

Finally, there is a notable lack of countries represented in Africa and South America. This is for a couple of reasons. First, the only country in South America with a notable history of hosting U.S. military personnel is Panama. However, this is largely due to a brief spike in deployments following the U.S. invasion. In general, most Latin American countries have not played host to large long-term deployments in the way that countries in Europe and the Asia-Pacific region have. Those deployments that do occur in Latin America are generally short-term military exercises that tend to occur outside of the public's view. Though there is some limited interaction with the public during some of these deployments, it is not of the same frequency or intensity as in other cases. Members of our team have addressed the effects of these types of deployments in other work, but we have opted to exclude them from our current survey effort so as to focus on other cases with a history of hosting larger deployments over long periods of time. 

However, we have conducted fieldwork in two Latin American countries as a part of this project---Panama and Peru. Panama is of clear historic importance given the U.S. invasion in 1989, and Peru has hosted multiple rounds of military exercises conducted by the U.S. military in conjunction with other partner countries throughout Latin American on an annual basis. In each case we interviewed U.S. military personnel, local politicians, journalists, and policymakers in an effort to better understand the nature of the U.S. military's activities in these countries, as well as how the U.S. military relates to the host-state public. Given the smaller and more episodic nature of the deployments, we believe these interviews were a more effective approach for covering Latin America than the use of large-N surveys.

Similarly, U.S. longer-term deployments in Africa are relatively new, but most are still relatively small in scale. None of the countries on our base list counting those that averaged $\geq$ 100 personnel per year were in Africa. Furthermore, those that are currently in Africa tend to be oriented towards military training and counter-terror operations. Unlike the larger deployments in Western Europe, these deployments are smaller and more focused in purpose, and do not tend to interact with the host-state population in the ways that deployments in other regions historically have.


\clearpage

\end{document}

\end{document}