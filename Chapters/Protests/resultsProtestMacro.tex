\section*{Results}

In this section we present our models' results. First, we review the results of our models that estimate the number of observed protest events in countries across time. Our main finding is that larger US military deployments do seem to cause protests against both the US and the US military in particular. As is to be expected, military deployments have a larger positive effect on anti-US military protests than they do on broader anti-US protests.

Next, we review the results of the models that focus on individual-level protest involvement as reported by the respondents in our cross-national survey data. In general we find that the greatest gains in predictive accuracy occur when we include individuals' experiences and attitudes, not just demographics, in the models. This finding carries significant weight for policy, as policymakers can more easily influence experiences than demographic attributes. 

\subsection*{Macro-Behavior Results}

The appendix contains table \ref{tab:antiusprotesteventmodels} and shows the results for the four models of country-year protest events. To ease interpretation, we also present a coefficient plot in Figure \ref{fig:coefplotcountryprotestmodels}. Models 1 and 3 present the basic models, while Models 2 and 4 include interaction terms designed to capture the expected conditional relationships between troop deployment size and protest activity. Across all four models we find that there is a large positive correlation between the size of the US military deployment in a country and the number of anti-US and anti-US military protest events. Notably the magnitude of this coefficient is roughly 3.5 times larger in Models 3 and 4 as compared to Models 1 and 2, suggesting that countries that  host larger US military deployments are more likely to see an higher number of protest events in general, but that the difference is even greater for protest events aimed specifically at the US military. We will return to a fuller evaluation of the causal nature of this relationship below.


%fix appendix references


We find that several other variables appear to correlate with anti-US protest events. Although relatively rare during this time period, countries involved as primary participants in a war that also involves the US tend to have a higher expected protest count than countries not at war with the US\footnote{Note that we use the countries listed in the PRIO/UCDP data's ``gw\_loc'' field \cite{Gleditschetal2002,Pettersson2019}. This lists the countries central to the incompatibility at the center of the conflict. This is a slightly broader way to group countries than looking only at countries where the US has invaded, but narrower than including all countries who participate in broad military operations like the ones we see in Afghanistan and Iraq.} Still, the models indicate that a US war leads to a fairly sizable increase in the expected count of protest events. This difference applies to both the general anti-US protest models and the models focusing only on protests against the US military. As with the troops coefficient, the coefficient is larger for the military protest models as compared to the more general protest models. US wars in the region surrounding the country in question, however, do not appear to correlate with a similar increase in protest events. More generally, domestic conflict within a state also appears to correlate positively with anti-US military protests during this period. However, we find no differences between conflict and non-conflict states in terms of the number of more general anti-US protest events. This is likely due to the strong relationship between US wars and domestic conflict during this time period---in our data, there are 0 observations where a state is at war with the United States and \textit{not} coded as experiencing a domestic conflict as well. 

The variable for the broader protest environment in a country also produces a positive coefficient. This coefficient indicates that countries where protest in general is more common tend to also have a higher number of anti-US protests. The magnitude here is fairly small, however, and we only find a clear positive coefficient for the more general anti--US protest models. The coefficients from the anti-US military protest models are close to zero and a large share of distribution of the coefficients falls above and below 0. Though a more precise causal analysis of this particular claim is beyond the scope of this project, this lends some support to the claim that a more robust protest environment can be leveraged to support anti-US protest movements. Similarly, we find some evidence of spatial clustering of anti-US protests. Countries with a higher number of anti-US protests in the surrounding region also tend to have a higher expected count of anti-US protest events.


\begin{figure}[t]
	\centering\includegraphics[scale=0.7]{../../Figures/Chapter-Protests/fig-country-coefficients.png}
	\caption{Coefficient plots for predictor variables of different protest events. Points represent the median point highest density posterior estimate and 50\%, 80\%, and 95\% highest density intervals. The histograms represent the distribution of the simulated draws from the model.}
	\label{fig:coefplotcountryprotestmodels}
\end{figure}

There are also some structural features of countries that correlate with higher expected protest counts. Countries with larger populations tend to have a higher expected count of anti-US protests than less populous countries. However, we find that this is limited to the more general anti-US protest models and find some evidence that these more populous countries simultaneously see \textit{lower counts} of protests against the US military. We also find some moderate evidence that more strongly democratic and more strongly autocratic regimes both tend to see a higher count of protests against the US and US military. The posterior distribution for these coefficients does overlap with 0, denoting some uncertainty around these results, but a large portion of the distribution falls on the positive size. 

%%Note: I think we need some summary
%%added a bit above.CM 

\subsubsection{Assessing the Causal Effect of Troop Deployments on Protest Events}

The results presented in the previous section point to a positive relationship between the presence and size of US troop deployments in a country and protest events against the United States. As we discuss above, we take additional steps to explore the degree to which this relationship may be causal, meaning that the US deployments are actually causing the increase in protest, not just that the two are correlated. Using the marginal structural models we discuss in the research design, Figure \ref{fig:atetroops} shows the estimated average treatment effects (ATE), and also the effect of the treatment history of troop deployments, for the two sets of models we estimate. The results of our marginal structural models indicate that US military deployments do, on average, have a positive contemporaneous effect on both anti-US protest events, and anti-US military protest events more specifically. The ATE estimates for the anti-US protest model appear in the bottom two panels and the estimates for the anti-US military protest model appear in the top two panels. In each panel, we display point estimates and credible intervals for the ATE, along with histograms showing the distribution of the ATE based on 10,000 simulations for each iteration of the model. As we note in the previous section, the fact that US military deployments can increase or decrease sharply in a particular set of cases makes estimating the propensity scores and treatment weights difficult. To assess how sensitive our results are to the choice of truncation points for the weights we present the estimates for each of six separate truncation points. 
%%This is something for later, but we probably will want to have a couple of sentences explaining what the truncation points actually mean, for our less methodsy readers. CM.


\begin{figure}[t]
	\centering\includegraphics[scale=0.7]{../../Figures/Chapter-Protests/figure-ate-troops.pdf}
	\caption{Plots show the predicted average treatment effect of US military deployments on protest events. Points represent the median point estimate and 50\%, 80\%, and 95\% highest density intervals. The histograms represent the distribution of the simulated draws from the model.}
	\label{fig:atetroops}
\end{figure}

Briefly, we find evidence of a positive average treatment effect (ATE) for the troop deployment variable in both models, indicating that countries that see an increase in the size of US military deployments tend to see an increase in the likelihood of an anti-US protest event, but also protests against the US military, more specifically. Notably, the multiple iterations of the model also show that the ATE estimate is indeed sensitive to the choice of truncation points, but the evidence generally indicates a positive effect. For example, in the case of the general anti-US protests we find average treatment effect coefficient estimates ranging from approximately $-$0.06 to $+$1.24. However, the only negative value appears in the model run using the 10,000 IPTW truncation point, which is extremely large. The IPTW truncation point of 10 produces a mean IPTW score of just over 1, which is relatively close to the guidance given by \citeasnoun{ColeHernan2008}. Otherwise we find that the models using the IPTW truncation points of 10, 50, 500, and 1,000 all produce relatively similar positive estimates, and we generally find that as we increase the truncation point the error around the ATE estimates narrows and the coefficient decreases slightly. 

When we look at the models predicting anti-US military protests, we see that across all of the IPTW truncation points the ATE is estimated to be positive, ranging from 2.68 at the lowest truncation point to 3.85 at the highest truncation point. As in the previous model we find some evidence that the dispersion around the ATE estimates shrinks as the truncation point increases, but nowhere near as dramatically as in the more general protest model. It is also notable that the estimates are considerably larger for protests against the US military as compared to the more general anti-US protest model.  And while we do see some shift in the median ATE estimate across the various models, there is a considerable amount of stability among the ATE estimates for the four highest truncation points. This is because there is far less variation in the anti-US military protest variable as compared to the more general anti-US protest variables, as can be seen in Figure \ref{fig:protesthistograms}. The maximum number of anti-US military protests is 4 in a given country-year, while we observe multiple cases where there are 5 or more protest events in a given country-year. Recalling the research design, the structural weights are calculated on the basis of the treatment variable, and these essentially tell the model how many copies of a given observation to make when generating the pseudo data for the model. What this tells us is that, after a certain point, the introduction of additional observations from the pseudo data does not fundamentally change the relationships between the predictor and the outcome variable (protest).

\begin{figure}[t]
	\centering\includegraphics[scale=0.7]{../../Figures/Chapter-Protests/fig-histogram-protests.png}
	\caption{Histograms showing the counts of anti-US and anti-US military protest events, 1990--2018.}
	\label{fig:protesthistograms}
\end{figure}


Notably, the treatment histories differ slightly in their expected effects. Larger deployment histories appear to cause a reduction in the expected number of anti-US protest events in a given country-year. As with the contemporaneous effects, higher IPTW truncation points produce estimates with less dispersion, but the median point estimates themselves are fairly stable. Alternatively, we find that evidence is more mixed when it comes to anti-US military protest events, where we find some indication of a positive effect. At the lowest IPTW truncation point we find the coefficient value is quite close to 0 with about 50\% of the posterior for the ATE estimate falling above and below 0. For the other IPTW truncation points, we find roughly an 87\% chance of a positive effect. Regarding the more general protest model, this negative effect of the treatment history makes some intuitive sense. Cases like Germany, South Korea, and Japan have long histories of hosting large US military deployments. After a period of time these deployments may provoke less general hostility towards the United States. However, the possibility of a positive effect of treatment history on protests against the US military also makes intuitive sense. In these same cases the presence of the military itself frequently remains highly contentious---even as individuals themselves frequently have close personal and professional relationships with US service personnel in and around base communities. Even as individuals may come to be more familiar with Americans, reducing more general hostility, the presence of a long-term military facility suggests continued exposure to various negative externalities of the sort we have discussed, including pollution, crime, environmental degradation, and more. These stimuli are understandably likely to provoke a continued reaction among the host-nation's population. However, the results also suggest that this dynamic is more uncertain than the more general protest model.

%Add something like this to explain what the ATEs mean: In other words, an increase of in troop deployments would lead to an increase of .21 to .27 protests in a given country-year.

