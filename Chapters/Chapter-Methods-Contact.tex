\chapter{Contact, Economic Reliance, and our Research Strategy \label{cha:meth}}




%\section*{Introduction}
\vspace*{-0.5cm}
\rule{\linewidth}{0.10pt} \\[-1cm]
{\footnotesize\paragraph{Summary:}  The previous chapter established our theoretical expectations for the subsequent chapters. This chapter examines the survey\index{survey} we have distributed to approximately 42,000 respondents. Additionally, it shows how both economic\index{economic effect} ties and contact\index{contact} with US military personnel can create positive and negative perceptions of the US military, the US government\index{government}, and the US people. One of our most important findings is that contact\index{contact}, both by individuals and people within those individual networks, is enough to push people to have ``informed'' views of the US military. Those contacts\index{contact} make it less likely for people to say they do not have views about the presence and more likely to express any view as a result. These findings advance our argument about how interpersonal experiences with US military members are one of the most important fronts in the domain of competitive consent.\index{Domain of Competitive Consent}} 
\\[-0.5cm] 
\rule{\linewidth}{0.10pt}

\vspace*{0.5cm}

%anecdote lead needed, something about general perceptions or people's views being tied to something. Carla or Andy will have the best idea as to what fits since they were in all four sets of interviews.


There is an old joke about the different U.S. military services interpreting a command to ``secure'' a building differently: The Army surrounds the building, the Navy, puts a lock on it, the Marines capture everyone inside. The punchline of the joke is that the Air Force builds a golf course on it. In a tone that was very much self-aware of that stereotype, a retired Air Force General told us how, when he served as Commander of U.S. Forces in Japan, he had a formula, passed on to him by his predecessor, of how to ``get things done'' with their Japanese counterparts. They would take them out to play golf on the much-coveted Air Force golf course, get drinks, hold a big dinner, and then negotiate.\cite{japantwo20211022} In another Asia interview, an American Public Affairs Officer told us that when he was deployed to Korea, he and his colleagues engaged in a concerted effort to get American service members off base. Some of it was done by encouraging activities like hiking, but in other cases they organized community outreach programs where US servicemembers would go to local orphanages and schools and practice English (a desirable skill that allows students to be admitted into university) with Korean children.\cite{koreatwo20211013}


While Japanese military officers are very different from Korean schoolchildren, these two narratives have a common theme: Personal contact with host country nationals as a way to build improved relations with the US military. In both cases, there was an expectation that sharing a round of golf and a drink, or taking the time to have a conversation with a child who had lost their parents, was a way to connect on a human level with individuals who otherwise might be suspicious of the US military presence in their country. While these are only anecdotes, they confirm the findings from our systematic evaluation of evidence. The Domain of Competitive Consent\index{Domain of Competitive Consent}, as we develop it throughout this book, has several underlying assumptions. Two of the key assumptions are that the perceptions of host-state populations matter to countries that deploy their militaries abroad. Second, those basing powers can affect perceptions in a variety of ways. The research we covered in the previous chapter from shows support for the first argument. We infer from the comparative politics\index{comparative politics} and international relations\index{international relations} research that doemstic politics shaping international relations\index{international relations} is an increasingly relevant feature of the evolving international system\index{international system} compared to the previous seven decades.

The second line of argument---that basing powers can create and mold people's perceptions in other countries---is what we seek to understand within this chapter. To this end, we briefly review the elements of our theory that motivate our argument within this chapter. Moving beyond the theoretical expectations that we established in the previous chapter, we develop our formal expectations in a series of hypotheses that relate both the role of contact\index{contact} and economic\index{economic effect} dependence in shaping the views of the US military. Subsequently, we discuss how we collected the survey\index{survey} data that underpin our research in this chapter and serve as the foundation for the remaining chapters in this volume. After reviewing the data,  we discuss our estimations and display the findings and inferences we draw from those models for the first time using our complete data set.

%%stopped here 10/24 CM

\section*{Contact and Perceptions}

We contend that the US military presence shapes the cultural\index{culture}, economic\index{economic effect}, and political institutions they become enmeshed while serving overseas. Through the deployment process, and even more so when those deployments are through permanent bases, the US presence has a gravitational effect that sends waves rippling through local communities. The larger the size of the presence, there are corresponding increases in the gravitational effect. As more US troops are in a region, there is an increased likelihood that the soldiers interact with locals, cause traffic\index{traffic} jams, and affect local prices with the influx of American dollars. The presence of troops is a social-institution disturbing presence that people will find hard to ignore.

The effect of this gravitational pull can be both positive and negative. In a benign setting, mutual interaction can humanize the presence of a single member of the military. Everyday interactions can break down existing negative or positive stereotypes and build new ones based on the outcome of those interactions. Humanizing the deployed forces can rewrite national political or media narratives through the process of intercultural exchange. In a negative interaction, victimization can overwrite democracy promotion\index{democracy promotion}, anti-communism\index{anti-communism}, national security\index{national security}, or other security cooperation narratives. Our expectations about the results of interaction follow these lines of logic. First, those who interact with the US military are less likely to say that they do not know whether they have a negative or positive view of the United States military, government\index{government}, and people. We present our first expectations as three subsets of hypotheses centered on the relationship between direct contact\index{contact} and views about American actors. 

\begin{subhyp}
	
	\begin{hyp}
		Individuals who report direct contact\index{contact} with a member of the US military will be more likely to express informed views of the American presence/government\index{government}/people. 
	\end{hyp}
	
	Research demonstrates that ignoring the people who profess not to have views on particular survey\index{survey} questions can bias the results of econometric estimation;\cite{Kleinberg2018} choosing to say that you do not know or refuse to answer a question is a choice that likely correlates with other demographic, ideological, or experiential indicators that may also correlate with the question researchers are interested in. In our case, it stands to reason that people who are younger, politically moderate, have fewer experiences with the military, and have other experiences may be less likely to say they have a positive or negative view of the US military. Not including these subsets of answers in our estimations would make such indicators less meaningful in their predictions since we would be excluding one of the things they predict: non-response.
	
	Beyond the methodological reason to include this type of response, there is a substantive issue at stake. We view positive or negative answers as more likely to be informed views (from whatever source, including predisposed biases). Those who say they do not know in response to one of our questions express a less informed view of the subject. While the informed and uninformed dichotomy is not a perfect mapping of reality, some people will refuse to answer a question for reasons other than information. Some people will respond in a directional matter but know very little about the issue serves as an approximation of knowledge. Additionally, repeatedly throughout this book, we find that people that report experiences with the US military are far more likely to have a view on the subject.
	
	For those that report positive and negative experiences, we expect that for a large number of respondents, the everyday interactions with US military personnel will have the contact\index{contact}-based effects that we discuss more thoroughly in Chapter \ref{cha:theory}. Experimental\index{experiments} evidence in political science\index{political science} shows that even a short, brief 10-minute conversation with someone of a maligned identity can dramatically break down stereotypes about a group.\cite{Broockman2016} Importantly, in the Broockman and Kalla study, those attitude changes endured when the researcher interviewed\index{interview} their subjects months later.\cite{Broockman2016} Additionally, some of those interactions will be beyond the everyday interactions and lead to positive experiences that encourage a civilian to think positively of the US presence.
	
	
	\begin{hyp}
		Individuals who report direct contact\index{contact} with a member of the US military will be more likely to express positive views of the American presence/government\index{government}/people. 
	\end{hyp}
	
	Those who have contact\index{contact} with the US presence are more likely to be at risk to experience the negative aspects of the presence. They may see military vehicles clogging their morning commute or experience constant disruptions from Air Force\index{Air Force, US} jets taking off. Seeing a rise of criminal\index{crime} activities in your neighborhood due to the increased demand among GIs or a decrease in the quality of beaches and parks due to base expansion may be factors that convince those that have contact\index{contact} with the US military that their presence is a force for worsening their lives. Beyond the environmental\index{environment} and social harms of the US presence, some interactions are directly negative. Traffic incidents\index{traffic}, public intoxication, fights, assault\index{crime!assault}, and sexual assaults\index{crime!sexual assault} are all events civilians in host-states have experienced from US military personnel. As such, these types of incidents and contact\index{contact} experiences will correlate with negative perceptions of US actors.
	
	Calder makes a strong hypothesis about the relationship between contact\index{contact} and negative perceptions.\cite{calder2007} Specifically, he argues that increasing familiarity creates points of friction and increases the likelihood of resentment, conflict, and protests\index{protest}. To test this, he compares country densities and anti-basing activity and finds support for his argument. While this supports our hypothesis, there is some caution in understanding these results. Notably, there is an issue of an ecological inference fallacy as Calder infers individual-level behavior from aggregated-level data.\cite{King2004} There are several reasons, however, why we expect density to correlate with opposition that may not derive from contact\index{contact}. Given that Calder compares activity within countries, more densely populated countries and cities have more access to media outlets that enable information sharing and recruitment. More people, public transportation, and cheaper communication make it easier to organize, mobilize\index{mobilization}, and collectively act\index{collective action} than in more rural deployments. While Calder supports our argument, we raise this point to highlight the importance of surveying\index{survey} individuals to assess if something like density is the cause of negative views, the enabler of collective action\index{collective action}, or both.\footnote{Calder's argument is more nuanced than contact\index{contact} alone, but given the focus of our argument throughout this book, it is important that we consider the existing argument for this relationship and whether we learn anything new from our research. Clearly, we also expect the opposite case here, but our confirmation gives us more depth in understanding the pathways to support and opposition of bases abroad.}  %new addition, revise and consider if we want to go after calder.
	
	\begin{hyp}
		Individuals who report direct contact\index{contact} with a member of the US military will be more likely to express negative views of the American presence/government\index{government}/people. 
	\end{hyp}
	
	Our views do not manifest purely from first-hand experiences, and humans learn lessons and develop opinions of the world from their family and friends. Indeed, evolutionary and social psychology\index{psychology} points out that the prevalence of people creating false memories where they remember other people's recounted stories as their memories might be an early adaptation for learning---by remembering someone else's story as our own, we effectively have learned a lesson without going through that experience.\cite{Ginsburg2010,Howe2011} Political psychology suggests that your social context, your friends and family, are essential in influencing your political beliefs.\cite{Campbell1960,Jasper1995,Settle2010} Consequently, we expect people who report that members of their social network have had contact\index{contact} with the US military will experience similar changes in their views about various US actors. We expect this effect to be less prominent than direct experiences, but our primary focus here is whether there is an effect from these sources of contact\index{contact}. 
	
\end{subhyp}


\begin{subhyp}
	
	\begin{hyp}
		Individuals who report network contact\index{contact} with a member of the US military will be more likely to express informed views of the American presence/government\index{government}/people. 
	\end{hyp}
	
	
	\begin{hyp}
		Individuals who report network contact\index{contact} with a member of the US military will be more likely to express positive views of the American presence/government\index{government}/people. 
	\end{hyp}
	
	\begin{hyp}
		Individuals who report network contact\index{contact} with a member of the US military will be more likely to express negative views of the American presence/government\index{government}/people. 
	\end{hyp}
	
\end{subhyp}

The deployment of military personnel comes with a whole host of economic\index{economic effect} shocks to local economies. The construction of a base and its maintenance requires extensive local resources and labor. US base commanders seek to source their supplies locally as it is both cost-efficient (as opposed to importing resources and labor). It builds up the local economy around the base.\cite{rafthree20190719,kaserneone20190725} Those that directly benefit from an overseas deployment of a base are more likely to have informed views about the military force and more likely to support its presence.

\begin{subhyp}
	
	\begin{hyp}
		Individuals who report economic\index{economic effect} benefits from the US military will be more likely to express informed views of the American presence/government\index{government}/people. 
	\end{hyp}
	
	
	\begin{hyp}
		Individuals who report economic\index{economic effect} benefits from the US military will be more likely to express positive views of the American presence/government\index{government}/people. 
	\end{hyp}
	
	
\end{subhyp}


It is feasible that the negative hypothesis would make sense here as those that economically\index{economic effect}  benefit from the military are likely to have close contact\index{contact} with service members and are more likely to experience first-hand negative events. However, we expect that, on average, the direct personal benefit will override other misgivings in a sum assessment about the effect of the military presence.

We expect the two same hypotheses for people who know others that have a personal benefit from the presence of the US military:

\begin{subhyp}
	
	\begin{hyp}
		Individuals who report economic\index{economic effect} benefits in their network from the US military will be more likely to express informed views of the American presence/government\index{government}/people. 
	\end{hyp}
	
	
	\begin{hyp}
		Individuals who report economic\index{economic effect} benefits in their network from the US military will be more likely to express positive views of the American presence/government\index{government}/people. 
	\end{hyp}
	
	
\end{subhyp}

To evaluate these hypotheses and those that we develop in later chapters, we now turn to the survey\index{survey} data we have collected over the past several years. 


\section*{Surveys\index{survey} and Strategies}

We seek to accomplish two sections in this section that we draw upon for the rest of the book. First, we deployed a survey\index{survey} in 14 countries over three years to assess how host-state civilians view US military personnel in their country. We discuss the deployment of this survey\index{survey}, the questions we asked respondents, and our choices in designing the survey\index{survey}. The second part of this section is a technical introduction to creating our statistical models to understand the relationships within our survey\index{survey}. For those technically inclined and want to know more details about how we set up our models, our Bayesian\index{Bayesian} priors, or other useful pieces of information, this section is where we go into the most detail about the models. If, however, you would rather avoid discussions of model specification, the second section on ``Estimation Strategy'' may be a safe section to skip. After discussing both the survey\index{survey} and our estimation strategies, we discuss the results of our model regarding both contact\index{contact} and economic\index{economic effect} reliance. 


\subsection*{survey\index{survey} Design and Implementation}

In 2017, we surveyed\index{survey} the literature on US overseas basing and military deployments and assessed areas that the scholarship had not covered sufficiently at that point in time. In 2017, there were quite a few areas still wide open, with several remaining open today, as the research on the effect of troops deployments was relatively thin. During the Cold War through the first few decades of the post-Cold War\index{Cold War} era, there were some research and debates about how troop deployments affected deterrence,\cite{Schelling1966} how troops affected local economic\index{economic effect} and social conditions \cite{Moon1997}, were involved with criminal conduct,\cite{Bryant1979} or how they reflect greater patterns of masculinity and patriarchy in domestic and global orders,\cite{Enloe1990}. Other books concerned themselves with debates over bases as the United States transitioned both its force position and some allied states sought to remove the US presence from their shores.\cite{calder2007,Cooley2008,Yeo2011} As we discuss in the introduction, the publication of data by Kane led to several different works looking at troops as a potential cause of various social, political, and economic\index{economic effect} outcomes.\cite{Kane2004} 


After reviewing the body of deployment knowledge, it is clear that we only have snapshots of how host-state civilians perceive bases in their societies. The best evidence in the literature is when perceptions manifest into discontent and mobilize\index{mobilization} into anti-basing movements\index{mobilization}. Often, the best-studied examples are those that succeed and shift national policy.  While movements in the Philippines\index{Philippines}, South Korea\index{South Korea}, Germany\index{Germany}, and Japan\index{Japan} deserve study, by focusing on cases that manifest in widespread movements, there is a bias in looking at those cases to exclude those that do not manifest in actions. \cite{Geddes1990} The technical terms for this focus are selection bias\index{selection bias} or survivorship bias, making it uncertain whether the inferences we make about these cases are generalizable. For example, do the protest\index{protest} movements represent the whole population or are they just the most vocal? Do political campaigns in the cases we observe just show where such mobilizatio\index{mobilization}n is feasible? Are political entrepreneurs that capitalize on anti-base sentiment exploiting a wedge issue or leading a movement that represents a quieter majority?

To get a better picture, we aimed to create a sample of countries representing a range of experience in hosting US military personnel on their territory. For the sake of inter-country comparison, we explicitly focused on countries that hosted troops during peacetime operations. The presence of an ongoing external or internal conflict shifts the security narrative to one of immediate or existential concern. It dominates secondary concerns about society, the economy, and the environment\index{environment}. Within each country we selected, we also strived to get a representative sample of individuals with different experiences, backgrounds, and beliefs. Ideally, the people within the country live throughout the sampled country, with some people living near bases and others living far away from a US presence. Likewise, we want people in both kinds of locations that both have and did not have contact\index{contact} with US service members. 

With this set of goals, we identified 14 countries where the United States either had a large historical or contemporary presence, and we could survey\index{survey}.\footnote{We identified two survey\index{survey} firms early in our research development process that could reach a large sample of the countries we wanted to survey\index{survey}. The two firms we used for this project were Qualtrics and Schmiedl Marktforschung.} The finalized list of countries includes Australia\index{Australia}, Belgium\index{Belgium}, Germany\index{Germany}, Italy\index{Italy}, Japan\index{Japan}, Kuwait\index{Kuwait}, the Netherlands\index{Netherlands}, the Philippines\index{Philippines}, Poland\index{Poland}, Portugal\index{Portugal}, South Korea\index{South Korea}, Spain\index{Spain}, Turkey\index{Turkey}, and the United Kingdom\index{United Kingdom}. Through our contracted firms, we distributed surveys\index{survey} to 1,000 people in each country. We used quotas to make sure the sample was nationally representative of age, gender, and income. We conducted the surveys\index{survey} in 2018, 2019, and 2020 for 42,000 people across three years in the 14 countries.\footnote{Age only includes people over the age of 18. Substantively, adults are our primary demographic since they are most likely to have economic\index{economic effect} connections to a base and have some ability to influence local or national politics. This makes our inferences about the bases more generalizable than including people under the age of 18. Additionally, including people under the age of 18 requires different restrictions on research for human subjects review.} Our goal in surveying\index{survey} all countries three years in a row was to isolate any phenomena within a country that may drive our results in isolation. For example, an election in a country may center around a left-right divide over security, and people's responses to our survey\index{survey} may reflect a polarized moment if we only examined a single year. We used the 2018 wave of our surveys\index{survey} to conduct initial tests of our hypotheses and found several interesting trends in that one year of data.\cite{Allen2020} This book project uses all three years for our analysis. 

\begin{figure}[t]
	\hspace*{-1cm}\centering\scalebox{.82}{\includegraphics{../../Figures/Chapter-contact/figure-map-survey-firms.png}}
	\caption{Surveyed\index{survey} countries and the firms we used to survey each country.}
	\label{fig:surveycoverage}
\end{figure}	

We localized our survey\index{survey} questions to each of the official languages of the countries we surveyed\index{survey}.\footnote{These translations include Arabic\index{Arabic}, Dutch\index{Dutch}, English\index{English}, Filipino\index{Filipino}, French\index{French}, German\index{German}, Italian\index{Italian}, Japanese\index{Japanese}, Korean\index{Korean}, Portuguese\index{Portuguese}, and Turkish\index{Turkish}.}  In addition to the surveys\index{survey}, we also conducted interviews\index{interview} with local activists\index{activists}, politicians\index{politicians}, US military personnel, journalists\index{journalists}, and US diplomats\index{diplomats} to add qualitative texture to our research questions. 

The survey\index{survey} itself had approximately 50 questions, and we estimated the completion time of the survey\index{survey} to be 10-15 minutes; the survey\index{survey} could be longer or shorter as some questions led to follow-up questions.\footnote{The survey\index{survey} remained consistent from year-to-year. In years 2 and 3, we did add a question about whether individuals trusted their government\index{government} as local trust would likely condition whether a respondent also trusted the security agreements their government\index{government} made with the United States.} The survey\index{survey} asked demographic questions (e.g., a respondent's age, gender, minority\index{minority} status, years of formal education, approximate income, and religion), ideological questions (e.g., their self-assessed left-right political ideology, whether they favor democracy\index{democracy}, views on security, trust in government\index{government}, and societal goals), and opinion questions (e.g., views on the American people, government\index{government}, influence\index{American Influence}, and military as well as questions about other actors). A remaining twenty questions focused on views of the US military presence in their country, its effect, and their experiences with the presence.\footnote{The full list of questions, our data, and several other supplemental projects we have worked on this project are fully available at \url{http://ma-allen.com/military-deployments/}.} Worth highlighting are the three questions that we draw several inferences on and estimate using our statistical models. Specifically, we ask:

\begin{quote}
	``In general, what is your opinion of the presence of American military forces in (respondent's
	country)?''
\end{quote}

We also ask what people's opinion is of the American government\index{government} and people.\footnote{During this process, we were concerned that respondents might conflate Americans with all of the Americas and not the United States specifically. After consulting with language and area studies experts, we concluded that this question phrasing mirrored our intentions, stayed consistent in translation, and would not confuse respondents.} Each respondent rated the military force, government\index{government}, or people on a 5-point Likert scale that included 1) ``Very favorable'', 2) ``Somewhat favorable'', 3) ''Neutral'', 4) ''Somewhat unfavorable'', 5) ''Very unfavorable'' or 6) ``Don't know/decline to answer.'' For our purposes, we group the unfavorable responses and the favorable responses. In constructing the survey\index{survey}, we were concerned that we might prime the respondents if we asked them about the benefits or costs of the US military presence before we asked them about their views on the US military. We made sure to first ask about the respondent's views on the US government\index{government}/people/influence\index{American Influence}/military before asking them about any benefits they may have received from or harms they have experienced by the US military.

We are also interested in whether people have had contact\index{contact} with or economically\index{economic effect}  benefited from the military presence for this chapter. We expect both contact\index{contact} and economic benefits\index{economic effect} to correlate with views on the US military presence and other actors. To get at these ideas, we asked people directly:

\begin{quote}
	``Have you personally had direct contact\index{contact} with a member of the American military in [respondent's country]?''
\end{quote}

People could answer this question yes, no, or don't know/decline to answer. The concept of ``economic\index{economic effect} benefit'' might be diffuse to respondents in our survey\index{survey}, so we provided a bit more context to this question to help the respondent reflect on possible connections they may have had:

\begin{quote}
	``Have you personally received a direct economic\index{economic effect} benefit from the American military presence in [respondent's country]? Examples include employment by the US military, employment by a contractor that does business with the US military, or ownership/employment in a business that frequently serves US military personnel.''
\end{quote}

Our theory posits that direct experiences are only part of the equation in determining beliefs. To account for the transmission of ideas from family members and friends, we asked people about the same events within their social networks. Our questions represent an active discussion of the military and help us capture a transmission of value-laden information from a social network to the res
pondent. Presumably, knowing whether your friend has had contact\index{contact} with a military member or if they economically\index{economic effect}  have benefited from the military comes with some anecdote about the experience. The anecdote may reflect routine or banal expectations, such as being held up in traffic\index{traffic} due to a large military convoy on the road or an influx of soldiers at a restaurant. It also may detail more personal interactions, such as having a child on a soccer team with kids of military personnel or getting into a altercation with an intoxicated sailor. Retelling these anecdotes to friends and family often comes with a subjective interpretation that our respondents may generalize into larger views about the military. We, therefore, inquire:

\begin{quote}
	``Has a member of your family or close friend had direct contact\index{contact} with a member of the American military stationed in (respondent's country)?''
\end{quote}

The economic\index{economic effect} benefit question about a respondent's social network mirrors our question about individual benefits as it primes the respondent with a few examples:


\begin{quote}
	``Has a member of your family or close friend received a direct economic\index{economic effect} benefit from the American military presence in (respondent's country)? Examples include employment by the US military, employment by a contractor that does business with the US military, or ownership/employment at a business that frequently serves US military personnel.''
\end{quote}

These questions instruments do not perfectly capture the social network effect on belief formation as some latent transmission occurs. For example, a family member may express beliefs about the US military presence that stem from an interaction or encounter that they have shared. More commonly, friends and family members share beliefs about political topics without any direct experience, and those processes transmit identity-shaping values. However, we posit that direct knowledge of other peoples' experiences represents a salient form of value-formation as it contextualizes the experiences to something immediate and affects their social network. 

We use other questions in our survey\index{survey} and mention those as they become relevant to our models. However, these seven questions form the core of our research as we turn to our statistical model design.


\subsection*{Data Analysis and Communication}

Throughout the book, we adopt a couple of different strategies for presenting and analyzing our data. We will discuss these approaches briefly here to provide readers with an overview of our general approach to analyzing the data and communicating results. As this book speaks to both non-academic and academic audiences, we want to spend some time discussing our approach up front so readers without a technical background can more easily process the material as it arises. 

\subsubsection*{Data Visualization}

First, as will be evident through the remainder of this chapter and the entire book, we use a range of data visualization tools to help communicate the insights from our data. We tend to rely more on figures and visualization to communicate patterns of interest than tables. For example, the descriptive figures we present later in this chapter can give us a quick snapshot of the most common responses to some of our key questions on contacts\index{contact} and benefits. These snapshots also give us a rough baseline for understanding how individual responses to our questions vary by country and reference group. They also provide us with some context to our more complicated models. Additionally, many of the questions we asked are unique to our survey\index{survey} as few other surveys\index{survey} attempt to deeply probe people's beliefs about the US military within their country. Showcasing some of these results provides new insight into how people in host countries see the United States and its security apparatus. 

While data visualization can provide useful insights into the basic descriptive patterns we observe in our data, this approach has limits. Over 42,000 individual survey\index{survey} respondents answered approximately 50 separate survey\index{survey} questions, ranging from basic demographic characteristics to personal views and experiences. The relationships between these different questions are both numerous and complex. Further, the individuals responding to these surveys\index{survey} exist within complex and diverse local and national contexts that invariably shape their views and experiences. This means that the already complex relationships between survey\index{survey} questions become further complicated by the possibility that they vary across these different contexts. To address these issues, we estimate a series of Bayesian\index{Bayesian} multilevel regression models to provide a fuller analysis of the relationships between our predictor variables and the outcomes of interest.


\subsubsection*{Why Bayes?}

In particular, the use of Bayesian\index{Bayesian} methods may benefit from some additional discussion as it is an approach that some readers may be less familiar with. Bayesian\index{Bayesian} statistics have a long history and several benefits that interest practitioners and policymakers, specifically. However, while researchers have used Bayesian\index{Bayesian} methods in comparatively limited applications over the last 200 years, they have seen more widespread use across several disciplines in recent decades as increases in computing power have reduced the costs of use for larger and more complex problems.\cite[For a fuller review of the history of Bayes' Rule and Bayesian\index{Bayesian} statistics see:][]{mcgrayne2012} A fuller treatment of Bayesian\index{Bayesian} models is beyond the scope of this book. Still, it is worth briefly discussing some of their advantages---particularly for policymakers---since they have seen less widespread usage than frequentist models. 

First, while there are numerous applications of Bayesian\index{Bayesian} methods, the common thread tying them together is the basic idea that uses new information to update expectations about how likely a particular event or relationship is. These methods offer a way for researchers to directly incorporate information obtained from prior research to inform expectations in new analysis. Some of the research presented in this book is closely tied to previously published work.\cite{Allen2020} Accordingly, these methods allow us to incorporate information from this previous work directly and can function as a built-in check to compare if and how the acquisition of new data and information has affected the relationships of interest compared to our previous studies.

Second, the Bayesian\index{Bayesian} framework provides us with a way to communicate the results of our models and the accompanying uncertainty in more direct probabilistic terms than frequentist alternatives. For example, we can make clearer claims regarding the probability of a positive or negative correlation between variables or an effect of a predictor on an outcome. We can also make more refined statements about the probability that an effect is of a given size or falls within a particular range, compared to traditional approaches using null hypothesis significance tests. Given that a goal of this book is to help present our results in a more intuitive and accessible way for non-technical audiences, this is a highly desirable trait.\footnote{All models estimated in R using {\tt brms}and Stan/CmdStanR. \cite{stan2021,cmdstanr2021,burkner2017}} 

Third, Bayesian\index{Bayesian} approaches offer us greater flexibility in modeling more complex relationships and structures in the data. Much of the data we work with in this book is survey\index{survey} data collected across 14 countries and three years. There is a clear grouping structure to this survey\index{survey} data---Individuals are nested within various territorial (and other) groups, like provinces, countries, etc. Both individual-level characteristics and group-level factors may drive respondents' views. For example, individual characteristics like gender, age, or income might affect how people view the US presence in their country. Still, they so too might group-level factors, like a US military base present within a particular province. Multilevel models are well-suited for dealing with data with grouping structures like this. They are also extremely useful for exploring the possibility of varying effects across different groups. For example, it may be the case that certain demographic traits, like education or income, exhibit different relationships with attitudes towards US military personnel depending on the country in question. Multilevel models provide a convenient and flexible tool for exploring these possibilities and are fairly easy to implement using Bayesian\index{Bayesian} software. 

The tables that we produce from these models are relatively large and technical. To make communicating their results easier, we have put the full tables into appendices available to the reader. Fuller diagnostic appendices will also be available online. Here we focus on a few key insights from the models that pertain to our primary theoretical expectations. We use the same general modeling strategy in Chapters \ref{cha:meth} through \ref{cha:protest}\index{protest}, though the specific model specification does change depending on the question at hand.


\subsubsection*{Additional Data Resources}

The cross-national surveys\index{survey} represent a major component of our analysis. Still, we also draw on other original data sets to help answer some of the critical questions at the heart of this book. Beyond the data on mass attitudes\index{public opinion} we have also collected data on media-reported crime\index{crime}, protests\index{protest}, military expenditures, and military construction\index{construction} to get a better sense of the footprint of the military and the stimuli people may or may not evaluate in developing their perceptions of the US military.\footnote{All of the data we collected, including the surveys\index{survey} and the data mentioned here, are available for public use at \url{http://ma-allen.com/military-deployments/}.} In our chapters focusing on crime\index{crime} and protests\index{protest}, we estimate additional models using these different data sets. Given that these models tend to be more narrow, we provide more detailed information on the specific measures and modeling strategies in the relevant chapters.



\section*{The Effect of contact\index{contact} and economic\index{economic effect} Benefits}


\begin{figure}[t]
	\centering\scalebox{.64}{\includegraphics{../../Figures/Chapter-Contact/figure-views-of-us-actors-combined.png}}
	\caption{Evaluations of the American military presence, government\index{government}, and people by country.}
	\label{fig:dvdesc}
\end{figure} %placeholder graph. Probably a 3x3x14 graph with 3 years as rows, 3 actors as columns, and each row will have all 14 countries, so basically a 9x9, but each cell has 14 charts within it.

We begin by using descriptive figures to provide a general overview of some of our key variables. First, Figure \ref{fig:dvdesc} shows the distribution of responses for the three reference groups of interest across the 14 countries included in our survey\index{survey}. This includes US military personnel stationed in the host country, the US government\index{government}, and US people in general. Lighter colors represent more favorable views, and darker colors represent more unfavorable views. We pool the three annual waves of the survey\index{survey} together in these figures. There are a few patterns that are worth highlighting. 

First, there is a substantial amount of variation across countries concerning the distribution of attitudes. When looking at attitudes towards US military personnel stationed within the host country, we can see that respondents in some countries tend to hold more negative views of those personnel than in other countries. For example, the overall favorability rates in Turkey\index{Turkey} and Germany\index{Germany} are fairly low. Only about 26\% of German respondents hold a ``somewhat favorable'' or ``very favorable'' view of the US personnel stationed in Germany\index{Germany}. Similarly, only 25\% of Turkish respondents gave similarly favorable responses. In contrast, other countries see far more positive views of US military personnel. Australia\index{Australia}, Belgium\index{Belgium}, the Netherlands\index{Netherlands}, Portugal\index{Portugal}, and South Korea\index{South Korea} have over 40\% favorable responses, and Kuwait\index{Kuwait}, the Philippines\index{Philippines}, and Poland\index{Poland} all have favorability rates of approximately 70\% or higher. 

Lower overall favorability rates do not necessarily translate to high rates of negative views. In Germany\index{Germany}, we see a 26\% overall favorability rate is mirrored by a 26\% unfavorability rate (for example, those who responded with ``somewhat unfavorable'' or ``very unfavorable''). The bulk of the remaining respondents reside in the ``neutral`` category. Spain\index{Spain} is similar with approximately 33\% of respondents expressing positive views and 33\% expressing negative views. However, in Turkey\index{Turkey}, we find that the prevalence of negative views is nearly double the positive views, with 47\% of respondents holding unfavorable views of US military personnel as compared to the 25\% of respondents who hold positive views.

These results are informative in that they suggest there is considerable variation in attitude formation towards US military personnel across countries. While this is perhaps not surprising, more refined cross-national data such as these are relatively lacking. Furthermore, the insight that the US military, while not viewed favorably by a host population, is also not necessarily viewed in an overwhelmingly negative light is not a trivial one. Low favorability rates may not damage the United States' ability to maintain a presence if negative views do not overwhelmingly offset those attitudes. Instead, large shares of neutral views may provide substantial political leeway for leaders to continue working with the US government\index{government}.

We explore the correlates of attitudes towards US personnel in more detail below. Before we move on it is also worth discussing how attitudes towards US troops compare to attitudes towards other US groups. In short, there is substantial variation across reference groups when it comes to attitudes \textit{within countries}. Figure \ref{fig:contactbreakdown} summarizes the percentage of favorable, neutral, and unfavorable views across countries to help illustrate this point. In this figure, we aggregate the ``somewhat`` and ``very`` categories to general overall favorable and unfavorable categories while also looking at the percentage of respondents who expressed neutral views. The benefit of this figure is that it helps to illustrate just how different attitudes towards the US government\index{government} compare with the other two reference groups.

In several countries, attitudes towards the US government\index{government} are substantially more negative than attitudes towards US military personnel and the US people. In the favorable panel in Figure \ref{fig:contactbreakdown} we can see that there is a substantial drop in the middle column reporting the percentage of respondents expressing a positive view of the US government\index{government}. Several countries, including Poland\index{Poland}, the Philippines\index{Philippines}, Australia\index{Australia}, Spain\index{Spain}, and Belgium\index{Belgium}, see sharp declines. Similarly, in the unfavorable panel, we can see a sharup increase in unfavorable views of the US government\index{government} compared to the other groups. 8 of the 14 countries in our sample see substantially higher rates of unfavorable attitudes towards the US government\index{government} when compared to US troops and the US people. Interestingly, we also see that some countries have similarly high unfavorable views when asked about the US military and US government\index{government}. For example, Turkey\index{Turkey}, the Philippines\index{Philippines}, and Poland\index{Poland} do not see such a spike in negative responses when asked about the US government\index{government}. 

\begin{figure}[t]
	\centering\scalebox{.55}{\includegraphics{../../Figures/Chapter-Contact/figure-contact-favorability-breakdown.png}}
	\caption{Breakdown of aggregate favorable, neutral, and unfavorable views by country and reference group. Favorable and unfavorable categories aggregate the ``very'' and ``somewhat' categories to provide a more general summary of views.}
	\label{fig:contactbreakdown}
\end{figure}

Since our data only comes from 2018-2020, we cannot assess whether the dislike of the US government\index{government} is a general trend over the last several decades or is a response to the Trump administration, specifically. We expect that the Trump administration played some role in the views of the US government\index{government} as a Pew survey\index{survey} of sixteen different countries shows the United States increasing its favorability among foreign populations by 28 points by the early part of the Biden administration\index{Biden administration}. \cite{Wike2021} More work is necessary to establish how responsive the mass attitudes of foreign publics are to changes in the composition of the US government\index{government} or particular political personalities. 

Overall, these preliminary glances at the data suggest substantial cross-national variation in attitudes towards US actors. The data also point to interesting within-country variation in attitudes towards these same actors. Negative views of one actor do not appear to spill over to determine attitudes of other related actors uniformly. 


What explains this variation? Why are some US actors more or less popular than others? What interpersonal factors affect mass attitudes towards these actors? And what explains variation in these dynamics across countries? We explore other predictors of mass attitudes towards US actors throughout this book. Still, in this chapter, we focus on how different types of interpersonal connections and the receipt of economic\index{economic effect} benefits affect mass attitudes towards US military personnel stationed within the host country and other US actors. This chapter largely builds off of our previous work on this subject \cite{Allen2020}. However, this previous study only had one year of data with which to work. Here we expand on this previous analysis to discuss the patterns we observed in our survey\index{survey} data three years. 

\begin{figure}[t]
	\hspace*{-0.5cm}\centering\scalebox{.62}{\includegraphics{../../Figures/Chapter-Contact/figure-contact-type-summary.png}}
	\caption{Responses to whether people have had contact\index{contact} with, their social network have had contact\index{contact} with, if they economically\index{economic effect}  benefit from, or if their social network economically benefits from the US military presence.}
	\label{fig:ivdesc}
\end{figure} %placeholder graph. 3x4x14?
\FloatBarrier

Figure \ref{fig:ivdesc} offers a snapshot of the responses to the survey\index{survey} questions concerning the different forms of contact\index{contact} individuals might have had with US military personnel stationed in the referent country. These questions ask respondents if they have had 1) any direct interpersonal contacts with US service personnel; 2) if anyone in the respondent's immediate social network has had any direct interpersonal contacts with US service personnel; 3) if the respondent has received any type of economic\index{economic effect}  benefits from the US military presence in the referent country; and 4) if anyone in the respondent's social network has received any type of economic\index{economic effect} benefits from the US military presence in the referent country. As above, the responses are broken down by contact\index{contact} type and country, with all three years pooled together. 

In stark contrast to the questions reviewed in the previous figures, the dominant response category is ``no'', meaning that across all of the countries in our sample, most respondents report no direct contacts\index{contact} or benefits and are not aware of any social contacts who have experienced such contacts, or received economic\index{economic effect} benefits. In this regard, all of the countries in our data are remarkably similar. However, this is not altogether surprising. The US often clusters military facilities in particular locations and isolates them from surrounding communities to minimize the opportunity for negative encounters. The US military may also implement other policies to reduce further the visibility of its personnel in the host country, meaning there may be some under-reporting by respondents unaware of contacts\index{contact}. Therefore, it is not surprising that most people do not report these kinds of contacts\index{contact} or benefits in the context of a nationwide survey\index{survey}. 


Among those who do report contacts\index{contact} or benefits, we do see some interesting variation across countries. Given its large military footprint, it is perhaps unsurprising that Germany\index{Germany} reports some of the highest interpersonal and network contacts\index{contact} rates-- approximately 25\% for both categories. Contrast this with Japan\index{Japan}, which also hosts a large US military presence but sees only 8\% of respondents reporting personal contacts\index{contact}. South Korea\index{South Korea}, another large host nation, falls roughly in the middle with 15\% of respondents reporting direct contact\index{contact} and 18\% reporting network connections. However, the highest rate of contacts is in Kuwait\index{Kuwait}, where 33\% of respondents reported personal contacts\index{contact} and 34\% reported network contacts. This is somewhat surprising compared to Germany\index{Germany} and Japan\index{Japan}, but Kuwait\index{Kuwait} has a mixture of small territorial size and a history of hosting large US deployments.

There appears to be more variation when we look at reported benefits. Interestingly, high rates of contact\index{contact} do not always translate into higher reported rates of benefits. In some countries like South Korea\index{South Korea} and Germany\index{Germany}, we see fairly high reported contact\index{contact} rates but comparatively low reported benefits---about 5\% in both countries. These rates are approximately one-fifth of the rate we see for reported contacts\index{contact} in Germany\index{Germany} and one-third for contacts\index{contact} in South Korea\index{South Korea}. That rate of reported contacts should exceed benefits is not necessarily surprising---there are likely to be far more opportunities to interact with US military personnel on an interpersonal basis than there are jobs that directly benefit from the US military. However, in some cases, like Kuwait\index{Kuwait} and Turkey\index{Turkey}, we see high rates of reported benefits that look similar to the rates of reported contacts\index{contact}. In Kuwait\index{Kuwait}, 33\% of respondents report receiving economic\index{economic effect} benefits, and 34\% report knowing someone who receives an economic\index{economic effect} benefit.
Similarly, in Turkey\index{Turkey}, we find that 11\%-12\% of respondents report contacts\index{contact} and benefits. Notably, the Philippines\index{Philippines} is the only country where reported benefits are higher than the rates of reported contacts\index{contact}. Here 15\% of respondents report personal economic\index{economic effect} benefits, and 22\% report knowing someone who receives economic\index{economic effect} benefits. This compares with 14\% reporting personal contacts and 19\% reporting network contacts\index{contact}. 


So far, we have reviewed respondents' assessments of various US actors and their reported interactions with US actors in isolation. Ultimately our goal is to see how these responses relate to mass attitudes towards US actors. Before moving on to more complex models, we can get another snapshot of these relationships by looking at ``typical'' responses across intersecting categories. For example, we can see what the most common assessment of US military personnel is among people who report not having any personal contacts\index{contact} with US personnel and compare those responses to people who do report having such contacts\index{contact}. While a range of other factors can contribute to attitude formation, this approach can provide us with a cursory overview and inform our expectations moving forward.



\begin{figure}[t]
	\centering\scalebox{.6}{\includegraphics{../../Figures/Chapter-Contact/figure-contact-type-group-summary-big.png}}
	\caption{Modal response towards reference group (row) by respondent country and type of contact\index{contact} with US military (column headers).}
	\label{fig:tileplot}
\end{figure}%replace with newer figure. If things change, then report down here. 

Figure \ref{fig:tileplot} plots these intersecting groups. This figure combines the data from the previous figures into a single plot, conveying a substantial amount of information to facilitate some basic preliminary comparisons. Each rectangle in the figure represents the modal response (the most common response) of respondents' views of the United States military, people, or government\index{government} in each of the 14 countries we surveyed\index{survey}. Each pair of columns looks at a specific form of contact\index{contact} or benefit. The left column is the modal response among those who reported no contacts\index{contact} or benefits, while the right column is the modal response among those who did report a contact\index{contact} or benefit. The labels on the right side represent the three reference groups we are interested in-- US military personnel, the US government\index{government}, and the US people in general. By comparing the pairs of columns within each contact\index{contact} type we can get a general sense of how those reporting personal or network contacts\index{contact} and benefits differ from those who do not report experiencing these. We can also make some initial comparisons as to how these differences themselves differ across countries.  

There are a few interesting trends within this figure. As we note above, the US government\index{government} sees the highest rate of unfavorable evaluations across the three reference groups. Conversely, the US people are generally the most well-liked, followed by US military personnel stationed within the host country. This figure's primary point of interest is the degree to which the modal response category changes when moving from the ``no'' to the ``yes'' columns. As can be seen in Figure \ref{fig:tileplot} the countries in our sample tend to see either equivalent modal responses or more positive modal responses among those who report having some form of personal contact\index{contact} with US military personnel, receiving economic\index{economic effect} benefits, or knowing someone who has experienced one of these outcomes. Importantly, in none of the 14 countries in our sample, we see a more negative modal assessment of any of the US actors among those responding ``yes'' to a contact\index{contact} or benefit question than those who responded ``no''. 

At the macro-level, these patterns suggest that heightened rates of contacts\index{contact} and benefits may improve overall attitudes towards US actors, or---at the very least---do not lead to a net decrease in favorability. However, caution is in order when attempting to draw inferences with these data. Relying on modal responses may generate undue confidence in an association that appears, \textit{prima facie}, to be positive. The modal response tells us nothing about how prevalent the modal category is relative to other categories, or how those responses distribute more broadly. A small number of responses may be sufficient to turn an apparent positive association neutral or a neutral association negative. Further, macro-level patterns are not an adequate basis on which to make judgments about the effect of contacts\index{contact} and benefits on \textit{individual respondents}. As we explore below, the factors that shape respondent attitudes are numerous, and the nature of an individual's interactions with the US military may be positive, negative, or both.


\subsection*{Modeling the Relationship Between Contacts\index{contact}, Benefits, and Mass Attitudes towards US Actors}

The descriptive figures provide a helpful initial overview of the relationships between the variables of interest. However, we are also interested in the relationships between the predictor variables and the outcomes of interest when we also condition on \textit{other} predictors of interest. For example, how do individuals reporting personal contacts\index{contact} compare to those who do not report personal contacts\index{contact} once we adjust for other things that may correlate with contacts\index{contact}, like the receipt of economic\index{economic effect} benefits? By modeling individual assessments as a function of multiple predictor variables, we can gain more insight into how two groups compare to one another after accounting for the possible influence of other factors. Our modeling strategy also allows for the possibility that the relationship between a given predictor and individuals' assessments of US actors may not be linear---encounters with US military personnel may yield both more positive \textit{and} more negative views. Ultimately this modeling approach can help us to generate more nuanced insights into the relationships between the factors of interest. 

We provide full tables containing the results of these models in the supplemental appendices. Here, we rely primarily on figures to help illustrate select relationships of interest resulting from these models. These figures can tell us something about both the consistency of the correlation coefficients and their relative magnitude of the relationships when looking at different types of contact\index{contact} and benefits across different outcome groups. 

\begin{figure}[t]
	\centering  \scalebox{.55}{\includegraphics{../../Figures/Chapter-Contact/figure-coefficient-plot.png}}
	\caption{Coefficient plot of the results from our estimations. 50\%, 80\%, and 95\% credible intervals shown.}
	\label{fig:coefplot1}
\end{figure}%replace with newer figure. If things change, then report down here. 

Figure \ref{fig:coefplot1} displays the coefficients from our multilevel choice models. Each column corresponds to one of the three reference groups we asked about in our survey\index{survey}. Each row represents one of the four questions about contact\index{contact} and economic\index{economic effect} benefits. The rows within each question block correspond to whether the respondent said ``yes'', or ``don't know/decline'' to answer in response to our questions. The ``no'' category is the reference category in all three models, meaning that the plotted coefficient values indicate the shift in the likelihood of a given response \textit{compared to those who said ``no''}. The colors of each point estimates and intervals correspond to the individual's response about a particular group---did they express a positive, negative, or non-responsive view of the actor in question? 

When interpreting these results, we are most interested in three things. First, we are most interested in the ``Yes'' responses to the contact\index{contact} and benefit questions. These coefficient values tell us how respondents reporting contacts\index{contact} or benefits compare to those who do not. Second, we are interested in the direction of the correlation coefficient---does it fall to the left or the right of the 0 line? This indicates the direction of the comparison. For example, are respondents with previous contact\index{contact} experience more or less likely to report a positive view of US military personnel? Third, we are interested in how far a point estimate is from the 0 line, which says something about the difference between the groups. 

To supplement the information provided in Figure \ref{fig:coefplot1} we also include Table \ref{tab:posteriordistribution}. This table breaks down the results to ease the interpretation of the primary comparisons of interest. The columns show the 1) type of contact\index{contact} or benefit reported, 2) the direction of the comparison, 3) the type of assessment, 4) the median coefficient estimate from the models, and 5) the percentage of the posterior distribution that falls in the median's direction. Regarding the final point, if we expect an increase in some attitude, and the median value is positive, then the credibility column shows the percentage of the distribution that falls above 0. This is useful because it gives us a sense of how confident we are about the difference in attitudes. For example, suppose we have a positive median value suggesting a higher probability of a particular response, and we see that 99\% of the posterior falls above 0. In that case, we are pretty confident in the differences we observe between the two groups. Alternatively, if we have a positive median value but see that only 55\% of the posterior falls above 0, we can conclude that the probability of the observed difference is only slightly better than a coin flip. 

\input{../Tables/Chapter-Contact/model-posterior-table.tex}

First, we can see that individuals reporting having had interpersonal contact\index{contact} with US military personnel have a higher probability of expressing a favorable attitude towards US military personnel stationed within their country. Similarly, we see that these individuals also have a higher probability of expressing a favorable view of the US government\index{government} and US people. We find similar results when looking at the network contact\index{contact} variable. Individuals who report knowing someone who has had interpersonal contact\index{contact} with US service personnel in their country also have a higher probability of expressing a favorable view of US military personnel, the US government\index{government}, and US people more broadly. 

Individuals who report having had personal contact\index{contact} with US military personnel, and individuals who report knowing someone who has had personal contact\index{contact} with US military personnel, also have a higher probability of expressing a \textit{negative assessment} of US military personnel, the US government\index{government}, and possibly the US people as well. This result matches the findings from our previous work focusing only on the first year of the survey\index{survey}.\cite[see]{Allen2020} Individuals who have had personal encounters with the US military or who know people who have had personal encounters are more likely to express both positive and negative views of all three groups. In substantive terms, contact\index{contact} likely generates more informed opinions. Seeing a decrease in the probability of a ``don't know/decline'' response among those who report having had contacts\index{contact} bolsters this idea. 

When it comes to benefits, we observe a slightly different pattern. Among those individuals who report directly receiving personal benefits from the US military presence in their country and individuals who report knowing someone who receives such benefits, we tend to observe a higher probability of expressing a positive view of US actors and a lower probability of observing a negative view. This holds primarily for evaluations of the US military and US government\index{government}, and to a lesser extent for evaluations of the US people. For the latter group, we find that individuals reporting someone in their social network has received economic\index{economic effect} benefits from the US correlates with an increase in positive and negative assessments of the US people. 

Overall, these results suggest that the receipt of economic\index{economic effect} benefits, more than interpersonal contacts\index{contact}, may contribute to more decidedly positive views of US actors. However, caution is in order here since we cannot disentangle the exact causal processes at work. Because we cannot track individuals' experiences over time, host-state residents who already view the US in a favorable light may be more likely to work for the United States. However, this selection dynamic may not account for those individuals who work for local firms that do business with the United States. Individuals may have an easier time choosing not to work directly for the US government\index{government} if they object to US policies, but it may be harder to avoid receiving economic\index{economic effect} benefits if local employers work with the US government\index{government} or military in a given locale. 

Tale \ref{tab:posteriordistribution} provides an overview of the different forms of contact\index{contact} and benefits, and individuals responding ``yes'' to these questions compared to those who respond ``no''. It provides a quick summary of the estimated coefficient size (such as the median column) and the percentage of the credible intervals that support the estimated direction. This may be useful as the variation in interval widths shown in Figure \ref{fig:coefplot1} may make it difficult to determine how strong the evidence is in favor of a given estimate. In general, we find that most of these distributions consistently suggest a higher or lower probability of a given response for individuals who report contacts\index{contact} or benefits. However, in a few cases, these estimates are less clear-cut. For example, when looking at the differences in views of US military personnel between those who report receiving personal benefits and those who do not, we find that there is only a 68\% probability of observing more favorable attitudes of US military personnel. Similarly, when looking at evaluations of the US people, the evidence linking personal contacts\index{contact} and personal benefits to more unfavorable attitudes is slightly weaker. Here we find only a 71\% and 55\% probability that individuals responding ``yes'' to these questions have more unfavorable attitudes than those who respond ``no.''


Before moving on to the next section, Figure \ref{fig:priorcomparison} compares the priors used in our models to the posterior distributions for the coefficients on the contact\index{contact} and benefit variables. We generate the priors using the coefficients from our earlier work using the first year of the surveys\index{survey}.\cite{Allen2020} This figure can help us understand how much our findings have changed as a result of including two additional years of surveys\index{survey} in our analysis. The larger points represent the prior values, and the smaller ones represent the posterior estimates from the models in this chapter. The different colors of the points correspond to the respondents' assessments of the reference groups. The overall structure of the figure is similar to that of Figure \ref{fig:coefplot1}, but to streamline the presentation and focus on the most relevant quantities, we only include the values corresponding to the ``yes'' responses for the contact\index{contact} and benefit variables. 

In general, the results of our current analysis largely reflect the results of the previous analysis. The priors and posteriors for the positive and negative responses are reasonably close to one another. When we focus only on the positive and negative response categories, we see that in roughly two-thirds of the coefficient estimates, the current analysis produces slightly smaller coefficient estimates than in our previous study. To put it differently, in many cases, the addition of the new data appears to have pulled the coefficients in towards 0, producing smaller absolute values of the estimates. In the remaining cases, the updated estimates are roughly equivalent to the prior values or larger. However, these differences are fairly small across the board.

Where we see greater levels of divergence, it tends to be in the estimates of the ``don't know/decline'' response categories. Further, in these cases, the current analysis generally produces smaller coefficient values for this response category. A smaller coefficient indicates that, across the different contact\index{contact} and benefit questions, individuals who respond ``yes'' to these questions are less different from those who responded ``no'' when giving a ``don't know/decline'' response than our first analysis suggested. 


\begin{figure}[t]
	\centering\scalebox{.55}{\includegraphics{../../Figures/Chapter-Contact/figure-coefficient-prior-comparison.png}}
	\caption{Comparison of the prior values based on and the posterior estimates from the full three year sample. Large point represents the prior value and the smaller points represent the posterior value. 50\%, 80\%, and 95\% credible intervals shown.}
	\label{fig:priorcomparison}
\end{figure}%replace with newer figure. If things change, then report down here.


Our takeaway here is that the results of the updated models are fairly consistent with our earlier modeling efforts. However the addition of new data and the tendency towards slightly smaller estimates suggests that the estimates in our previous models may have 

\subsubsection*{Do Contact\index{contact} and Benefits Vary by Country?}

% Maybe cite that Horiuchi paper here? Yes, definitely.
Thus far, we have assumed that the relationships between different forms of contacts\index{contact} and benefits were similar across countries. However, interactions between US service personnel and host-country locals may produce different assessments of US actors across different national and local contexts. Such varying could result from many sources. For example, the factors driving contact\index{contact} in different locations may vary in important ways. Alternatively, specific units stationed in specific areas might have specific missions or histories that have caused higher levels of local hostility towards various US actors. The deployment of Osprey\index{Osprey} aircraft in okinawa\index{Japan!Okinawa} and their reputation for accidents may cast a shadow over the interactions between residents and US military personnel in ways different from interactions between these same groups in different parts of the country. In other cases, US military facilities and operations may be further removed from civilian populations, meaning that interactions occur under more favorable conditions. 

To address this possibility, we ran the models in Table \ref{tab:contactfull} again. Here, we allowed the coefficients on our four contact\index{contact} and benefits variables to vary across the countries in our sample. We present the results from these varying coefficient models in Figure \ref{fig:contactcoefficientsvarying}. Due to the more complex nature of plotting the varying coefficient values for each of the 14 countries, we have arranged this figure differently than the previous figures plotting similar results. First, we restrict our focus to only ``Yes'' responses to the contact\index{contact} and benefits questions. Accordingly, these coefficients represent the differences between the ``Yes'' respondents and the ``No'' respondents for a given contact\index{contact} or benefit question. Second, we only look at the positive and negative outcome categories for the respondents' questions about their attitudes towards US military personnel, US government\index{government}, and US people. The top panel corresponds to the positive assessments of the three groups, and the bottom panel corresponds to the negative assessments of the three groups. As in the previous figures, each column corresponds to one of the models analyzing attitudes towards these three groups. Last, the colors of each coefficient correspond to the coefficients for each of the four contact\index{contact} and benefit variables. Again, each coefficient represents the coefficient for those who responded ``Yes'' when asked about 1) personal contacts\index{contact}, 2) network contacts\index{contact}, 3) personal benefits, or 4) network benefits.  


\begin{figure}
	\centering\scalebox{.62}{\includegraphics{../../Figures/Chapter-Contact/figure-coefficient-plot-varying.png}}
	\caption{Varying coefficients of contacts\index{contact} and benefits across country. 50\%, 80\%, and 95\% credible intervals shown. Coefficient estimate includes the fixed population-level coefficient value plus the country-specific error associated with the varying coefficients.}
	\label{fig:contactcoefficientsvarying}
\end{figure}

We will spend some time focusing on results specific to the US troops model and will discuss more general patterns from the varying effects models. We approach the discussion in this way for a couple of reasons. As is evident from Figure \ref{fig:contactcoefficientsvarying}, there are dozens of specific coefficient values to review given the four predictor variables, two outcome responses, and 14 countries in the sample. Our aim here is to highlight how the varying effects models differ from the models estimating the population-level effects presented above. The model predicting attitudes towards US military personnel deployed to the referent country exhibits the most variability when using this alternative modeling strategy. This variability merits the most scrutiny because it suggests that the most pronounced differences in country-level effects compared to the population-level effects presented above are when we look at US military personnel more than the other reference groups.



These models indicate that there is sometimes substantial variation in the size and direction of these coefficients when we allow them to vary across countries. In substantive terms, this means that the differences between individuals who report some type of contact\index{contact} or benefit are not the same in every country. For example, when we look at the coefficient for the personal contact\index{contact} variable, we can see that its relationship with positive assessments towards US military personnel is mostly positive across countries, but the magnitude of these positive values varies considerably. In South Korea\index{South Korea}, Poland\index{Poland}, the Philippines\index{Philippines}, the coefficient values are fairly small compared to the other countries. Alternatively, the coefficient values are relatively higher in Turkey\index{Turkey}, Spain\index{Spain}, Portugal\index{Portugal}, the Netherlands\index{Netherlands}, Japan\index{Japan}, and Belgium\index{Belgium}. 

In the case of Kuwait\index{Kuwait}, the median value for personal contact\index{contact} is negative. However, it is inaccurate to suggest that personal contact\index{contact} correlates with negative views of US service personnel in Kuwait\index{Kuwait}. The median value on this coefficient is close to 0, and a substantial amount of its posterior distribution falls above 0. Furthermore, the range of values contained within the credible intervals is relatively small compared with other countries. Overall this suggests a fairly high degree of \textit{uncertainty} and possibly mixed responses concerning how likely individuals who report interpersonal contacts\index{contact} are to give a favorable assessment of US military personnel stationed in Kuwait\index{Kuwait}.

We find similar patterns when looking at the network contacts\index{contact}, though the magnitude of the coefficients on these variables tends to be smaller than interpersonal contacts\index{contact}. In a couple of cases, we also find that the median of the network contact\index{contact} coefficient comes very close to 0. South Korea\index{South Korea} and Kuwait\index{Kuwait} have median values that fall close to 0, again suggesting a high level of uncertainty about how respondents reporting friends or family having contact\index{contact} with US military personnel compared to those who do not.  

When we look at the relationships between the receipt of personal economic\index{economic effect} benefits and positive assessments, we see more variability in the size and direction of the coefficients across countries. Most countries yield a positive coefficient value, though the magnitude of these coefficients varies considerably. Turkey\index{Turkey}, the Philippines\index{Philippines}, and Italy\index{Italy} all see larger and more consistently positive coefficients, indicating that those who receive personal economic\index{economic effect} benefits from the US military presence are more likely to have a favorable view of the US military presence in the referent country than respondents who do not receive these types of benefits. Other countries, like Portugal\index{Portugal}, Poland\index{Poland}, and Australia\index{Australia}, see smaller positive values and some more uncertainty regarding the positive direction of the coefficient. For example, only 83\% and 74\% of the posterior falls above 0 for Portugal\index{Portugal} and Australia\index{Australia}, respectively. In still other cases, like the United Kingdom\index{United Kingdom}, the Netherlands, Japan\index{Japan}, Germany\index{Germany}, and Belgium\index{Belgium}, we find that personal benefits correlate \textit{negatively} with more positive views of US military personnel. For the United Kingdom\index{United Kingdom}, only 66\% of the posterior falls below 0, but for other countries, like Germany\index{Germany} and Japan\index{Japan}, $\approx$80\% of the posterior falls below 0. In the Netherlands, 92\% of the posterior is below 0. Alternatively, network benefits do not appear to exhibit the same level of variability in the results, instead producing more consistently positive coefficients, suggesting that knowing someone who benefits economically\index{economic effect}  from the US presence is more likely to correlate with more favorable attitudes towards the US. 

Looking at the negative assessments panel, we see that the correlation coefficients are more homogeneous than the positive assessment outcome response. In general, we see that both personal contacts\index{contact} and network contacts\index{contact} yield positive correlation coefficients, suggesting that those who respond ``Yes'' to these questions are more likely to express negative assessments of the US military personnel deployed to their country than those who say ``No.'' Overall this is in line with the results from our previous models, which suggest that individuals reporting contacts\index{contact} appear to have more informed opinions in general. However, the magnitude of these coefficients tends to be fairly small compared to the magnitude of the coefficients on the contact\index{contact} variables in the positive assessment part of the model.

Similarly, we see that the personal and network benefit questions tend to yield negative coefficients when looking at the negative assessment part of the model. Here we also tend to see more variability across countries as compared to the contact\index{contact} questions. The receipt of personal benefits appears to more reliably correlate with a lower probability of expressing a negative view of US military personnel across countries than knowing someone who receives such benefits. Several countries yield coefficient estimates where 100\% of the posterior falls below 0, indicating a high probability of a negative correlation. Like Poland\index{Poland}, the Philippines\index{Philippines}, the Netherlands, Kuwait\index{Kuwait}, Italy\index{Italy}, and Belgium\index{Belgium}, other countries show a very high percentage of the posterior falling below 0. Importantly, there may be substantial overlap in countries like Italy\index{Italy} with 0, 95\% of the posterior falls below 0. Accordingly, we have a fairly high degree of certainty that these correlations are negative. Notably, Japan\index{Japan} is the only country where the personal benefit question yields a coefficient with significant overlap with 0---only 57\% of the posterior falls below 0, indicating that, in Japan\index{Japan}, there is fairly substantial uncertainty regarding whether the receipt of personal economic\index{economic effect} benefits correlates with a reduced probability of expressing a negative view of US military personnel. 

When looking at the network benefits question, we can see that in several cases, the median of the posterior falls below 0, but large portions of the posterior fall above 0. For example, when looking at South Korea\index{South Korea}, only 65\% of the posterior falls below 0 for the network benefits question. For Poland\index{Poland}, only 54\% of the posterior falls below 0. Other countries like Japan\index{Japan} and Kuwait\index{Kuwait} follow similar patterns. Looking across countries, there is less consistent evidence that knowing someone who receives economic\index{economic effect} benefits of some kind correlates with a lower probability of expressing a negative view of US military personnel.

Overall, these results point to some interesting patterns, suggesting substantial variation in how contacts\index{contact} and benefits translate into assessments of US military personnel in the referent countries. But what about the other reference groups? The coefficients for the contact\index{contact} and benefits variables appear to exhibit the most variation across countries when looking at views of US military personnel stationed in the respondents' country. However, there is some variation worth discussing in the other models as well. 

The coefficients for all four predictor variables exhibit a greater consistency when predicting views of the US government\index{government}. Though the coefficient values tend to be fairly small compared to the US troops models, they tend to correlate more uniformly with more positive views of the US government\index{government} than people who respond ``No''. Notably, exceptions appear in Portugal\index{Portugal} when looking at the relationships between network benefits and positive views of the US government\index{government}---here, we see this coefficient is negative. In contrast, the other coefficients yield positive median values. The network benefits question is the only question that produces a negative coefficient value with a fairly high level of certainty.  Similarly, Kuwait\index{Kuwait} is the only country to produce estimates that uniformly fall below 0. Again we see that the median estimates tend to be small in magnitude, and only the benefit questions appear to yield negative coefficients with a high level of certainty. Interestingly these results suggest that in Kuwait\index{Kuwait} at least, those who receive economic\index{economic effect} benefits from the US military presence are less likely to express positive views of the US government\index{government}. 

Looking at negative assessments of the US government\index{government}, we see a similar level of consistency on the coefficients for the contact\index{contact} questions. In general, these coefficients tend to be small and positive, suggesting that contact\index{contact} with US service personnel correlates with a higher probability of expressing a negative view of the US government\index{government}. However, the benefits questions yield more mixed coefficients---particularly the personal benefits question. Here we can see that for several countries, like the United Kingdom\index{United Kingdom}, Portugal\index{Portugal}, the Netherlands, Germany\index{Germany}, Belgium\index{Belgium}, and Australia\index{Australia}, we tend to see moderate to strong evidence that individuals reporting the receipt of personal economic\index{economic effect} benefits are less likely to express a negative view of the US government\index{government} as compared to those who do not. In other cases, like Turkey\index{Turkey}, South Korea\index{South Korea}, and Poland\index{Poland}, we find coefficient values very close to 0, indicating a high degree of uncertainty over the direction of the comparison. However, in the Philippines\index{Philippines} and Japan\index{Japan}, we find stronger evidence that those who report receiving personal economic\index{economic effect} benefits are more likely to express negative views of the US government\index{government}---the opposite of what we find in many other countries. 

Finally, the models predicting attitudes towards the US people yield the most consistency. Looking at positive responses, we see that all of the coefficients yield positive values. However, some see larger shares of the posterior falling below 0, suggesting some uncertainty over the direction of the comparison. In the case of personal contacts\index{contact}, we see more evidence of some stronger positive correlations in the United Kingdom\index{United Kingdom}, Turkey\index{Turkey}, and Australia\index{Australia}. 

Turning to the US people's negative assessments, we see a slightly more mixed picture, but this largely stems from the higher level of variability on the personal benefit coefficient. In general, the personal contact\index{contact} and network contact\index{contact} questions produce positive coefficients, which is in line with the results from the other models. Again, respondents who report having personal contact\index{contact} with US personnel or who know someone who has personal contact\index{contact} with US personnel are more likely to express both positive and negative views of the US people. However, when we look at the personal benefits question, we can see a greater variation in the direction of the coefficients. Some countries have clearer positive coefficient values, some with clearer negative coefficient values, and some with a high level of uncertainty. Portugal\index{Portugal}, the Philippines\index{Philippines}, and Australia\index{Australia} appear to have more reliably negative coefficient values, while the United Kingdom\index{United Kingdom} and Kuwait\index{Kuwait} yield more clearly positive coefficients. The rest produce coefficients that are very close to 0 and suggest a high level of uncertainty over the direction of the differences between groups. 

Overall, these results point to the importance of considering country-level factors and local contexts in considering how contact\index{contact} with US personnel and receiving economic\index{economic effect} benefits shape mass attitudes towards various US actors.



\section*{Conclusions}

While previous research shows that contact\index{contact} between populations can break down the barriers that create stereotypes, mistrusts, and conflict, very little research has examined this in the context of the relationship between the foreign military and host-state civilian populations. Our results offer a picture of how interpersonal contact\index{contact} and economic\index{economic effect} relationships may help to create more positive views of US actors. In general, interpersonal contact\index{contact} and the flow of economic\index{economic effect} resources from the US military into the host state may, directly and indirectly, create more informed and more positive views of the United States.

However, our results suggest that there are some important conditions and nuance to these patterns. People reporting interpersonal contact\index{contact} with US military personnel are more likely to express various US actors' positive and negative views. Overall, this suggests that contact\index{contact}---whether in direct personal experience or by knowledge of others' experiences---can create more informed views of the US military and other US actors. On balance, however, the correlation coefficients suggest that more individuals responding ``Yes'' to the contact\index{contact} questions are slightly more likely to have a favorable view of these US actors than not. Importantly, however, people who have had direct, negative experiences with US military personnel may be more likely to think ill of the United States and its people. While part of the process may be from increased social friction and harmed by general social ills of traffic\index{traffic}, environmental\index{environment}, and noise pollution, some subset of these people has experienced direct incidents, including motor vehicle accidents and personal violence. Some subset of these direct and general issues can gain traction to a broader national audience and become the catalyst for anti-basing movements that seek to expel the United States.

Furthermore, these results are not constant across countries. While direct interpersonal contact\index{contact} and knowledge of others' interactions with US military personnel correlate with a higher probability of expressing both positive and negative views of US actors across most countries, there are some notable exceptions. In some cases, the magnitude of these coefficients is smaller, and there is greater uncertainty over their direction, indicating that, at a country-specific level, these broader patterns may or may not hold. In more extreme cases, like Kuwait\index{Kuwait}, there is at best a high level of uncertainty concerning the correlation between forms of contact\index{contact} and the views individuals are likely to express. At worst, again as in the case of Kuwait\index{Kuwait}, it may be the case that contact\index{contact} with US military personnel is more likely to correlate with generally less favorable views of US military personnel and the US government\index{government}. 

Alternatively, it is often the case that second-order contacts\index{contact}---that is, knowing someone who has had direct personal contact\index{contact} with the US personnel stationed within a country---often correlate more strongly with positive and negative assessments. This is especially true when we look at the views of the US government\index{government} and the US people. When asking people about their views of the US people, the personal contact\index{contact} coefficients are generally small. There is a slightly lower level of certainty regarding their direction in most countries. In contrast, the network coefficients are both positive and have a higher certainty regarding their direction.

Compared to contact\index{contact}, the benefits questions tend to produce far more variable results. When looking at all of the countries in our sample together, we find some indication of a tradeoff-- benefits appear to correlate with a higher probability of positive responses and a lower probability of negative responses. However, when we look at how these questions vary across countries, we see a mixture of positive and negative coefficient values. Substantively, this means that the receipt of personal benefits can correlate with a higher probability of positive responses in some countries and a lower probability of positive responses in other countries. This is particularly the case when we look at attitudes towards US military personnel. Still, we also find similar patterns when looking at attitudes towards the US government\index{government} and US people.

We emphasize that these quantitative results alone are not as causal. Our underlying survey\index{survey} is not designed to tease out the causal effects of contact\index{contact}. Instead, a better way to conceptualize their meaning is simply through inter-group comparisons, accounting for the correspondents' various other demographic and attitudinal features. That said, these results, combined with our interviews\index{interview} and existing qualitative work, suggest the processes that likely play a role in driving mass attitudes towards the US military presence in a country and other US actors. Different types of contact\index{contact} and experiences likely shape views of US actors. This is likely not a contentious statement. But these views and attitudes also likely play an important part in shaping individual decisions over whether, and how, to engage with US military personnel in the future. Individuals with positive experiences may come to hold positive views of US personnel and may decide to seek out more interactions. However, individuals with negative experiences may also choose to seek out further opportunities for interaction, though the nature of these encounters may vary dramatically. Individuals who are victims of crimes\index{crime} or accidents may be more likely to seek out events where opportunities for conflict and escalation are higher, like protests\index{protest}, potentially strengthening existing animosities. 

%Our results also suggest there is no simple answer to how contact\index{contact} and benefits shape attitudes. The results presented in Figure \ref{fig:contact\index{contact}coefficientsvarying} suggest that we see a fairly substantial amount of variation in the magnitude and direction of the coefficients on the contact\index{contact} and benefit variables. In particular, the benefit variables are prone to considerable variation, with the personal receipt of economic\index{economic effect} benefits correlating with more positive attitudes of US actors in some countries and more negative attitudes in others.
%redundant with the above, so cutting for space


%Ultimately, teasing out these causal pathways will be challenging, especially when considering the cross-national variation discussed above. Survey experiments may provide a way to understand these dynamics better, but qualitative research methods will also prove to be vitally important in developing a fuller understanding of how, when, where, and how people interact and how these interactions shape future attitudes and decisions.
%I think we cover this in the previous paragraph


Our results also highlight some important challenges for policymakers. When a new local crisis emerges from an altercation between a service member and a civilian, the source of negative attitudes towards the US military appears readily apparent to decision-makers. The policy response has often been to reduce the friction points between the military and civilians and draw service members back to the base so that future points of conflict become less likely. Gillem discusses the withdrawal of service member contact\index{contact} with civilians by the branches of the military.\autocite{Gillem2007} While the Navy\index{Navy, US} has had fewer points of contact\index{contact} historically, the development of bases over the last few decades have attempted to limit any points of contact\index{contact} between service members and host-state civilians. However, as Gillem notes, ``Relocating the US military to deserts or isolated islands (artificial or real) to be ``safe'' from terrorism\index{terrorism} is only one justification for policies of avoidance. These moves are also ways to remove American soldiers from places like American Village.'' \autocite[p. 262]{Gillem2007}. Such policies became more widespread after the 9/11 terrorist\index{terrorism} attacks to reduce service member exposure to security risks and kidnapping\index{crime!kidnapping} threats. The immediate and loud negative attention from negative interactions generally dwarfs the small, every day, passive effects of interpersonal contact\index{contact}---few media stories gain national or international attention about military and civilian parents both having kids on the same soccer team. 


There is an inherent problem with the strategy of isolating the military from civilian populations. Suppose negative views about the US presence comes from experiencing the problems caused by the base itself, and negative contact\index{contact} experiences are largely incidental to those with negative views. In that case, there becomes an asymmetric feedback problem for the US military. Gillem argues that ``despite widespread media attention focusing on the tragic stories of rapes\index{crime!rape}, deadly accidents, and environmental\index{environment} damage, surveys\index{survey} of residents near some of these outposts reveal not so much an all-consuming desire for their demise but disgust, above all, with the excessive use of land by American forces.'' \autocite[p. xv]{Gillem2007} Thus, media attention may amplify negative events and anti-US sentiment that do not represent the modal type of interaction or attitude. Consequently, a drawdown of contact\index{contact}, a policy of service member isolation, means that the negative externalities increase in local communities. At the same time, one of the few paths for building support is behind fortress walls. A separation policy against inter-community contact\index{contact} limits the United States' ability to maintain long-term bases in the evolving domain of competitive consent\index{Domain of Competitive Consent}.

Further complicating matters, however, is the fact that our results emphasize that policymakers need to be sensitive to local realities. Contact\index{contact} or benefits can correlate with less favorable attitudes in some countries and more favorable attitudes in others. Local conditions, and mission-specific functions, all play a role in shaping what the ``typical'' interaction looks like. What works in Germany\index{Germany} or Spain\index{Spain} may not work in the United Kingdom\index{United Kingdom} or Japan\index{Japan}. Even within the European\index{Europe} context, we find very different patterns prevailing across countries. Better understanding the local factors that shape mass attitudes towards the US military and other US actors is pivotal to contextualize the influence\index{American Influence} of basing. Our findings should also serve as a warning to policymakers and military commanders for using overly broad proxies like ``Western culture''\index{culture} or geography in inferring what effect contacts\index{contact} or benefits are likely to have.

The everyday behavior of US military personnel living in foreign communities builds the soft power\index{soft power} upon which support for the US, its mission, and its strategy for the international arena accumulate. As the US moves towards increased domains of competitive consent\index{Domain of Competitive Consent} as basing powers, compete for new basing sites, and maintain old alliances, the passive spread of stereotype deconstruction and building of goodwill is a vital front that is easy to overlook. Still, base commanders facing increased public backlash are unlikely to weigh the long-term effects of retreating to the proverbial motte versus the short-term gain of decreasing community friction. Retreat is an easy policy with a seemingly obvious effect. The harder and costlier strategy involves cultural change, behavioral nudging, and effective monitoring of off-base activities by service members. Additionally, such efforts will never be 100\% effective as service members are human---harmful interactions will happen, and the result will at times fuel anti-base or anti-American movements. Notably, returning to figures \ref{fig:coefplot1} and \ref{fig:contactcoefficientsvarying}, it is evident that there is often a marked shift for positive views relative to negative views as a result of contact\index{contact}. The economic\index{economic effect} effects within social networks are even clearer. While hiring local construction\index{construction} workers is part of that story, soldiers patronize local establishments when they are off-duty.

We argued in the introduction that troop deployments had been the historical microfoundations of US power and grand strategy\index{Grand Strategy}. We still stand by that argument as American personnel deployed overseas have been vital in shaping international and domestic politics\index{international relations}. The US is currently in a military transition where it seeks to rely more heavily on the capital-intensive parts of its force, such as drones\index{drone}\index{Remotely Piloted Aircraft}, and favoring smaller bases rather than larger facilities with massive geographic footprints in the middle of a country. However, this trend will not eliminate all major bases as human power is still fundamental to adequate force projection, the defense of territory, and actual warfighting. Additionally, longstanding allies continue to petition the United States to maintain its larger deployments in countries like South Korea\index{South Korea} and Germany\index{Germany}. These trends, combined with the research in this chapter, suggest that the people of the armed services are not only the microfoundations of US power of the past, but will be the primary assets in competing for consent\index{Domain of Competitive Consent} for the US to maintain its basing network into the future.
