
%NOTE: Now that this chapter has more of a negative spin maybe we should think of a different title for this chapter

%\section*{Introduction}
\vspace*{-0.85cm}
\rule{\linewidth}{0.10pt} \\[-1.25cm]
{\footnotesize\paragraph{Summary:} Conventional accounts hold that minority communities often bear the brunt of the negative externalities of US military bases. Others suggest that US military bases can provide opportunities for social and economic mobility that may not otherwise be present for minority groups in some host countries. We show that the relationships between minority populations and their attitudes towards US actors are complex. First, minority self-identification alone correlates with a higher likelihood of expressing a positive view of US actors. Second, once we adjust for demographic and attitudinal variables that correlate with minority status, like income and education, this relationship disappears in the population-level effects and minority self-identification correlates with less positive views of all three groups. Finally, we find that there is substantial local variability---adjusting for other individual-level factors the difference between minority and non-minority views of US actors depends on the national context, as well as individual experiences.} 
\\[-0.5cm] 
\rule{\linewidth}{0.10pt}

\vspace*{0.5cm}

%%%%NOTE: Just went through the interview with the Wiesbaden mayor's office liaison and there's a ton about ethnic minorities that we haven't used yet. Let's make sure and go through and add some of that to this chapter. CM. 


``Okinawan people have a strong feeling that they bear too much of the cost,'' said the retired Japanese ambassador, sitting in his home in Tokyo \cite{tokyoone20200427}. In a polite, diplomatic fashion that is typical of his profession, and that on our third year of fieldwork we were starting to become familiar with, he further explained this sentiment to us, ``I have no intention of criticizing the US forces, but many crimes have happened because of the US soldiers.'' As we will discuss in more detail in Chapter \ref{cha:protest}, he noted how, among the general population, a single crime can undo years of public diplomacy efforts. He referenced the example of the 1995 rape of an Okinawan schoolgirl by three American service members. He noted that even though the rape ``has nothing to do with the US security situation,''  it still has a negative impact on Okinawans' perceptions of the US military. 

The US provides external security for Japan, yet 70 percent of US bases are in Okinawa, taking up 15 percent of the island's territory (Okinawa is only 0.6 percent of Japan's total territory) \cite{JPTimes2020,tokyoone20200427}. The sentiment among many Okinawans is that if US forces offer security to all of Japan, then other parts of the country should carry a more proportional share of the burden. In this respect, the experience of Okinawans is not much different from that of other minority populations around the world. The asymmetry between diffuse, collective goods and the small sector of the population bearing the burden for their provision is common in overseas basing for a variety reasons, including geography, logistics, and national policy.

Okinawans have long had grievances about discrimination by the Japanese government. The location of US bases in their region, along with the problems they bring, is a crucial point of contention between many in the region and the government. Okinawans are a minority ethnic group whose home island was annexed by Japan in 1879.\footnote{Though suppressed by the Japanese, the Ryukyuan languages (one of which is Okinawan) are actually distinct from Japanese, and those looking to preserve Okinawan identity are engaging in concerted efforts to teach the languages to the new generations to prevent their extinction \cite{Heinrich2004,Fifield2014,UNESCO2010}.} The topic of Okinawa's disproportionate burden remains relevant, with Okinawa's Governor Denny Tamaki noting in May of 2020, as annual anti-US military base protests had to be canceled due to the COVID-19 outbreak, ``I will completely devote myself to resolving issues [in Okinawa] including the heavy burden [of hosting US bases]'' \cite{JPTimes2020}.

Though the United States encouraged the Ryukyuan language and independence during American rule over the Ryukyu islands from 1945 to 1972, the US occupation resulted in increased Japanese nationalism in this area. It is of note that the Ryukyu islands saw bloody battles against American forces on their territory \cite{tokyoone20200427}. The territory that the Americans occupied was one in which the virtual totality of the population became displaced and impoverished \cite{Heinrich2004}. Thus, rather than Americans being viewed positively for their encouragement of Ryukyuan independence and culture, nationalized Japanese language and identity continued to expand and, after the occupation, tensions between Okinawa and Tokyo arose over the continued US military presence in Okinawa. Especially heinous offenses by service members against Okinawan civilians served to spark protests and turn public opinion against the US presence. In our sampling of reported criminal offenses by service members against civilians, 37.5\% of reported events occurred in Okinawa. These offenses include DUIs, drug-related incidents, manslaughter, kidnapping, and sexual assault.

In a stark contrast to the case of Okinawa, we also talked to US service members and locals at the RAF Lakenheath military base in Suffolk in the United Kingdom. In their glowing descriptions of each other, the English locals and the American service members cited a shared language and culture as part of what enables that positive connection. Over coffee at a series of discussions held at the base with American non-commissioned officers and two Britons representing the Ministry of Defense and the Royal Air Force, we heard a stream of fondly-recalled positive anecdotes: ``I live in government housing in the village. Local kids come through for trick or treating because they know they'll get American candy'' \cite{rafsix20190719}. ``Oddly enough, the locals like to celebrate the 4th of July with us. Any chance to drink'' \cite{rafeight20190719}. ``On the macro level, the base is well received in the local area. Why wouldn’t it be? It's been here so long'' \cite{rafthree20190719}. ``There is a sushi place by my house that I religiously eat at. One day they were getting their daily fresh fish run. I made a joke about whether they had any extra, the owner went back inside and gave me a whole lobster tail, and three pounds of shrimp. He said it was to thank me for everything the US does'' \cite{raffive20190719}. The locals were no less effusive, noting the importance of the shared language and culture between the British and Americans, highlighting the stark contrast between the relationship with the local population in England and in Okinawa \cite{councilone20190718}.

Beyond the shared culture, interviewed service members considered England a desirable assignment in part because locals are friendly regardless of the service members' race, ethnicity, gender, or sexual orientation. During our discussions, several service members mentioned that in England, ``they treat everyone exactly the same'' \cite{raffive20190719}.  There was a clear consensus that the type of discrimination that service members faced when deployed to other locations was uncommon in England. But what about \textit{local} minority groups?  Could the positive perceptions we observed in England be related to the fact that US service members interact mostly with members of the majority ethnic group, as opposed to Okinawa, where many of the locals interacting with the US military belong to an ethnic minority? Would ethnic minorities in England perceive a US military presence in their area differently from majority groups? Given that most of the locations that the US deploys to also have significant minority populations, this becomes a key question to explore in understanding relations between the US military and local host populations. 

%Add this back in somewhere for book version.
%A lesbian service member noted that she walks around holding her wife's hand and does not get treated differently at all, observing that the British are actually more accepting of being gay than the Americans are. She contrasted this with her experience being stationed in South Korea, where she noted that locals often misgendered her, calling her ``sir'' and changing her name to a male name \cite{rafone20190719}. Another service member recalled more overt racism in Eastern Europe, recounting that going to dinner with African-American friends was ``difficult'' \cite{rafeight20190719}.

%I'm not sure about this paragraph, since it seems to be about some negative local attitudes toward American minorities, not necessarily minority attitudes toward the American presence. I'm not sure the local attitudes described here have any real relationship to the minority populations within the country either, especially in the SK or Eastern Europe examples. Might be good to save for something more about American experiences across contexts rather than minority vs. majority on the local end. - AS

It is of note that we conducted the Lakenheath interviews in the aftermath of Britain's vote to leave the European Union in 2016 (``Brexit''), as the country negotiated the terms of Britain's exit, and just days before Boris Johnson's election as Prime Minister in July 2019.\footnote{At this point in the process, people widely assumed that Johnson would become the next Prime Minister and the country was just awaiting the formal institutional processes to formalize the decision.} Voters in Great Britain voted to exit in large part because of concerns about immigration from the European continent (especially Central and Eastern Europe) and changes to British identity \cite{Goodwin2016,Goodwin2017}. The large influx of refugees into Europe in 2015 exacerbated these concerns as people sought asylum from, among other places, the conflicts in Syria, Iraq, and Afghanistan (and broadly from Africa, the Middle East and South Asia) \cite{Chan2015}. Research on the Brexit vote found that areas experiencing the highest increases in immigration in the years prior to the vote were most likely to vote to leave. This was particularly true in areas that had been previously ethnically homogenous and experienced sudden demographic changes due to immigration. Britons' fear of how these sudden changes would alter British identity affected their likelihood of voting ``Leave'' \cite{Goodwin2017}.

We confirmed some of these observations when talking to British locals in both London and Suffolk. A Lakenheath Council Member told us that he had voted ``Leave'' in the referendum, though he was starting to change his mind on it as he saw the economic impact Brexit was having on Britain. The town of Lakenheath voted for ``Leave'' (18,160 votes in favor of leaving, 65 percent of the vote, with 72.5 percent turnout). When we asked what it was that had driven the high turnout and ``Leave'' vote, he responded, without hesitation, that it was immigration. He recounted how in Brandon, a town about 6 miles away from Lakenheath, immigrants (especially Eastern European and Portuguese immigrants) were more likely to frequent the local libraries than the English. He noted, ``wanting to avoid racial overtones; it throws people because it's a change in culture. I'm in the supermarket and I can't hear English being spoken.'' He was quick to clarify that ``it's not bad that [demographic change] happens, it's that it happens too fast,'' summarizing the sentiment as, ``it made people feel like England wasn't England anymore'' \cite{councilone20190718}.

The Council Member contrasted interactions with immigrants and refugees to ones with the American military personnel, noting that when it comes to the Americans, ``because our cultures are similar, it is not as big of an issue. The fact that our cultures are very similar helps. It would be different without that.'' He also highlighted the importance of the common language in building positive relations between locals and the US military base. Yet, when we asked him about relations between the base and minority populations in the area, he just said that ``It hasn't come up'' \cite{councilone20190718}.\footnote{We note that despite many active efforts to reach out and interview civil society leaders for minority groups in the area, we received no response. The lack of response is perhaps telling in and of itself, as we identified ourselves in all of our correspondence and calls as researchers based at US universities studying perceptions of the US military abroad.}

The interview subjects at RAF Lakenheath were equally ambiguous when we asked them about relations between the base and minority populations in the region. We found that very few of them had interacted with members of local minority communities, and that they mostly held positive but vague views of them. For example, when we asked about their interactions with Portuguese or Eastern European locals, only one out of six service members volunteered an anecdote, noting that her ``lash lady'' (meaning the technician who applies fake eyelashes) was Portuguese and was ``very nice,'' and that the ``night life is diverse.'' She recalled having friendly conversations with immigrants while out at nightclubs and bars \cite{raffive20190719}.\footnote{It is of note that this individual was the youngest in the group of service members we interviewed at Lakenheath, describing herself as ``mid-20s'' and self-identified as an ethnic minority in the United States (``I'm half Filipina'').} 
%One of the Britons interviewed at the base noted that the immigrants mostly do agricultural work in the area and told us about how in his village once a year they have a yard sale and ``usually the rubbish gets bought by the Eastern Europeans'' \cite{rafthree20190719}.
%NOTE: I went back and forth on that last quote about the rubbish. It's very telling of attitudes towards minorities, but the commander is very easily identifiable and it makes him seem kind of shitty. I'm leaning towards scrapping it to not burn any bridges.CM.
%NOTE: I agree. I'd like to keep it but I can't think of a way to refer to it without referencing him or the specific thing he said, since it's pretty specific. - AS

People living in Japan and in England had very different experiences in their interactions with American service members, but there was a clear common thread across both sets of interviews: The experiences that ethnic minorities had in dealing with the US military were vastly different from those that dominant ethnic groups held. Generally, minorities seemed to be having more negative, or at least less positive, interactions. However, the few comments from American service members about local minorities were largely positive. This highlights the importance of not homogenizing local experiences with the US military and acknowledging that an individual's perception may be significantly influenced by whether they see themselves as being part of a state's dominant ethnic group or belonging to a minority. 

%We don't actually test anything related to the ethnicities of service members, so taking this out. 
%Given that some positive comments about relationships with local minority populations also came from service members who considered themselves minorities, the presence of minority groups within populations of US personnel on installations may also be an important factor in determining local minority views. 

\subsection*{Expectations}
 
Ethnic and minority group relations related to US military deployments abroad have been a concern since the US first started deploying its troops abroad as a major power. The United States often portrays itself as an ally of minority groups abroad. For example, through its interventions in the Balkans, in different parts of Africa, and its support of the Kurds in Iraq, American military power has been brought to bear to protect minority groups from hostile and repressive governments. Though these interventions have sometimes been long-term and costly, they have helped some groups realize ambitions for an independent national state; for others, it has led to the cessation of brutal oppression by majority-led governments. Other times, the United States has used its non-military power to press governments around the world for an increased respect for human rights, particularly in the treatment of minority populations. In fact, states who host US military forces have been found to see an increased respect for human rights \cite{bell2017}.

However, the United States itself has also had problems with violating the rights of ethnic minorities within the United States and has struggled with systematic discrimination against ethnic minorities \cite{Williamson2018}. Beyond its borders, scholars, activists, and journalists have accused the United States of exporting attitudes of racial inferiority, stereotypes, and hierarchies into foreign states. In particular, South Korea and Saudi Arabia, two states that have hosted large numbers of US military forces, demonstrate the proliferation of US racial attitudes \cite{Moon1997,Vitalis2007}. Furthermore, after conducting long wars in Iraq and Afghanistan, and recently curtailing immigration and refugee flows from predominantly Islamic countries, Muslims, whether in majority- or minority-Muslim contexts, may not see the United States as an ally against oppression. 
%The history of Diego Garcia, as covered in Chapter 1, gives further credence to the idea that the United States would rather not deal with human populations at all if it can be avoided entirely.
 
These two trends and narratives leave us with a puzzle. Do minority populations in states that host the US military see it as a friend or a foe? There are several mechanisms by which either of these two narratives could be true. Despite US efforts, long-term deployments in peaceful countries could create direct and indirect paths toward minority discontent with the United States. Directly, Status of Forces Agreements between host states and the United States place some of the basing burden on minority groups not well represented by host-state governments. Okinawa, which in a 2019 referendum voted 72\% against the construction of a new US military base on the Okinawan town of Henoko (a project that involved various environmental concerns) is an example of this dynamic \cite{tokyoone20200427,McCurry2019}.  

%changing language to be more hypothetical and still open to adjudication

Indirectly, the very act of basing also sends credible signals about US commitment to the host state and could empower it to maintain status quo policies towards disenfranchised groups. As an example of this, in Bahrain, where the United States has a significant basing presence by housing both the United States Naval Forces Central Command (NAVCENT) and the 5th fleet, the US seemed to turn a blind eye to Shia-led protests during the Arab Spring, which the Sunni government put down violently \cite{McDaniel2013}. While the Shia population is not a numerical minority in Bahrain, they are the non-dominant ethnic group. These combined direct and indirect mechanisms could compel minority groups to see the United States in less positive terms than the dominant group does. Contrarily, the United States spent decades in Iraq protecting Kurdish populations from Saddam Hussein's central government. The US also attempted to restrain Turkish cross-border incursions into Kurdish areas. Overall, the United States went to great lengths to shield this minority population from adversarial governments and groups throughout the region. 

This chapter thus focuses on studying the differences between the views of individuals who self-identify as ethnic minorities in the 14 countries in our survey sample from the rest of the population in order to adjudicate between these two potential lines of thinking. This chapter goes further than previous ones by contextualizing the contact hypotheses from chapter \ref{cha:meth}. Specifically, we expect that contact will be conditionally different for minority groups relative to the general population. Given the different relationships between minority and majority populations and their governments, and the potentially systematically distinct views of the role the United States plays in both the host country and the international system, we expect that those self-identifying as minorities will have consistently different attitudes toward the US military across host nations. 
%Expand a bit on the Kuwait stuff later.   

%\footnote{We address the case of Kuwait in more detail in our research design, as this is also a case in which the dominant ethnic group, ethnic Kuwaitis, is a numeric minority. Ethnic Kuwaitis, the dominant ethnic group, make up 30.4\% of the population, as compared to 40.3\% Asians, many of whom are immigrant labourers. This makes inference of our variables related to being in a ``minority'' challenging. All other cases consist of countries in which the ethnic majority is also the dominant group, necessitating that Kuwait be considered separately.} 


% We expect, conditionally, that contact with minority members of the population will moderate the positive effects of contact and have a distinctly more negative effect than traditional contact with majority groups.  %change amplify/moderate and negative/positive.

%[ADD a summary of the findings when we have them]
%This chapter's findings thus help to add nuance to policy decisions regarding military engagement with communities around military installations. Briefly, across the countries in our survey sample we find that there is a general pattern of self-identified minority group members holding less positive views of the US than non-minority individuals. However, when we look at the countries individually we find that context matters, and the relationships between minority self-identification and views of the US vary across countries. In some cases minority groups appear to hold more favorable views of the US, while in other cases minority groups hold less positive views of those same actors. In still other cases, we find that that is a stronger tendency towards neutral views among members of minority groups.

%Taken together our quantitative and qualitative findings demonstrate the importance of context-specific factors related to individual host societies, and highlight the importance of reaching out to minority groups who will often perceive the United States as supporting a system that discriminates against or represses them. It also shows that generally positive perceptions of the US military in the host country may obscure systematic differences across different racial, ethnic, or religious groups. Because minority groups tend to hold less political power in their home countries, it is up to the US and its military to create positive relations with these communities and in some cases intercede on their behalf when dealing with host governments. Particularly in cases where the US messaging system toward a host country is run at the national level, through embassy staff, or through the national host government, information systems related to the US presence may not be adequately reaching minority populations. This may both reduce the effect of messaging about the purpose and opportunities associated with the US presence and also limit the potential for fruitful interpersonal connections and economic relationships with minority populations that exist with the rest of the host country.

%To test our hypothesis, we conducted a survey in 14 different countries with approximately 1,000 respondents in each country. Using a multilevel fixed-effects logistic regression, we find that minority population support for US troops to be less than majoritarian populations. However, this effect does not hold when we ask about minority perceptions about US people or the government. Such findings for research on United States Foreign Policy, Human Rights and Democracy promotion, and for policymaking directly. US decision-making in the security realm affects US objectives in other areas and may undermine efforts towards liberal institution building. 