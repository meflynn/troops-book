
\section*{Ethnic Minorities and the US Military}

Research on the relationship between the US military and local minority populations is relatively thin, especially with regards to quantitative, large-sample research. The work that has been done in this area largely consists of episodes in which minority populations have played a key role in American combat operations somewhere in the world, theories on how minority populations can use third-party support to accomplish their goals, or case study research on specific episodes in US-host state relations. The specific work on the US military's treatment of minorities in the context of basing focuses on individual cases, such as those previously mentioned in South Korea and Saudi Arabia. 

Katherine McCaffrey has studied this issue on the island of Vieques, Puerto Rico, where until recently the US military maintained a bombing range \cite{Mccaffrey2002}. McCaffrey states in her work that ``bases are frequently established on the political margins of national territory, on lands occupied by ethnic or cultural minorities or otherwise disadvantaged populations'' (p. 9-10). As examples, she points to the US military bases in the Philippines that were located on land reserved for indigenous populations, along with Okinawa. She describes how the local minority populations often had to comb through military trash to survive after their governments evicted them from their land. While these are certainly powerful examples, to our knowledge there are no systematic and cross-national studies that test whether the dynamics illustrated by these examples are consistent and widespread across contexts. Taking it a step further, we ask the question of whether these examples form a representative sample of how minority populations view an American military presence, or whether these stark cases of mistreatment are the exception to the rule of generally positive relationships. 

Counter to these negative precedents, the United States also has a long history of intervening in civil wars and on the side of minority populations around the world. Particularly since World War II and the realization of the horrors of the Holocaust, there has been a growing consensus within the American foreign policy community that the United States has the ability and, therefore, the responsibility to protect populations around the world from extreme abuses by their governments. This idea culminated in the ``Responsibility to Protect'' (R2P) doctrine, endorsed by the United States and United Nations in 2005, after American interventions in the Balkans, along with the establishment of no-fly zones to protect Kurdish populations in Northern Iraq and the protection of rebel groups from government forces in Libya. While those episodes were largely seen as successfully protecting minority groups, the failure to intervene in Rwanda and the resulting genocide also stood as a strong counterexample of what could happen in the absence of American military involvement. 

If we extrapolate from these interactions that take place in the context of conflict, we might conclude that minority ethnic groups will be likely to have positive perceptions of a peacetime US military presence in their home country.  An initial expectation might be that minority groups are likely to view the US positively because the US military presence will make the host government less likely to repress them. If the US actively promotes human rights abroad, and is active in naming and shaming human rights violators and pressuring them to change their repressive behaviors, then minority groups, which majority governments often try to repressed, should be glad to have the US military present to act as a restraining influence on the host government. After all, previous work finds a positive correlation between a US military presence and respect for physical integrity rights in the host country \cite{bell2017}. The presence of US forces could act as an implicit threat against defection from the US hierarchical system, which endorses respect for human rights. As occurred in Panama in 1989, these troops can be used against the host government if it is seen as acting against the wishes of the United States. This type of implicit threat can keep host state behavior within the bounds of what is considered acceptable by norms established by the United States, as the leader of a global hierarchical system \cite{Towns2012}. 

%The United States can be a positive influence on central governments in their treatment of minorities.

Yet, this dynamic may not play out in practice in countries in which the United States has a large military presence. The decrease in human rights violations that accompanies a US military presence only occurs in states that are not considered strategically important to the United States \cite{bell2017}. In fact, \citeasnoun{StraversElKurd2018} find that a US military presence in autocratic states that are also strategically important correlates with increased autocratization. A good example of this is in the aforementioned case of Bahrain during the Arab Spring protests, when the presence of American military forces did not, in fact, restrain the government from violating the human rights of Shia protesters. With these theories and types of cases in mind, minorities may not only have their everyday lives disrupted by the US military presence, but may be made to feel less secure through the US support of a government that engages in discrimination against them. Through the need to stay in the good graces of a government led by the more powerful majority in order to maintain the basing apparatus, the United States often turns a blind eye to discrimination in host countries and in some cases contributes to it. This process may then result in a systematically more negative view of the US military presence among minority populations. 

First, we argue that the US military presence will be more disruptive for minority members' everyday lives than the rest of the population. While the host country gains economic, security, and political benefits from the US military presence, it can also face domestic political costs for it. In particular, military installations that are located close to major cities and population centers are most likely to encounter opposition. These installations are more likely to serve as a constant reminder to the population of the host's subordinate role relative to the United States, and of the United States' ``imperialism'' \cite{cooley2008}. As we note in Chapter \ref{cha:protest}, cities are also more likely to facilitate anti-base activists' coordination capabilities, with more access to transnational anti-basing organizations and demographic groups that are predisposed to protest, such as students. This is in addition to the fact that coordination is easier in cities, which helps reduce collective action problems involved in protest. Thus, host states will often choose to locate American bases in less contentious, more remote locations \cite{cooley2008}.

The case of the US Thule Air Base in northern Greenland, established in 1951 and expanded to add missile defense in 2004, is an example of this type of dynamic. Thule's strategic location close to the Arctic is of course the major part of its appeal to the Americans as a site for missile defense, yet Greenland's position within the Danish government also makes it a politically advantageous location from Denmark's perspective. Greenland was a Danish colony until 1953, when it became a Danish county, with representation in the Danish Parliament. Though Greenland has expanded its self-rule in 2008 and 2009, there are still many aspects of Greenlandic policy that remain under Danish control \cite{Dragsdahl2005}. 

Ethnically, Greenland's population of 57,000 is 89\% Greenlandic Inughuit (Inuit). This contrasts with Denmark's population of which 83\% are ethnically Danish of Nordic European ancestry. Thule Air Base has had a disproportionately negative effect on the Inughuit people's livelihood, while benefiting the general Danish population. The Air Base was established in 1951 in the town of Thule and, when the military added anti-aircraft guns in 1953, the Danish government forcibly relocated the Inughuit people of Thule to the nearby town of Qaanaq \cite{Spiermann2004}. In 2004, the United States added missile defense systems to the base, increasing concerns by the Inughuit population of being targeted by nuclear weapons. This added to existing grievances of the base harming the environment through toxic waste and affecting the hunting fields traditionally used by the Inughuit \cite{Dragsdahl2005}.

In May of 2004, the Inughuit group Hingitaq 53 (meaning ``The Expelled of 53'') brought a case to the European Court of Human Rights asking for the authority to return to their original home in Thule. While the court did not grant them the ability to return, it did order the Danish Prime Minister to compensate the Thule Tribe \cite{Spiermann2004}. Meanwhile, Denmark, knowing the strategic importance of Thule Air Base, has been able to use the base as a bargaining chip with the United States. As a NATO member, Denmark often uses the base as evidence of its contribution to the alliance \cite{Dragsdahl2005}. Thus, the benefits from the base distribute to the Danish population generally, while the costs are borne disproportionately by an ethnic minority that lacks political power. This dynamic has continued up to the present day with the President of the United States publicly proposing that the United States purchase Greenland. Senator Tom Cotton, a close administration ally, discussed the idea with the Danish ambassador to the United States, as well \cite{hart2019,wu2019}. While the Danish government's response was swift and dismissive, this series of events took place without the United States ever involving the local population in the decision-making process, which highlights the degree to which minority populations can often be sidelined. 

%%This sounds more like actually emphasizing Greenlandic autonomy, which is not really the point we're making, so I'm commenting it out for now.
%Even the Prime Minister of Denmark insinuated as much about the American approach, saying ``Greenland is not Danish, Greenland is Greenlandic \cite{jorgensen2019}.'' One of the two Greenlandic representatives in the Danish Parliament supported this sentiment by commenting that ``Greenland is not a commodity which can just be sold'' \cite{thelocal}. 

%and ``I'd be concerned about the type of society we'd have if Greenland becomes American rather than Danish." 

This example highlights the role of host governments in basing access and its relationship to minority populations. When host governments negotiate with the United States over base access, they are doing so on two-levels; directly with the United States government, but also needing to gain enough support from their selectorate at home \cite{Putnam1988,demesquita2005,cooley2008}. They thus need to not only obtain benefits from the United States, but also secure the support of key political players at home. Leaders can do this by maximizing the net benefits received by the leader's winning coalitions (maximizing their gains while minimizing their costs) \cite{Mesquitaetal2005}. In determining where to place US military facilities, a common solution is to have them in remote locations that are often homes to politically powerless ethnic minorities. This allows the dominant populations in the country to receive the benefits of a US military presence without experiencing the full weight of its negative consequences.

Okinawa, exemplifies another asymmetry between the collective benefit of national security versus the target costs of hosting bases in territory with a concentrated minority group. The movement against bases in Okinawa started in the Cold War and negative interactions between troops and civilians build support for protests and national action against the heavy Marine presence in the region. In classic Olsonian collective action problems, concentrating the costs onto a smaller section of the population makes collective action against the presence of bases more likely than a strategy of diffusing bases nationally and, thereby, diffusing the costs of hosting bases in Japan \cite{Olson1965}. However, as we will discuss in further detail in the chapter on protest, this also potentially limits how cross-cutting protest demographics are, which reduces the likelihood of their long-term success \cite{Yeo2011}.

Second, the US has a troubled history with domestic ethnic relations and government respect for rights among minority groups. While domestic issues do not always translate into international behavior, the United States has exported its domestic turmoil to other countries it bases in. US President Harry Truman formally integrated the armed forces in 1948, yet this did not prevent soldiers' ideas about segregation from being perpetuated in other countries.\footnote{The various branches up through World War II had differing policies on racial segregation. Truman's executive order 9981 uniformly removed segregation across the branches.}  Moon \citeyear{Moon1997} goes into detail about how US racial beliefs conditioned how South Korean business establishments treated white and Black soldiers differently in the post-Korean War era. Businesses would exclusively serve white soldiers, play music more specifically targeting white clientele, and create conditions that were generally hostile to Black soldiers. White soldiers would reinforce these conditions both in how they patronized establishments as well as in high-profile cases of soldier behavior amplifying such messages. In one particular case, a white soldier murdered a sex worker after he saw her working with both white and Black clientele \cite{Moon1997}. 

The US propagation of racial attitudes has a direct negative impact on US operations and the de facto segregation encouraged in the 1950s and 1960s required decades of dedicated work to begin to reverse. These attitudes were not simply confined to how US personnel and host populations related to each other. Instead, they became pervasive in host societies and oftentimes altered majority relations with minority populations, providing a template for how racial and ethnic discrimination can be conducted. As such, individual soldier attitudes, even if they do not reflect government policy (but especially when they do), can do further damage to minority perceptions about the United States' mission in their country by impacting not only their perceptions of US and US military attitudes toward minorities, but also affecting their place within their own societies. 

Third, by virtue of the military presence in a host state, the US government and the US military have a preexisting, institutional, and strategic relationship with the host state. Given the basing relationship, the United States and the central government in host states are already on friendly terms. The presence of US forces also signals that the location or country is strategically valuable to the United States and that maintaining basing access will likely be a priority of US policy \cite{StraversElKurd2018}. These dynamics show that the starting point in any tension between substate actors and the central government will see bias toward the central government from the United States. While minority populations repressed by the host government could certainly draw US support under certain circumstances, it is key to note that the United States is not a neutral party. 

When the US places troops in a foreign country, it is making an implicit (or explicit) commitment to the security of that country. Whether US troops are placed there to directly defend the host country or serve as a trip-wire mechanism that would trigger a US intervention if the host country were to be attacked and US troops threatened, troops stationed consensually abroad protect the host country from external threats \cite{Schelling1966}.  Though in most cases US troops are not there to protect the host country from internal threats, the very fact that US troops are providing security from external threats means that the host country can redirect its own resources from protecting against external threats to protecting against internal threats \cite{machainandmorgan2013,allenetal2016,allenetal2017}.

In addition, a US military presence often comes with tangible material benefits for the government. For instance, a basing relationship with the United States routinely brings increased bilateral military training exercises that increase the coercive capability of the central government \cite{Ruby2010}. The United States also tends to provide more access to foreign military sales for countries that host US military forces, along with sharing existing technology that allows for the growth of a domestic arms industry. These kinds of training, arms transfer, and technology benefits also empower the central government in relation to the rest of the country and potentially make challenges from substate actors easier to manage. 

If the host country views minority groups as an internal threat (something that is common among states with significant minority populations), then it is likely to use repression as a way to protect itself from the perceived internal threat \cite{Regan2005,Jakobsen2009,Brathwaite2014,Hendrix2019}.  If the central government has more resources available to it (because it is now spending less on external security or because the US is directly providing it with resources), then it can devote more of those resources to repression.  A basing relationship with the United States can then have the effect of materially enhancing the power of the central government in relation to substate actors and therefore make challenges to the central government less likely. Thus, a US military presence is increasing the resources available for the repression of minorities.  This will naturally condition minority groups to be less supportive of a US military presence that is not only legitimizing the central government, but also allowing for the expansion of its repressive power.

Furthermore, the mere presence of US military forces on host territory provides legitimacy to the central government from the most powerful country in the international system. Lending legitimacy to central governments can have profound effects that further empower the central government by backing the idea that the government is the rightful power in the country and dissuade challenges to its authority.  For example, the US placing military forces in Spain through the 1953 Madrid Pact legitimized General Francisco Franco's rule in Spain at a time when the international community had generally ostracized him \cite{cooley2008}. The presence of US forces also gives a sense that there is a significant external security concern for the country, which may be necessary to defend against before challenging the central government directly. This is particularly important in cases where a minority group feels that a central government is mistreating them. Other groups within the country and even within the minority group itself will see the legitimacy and authority conferred on the host government via the presence of US military forces and be less likely to challenge the central government's authority. 

By enhancing the power of the central government, the US basing apparatus may further centralize power in host states. The United States deals mainly with the central governments in host state relationships, not regional or local governments. For example, when determining where within a country to send humanitarian deployments, the US consults first with the central government, which presents a series of options \cite{Flynn2018}. By dealing directly with it, the US not only enhances the power of the central government in relation to minority groups, but it also enhances the power of the central government in relation to regional or local governments. Decentralization reduces ethnic conflict and secessionism by increasing opportunities for minority groups to participate in government \cite{Brancati2006}. By empowering the central government both materially and in its legitimacy, the presence of US forces may inadvertently centralize power in a way that disempowers minority groups from effectively engaging in politics. 

Even in democratic countries, minority groups may see themselves at odds with the majority groups that control the government and view a US military presence as supporting the majority population rather than the whole population equally. They may not always be wrong to assume this. While the United States may not be supportive of repression and discrimination against ethnic minorities, US military deployments' efforts to build goodwill are most often focused on those members of society whose support is needed to maintain the presence. This usually is the ethnic majority, which controls the government and to whom the government answers.

Many of these dynamics came to light when we visited the Clay Kaserne Army Post in Wiesbaden Germany, where we heard a lot about outreach efforts targeted at local government officials. The base's Government Relations Officer noted that in less than a year since she had taken that position, she had already met with every mayor of each town surrounding the base. The mayors all get permanent passes that allow them to drive onto the base without an escort. The Government Relations Officer also offers tours of the base for local council members, as well as collaborating with the police and fire departments in neighboring towns. She noted the importance of cultivating relationships with local governments and making the relationship feel less transactional for them. A gregarious and extroverted individual, she knew that she was well-suited for the job: ``You have to stroke and smooch them,'' she said to us in a jovial tone, ``and I am a good smoocher'' \cite{kaserneone20190725}.   

Given the very different views towards ethnic minorities and majorities that we had seen in Lakenheath, we were particularly eager to hear about the base's relations with and outreach efforts towards refugee and immigrant communities in Germany, which had dramatically increased in size since the country began accepting large numbers of refugees from the wars in Syria and Iraq. The topic actually came up on its own when we asked the Government Relations Officer about locals' views of economic benefits from the base. She told us that while there had been a trend of younger German generations (who had not lived through the Marshall Plan) becoming increasingly frustrated with American service members for not learning German, the trend was somewhat reversing with the recent influx of refugees. She noted that Germany was becoming more like the United States and experiencing ``crime, rape, and pedophilia'' \cite{kaserneone20190725}. 


Research on in-group/out-group dynamics shows that crises and the introductions of an outside threat can redefine in-groups and out-groups \cite{Coser1998,Levy1989,Simmel2010}. This dynamic was clearly at play in Germany. The Government Relations Officer put it plainly, ``Before the refugees, Americans were seen as outsiders.'' As an illustration, she noted that, as refugee populations have increased, ``You don't see as much `US Go Home,' but instead `Country X go home' '' (with ``Country X'' referring to the home countries of refugees) \cite{kaserneone20190725}. There is thus a redefinition of the outgroup occurring with the native Germans viewing the Americans as part of their in-group and perceiving refugees and immigrants as outgroup members.

From the American perspective, this dynamic is also at play. While the Americans are unlikely to feel threatened by the influx of refugees in the way that the Germans are, they also know that policy concessions will come from those Germans who control government institutions, not the politically powerless refugees or other ethnic minorities. We specifically asked if the base did any outreach directed at the civil society groups of different ethnic populations, similarly to how it reaches out to the mayors and council members of surrounding towns. The government relations noted that such groups ``probably exist, but I don't meet with them.'' Regarding the Turkish community, she noted, ``They are not my priority to reach out to. Our primary target is the city and mayors'' \cite{kaserneone20190725}.

As far as the refugees go, she anecdotally confirmed that they are less likely to feel positively towards the United States than other Germans: ``The refugees that come from the countries that we are at war with, they don't like us.'' Interactions between refugees US service members are thus more likely to be negative ones. She gave the example of young people going to clubs, getting drunk, and fights breaking out between US service members and ``a foreign national'' (meaning a non-German) after people from countries that the US is in conflict with say things like ``You killed my parents, you bombed my town.'' The alcohol consumption and the Americans being ``somewhat arrogant'' and patriotic exacerbates the confrontations. 

Previously we have shown that direct and indirect contact with the US military in a non-combat deployment setting can lead to both a positive and a negative effect on perceptions of US actors (with the positive effect generally being larger than the negative one).  It is thus of interest to us to understand the determinants of the direction of the effect. As noted by \citeasnoun{cooley2008}, a US military presence in a host country can mean different things to different actors within society. In this case, we believe that an ethnic minority identification will be one of the factors that make it more likely to observe the negative, instead of positive, effect of interactions on perceptions of US actors for the theoretical reasons described above, along with supporting testimony from local populations and US officials. 

Given both direct and indirect mechanisms, we draw the following hypothesis:

\begin{hyp}
Minority populations in states that host US military forces will be more likely to have negative perceptions of US military forces, the United States government, and the United States people. 
\end{hyp}

% I don't think we need this. If we can reject the null in a positive direction, we can then also reject the first hypothesis. - AS
%The start of this section noted that there is an outstanding belief that the United States is a force for minority empowerment and protection globally. As such, we allow for the opposing hypotheses as well. Theoretically, we are less convinced by the arguments supporting it, but include it for completeness of the theoretical record:

%\begin{hyp}
%	Minority populations in states that host US military forces will be more likely to have positive perceptions of US military forces, the United States government, and the United States people.
%\end{hyp}

Combining our research in chapter \ref{cha:meth} about the role of contact, interactions with minority members of the community is likely to have a uniquely different effect on perceptions alone. As such, while we include contact in the models, we also include an interactive effect between minority respondents that report contact. Given that that negative contact with troops generates movements against the US presence in places like Okinawa, we expect that this type of contact will amplify the negative view that minority populations have of the US presence and make it even more negative.

\begin{hyp}
	Minority populations in states that host US military forces will be more likely to have more negative perceptions of US military forces, the United States government, and the United States people as a result of interacting with the US military. 
\end{hyp}

%Same thing here. I don't think we need the reverse hypothesis. Can deduce it from what we find. - AS
% However, both the positive view of the United States and the social theory we express in contact, it is possible that contact has a mitigating effect on people's perception. As such, we test whether contact reduces the expected negative perception. 

%\begin{hyp}
%	Minority populations in states that host US military forces will be more likely to have less negative perceptions of US military forces, the United States government, and the United States people as a result of interacting with the US military. 
%\end{hyp}

%NOTE: Do we think this effect will carry over to perceptions of the US government and US people? Or is it just about the US military being there and disrupting their lives? Maybe the direct effect of having the military disrupting your day to day life affects your views of the military but not the other two, whereas the indirect effect of having the US support the oppressive state also affects views of the US govt and people? Is there a way we can think of to distinguish between the two when testing? Maybe look at the base locations? If you have a base in your subnational unit you're more likely to face everyday disruptions?

 %add additional hypotheses here.



















