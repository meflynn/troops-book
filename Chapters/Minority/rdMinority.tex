\section*{Research Methodology}
%\ref{cha:theory} and \ref{cha:meth}
Our research design follows the basic approach we discuss in chapters 2 and 3, but here we shift the focus to thinking more specifically about how minority populations within each country view various US actors. We begin by estimating a series of multilevel Bayesian categorical logit models with varying intercepts by country, which allow us to consider the possibility of different dynamics in each country. The analysis in this section proceeds in three stages. First, we estimate a set of models using a single predictor variable---minority self-identification. Minority status is potentially determined by factors unrelated to other individual-level characteristics that we include in our data (meaning that minority status is \textit{exogenous} to other individual characteristics). Accordingly, it can be informative to begin by looking simply at basic differences between individuals who self-identify as belonging to some racial, ethnic, or religious minority, and those who do not. 

Second, because minority status likely correlates with a number of other individual-level and group-level traits, we estimate another set of models where we adjust for some of these other considerations. For example, members of minority groups may have less access to education, lower incomes, or may be geographically concentrated in areas that are more vulnerable to the negative externalities of US basing, like noise, crime, or pollution. The aim here is to better understand how the conditions of self-identified minorities differ from other individuals with respect to attitudes towards US basing and troops, and what additional information an individual's minority status conveys once we've adjusted for these other considerations. 

Third, we estimate these same fuller model specifications again, but this time allow the effect of the minority self-identification variable to vary across countries (by allowing its coefficient to vary across countries). This will provide us with some insight into whether the differences between self-identified minorities and non-minorities are similar, or if they vary, across countries. As we discuss above, it is possible that relations between US service members and minority populations are dependent upon various contextual factors. In countries where minority populations disproportionately experience the negative effects of US deployments, we might expect to see more negative attitudes among minority groups towards the US military presence in their country. However, if some countries are home to minority groups who have greater equality in terms of income, education, and geographic mobility, we might expect more positive attitudes among those minority populations, or even no discernible difference from the rest of the population with respect to their views of US military personnel and other US actors. 

\subsection*{Outcome Variable}

As we outline in Chapter \ref{cha:meth}, we have three outcome variables of interest: Individuals' views of the United States military, the United States government, and the United States people. We collapse the survey responses down into four categories to estimate our models: 1) Positive, 2) Negative, 3) Neutral, and 4) Don't know/Decline to answer. Our primary focus here is on how minority self-identification affects individual attitudes about the US military presence  in each country. However, we are also interested in how these relationships compare across different groups, and so we run two additional sets of models using views of the US government and US people as outcome variables. Using these additional outcomes will help us to determine how general the effects of the key predictor variables are and will help us to better understand the contours of attitudes towards the US in general. 

\subsection*{Predictor Variables}

Our main predictor variable is the respondent's self-identification as a member of a minority group within the referent country. Our survey asked the question, ``Do you identify as a racial, ethnic, or religious minority?'' Respondents could answer ``Yes,'' ``No,'' or ``Don't Know/Decline to Answer.'' Out of our full sample of 41,545 respondents, 22\% percent responded ``Yes'' to identifying as a minority.  

We emphasize two vital details here. First, readers should note that this includes racial, ethnic, and religious groups. Accordingly, this question casts a broad net when considering an individual's status as a minority group. Second, this represents the respondent's self-assessment of their status within a given country. The implications of this are important in this context---the rates of minority self-identification in our survey data may differ from various official categorization schemes as embodied in national census counts. For example, in our survey data 16.2\% of Japanese respondents self-identify as a minority of the type mentioned above. At first glance this seems at odds with common stereotypes of Japanese homogeneity. However, some authors have discussed the idea that such ideas are often based on oversimplifications that mask variation in group identities \cite{Johnson2019}. While a fuller discussion of this topic is beyond the scope of this book, readers should keep this point in mind as they proceed through the rest of the chapter. 

% Check this. We probably want to add more details here about demographics as we get them.
Table \ref{tab:minoritypercent} shows the percentage of individuals who self-identified as belonging to a minority group in each country across the three years of our survey. Most countries in our sample tend to cluster around a 10--20\% minority self-identification rate. Germany is considerably lower across all three years of our survey. In contrast, the Philippines and Kuwait are consistently among the highest. The Philippines consistently returns 70--84\% minority self-identification rates. Kuwait exhibits by far the most variation in our sample, with a low of 30\% in 2018 and a high of 98\% in 2020. We expect this is the result of a relatively large population of migrant workers in Kuwait and a bias in the survey firm's ability to reach more affluent Arab Kuwaiti citizens as respondents. Notably, this also highlights one of the benefits to using the multilevel modeling framework---outliers like Kuwait are less likely to exert considerable influence on the general results, particularly when we allow effects to vary across countries. That said, we are still sensitive to the possibility that the data collection process for Kuwait is particularly skewed, and so we run a series of robustness checks to further assess its influence on the results.


\input{../Tables/Chapter-Minority/table-minority-percent.tex}

%%%NOTE check the sentence below, it said no before, but i think it needed to say yes 

In general, we expect that individuals that respond ``yes'' to the question of whether they identify as a minority will be more likely to have a negative view of the United States military. Beyond that, given the theoretical interest and empirical support for ``contact'' thus far, we expect there to be an interactive effect between contact and self-reported minority status. We include the contact variable (see chapters \ref{cha:theory} and \ref{cha:meth}) as a constituent variable as well as a multiplicative interaction term. Jointly, these variables will amplify or dampen the effects of the constituent variables on the dependent variable (views on various actors). Likewise, we maintain the previous control variables discussed in previous chapters. 

%add more if we add anything specific. 




  






