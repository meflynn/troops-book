\section*{Results}

First, we discuss the results for the population-level effects from our most basic model that includes only the minority self-identification variable, shown in Table \ref{tab:minoritybase} in the Appendix. These basic models give use a first look at how minority compare to non-minority populations without adjusting for other factors. To ease interpretation of the results throughout this chapter, we include figures showing the predicted coefficient values. Figure \ref{fig:minoritycoefbase} shows the results of this basic model. This plots the predicted coefficient of minority self-identification on attitudes towards each of the three groups. Each coefficient tells us, given the data we have, how strongly a variable correlates with the outcome, and whether that correlation is positive or negative. For each coefficient our model then generates a range of possible values that are more or less likely. Following a typical bell curve, values closer to the middle are more likely, while values at the edges are less likely The three panels each represent the effects associated with the three outcome variables, and each row represents the effect on the possible responses (e.g. a positive response, a negative response, or a don't know/decline response). Each estimate that we depict shows the median point estimate and the 90\% and 95\% high posterior density credible intervals from the posterior distribution\footnote{The posterior predictive distribution refers to the probability of an individual giving a particular response after we have taken other relevant information into account. See Chapter \ref{cha:meth} for a more detailed discussion of this statistical method.}. 

In these basic models we see that, compared to non-minority individuals, those identifying as belonging to a minority group tend to have a higher probability of expressing a positive view of all three US groups. Alternatively, self-identified minorities have a lower probability of expressing a negative attitude of US troops in the host country, and also the US government. However, there is some evidence that they also have a higher likelihood of expressing a negative view of the US people---81\% of the probability mass of this coefficient's distribution falls above 0 (a zero coefficient is essentially like a 50/50 coin flip, in terms of probability). Finally, minority respondents tend to have a lower probability of responding with ``Don't know/Decline to answer'' compared to non-minorities. This holds except for the question on views of the US people in general.

\begin{figure}[t]
	\includegraphics[scale=0.8]{../Figures/Chapter-Minority/fig-coefplot-base.png}
	\caption{Coefficient estimates for minority self-identification and views of US actors. Credible intervals show the 50\%, 80\%, and 95\% highest posterior density intervals around the median point estimate. }
	\label{fig:minoritycoefbase}
\end{figure}


As we note above, these results present an initial look at exploring how minority groups view US military deployments and other US actors and groups. Minority status often correlates with a range of other important characteristics that are also likely to influence views and personal experiences. Once we adjust for these other factors, we may find that minority experience itself yields a different picture of individuals' attitudes. The models presented in Table \ref{tab:minorityfull} in the appendix replicates the first model we estimate, but this time we include several of these other individual-level and group-level variables. As we do above, Figure \ref{fig:minoritycoeffull} plots the coefficient for the minority self-identification variable along with a few other demographic variables to ease interpretation and comparisons across models. 


\begin{figure}[t]
	\includegraphics[scale=0.8]{../Figures/Chapter-Minority/fig-coefplot-full.controls.png}
	\caption{Coefficient estimates for minority self-identification and views of US actors. Models include various individual demographic and attitudinal variables. Credible intervals show the 50\%, 80\%, and 95\% highest posterior density intervals around the median point estimate.}
	\label{fig:minoritycoeffull}
\end{figure}



There are some important changes when we include these other variables in our models. The bottom row in Figure \ref{fig:minoritycoeffull} shows the coefficients on the minority variable, with the different colors representing the three outcomes (e.g. positive, negative, or don't know). Across all three models we see that after adjusting for these other variables individuals who self-identify as belonging to a minority group are less likely to report a positive view of US military personnel deployed to their country, the US government, and the US people in general. They are also less likely to report a \textit{negative} view of the US government. Overall the results suggest that minority populations are less likely than non-minorities to see US actors in a positive light, but this does not necessarily translate into more negative attitudes. Across all three reference groups it appears that the lower likelihood of positive views is more offset by an increase in the propensity to have neutral attitudes. 


Comparing the minority variable coefficients with coefficients for the other variables is informative. First, looking at the age variables we can see that older age cohorts tend to have a higher likelihood of expressing a positive view of all three US groups as compared to younger age cohorts, and a lower probability of responding with ``I don't know''. Older age cohorts also appear to be less likely to express a negative view of US actors as compared to younger groups.  The reference category here is the 18--24 age cohort, meaning that we should evaluate all of the the coefficients with respect to that baseline. However, we can still compare the coefficient values of the other groups but need to be cautious where there is significant overlap between the posterior distributions. 

Individuals with more education appear to be more likely to express both positive and negative views of US actors. However, there is no clear indication that more education correlates with larger differences in negative versus positive views. As we discuss in the Chapter \ref{cha:theory}, this likely reflects a similar process whereby individuals with more education simply have more informed opinions, one way or the other. 

%fix references to the appendix

Regarding gender, individuals identifying as female tend to be less likely to express positive views as compared to those who identify as male. Female respondents also appear more likely than male respondents to give a ``Don't know/Decline to Answer'' response when asked their views of these three groups.

Finally, when comparing across income quintiles we find some interesting patterns. First, the baseline category here is the lowest income grouping---the 0--20\% quintile. Compared to this group, there is some indication that higher income groups tend to have slightly less favorable views of US military personnel, but these differences are very small. We find a similar pattern when looking at views of the US government, where only the highest income group shows a slightly positive coefficient value. When looking at views of the US people, groups above the 40\% income group appear to have slightly more positive probability of expressing positive views as compared to the bottom two income categories. We find somewhat similar patterns when looking at negative views, though we do see larger differences in the probability of expressing negative views among the highest income groups when looking at views of the US government. Perhaps the most notable feature here is that income appears to correlate quite strongly with a lower probability of respondents saying ``Don't Know/Decline to Answer'' when asked about any of the three US groups. Only 2.7\% of the highest income groups across all 14 countries replied ``Don't Know/Decline to Answer'' when asked about the US military presence in their country. On the other hand, 7.7\% of respondents in the lowest income group gave the same response. Like education, income appears to correlate more strongly with more informed opinions, as opposed to clearly lining up with more positive or more negative attitudes. 

Next, we discuss the results of a model that is identical to the one we developed in Table \ref{tab:minorityfull}, but this time we allow the coefficient on the minority variable to vary across countries. We show the results from this model in Appendix Table \ref{tab:minorityvarying} and we plot the minority variable coefficient in Figure \ref{fig:minoritycoefvarying}. This figure shows that there is considerable variation in the coefficient on the minority variable across countries. Put differently, the differences between minority and non-minority populations with respect to their views of US actors, is highly context dependent. 

Beginning with the model predicting views of US troops, minority self-identification correlates with a higher probability of a positive response in Turkey, Portugal, and Italy. The posterior distributions for the minority coefficient for Spain and South Korea are both largely positive, with a 93\% and 92\% chance of a positive coefficient given the data. Alternatively, we see minority self-identification correlates with a higher probability of a \textit{negative} response in the United Kingdom, the Netherlands, and Belgium, with slightly more mixed evidence in South Korea, the Philippines, and Australia. In the case of South Korea there's an 83\% chance of a positive effect given the data.

%fix references to the appendix

In the model predicting attitudes towards the US government we find that minority self-identification leads to an increase in the probability of a positive response in several countries, including Turkey, Spain, South Korea, and Belgium. We see slightly more limited evidence of this effect in Italy and Germany as well. There is also evidence that minority self-identification leads to a \textit{decrease} in the probability of a respondent expressing a \textit{negative view} of the US government. That is, individuals are more likely to report that they do not know, decline to answer, or a positive view. This is clearest in Spain, Portugal, the Netherlands, and Australia, with largely negative (but slightly more mixed) coefficients in Belgium, Italy, the Netherlands, and Turkey. 

Looking at the models of attitudes towards US people, minority status correlates with a higher probability of a positive response in Turkey and Portugal, with some evidence of a smaller effect in South Korea, Belgium, and Germany. We see some evidence of a higher probability of a negative response across multiple countries, but fairly sizeable portions of the posterior distributions overlap with 0 in many of these cases, indicating slightly more uncertainty over the direction of the effect on negative responses. The clearest cases of a higher probability of negative views among minority communities are in Belgium, Germany, Japan, Kuwait, the Netherlands, the Philippines, South Korea, and the United Kingdom. These countries all yield posterior distributions for the minority coefficient that are greater than or equal to $70\%$.  Further, we again see some evidence of minority self-identification correlating with a lower probability of positive responses, as in the United Kingdom, Spain, the Philippines, the Netherlands, Kuwait, and Australia.

\begin{figure}[t]
	\includegraphics[scale=0.8]{../Figures/Chapter-Minority/fig-coefplot-varying.png}
	\caption{Coefficient estimates for minority self-identification and views of US actors across countries. Models include various individual demographic and attitudinal variables. Credible intervals show the 50\%, 80\%, and 95\% highest posterior density intervals around the median point estimate.}
	\label{fig:minoritycoefvarying}
\end{figure}


Taken together these models highlight a few key points. First, the way that minority communities view various US actors as compared to non-minority communities varies across countries and can be quite a bit more complicated than our basic expectations suggest. This cross-national variation is interesting not just because minority groups in some countries have more favorable attitudes of US actors than minority groups in other countries, but because minority self-identification does not appear to cause a clear linear increase or decrease in views of the United States. In some cases, like in Portugal, we see clearer evidence of minority self-identification leading to more positive views and less negative views than non-minority respondents. In the United Kingdom, we see the exact opposite pattern, with minority respondents more likely to express negative views of US military personnel and less likely to express positive views. However, in South Korea we see some evidence that minority status correlates with increases in \textit{both} positive and negative views, suggesting that minority respondents have stronger opinions about US military personnel and the US people, and fewer individuals expressing neutral views, as compared to non-minority individuals. In still other cases, like Poland, we find evidence that minority respondents are less likely to express positive views of US actors as compared to non-minority respondents, but minority status does not appear to affect negative views in the same way. 


\input{../Tables/Chapter-Minority/table-varying-coefficient-posterior-percent.tex}


Thus far we have only examined the effect of the minority self-identification variable alone. However, we also expect that differences between minority and non-minority groups may also be conditional upon the level of interpersonal contact individuals report. Specifically, we believe there is reason to expect that minority groups will be more likely to hold negative attitudes towards US groups if they report more interactions with US personnel. As we discuss above, it is often the case that minority groups are exposed to more of the negative externalities associated with US basing and military deployments, and so the nature of their interpersonal contacts may also be different, on average, as compared to other groups. 


\begin{sidewaysfigure}[h!p]
	\centering\includegraphics[scale=0.48]{../Figures/Chapter-Minority/figure-posterior-predictive-varying.png}
	\caption{Posterior predictive check for varying effect models. Darkened dots with credible intervals show the mean predicted count of the outcome categories based on 1,000 simulations from the model. Light blue vars show the actual count of each outcome category as observed in the data. Better fitting models should produce simulated values that are closer to the actual observed data.}
	\label{fig:minorityvaryingppcheck}
\end{sidewaysfigure}



We present the results of these models in Figure \ref{fig:minorityinteractionvaryingeffect}.\footnote{We generate these comparisons by holding other variables constant at specific categories and values. Age is set to ``35--45 years'', Income to the ``41$^{st}$--60$^{th}$'' percentile, Gender to ``Female'', Personal Benefits to ``No'', Experience with Crime to ``No'', the Amount of American Influence in the referent state to ``Some'', the Quality of American Influence to ``Neither positive nor negative'', and the remaining variables---education, ideology, the count of US bases, GDP, population, and troop deployment size---to their mean values.} For the sake of simplicity we present only a figure showing the primary comparisons of interest for the positive and negative outcome equations. Here we seek to illustrate two primary comparisons: 1) how minority respondents reporting interpersonal contact with US personnel compare to minority respondents who do not report such contacts, and 2) how minority respondents compare to non-minority respondents among those who report having interpersonal contacts with US military personnel. We do this by generating simulated posterior predictions of the probability of a positive or negative response for each of these groups and then comparing the posterior distributions across different groups. These models are the most flexible with respect to allowing the coefficients to vary across countries and factor groupings, and therefore allow for the most variation in inter-group comparisons. Accordingly, we focus on what we think are some of the more notable points, but cannot claim to provide an exhaustive overview of all possible comparisons.

Most generally, it is worth observing that the combination of interpersonal contact and minority self-identification can, and often does, generate substantively large changes in the predicted probabilities across all three models. We see the most variation in the US troops model, where personal contact with US military personnel among self-identified minority populations tends to produce larger positive changes in the predicted probability of a positive response. In contrast, when contact is held constant at a ``Yes'' value, toggling minority self-identification on tends to produce more mixed changes, with several countries seeing some reduction in the predicted probability of a positive response, but a few cases, like Turkey, Spain, Portugal, and Italy yielding increases in the probability of a positive response. 

Looking at the negative responses we find that there is a mixture in how minority respondents compare to non-minority respondents when both groups report interpersonal contact with US personnel. In some countries like Turkey, Spain, and Italy we find that minority respondents tend to be less likely to give a negative response, whereas in other countries, like Poland, the Philippines, and to a lesser extent the Netherlands and Belgium, we find that minority respondents are slightly more likely to give a negative response. When we focus on minority respondents and compare those reporting contact versus those not reporting contact, we find that the contact group has a lower probability of giving a negative response in several countries, including the Untied Kingdom, Turkey, Spain, and, Belgium. We find smaller differences in Italy and Australia where the comparisons of the posterior distributions do not uniformly show the contact group to be less likely to give a negative response.

\begin{figure}[t!]
	\centering\includegraphics[scale=0.7]{../Figures/Chapter-Minority/fig-contact-minority-interaction.png}
	\caption{Figure shows the change in predicted probability of positive and negative responses according to changes in the personal contact and minority self-identification variables. These changes were derived from varying effects models identical to those shown in Table \ref{tab:minorityvarying}, except we now include an interaction term between minority self-identification in the population-level equation, and allow the effects of these three terms to vary across countries. 50\%, 80\%, and 95\% credible intervals shown around the predicted values.}
	\label{fig:minorityinteractionvaryingeffect}
\end{figure}

When we look at the US Government models we see more consistent increases in the probability of a positive response when we toggle both variables ``on'' while holding the other constant in the ``on'' position across most countries in our data. Poland, the Philippines, and Kuwait are the notable exceptions here. As we have discussed elsewhere, however, Kuwait and the Philippines in particular are significant outliers with respect to the size of the minority populations in our survey data. While this over-representation is not necessarily a problem in and of itself, the extreme differences we observe in the case of Kuwait may reflect the more unique status of the large migrant worker communities rather than ``domestic'' minority communities.

We find slightly more variation when we look at the negative response comparisons. In general we find that, among those reporting interpersonal contacts with US personnel, minority respondents are often less likely to give a negative response as compared to non-minority respondents. These differences are most evident in the United Kingdom, Spain, Portugal, the Netherlands, Germany, and Belgium. Among minority respondents only, we find that those reporting interpersonal contacts with US service personnel are less likely to give a negative response in the United Kingdom, Portugal, and Belgium, with more limited evidence of similar differences in countries like Japan and Germany.

Finally, when we look at the ``US people'' models we find more a more consistent pattern of minority respondents with personal contact reporting a higher probability of a positive response in most countries. We find these differences are more muted in Poland, Japan, and Italy. In general, the differences between minority respondents who report contact and those who do not is largest in Australia, and Belgium, with Turkey, Portugal, and Kuwait all showing changes of approximately 10--15 percentage points in the predicted probability of a positive response. And of those individuals reporting having had personal contact with US military personnel, we find that minority respondents tend to have a lower probability of expressing a positive view of the US people in most countries. The notable exceptions here are Turkey, Portugal, Germany, and Belgium where the differences between these groups are more muted.

The comparisons are decidedly different when we look at the negative response models. To facilitate comparisons across response equations and models we held the range of X-axis values constant across panels in the figure. That is, we are using the same scale for the negative models for comparison even though this compresses the perceivable range of the values for these models.  We do indeed find that there are differences between minority and non-minority respondents who report contacts, as well as among minority respondents who do and do not report interpersonal contacts, but the magnitude of these differences is considerably smaller in this model and equation. For example, we find that minority respondents are generally less likely to give a negative response in Turkey and Portugal, with some more limited differences in Belgium, and to a lesser extent Germany. Among minority respondents, we also find that those reporting interpersonal contacts are less likely to give a negative response in Turkey, Portugal, Belgium, and Australia, and again find more limited differences in the United Kingdom, the Netherlands, Spain, Kuwait, and Germany. Importantly, while we do observe differences, the magnitude of these differences as compared to other models and outcomes is much smaller. In general the differences we see here are in the 3--6 percentage point range. 

Regarding broader model performance, the varying coefficient models generally perform well. Figure \ref{fig:minorityvaryingppcheck} shows the results of posterior predictive checks on the three sets of varying coefficient models shown in Appendix Table \ref{tab:minorityvarying}, broken down by the country groups in each model. These checks function by using the models to generate a number of simulated ``data sets'' for the outcome variable. For example, we use the model to generate $\sim$ 38,000 simulated response values for each respondent in the data, and then we do this iteratively 1,000 times. We can compare relative frequency of these simulated values to the actual frequency of the outcome variable categories as observed in the actual data. In this plot the dark dots represent the mean of 1,000 different simulations and 95\% credible intervals. The lighter blue bars represent the actual count of each category observed in the data for a given country. Overall the models perform well and the simulated counts fall very close to the observed counts for all levels of the outcome across all countries.




\subsection*{A look at Japan}

Japan is well-known for the hotly contested presence of US military facilities, particularly in areas like Okinawa. However, US military facilities exist all throughout the country, and so the relationship between minority groups and US facilities may vary depending on geographic context. For example, while many people associate minority-military interactions with Okinawa it is often the case that there is overlap between minority communities and US military facilities in general. That is, aggregating to the regional or province level and looking for relationships between the location of US facilities and regions with high minority populations can be misleading. 

Figure \ref{fig:japanminoritymap} shows the breakdown of minority respondents across regions within Japan (shaded green), and the distribution of US military bases (blue dots).\footnote{Note that this map aggregates Okinawa into the Kyushu region. In the analysis that follows we break Okinawa out into a separate region given its historical importance. Unfortunately we only have data parsing Okinawa out in this way for two of the three years of our survey.} As the map shows the largest proportion of self-identified minority group members in our survey are located in the Kanto Region of Japan, which includes Tokyo as well as some key US military installations. While our survey is not designed to be geographically representative there is still an important point to be made here---while there is often a correlation between geographic region and minority status, these two variables do not always map neatly on to one another. Even in regions that are not normally synonymous with particular minority groups there may still be large minority populations, and those populations may have very different interactions with US military personnel than individuals who do not belong to a minority group. For example, occupying space close to bases, or spaces that are particularly exposed to the negative externalities of bases.


\begin{figure}[t]
	\centering\includegraphics[trim = 1.75cm 1cm 1cm 1cm, scale=0.85]{../Figures/Chapter-Minority/fig-map-japan-min-location.png}
	\caption{The distribution of Japanese respondents who self-identified as belonging to a minority group and the location of US military bases. The shading intensity represents the share of all minority respondents present in each region.}
	\label{fig:japanminoritymap}
\end{figure}


To examine how minority compare to non-minority groups in Japan we estimate a model identical to the one shown in Appendix Table \ref{tab:minorityvarying}, but this time include only Japan. As in the full cross-national version of this model, we include several predictor variables that may be associated with minority status, and we allow the minority self-identification variable to vary across regions. Figure \ref{fig:coefplot-japan} shows the coefficients for the minority self-identification variables for the Positive, Negative, and Don't know/Decline to answer equations for the three models and reference groups.

Across all three models, we find little difference between minority and non-minority respondents with respect to the probability of a respondent expressing a positive view of US actors. The predicted coefficient value is extremely close to zero and a substantial amount of the posterior distribution falls on either side of 0. Only in the Kanto region is there some limited evidence of minority respondents having a slightly higher probability of expressing a positive view of the US people, specifically. 

We find slightly more evidence that minority respondents differ from non-minority respondents when we look at the negative outcome equations in each model. We also see some evidence that the differences between minority and non-minority respondents are partly conditional upon the region in which respondents reside. Most of the median point predictions for the posterior distributions fall below 0, and in some cases fairly large proportions of those distributions fall below 0 as well. In Kanto, for example, over 96\% of the posterior distribution falls below 0, indicating that minority respondents there are slightly less likely to express a \textit{negative} view of US troops as compared to non-minority respondents, once we adjust for various other individual-level and environmental characteristics. Similarly, approximately 90\% and 92\% of the posterior falls below 0 in the Chuugoku and Hokkaido regions, respectively. Alternatively, in the Kinki region we find that approximately 94\% of the posterior falls \textit{above} 0, indicating that minority populations there are slightly more likely to express a negative view of US military personnel than non-minority groups.


\begin{figure}[t]
	\centering\includegraphics[scale=0.75]{../Figures/Chapter-Minority/fig-coefplot-2-japan.png}
	\caption{Varying coefficient estimates for minority self-identification and views of US actors in Japan. Coefficients vary across regions. 50\%, 80\%, and 95\% credible intervals shown around point estimate. Credible intervals are calculated using the highest density posterior interval.}
	\label{fig:coefplot-japan}
\end{figure}

Shifting to views of the US government the results look very similar to the model predicting views of US military personnel with respect to the differences between minority and non-minority groups in giving positive responses. For all regions the median posterior prediction is close to zero, with substantial portions of the distribution falling on either size of 0. Ultimately this indicates that there is little difference between minority and non-minority groups with respect to positive responses. Shifting to negative views, we get a slightly more homogeneous pattern wherein minority groups generally appear \textit{more} likely than non-minorities to give a negative response when asked their views of the US government. In Kinki we again find the clearest evidence that minority groups tend to be more likely to express a negative view than non-minorities, with approximately 95\% of the posterior distribution falling above 0. However, in Chubu, Kyushu, Shikoku, and Tohoku we find 75\% or more of the posterior falling above 0 in each case.

Finally, looking at the model predicting attitudes of US people we find slightly stronger evidence of a general positive trend, wherein minority respondents are more likely to express a favorable view than non-minority respondents. This is clearest in the Kanto region, where approximately 90\% of the posterior distribution for the coefficient falls above 0. In nearly every other region we find 70\% or more of the posterior distribution falling above 0, with only Kinki and Tohoku having values in the 60\% range. Interestingly, we also find a more consistent pattern indicating that minority respondents are also more likely to express a negative view of the US people. The posterior distribution on the minority coefficient is positive across every region, but in Chubu, Chugoku, and Tohoku we find 90\% or more of the posterior falling above 0. We find slightly more muted patterns in Hokkaido, Kanto, Kinki, Okinawa, and Shikoku, where approximately 70\%--83\% of the posterior falls above 0.

Overall, it is somewhat surprising that we do not find more consistent relationships between minority self-identification and anti-US attitudes, as this is what many historical narratives and anti-base activists would suggest we should expect to find. Some of the large variation (credible intervals) may be due to the relatively small size of the minority sample---with only about 500 Japanese respondents identifying as belonging to a minority group and many regions not having large minority respondent rates, we must allow for the possibility that some of variation we see here is mostly noise rather than genuine variability in effects. One advantage of the multilevel modeling approach is that it allows us to draw information from the general population to help inform estimates in cases where we have fewer observations for particular groups. While this is certainly an advantage compared to other statistical approaches, like using binary fixed effects to capture differences in group-level intercepts, we are cautious given that we expect the differences between minority and non-minority respondents to vary across groups. The shrinkage induced by our approach here may also tend to compress some of the between-group variation we expect to see where we lack more observations.