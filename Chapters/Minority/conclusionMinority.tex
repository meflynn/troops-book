\section*{Conclusions}

When conducting interviews, we spoke with an anti-base activist in Berlin. During the conversation, he made it clear that minority populations, particularly immigrants and refugees, were largely an afterthought in the dynamics of basing: ``They are not playing a key role in the base discussion. The Germans and the US have security concerns related to hiring immigrants and refugees to work on the base. They are afraid to hire immigrants. They have strong security checks and controls. They are afraid to hire someone from al-Qaeda'' \cite{berlinone20190723}. At the beginning of the chapter, we discussed the idea that many minority populations may see US military installations positively through the potential economic opportunity that they provide. However, as both this anti-base activist and the Government Relations Officer at the US installation in Wiesbaden attest, there was little consideration given to local minority populations at all, let alone any concerted efforts undertaken to reach out to such communities. 

%To the extent that they consider minority populations, it is to sideline them from the types of employment otherwise available to local people on the base.  - This statement seems strong, do we have evidence of active sidelining for employment?

The data that we collected from countries around the world offers a glimpse into how these issues present themselves in public opinion. In our most general models, minority groups are less likely to report positive views of US actors than majority groups. When we allow the effect of minority self-identification to vary across countries, however, we find that the effect of minority status on views of the US is more mixed. In many countries, minority groups are less likely to report positive views of US actors than non-minority respondents, but often this simply translates into more neutral views, while in other cases there is a clearer increase in actual negative sentiment. This complexity holds true when we look specifically at the effect of interpersonal contact, where in some countries, self-identified minorities who report interpersonal contact with US forces have more positive views and in others they have more negative views. 


%\ref{cha:meth}
What we have demonstrated in this chapter is that it is important for decision makers to see host nation populations in more complexity. For example, contrary to many popular assumptions, \citeasnoun{Johnson2019} describes the relationships between the Okinawan population and the US military personnel as a deep, multilayered subject, characterized by both affection and nostalgia, but also resentment and anger. The people in host nations do not have uniform experiences of the US military, and they do not have uniform views. As discussed in Chapter \ref{cha:meth}, views can depend on whether individuals have had interpersonal contact with members of the US military, and in this chapter, we have uncovered another complicating factor --- minority status. Minority communities in some cases may be on the receiving end of more negative externalities than majority communities. In such cases, the negative relationship between minorities and the US military is direct. In more indirect relationships like the one described in Germany, minority populations are given very little consideration in the dynamics of military basing. In such circumstances, a more circuitous causal mechanism is likely at work, in which the American relationship with the host central government empowers both the government and the dominant ethnic group. By intertwining itself with the ethnic-majority central government, the United States becomes conflated with groups contributing to minority discrimination and reducing pathways to minority representation through the centralization of power. 

Much as we would expect from public opinion polling in the United States, an individual's identity matters a great deal in how they view the world. This is because an individual's experiences will influence their views. The same is true in other states and researches should consider it in any analysis of how the United States interacts with societies in the international arena. While the US military may see a high degree of support from majority groups in some countries, not all populations have uniform experiences and views of the American military presence. Unique experiences with the US military, the host government, and the majority population in the host country will influence individuals' views. From a policy perspective, this issue is vital  to understand and apply to the US military's relations with local populations in host countries around the world. Even in societies that see high levels of favorable views toward the US military, negative views among a cohesive minority population can cause any number of issues --- from security concerns to land use issues, and threaten the stability of political support for a US presence. In previous chapters, we argue that outreach to surrounding communities is a way in which the US military can improve relations with surrounding communities. Extending these outreach efforts to minority groups, and specifically tailoring to their wants and needs, would be a way to build better relationships with minority populations and promote better relations between them and central governments.  

%outreach recommendation might depend on what the contact variable looks like for minorities. - AS
These results are a call to those engaging in policymaking surrounding US military bases and their relationship to ethnic minorities to more deeply understand the communities that will be engaged through the base-making and base-sustaining processes. Many national governments have vastly different relationships with minority populations within their borders than do others, and many ethnic minority groups within the same country can relate very differently with both their national governments and an outside military force. While some will see an American presence as a sign of potential economic prosperity, others may see it as an unwelcome intrusion. Policymakers must deeply engage in the history and preferences of the individual countries and groups involved in basing decisions to understand whether and how to engage in outreach efforts, so that potential basing relationships between the United States and minority populations can be healthy and productive.