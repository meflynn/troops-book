


%\section*{Introduction}
\vspace*{-0.5cm}
\rule{\linewidth}{0.10pt} \\[-1cm]
{\footnotesize\paragraph{Summary:}  The previous chapter established our theoretical expectations for the subsequent chapters. This chapter examines the survey we have distributed to approximately 42,000 respondents. Additionally, it shows how both economic ties and contact with US military personnel can create positive and negative perceptions of the US military, US government, and the US people. One of our most important findings is that contact, both by individuals and people within those individuals network, is enough to push people to have ``informed'' views of the US military. Those contacts make it less likely for people to say they do not have views about the presence and more likely to express any view as a result. These findings advance our argument about how interpersonal experiences with US military members are one of the most important fronts in the domain of competitive consent.} 
\\[-0.5cm] 
\rule{\linewidth}{0.10pt}

\vspace*{0.5cm}

%anecdote lead needed, something about general perceptions or people's views being tied to something. Carla or Andy will have the best idea as to what fits since they were in all four sets of interviews.

While this experience is a single case and not generalizable to the whole population, it does provide insight that confirms what we are finding in our systematic evaluation of the evidence. The Domain of Competitive Consent, as we develop it throughout this book, has several underlying assumptions. Two of the key assumptions are that the perceptions of host-state civilians matter to regimes. Second, those basing powers can affect those perceptions in a variety of ways. The research in comparative politics and international relations shows support for the first argument. While it is a fascinating study area, it is beyond this project's scope to test that argument more directly. Additionally, we think this will be a feature of the future international system and not a product of the past, so testing whether popular consent manifests in changes in basing will become more relevant in the upcoming decades. 

The second line of argument that basing powers can create and mold people's perceptions in other countries is what we seek to understand within this chapter. Moving beyond the theoretical expectations that we established in the previous chapter, we develop our formal expectations in a series of hypotheses that relate both the role of contact and economic dependence in shaping the views of the US military. To this end, we briefly review the elements of our theory that motivate our argument within this chapter. Subsequently, we discuss in detail the collection of our survey and interview data that underpin our research in this chapter and serve as the foundation for the remaining chapters in this volume. After reviewing the data,  we discuss our estimations and display the findings and inferences we draw from those models for the first time using our complete data set.
