\section*{The Effect of Contact and Economic Reliance}


\begin{figure}[t]
	\centering\scalebox{.85}{\includegraphics{../../Figures/Chapter-Contact/old/figure-american-government-country.pdf}}
	\caption{Evaluations of the American military presence, government, and people by country by year.}
	\label{fig:dvdesc}
\end{figure} %placeholder graph. Probably a 3x3x14 graph with 3 years as rows, 3 actors as columns, and each row will have all 14 countries, so basically a 9x9, but each cell has 14 charts within it.

To begin the discussion of our survey and how both contact and economic reliance condition how people perceive the US military, we review the distributions of our data. These first-level results given some keen insights into how countries differ from each other and how answers to these questions vary over time.	


Figure \label{fig:dvdesc} presents how people have responded to our primary outcomes across all three years. These charts offer some descriptive variation of our questions that we can make some preliminary inferences from. Notably, the American people tend to be more popular (and less disliked) than the American government. The US military presence is somewhere in the middle of these two actors. Some countries are keener on Americans overall than others. Kuwait stands out as overly positive on Americans, with a small percentage of those surveyed holding negative assessments. On the other hand, Germany has far fewer people enthusiastic about American actors and many more that are not supportive. For all countries, the number of positive, neutral, and negative evaluations vary by which actor we asked about. %discuss any annual variations if they appear here. Good talking point.


\begin{figure}[t]
	\centering\scalebox{.80}{\includegraphics{../../Figures/Chapter-Contact/old/aspr-figure-descriptive-iv-distribution.pdf}}
	\caption{Responses to whether people have had contact with, their social network have had contact with, if they economically benefit from, or if their social network economically benefits from the US military presence.}
	\label{fig:ivdesc}
\end{figure} %placeholder graph. 3x4x14?
\FloatBarrier
We created similar charts for the questions we think are causally related to people's perceptions of the US military and other actors. Figure \ref{fig:ivdesc} offers a snapshot of whether people say they have had contact with, their social network has had contact with, if they economically benefit from, or if their social network economically benefits from the US military presence. Generally, the people with contact and personal benefits from the military are in the vast minority within each society. There are some stark outliers. The Philippines sees, relatively to other countries, quite a bit of economic flows and contact with the United States military. Around 25\% of respondents have friends or family that economically benefit from the US presence. Germany is in the top tier of countries that have quite a bit of contact with the US military, and, given the size of the deployment to Germany during this period, that follows what we would expect. However, it is in the lower tier of the respondent percentages with personal or network economic benefits. Japan follows a similar pace where the number of contact reports is far higher than those reporting economic benefits from the presence. We generally expect to see this in every country, but Japan is notable given that the concentration of the military presence is primarily in the Okinawan Prefecture. Year to year, there are not many changes in how people respon to the questions. This is something we expected as the patterns of contact and economic benefits are not likely to change dramatically in one-year to the next. %change based on actual yearly results. Or maybe just combine all the years to one graph as this shouldn't vary much. Make sure to alter the text above.


Our goal is to see how these responses relate to each other, and we can get another snapshot of this in Figure \ref{fig:tileplot}. This figure combines the data we have from the previous figures into a wealth of information. Each rectangle in the figure represents the modal response for a country on their views of the United States military, people, or government. Each column represents whether they had contact with the military, their network had contact with the military, received economic benefits from the military, or if someone in their social network receives economic benefits from the military. We divide each column into whether people report a no or yes response to the four categories, each row of 14 countries represents questions about a different actor. Finally, the shade of the box represents the intensity of the modal category. 

\begin{figure}[t]
	\centering\scalebox{.80}{\includegraphics{../../Figures/Chapter-Contact/old/fig3_apsr-figure-contact-combined.pdf}}
	\caption{Modal response towards reference group (row) by respondent country and type of contact with US military (column headers).}
	\label{fig:tileplot}
\end{figure}%replace with newer figure. If things change, then report down here. 


There are a few interesting trends within this figure. The most disliked actor is the US government, and the US people are the most well-liked. Since our data only comes from 2018-2020, we cannot assess whether the dislike of the US government is a general trend over the last several decades or is a response to the Trump administration. We expect that the Trump administration played some role in the views of the US government as a Pew survey of sixteen different countries shows the United States increasing its favorability among foreign populations by 28 points by the early part of the Biden administration \cite{Wike2021}. While our data cannot make inferences about the change from administration to administration, one noticeable pattern is the change of results from several neutral and negative modal categories for the no responses to nearly universal positive views for those who report contact or economic benefits.  


Moving from the descriptive comparison of our data, we move to our multilevel categorical Bayesian logistic regression, allowing us to see if the patterns we began to notice in the above figures persist when we account for confounding but important factors. The full description of the models we used and the results from those models are in our online appendix. Here, we turn to graphs that we generated from the models. These resulting figures can give both the consistency of the effects and their relative magnitude of the relationships. Figure \ref{fig:coefplot1} displays the coefficients from our models. Each column offers the outcome variable of the US actor we asked about. Each row represents our questions about contact and economic benefits. The rows in the chart are whether they said yes, no, or don't know/decline to answer in response to our questions.
Additionally, we estimate whether they are likely to have a positive, negative, or non-responsive view above the various actors, given their responses to our questions about contact and economic benefits. In interpreting these data, we are most interested in results far from the middle line and whether the coefficient is on the negative or positive side of the line. The number above each line is the number of estimations that fall above or below zero based on whether the average effect is negative or positive.

In the first cell, under the row of Personal Contact and the column of US Troops, we already begin to have some interesting findings. First, those who responded to the question in the affirmative where more likely to have both positive and negative views of the US presence. Additionally, they were far less likely to say that they did not have an opinion at all (don't know/decline to answer). This follows our first set of hypotheses where we expect that contact produces more informed responses to questions about people's views on US actors. Notably, for contact, these views do not transfer to the US government, but we do find a similar effect for US people and the effect appears to be similar to troops. The role of network contact generally under-performs relative to contact. Still, we do see a consistent positive relationship between those that report their social network having contact with US service members and positive views of US troops, government, and people. There is a similar relationship between contact and negative views of the US government. 

Receiving economic benefits from the US military has a more dividing effect. While contact seemed to produce more positive and negative views about the military, those who report receiving economic benefits from the military were less likely to negatively view of the US military, government, and people. Interestingly, this question positively correlates with views of the US government, but we have less evidence for its relationship with the military presence and people. The network effect of economic benefits is stark. There is a clear dividing line in the data that shows those who report their friends and family receiving economic benefits from the US military are more likely to have a positive view of the presence and the people and less likely to report negative views of all three actors. 

In general, we see the effects that we expected to see even when we control for other personal, regional, and state-level characteristics.





\begin{figure}[t]
	\centering  \scalebox{.55}{\includegraphics{../../Figures/Chapter-Contact/old/fig4_apsr-figure-coefplot-cat-bayes-20190924.pdf}}
	\caption{Coefficient plot of the results from our estimations. 90\% and 95\% credible intervals shown. Percentage values show the percent of the overall coefficient distribution that falls above or below 0.}
	\label{fig:coefplot1}
\end{figure}%replace with newer figure. If things change, then report down here. 

The final figure for this chapter, figure \ref{fig:catbayesintercepts}, examines whether there are systematic differences between countries in reporting their views of the US presence and other actors. Panel A shows the population-level intercepts for the three outcome variables and the three response equations associated with each. Panel B shows country-level error for each model/group and response.  Given the distribution of the views across countries in the earlier graphs, we expect that there might be some national-level variation in perceptions. We do find some evidence of that when we specifically include country-level influences into our models. Germany, for example, seems to have a higher than average propensity to have negative views of the US government and people. South Koreans are less likely to have negative views of the US government. Overall, however, the other controls we include in our models, including personal demographics and ideology, seem to account for most of the variation between respondents and their views of US actors. The variation by country, when we consider all other items, is relatively minor. 



\begin{figure}[t]
	\centering\scalebox{.55}{\includegraphics{../../Figures/Chapter-Contact/old/fig5_apsr-figure-intercepts-cat-bayes-20190924.pdf}}
	\caption{Predicted population-level intercepts (Panel A) and predicted country-level error (Panel B) from models shown in our appendices. 90\% and 95\% credible intervals shown.}
	\label{fig:catbayesintercepts}
\end{figure}%replace with newer figure. If things change, then report down here.
