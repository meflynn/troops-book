\section*{Conclusions}

While previous research shows that contact between populations can break down the barriers that create stereotypes, mistrusts, and conflict, very little research has examined this in the context of the relationship between foreign military and a host-state civilian populations. Our results offer a clear picture into how contact and economic relationships can create more positive views even in areas where there are a predisposed bias against US military personnel. Additionally, contact between personnel does affect people's perceptions of actors beyond just the military. People use their interactions with US military personnel to inform their views of how they view the US people as a whole as well as the US government.

Our results are not purely positive. There is a strong undercurrent where people who report increased contact with US military personnel are more likely to have negative views of the US military presence. People who have had direct, negative experiences with US military personnel are more likely to think ill of the United States and its people. While part of the process may be from increased social friction and harmed by general social ills of traffic, environmental, and noise pollution, some subset of these people has experienced direct incidents as well, including vehicular incidents and violence. Some subset of these direct and general issues can gain traction to a broader national audience and become the catalyst for anti-basing movements that seek to expel the United States.

This bifurcation of reactions to the US troop presence has created competing influences for basing policy and the decisions base commanders face. When a new local crisis emerges from an altercation between a service member and a civilian, the active process that generates disdain for the US military becomes readily apparent to decision-makers. The policy response has often been to reduce the points of friction between the military and civilians and to draw service members back to the base so that future points of conflict become less likely. \citeasnoun{Gillem2007} discusses the withdrawal of service member contact with civilians by the branches of the military. While the Navy has had fewer points of contact historically, the development of bases over the last few decades have attempted to limit any points of contact between service members and host-state civilians. However, as Gillem notes, ``Relocating the US military to deserts or to isolated islands (artificial or real) in order to be ``safe'' from terrorism is only one justification for policies of avoidance. These moves are also ways to remove American soldiers from places like American Village'' \citeyear[p. 262]{Gillem2007}. Such policies became more widespread after the 9/11 terrorist attacks to also reduce service member exposure to security risks and kidnapping threats. The immediate and loud negative attention from negative interactions generally dwarf the small, every day, passive effects of interpersonal contact---there are few media stories that gain national or international attention about military and civilian parents both having kids on the same soccer team. 


There is an inherent problem with the strategy of isolating the military from civilian populations. If negative views about the US presence comes from experiencing the problems caused by the base itself, and negative contact experiences are largely incidental to those with negative views, then there becomes an asymmetric feedback problem for the US military. Gillem argues that ``despite widespread media attention focusing on the tragic stories of rapes, deadly accidents, and environmental damage, surveys of local residents near some of these outposts reveal not so much an all-consuming desire for their demise but disgust, above all, with the excessive use of land by American forces'' \citeyear[p. xv]{Gillem2007}. While media attention may provide a short-term catalyst, it is not the cause of the problem but a spurious relationship between negative events and anti-US sentiment. Consequently, a draw down of contact, a policy of service member isolation, means that the negative externalities continue to proliferate in local communities while one of the few paths for building support are kept behind fortress walls. A policy of separation and against inter-community contact limits the United States' ability to maintain long-term bases in the evolving domain of competitive consent.


The everyday behavior of US military personnel living their lives in foreign communities builds the soft power upon which support for the US, its mission, and its strategy for the international arena accumulate. As we move towards increased domains of competitive consent as basing powers compete for new basing sites and to maintain old alliances, the passive spread of stereotype deconstruction and building of goodwill is a vital front that is easy to overlook. Still, base commanders facing increased public backlash are unlikely to weigh the long-term effects of retreating to the proverbial motte versus the short-term gain of decreasing community friction. Retreat is an easy policy with an obvious effect. The harder and costlier strategy involves cultural change, behavioral nudging, and effective monitoring of off-base activities by service members. Additionally, such efforts will never be 100\% effective as service members are human, harmful interactions will happen, and the result will fuel anti-base or anti-American movements. Notably, returning to figure \ref{fig:coefplot1}, it is evident that there is a marked shift for positive views relative to negative views as a result of contact. The economic effects within social networks are even clearer. While hiring local construction workers is part of that story, so are soldiers patronizing local establishments when they are off-duty.

We argued in the introduction that troop deployments had been the historical microfoundations of US power and grand strategy. We still stand by that argument as American personnel overseas have shaped international and domestic politics globally. The US is currently in a military transition where it seeks to rely more heavily on the capital-intensive parts of its force, such as drones, and favor bases as lily pads instead of massive footprints in the middle of a country. However, this trend will not eliminate all major bases as human power is still fundamental to adequate force projection, defense of territory, and actual warfighting. Additionally, longstanding allies continue to petition the United States to maintain its larger deployments in South Korea and Germany. These trends, combined with the research in this chapter, suggest that the people of the armed services are not only the microfoundations of US power of the past but will be the primary assets of competing for consent for the US to maintain its basing network in the future.
