
\section*{Surveys and Strategies}

We seek to accomplish two sections in this section that we draw upon for the rest of the book. First, we deployed a survey in 14 countries over three years to assess how host-state civilians view US military personnel in their country. We discuss the deployment of this survey, the questions we asked respondents, and our choices in designing the survey. The second part of this section is a technical introduction to how we create our statistical models to understand the relationships within our survey. For those that are technically inclined and want to know more details about how we set up our models, our Bayesian priors, or other useful pieces of information, this section is where we go into the most detail about the models. If, however, you would rather avoid discussions of model specification, the second section on ``Estimation Strategy'' may be a safe section to skip. After discussing both the survey and our estimation strategies, we proceed to discuss the results of our model in regards to both contact and economic reliance. 


\subsection*{Survey Design and Implementation}

In 2017, we surveyed the literature on US overseas basing and military deployments and assessed areas that the scholarship had not covered sufficiently at that point in time. In 2017, there were quite a few areas still wide open, with several remaining open today, as the research on the effect of troops deployments was relatively thin. During the Cold War through the first few decades of the post-Cold War era, there were some research and debates about how troop deployments affected deterrence \cite{Schelling1966}, how troops affected local economic and social conditions \cite{Moon1997}, were involved with criminal conduct \cite{Bryant1979}, or how they reflect greater patterns of masculinity and patriarchy in domestic and global orders \cite{Enloe1990}. Other books concerned themselves with debates over bases as the United States transitioned both its force position and some allied states sought to remove the US presence from their shores \cite{calder2007,cooley2008,Yeo2011}. As we discuss in the introduction, the publication of data by \citeasnoun{Kane2004} led to several different works looking at troops as a potential cause of various social, political, and economic outcomes. 


After reviewing the body of deployment knowledge, it is clear that we only have snapshots of how host-state civilians perceive bases in their societies. The best evidence in the literature is when perceptions manifest into discontent and mobilize into anti-basing movements. Often, the best-studied examples are those that succeed and shift national policy.  While movements in the Philippines, South Korea, Germany, and Japan deserve study, by focusing on cases that manifest in widespread movements, there is a bias in looking at those cases to the exclusion of those that do not manifest in actions \cite{Geddes1990}. The technical terms for this focus are selection bias or survivorship bias, making it uncertain whether the inferences we make about these cases are generalizable. For example, do the protest movements represent the whole population, or are they just the most vocal? Is dissatisfaction with basing widespread and the cases that manifest into political campaigns show where such mobilization is feasible? Are political entrepreneurs that capitalize on anti-base sentiment exploiting a wedge issue or leading a movement that represents a quieter majority?

To get a better picture, we aimed to create a sample of countries representing a range of experience in hosting US military personnel on their territory. For the sake of inter-country comparison, we explicitly focused on countries that hosted troops during peacetime operations. The presence of an ongoing external or internal conflict shifts the security narrative to one of immediate or existential concern. It dominates secondary concerns about society, the economy, and the environment. Within each of the countries we selected, we also strived to get a representative sample of individuals with different experiences, backgrounds, and beliefs. Ideally, the people within the country live throughout the sampled country, with some people living near bases and others living far away from a US presence. Likewise, we want people in both kinds of locations that both have and did not have contact with US service members. 

With this set of goals, we identified 14 countries where the United States either had a large historical or contemporary presence, and we had the capacity to survey.\footnote{We identified two survey firms early in our research development process that could reach a large sample of the countries we wanted to survey. The two firms we used for this project were Qualtrics and Schmiedl Marktforschung.} The finalized list of countries includes Australia, Belgium, Germany, Italy, Japan, Kuwait, the Netherlands, the Philippines, Poland, Portugal, South Korea, Spain, Turkey, and the United Kingdom. Through our contracted firms, we distributed surveys to 1,000 people in each country. We used quotas to make sure the sample was nationally representative of age,\footnote{This only includes people over the age of 18. Substantively, adults are our primary demographic since they are the ones most likely to have economic connections to a base and have some ability to influence local or national politics. This makes our inferences about the bases more generalizable than including people under the age of 18. Additionally, including people under the age of 18 requires a different set of restrictions on research for human subjects review.} gender, and income. We conducted the surveys in 2018, 2019, and 2020 for a total of 42,000 people across three years in the 14 countries. Our goal in surveying all countries three years in a row was to isolate any phenomena within a country that may drive our results in isolation. For example, an election in a country may center around a left-right divide over security and people's responses to our survey may reflect a polarized moment if we only examined a single year. We used the 2018 wave of our surveys to conduct initial tests of our hypotheses and found several interesting trends in that one year of data \cite{Allen2020}. This book project uses all three years for our analysis. 

	\begin{figure}[t]
	\centering\scalebox{.85}{\includegraphics{../../Figures/Chapter-Contact/old/figure-map-survey-coverage.pdf}}
	\caption{Surveyed countries and the firms we used to survey each country.}
	\label{fig:surveycoverage}
	\end{figure}	

We localized our survey questions to each of the official languages of the countries we surveyed.\footnote{These translations include Arabic, Dutch, English, Filipino, French, German, Italian, Japanese, Portuguese, and Turkish.}  In addition to the surveys, we also conducted interviews with local activists, politicians, U.S military personnel, journalists, and US diplomats to add qualitative texture to our research questions. 

The survey itself had approximately 50 questions, and we estimated the completion time of the survey to be 10-15 minutes; the survey could be longer or shorter as some questions led to follow-up questions.\footnote{The survey remained consistent from year-to-year. In years 2 and 3, we did add a question about whether individuals trusted their government as local trust would likely condition whether a respondent also trusted the security agreements their government made with the United States.} The survey asked demographic questions (e.g., a respondent's age, gender, minority status, years of formal education, approximate income, and religion), ideological questions (e.g., their self-assessed left-right political ideology, whether they favor democracy, views on security, trust in government, and societal goals), and opinion questions (e.g., views on the American people, government, influence and military as well as questions about other actors). A remaining twenty questions focused on views of the US military presence in their country, its effect, and their experiences with the presence.\footnote{The full list of questions, our data, and several other supplemental projects we have worked on this project are fully available at \url{http://ma-allen.com/military-deployments/}.} Worth highlighting are the three questions that we draw several inferences on and estimate using our statistical models. Specifically, we ask:

	\begin{quote}
		``In general, what is your opinion of the presence of American military forces in (respondent's
		country)?''
	\end{quote}

We also ask what people's opinion is of the American government and people.\footnote{During this process, we were concerned that respondents might conflate American with all of the Americas and not the United States specifically. After consulting with language and area studies experts, we concluded that this phrasing of the question mirrored our intentions, stayed consistent in translation, and would not confuse respondents.} Each respondent rated the military force, government, or people on a 5-point Likert scale that included 1) ``Very favorable'', 2) ``Somewhat favorable'', 3) ''Neutral'', 4) ''Somewhat unfavorable'', 5) ''Very unfavorable'' or 6) ``Don't know/decline to answer.'' For our purposes, we group the unfavorable responses and the favorable responses together. In constructing the survey, we were concerned that we might prime the respondents if we asked them about the benefits or costs of the US military presence before we asked them about their views on the US military. We made sure to first ask about the respondent's views on the US government/people/influence/military before asking them about any benefits they may have received from or harms they have experienced by the US military.

For this chapter, we are also interested in whether people have had contact with or economically benefited from the military presence. Our expectation is that both contact and economic benefits correlate with views on the US military presence and other actors. To get at these ideas, we asked people directly:

	\begin{quote}
		``Have you personally had direct contact with a member of the American military in [respondent's country]?''
	\end{quote}

People could answer this question yes, no, or don't know/decline to answer. The concept of ``economic benefit'' might be diffuse to respondents in our survey, so we provided a bit more context to this question to help the respondent reflect on possible connections they may have had:

	\begin{quote}
		``Have you personally received a direct economic benefit from the American military presence in [respondent's country]? Examples include employment by the US military, employment by a contractor that does business with the US military, or ownership/employment in a business that frequently serves US military personnel.''
	\end{quote}

Our theory posits that direct experiences are only part of the equation in determining beliefs. To account for the transmission of ideas from family members and friends, we asked people about the same events within their social networks. Our questions represent active discussion of the military and helps us to capture if there is a transmission of value-laden information from a social network to the respondent. Presumably, knowing whether your friend has had contact with a military member or if they economically have benefited from the military comes with some anecdote about the experience. The anecdote may reflect routine or banal expectations such as being held up in traffic due to a large military convoy on the road or an influx of soldiers at a restaurant. It also may detail more personal interactions, such as having a child on a soccer team with kids of military personnel or getting into a altercation with an intoxicated sailor. Retelling these anecdotes to friends and family often comes with a subjective interpretation that our respondents may generalize into larger views about the military. We, therefore, inquire:
	
	\begin{quote}
		``Has a member of your family or close friend had direct contact with a member of the American military stationed in (respondent's country)?''
	\end{quote}

The economic benefit question about a respondent's social network mirrors our question about individual benefits as it primes the respondent with a few examples:

	
	\begin{quote}
		``Has a member of your family or close friend received a direct economic benefit from the American military presence in (respondent's country)? Examples include employment by the US military, employment by a contractor that does business with the US military, or ownership/employment at a business that frequently serves US military personnel.''
	\end{quote}

These questions instruments are not a perfect capture of the social network effect on belief formation as some latent transmission occurs. For example, a family member may express beliefs about the US military presence that stem from an interaction or encounter that they have shared. More commonly, friends and family members share beliefs about political topics without any direct experience and those processes transmit identity-shaping values as well. However, we posit that direct knowledge of other peoples' experiences represents a salient form of value-formation as it contextualizes the experiences to something immediate and affects their social network. 

We use other questions in our survey and mention those as they become relevant to our models. However, these seven questions form the core of our research as we turn to our statistical model design.


\subsection*{Estimation Strategy}

We have a few strategies for discussing our findings. First, as will be evident through this chapter, we highlight some of the simple results we have from basic questions. These charts we present in the next section can give a snapshot of where people are, on average, within their country on a given question. These snapshots provide some of the baseline responses to our questions and provide context to our more complicated models. Additionally, many of the questions we asked are unique to our survey as few other surveys attempt to deeply probe people's beliefs about the US military within their country. Showcasing some of these results provide novel insight into how people see the United States and its security apparatus. 

Beyond these graphs, we also seek to explore the relationships between the different questions that we asked in our survey. When someone answers a question in our survey, they are doing so enmeshed in a particular individual, regional, and national context. Several factors likely influence their answer. To assess the strength of the relationship between the various questions, we turn to a complicated statistical model to tease out these competing effects at different levels of interaction. The specific model we use for our models dealing with the survey results is a multilevel categorical Bayesian logistic regression. Using this model, we are estimating the discrete choices people can make within the survey and accounting for both personal attributes we know about them given the demographic and ideological questions while also controlling for regional and national factors that may influence their answer. This multilevel approach allows us to include all fourteen countries, all 42,000 respondents, in the same model while also allowing the model to treat people from different countries as if they have different experiences.

The Bayesian framework of statistics allows us to ask the question of ``what parameters are likely to have generated these correlations between the variable?'' We instruct the estimation to run for 1,000s of iterations to provide a distribution of possible parameters that are likely to have produced the relationships within the data. This provides us with a credibility interval that we can interpret to determine how likely two different answers on a survey (or other inputs into the estimation) positively or negatively associate with each other. The tables that we produce from these estimations are relatively large and technical. For reading ease, we have put the full models into appendices available to the reader. We will take a few of the key highlights from each of the models and discuss them in reference to our argument and our findings. We use this model for the survey data to draw inferences in Chapters \ref{cha:meth} through \ref{cha:protest}.

Additionally, in some of our chapters, we turn to data from outside of our models. We have collected data on media-reported crime, protests, military expenditures, and military construction to get a better sense of the footprint of the military and the stimuli people may or may not evaluate in constructing their perceptions of the US military.\footnote{All of the data we collected, including the surveys and the data mentioned here, are available for public use at \url{http://ma-allen.com/military-deployments/}.} In our chapters dealing with crime and protests, we use some additional models for estimation and discuss the construction of those models in more detail for those chapters.
