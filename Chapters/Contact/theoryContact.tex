\section*{Contact and Perceptions}
%this section should not be long, we frontloaded most of this in the theory section

We contend that the US military presence shapes the cultural, economic, and political institutions that they become enmeshed in while they serve overseas. Through the process of deployment, and even more so when those deployments are through permanent bases, the US presence has a gravitational effect that sends waves rippling through local communities. The larger the size of the presence, there is corresponding increases in the gravitational effect. As more US troops are in a region, there is an increased likelihood that the soldiers interact with locals, cause traffic jams, and affect local prices with the influx of American dollars. The presence of troops is a social-institution disturbing presence that people will find hard to ignore.

The effect of this gravitational pull can be both positive and negative. In a benign setting, mutual interaction can humanize the presence of the behaviors of a single member of the military. Everyday interactions can break down existing stereotypes, whether negative or positive and build new ones based on the outcome of those interactions. Humanizing the deployed forces can rewrite national political or media narratives through the process of intercultural exchange. In a negative interaction, victimization can overwrite beliefs about democracy promotion, anti-communism, national security, or other security cooperation narratives. Our expectations about the results of interaction follow these lines of logic. First, those who interact with the US military are less likely to say that they do not know whether they have a negative or positive view of the United States military, government, and people. We present our first expectations as three subsets of hypotheses centered on the relationship between direct contact and views about American actors. 

\begin{subhyp}
	
	\begin{hyp}
		Individuals who report direct contact with a member of the US military will be more likely to express informed views of the American presence/government/people. 
	\end{hyp}
	
	Research demonstrates that ignoring the people who profess not to have views on particular survey questions can bias the results of econometric estimation \cite{Kleinberg2018}; choosing to say that you do not know or refuse to answer a question is a choice that likely correlates with other demographic, ideological, or experiential indicators that may also correlate with the question researchers are interested in. In our case, it stands to reason that people who have fewer experiences with the military, younger, politically moderate, or other factors may be less likely to say they have a positive or negative view of the US military. Not including these subsets of answers in our estimations would make such indicators less meaningful in their predictions since we would be excluding one of the things they predict: non-response.
	
	Beyond the methodological reason to include this type of response, there is a substantive issue at stake. We view positive or negative answers as more likely to be informed views (from whatever source, including predisposed biases). Those who say that they do not know in response to one of our questions express a less informed view about the subject. While the informed and uninformed dichotomy is not a perfect mapping onto reality, some people will refuse to answer a question for reasons other than information. There are people who will respond in a directional matter but know very little about the issue serves as an approximation of knowledge. Additionally, repeatedly throughout this book, we find that people that report experiences with the US military are far more likely to have a view on the subject.
	
	For those that report positive and negative experiences, we expect that for a large number of respondents, the everyday interactions with US military personnel will have the contact-based effects that we discuss more thoroughly in Chapter \ref{cha:theory}. Experimental evidence in political science shows that even a short, brief 10-minute conversation with someone of a maligned identity can dramatically breakdown stereotypes about a group \cite{Broockman2016}. Importantly, in the \citeasnoun{Broockman2016} study, those attitude changes endured when the researcher interviewed their subjects months later. Additionally, some of those interactions will be beyond the everyday interactions and lead to positive experiences that encourage a civilian to think positively of the US presence.
	
	
	\begin{hyp}
		Individuals who report direct contact with a member of the US military will be more likely to express positive views of the American presence/government/people. 
	\end{hyp}
	
	Those who have contact with the US presence are more likely to be at risk to experience the negative aspects of the presence. They may see military vehicles clogging their morning commute or experience constant disruptions from Air Force jets taking off. Seeing a rise of criminal activities in your neighborhood due to the increased demand among GIs or a decrease in the quality of beaches and parks due to base expansion may be factors that convince those that have contact with the US military that there presence is a force for worsening their lives. Beyond environmental and social harms of the US presence, some interactions are directly negative. Traffic incidents, public intoxication, fights, assault, and sexual assaults are all events that civilians in host-states have experienced from US military personnel. As such, these types of incidents and contact experience will correlate with negative perceptions of US actors.
	
	\citeasnoun{calder2007} makes a strong hypothesis about the relationship between contact and negative perceptions. Specifically, he argues that increasing familiarity creates points of friction and increases the likelihood of resentment, conflict, and protests. To test this, he compares country densities and anti-basing activity and finds support for his argument. While this supports our hypothesis, there is some caution in understanding these results. Notably, there is an issue of an ecological inference fallacy as Calder infers individual level behavior from aggregated level data \cite{King2004}. There are several reasons, however, why we make expect density to correlate with opposition that may not derive from contact. More densely populated countries, and cities given that Calder compares activity within countries as well, have more access to media outlets that enable information sharing and recruitment. Having access to more people, public transportation, and cheaper communication makes it easier to organize, mobilize, and collectively act than in more rural deployments. While Calder lends support to our argument, we raise this point to highlight the importance of surveying individuals to assess if something like density is the cause of negative views, the enabler of collective action, or both.\footnote{Calder's argument is more nuanced than contact alone, but given the focus of our argument throughout this book, it is important that we consider the existing argument for this relationship and whether we learn anything new from our research. Clearly, we also expect the opposite case here but what confirmation we do find gives us more depth in understanding the pathways to support and opposition of bases abroad.}  %new addition, revise and consider if we want to go after calder.
	
	\begin{hyp}
		Individuals who report direct contact with a member of the US military will be more likely to express negative views of the American presence/government/people. 
	\end{hyp}
	
	Our views do not manifest purely from first-hand experiences, and humans learn lessons and develop opinions of the world from their family and friends. Indeed, evolutionary and social psychology points out that the prevalence of people creating false memories where they remember other people's recounted stories as their memories might be an early adaptation for learning---by remembering someone else's story as our own, we effectively have learned a lesson without going through that experience \cite{Ginsburg2010,Howe2011}. Political psychology suggests that your social context, your friends and family, are essential in influencing your political beliefs \cite{Campbell1960,Jasper1995,Settle2010}. Consequently, we expect people who report that members of their social network have had contact with the US military will experience similar changes in their views about various US actors. We expect that this effect is likely less prominent than direct experiences, but our primary focus here is whether there is an effect from these sources of contact. 
	
\end{subhyp}


\begin{subhyp}
	
	\begin{hyp}
		Individuals who report network contact with a member of the US military will be more likely to express informed views of the American presence/government/people. 
	\end{hyp}
	
	
	\begin{hyp}
		Individuals who report network contact with a member of the US military will be more likely to express positive views of the American presence/government/people. 
	\end{hyp}
	
	\begin{hyp}
		Individuals who report network contact with a member of the US military will be more likely to express negative views of the American presence/government/people. 
	\end{hyp}
	
\end{subhyp}

The deployment of military personnel comes with a whole host of economic shocks to local economies. The construction of a base and its maintenance requires extensive local resources and labor. US base commanders seek to source their supplies locally as it is both cost-efficient (as opposed to importing resources and labor). It builds up the local economy around the base \cite{rafthree20190719,kaserneone20190725}. Those that directly benefit from an overseas deployment of a base are more likely to have informed views about the military force and more likely to support its presence.

\begin{subhyp}
	
	\begin{hyp}
		Individuals who report economic benefits from the US military will be more likely to express informed views of the American presence/government/people. 
	\end{hyp}
	
	
	\begin{hyp}
		Individuals who report economic benefits from the US military will be more likely to express positive views of the American presence/government/people. 
	\end{hyp}
	
	
\end{subhyp}


It is feasible that the negative hypothesis would make sense here as those that economically benefit from the military are likely to have close contact with service members and are more likely to experience first-hand negative events. However, we expect that, on average, the direct personal benefit will override other misgivings in a sum assessment about the effect of the military presence.

We expect the two same hypotheses for people who know others that have a personal benefit from the presence of the US military:

\begin{subhyp}
	
	\begin{hyp}
		Individuals who report economic benefits in their network from the US military will be more likely to express informed views of the American presence/government/people. 
	\end{hyp}
	
	
	\begin{hyp}
		Individuals who report economic benefits in their network from the US military will be more likely to express positive views of the American presence/government/people. 
	\end{hyp}
	
	
\end{subhyp}

To evaluate these hypotheses and those that we develop in later chapters, we now turn to the survey data we have collected over the past several years. 