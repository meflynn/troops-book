\chapter{Introduction: US Service Members as the Microfoundation of US Power \label{cha:intro}}

\setcounter{page}{1}


\begin{quote}
	``You know, you're studying international relations\index{international relations} and you hear about what the most important elements of good relations are.  In international relations, all politics is personal---I presume to say to you professors---because it's all ultimately based on trust.  And trust only flows from personal---not friendly---personal, candid relationships with your counterpart, so you don't have to wonder about intentions.''
	
	\begin{flushright}
		\begin{minipage}{3in}
			---Vice-President Joseph R. Biden \index{Biden, Joseph R.} in 2013 at \hspace*{8pt} Yonsei University in Seoul, South Korea.\index{South Korea}
		\end{minipage}
		
	\end{flushright}
	
\end{quote}

\doublespacing


\noindent It was pouring rain in Naha, Japan \index{Japan} on August 11, 2018. This weather is not atypical for the capital of the Okinawa\index{Japan!Okinawa} Prefecture---the city receives twice the annual rainfall of Seattle, with 80 inches of precipitation per year. Atypical, however, were the 70,000 people gathered that Saturday to observe a moment of silence for the governor of Okinawa\index{Japan!Okinawa} who had died from cancer earlier in the week.\autocite{cnbc2018,tjt2018} Former Governor Takeshi Onaga was an outspoken critic of the US military presence in Okinawa\index{Japan!Okinawa} and campaigned to remove the US forces stationed there. The thousands of people presently standing silent in the rain were at the state park to protest the presence of US bases. More specifically, they called for the cessation of plans to move a base to a new location in Okinawa\index{Japan!Okinawa}. 

The United States has provided external security for Japan\index{Japan} since the end of World War II\index{World War II} through a mixture of defense cooperation agreements, arms sales, and the stationing of US military forces on Japanese soil. While the threat of communism\index{communism} from the Cold War\index{Cold War} period has waned, a nuclear-armed North Korea\index{North Korea} and a rising China\index{China} has renewed emphasis on US security ties with Japan.\index{Japan} According to a 2018 Pew Research survey, 81\% of Japanese respondents preferred the US over China\index{China} to lead the world.\autocite{StokesandDevlin2018} However, the existence of political movements against that US presence---in Okinawa\index{Japan!Okinawa} in particular---dates to the earliest years of the US occupation.\autocite{cottrellmoorer1977,akibayashi2009} While many in Japan \index{Japan} may view the US in generally favorable terms, the US-Japanese security arrangement is deeply unpopular in the place where most US service members reside: Okinawa\index{Japan!Okinawa}. Resident grievances with the US presence stem from a long list of harms, including crime\index{crime}, drunken behavior of US service members, violence, sexual assaults\index{crime!sexual assault}, land-use by the bases, pollution, a threat to endangered wildlife, habitat destruction, traffic\index{traffic}, traffic accidents, and noise pollution\index{noise pollution}.   The history of the US military in Japan\index{Japan} is a story of these competing forces, with protests\index{protest} against the US military occurring on a fairly frequent basis at the local level and the occasional event triggering larger mass protests\index{protest} that draw crowds of tens-of-thousands of people. 

On June 26, 2019, a few weeks before we arrived in Germany \index{Germany} to carry out fieldwork for this project, another protest\index{protest} occurred around Ramstein\index{Germany!Ramstein Air Base} Air Base, found in the state of Rheinland-Pfalz in the southwest of the country. Like Japan,\index{Japan} the US military presence in Germany\index{Germany} dates to the US occupation at the end of World War II\index{World War II}. In both cases, the US established the bases in the aftermath of World War II\index{World War II} to occupy and rebuild the two countries. While the anti-US base movement is also decades old in Germany,\index{Germany} the activists\index{activists} could only muster approximately 5,000 people to attend the protest\index{protest} against the base, according to one estimate.\autocite{berlinone20190723} This protest\index{protest} marked the fifth year of this now-annual protest\index{protest} against Ramstein\index{Germany!Ramstein Air Base}. The event has generally been peaceful and includes workshops for activists\index{activists} and some acts of civil disobedience. Overall, the planned demonstration lasted just over an hour.

% We're not comparing apples to apples here with the two polls. We should find another source that directly addresses this.
%I think it's okay if it's not a perfect comparison; we're just noting that the two cases are different. We're just setting up a story through anecdotes at this point.

According to one poll, about 52\% of Germans believe that the presence of US bases is somewhat or very important to German national security\index{national security}.\autocite{Gramlich2020} Compared to Japan,\index{Japan} there appears to be less national support for the US presence, but domestic opposition to particular bases is relatively weak. One German peace activist we spoke with suggested that the movement in 2019 was a high watermark for anti-base activism and expected to see continued growth in the movement. In Japan,\index{Japan} however, the US mission in the country has strong popular support, but there tends to be more intense local opposition closer to basing sites. 

% I think we should work in some language here, noting that systematic study of these questions is largely lacking.

How is it that, despite such similar histories, the US military deployments in these countries have tended to provoke such radically different reactions among the host-state populations? This question matters not just in explaining political movements opposed to a US military presence but also for better understanding the relationships that the US military builds with the communities in which it becomes embedded when deployed abroad. These relationships have serious implications for where the US military has the authority to deploy, the latitude with which it operates overseas, and the freedoms that US personnel enjoy within host countries. More generally, lower levels of popularity can lead to greater constraints on US personnel and decreased levels of burden-sharing by formerly willing allies. A widespread backlash against the US military's presence within a given country frequently correlates with events like criminal activity or environmental degradation\index{environment}. Importantly, this suggests that mass attitudes towards the US military are not fixed but shaped by various causal factors. This relationship raises a key question: What factors increase or decrease support for US military deployments in countries worldwide?  %work on this more

In this book, we argue that increased contact\index{contact} between deployed military personnel and host-state civilians has a net-positive effect on host-state civilian perceptions of the US and US actors. Everyday interactions build bridges and bonds with civilians that break down stereotypes and negative perceptions. These effects transmit through social networks, and the economic effects\index{economic effect} of deployments likewise build support in host-state populations. Not all interactions produce positive sentiments, and some negative interactions may become the catalyst for anti-US movements\index{protests}. Importantly, we argue that the US overseas security strategy is increasingly dependent upon cultivating popular support; the US is entering a ``Domain of Competitive Consent'' where it will compete with other states for security arrangements, the ability to deploy, and the ability to base. If the US hopes to maintain a vast network of deployments overseas, it increasingly needs to convince the people of other countries that the US is acting within their interests and that they want the US within their country. The personnel of the US military help to uniquely cultivate the requisite soft power\index{soft power} in enabling their deployments. Efforts to isolate those forces allow existing negative effects of basing and deployments to continue while limiting opportunities for positive interactions to occur.

Studies have examined the negotiations that the United States carries out with partner states hosting US military deployments.\autocite{McDonald1990} There are also many frameworks for understanding various aspects of major power relations with minor powers (prot\'{e}g\'{e}), and the willingness of the prot\'{e}g\'{e} to cede autonomy to the major power---in this case, the United States. In general, these studies all adopt a macro-level perspective of these relationships as they portray the United States' constructing new hierarchies\index{hierarchy}, security arrangements, grand strategies, or empires.\autocite{Ikenberry2004,Johnson2004,Bacevich2009,Lake2009a} We still lack equally extensive cross-national explanations of what it is that makes an individual accept a foreign military stationing in their hometown and what makes another individual resist it. Regardless of what US politicians\index{politicians} say or promise to host country populations, the Americans who come into contact with host state populations are US service members. They are not diplomats\index{diplomats}, yet they consciously or unconsciously engage in public diplomacy\index{public diplomacy} every time they interact with a host-country resident. While these individual-level interactions may seem insignificant to international politics, a single US soldier violently killing his Panamanian\index{Panama} girlfriend can create consequences that affect the ability of hundreds of other US diplomats\index{diplomats} and military officers to effectively carry out their jobs.\footnote{This is in reference to a case cited by one of our interviewees \autocite[see:][]{amb20180713}. In 2015, police arrested Master Sgt. Omar Velez-Pagan for murdering\index{crime!murder} his Panamanian\index{Panama} girlfriend and attempting to dispose of her body in the Panamanian jungle. He was sentenced to 30 years in prison in 2016 \autocite[see:][]{ChicagoTribune2016}.} Alternatively, it is also possible that the innumerable small financial transactions between US airmen and English pub owners can compound over time, creating social and economic pressures in support of a US Air Force base. It is easier to assess causal effects in a single interaction or particular event that produces a decidedly positive or negative reaction, but can we identify more general patterns that determine the tenor of civil-military relations when the US military deploys abroad? 


\section*{US Military Deployments and Bases Since 1950}

In 2020, the United States deployed service members to 175 countries and territories in the world.\autocite{DMDC2020} This includes nearly every country in the world, as can be seen in Figure \ref{fig:intromap2}.\footnote{Maps generated using the {\tt cshapes} package in {\tt R} \autocite[see:][]{weidmannetal2010}.}  The deployment of service members to countries around the globe serves a multitude of roles, functions, and goals. At the macro level, the deployment of troops abroad has been the backbone of US strategy since the end of World War II\index{World War II}. Even as the largest deployments to locations like Germany,\index{Germany} Japan,\index{Japan} and South Korea\index{South Korea} have shrunk considerably since the immediate post-WWII era\index{Cold War}, they remain home to tens of thousands of US service personnel and their families. Service members act as credible commitments to allies and are a deterrent to would-be aggressors.\autocite{Schelling1966} Additionally, the presence of service members allows the United States to have a rapid military response, or the threat of one, to developing situations in every region throughout the world.  Since the collapse of the Soviet Union\index{Soviet Union}, and arguably including it, no country has competed with the United States' military presence.

\begin{figure}[t]
	\centering\includegraphics[scale=0.8]{../../Figures/Chapter-Intro/figure-map-troops-2020.png}
	\caption{US soldiers deployed to overseas territories in 2020 using data we collected from the Defense Manpower Data Center.\autocite{DMDC2020} Estimated figures for active operations collected from secondary sources.}
	\label{fig:intromap2}
\end{figure}


%explicit statement of questions asked in the book, one para

The United States formed its contemporary basing network in the aftermath of World War II\index{World War II}. As the Cold War\index{Cold War} developed, the United States shifted from a drawdown period to its acquired wartime basing network to develop a global presence and dramatically expand its military reach. The drive to counter communist expansion\index{anti-communism}, reassure allies, and logistically support its expanded position deepened the United States' locations for permanent deployment.\autocite{Huston1988} Figure \ref{fig:intromap1} shows the geographic distribution and size of US military deployments in 1950. Some countries, like Australia\index{Australia} or Romania\index{Romania}, hosted a small number of US troops and equipment. In contrast, other countries, such as Germany,\index{Germany}, Japan,\index{Japan} and South Korea,\index{South Korea} garrisoned tens of thousands of US military personnel, along with their families, equipment, supplies, and more. 


\begin{figure}[t]
	\centering\includegraphics[scale=0.8]{../../Figures/Chapter-Intro/figure-map-troops-1950.png}
	\caption{US soldiers deployed to overseas territories in 1950 using data from.\autocite{Kane2004}}
	\label{fig:intromap1}
\end{figure}


During the Cold War\index{Cold War}, the United States sought out stable governments\index{government} to host bases. An overseas base is a fixed asset with a high sunk cost, and the United States needed to ensure a base would endure for decades if possible.  Often, during this period, the framing of the choices between different basing sites encouraged the President to select more autocratic\index{autocracy} governments\index{government} (such as Franco-era Spain\index{Spain} or the Philippines\index{Philippines}) over states with more contentious democratic politics that could threaten the US access or control over a base.\autocite{Halperin2007} The Department of Defense\index{Department of Defense} wanted to maximize the latitude it had for operations. Given that it is the primary provider of information to the President, it portrayed particular basing sites as necessary or the only real option while downplaying sites that would prove more politically challenging to navigate for the DoD\index{Department of Defense}. 

Of course, an autocratic country\index{autocracy} is not guaranteed to remain autocratic either, and determining which autocracies\index{autocracy} are the most stable is difficult even for the DoD\index{Department of Defense}. Illustrating this dynamic, democratic\index{democracy} transitions\index{democratic transition} led to the removal of some US forces in both Spain\index{Spain} and the Philippines\index{Philippines}.\autocite{Cooley2008} While the United States could not avoid the democratic transitions in these countries, it still shows the volatility of democracies\index{democracy} for hosting bases. If the public is not on your side, they can galvanize the political process to expel your presence.  From the perspective of the United States, the contemporary period has diminished opportunities for the US to retain total control over its bases and troops in Status of Forces Agreements (SOFAs)\index{Status of Forces Agreements}, and the majority of overseas US bases are now in democratic polities\index{democracy}. 

While the Cold War\index{Cold War} did present some opportunities for challenges from the Soviet Union\index{Soviet Union}, three conditions made that era fundamentally different from the contemporary situation. First, the United States initially expanded its basing network during power transition when allied powers reduced their base holdings, with many ally bases becoming new US basing sites. Second, the Soviet Union\index{Soviet Union} confined most of its basing activities to proximate allies, which were not places where the US could feasibly compete for access due to Soviet influence\index{Soviet Union}. Competition for basing was the exception rather than the rule.\autocite{Nieman2020} Third, the responsiveness of governments\index{government} to their populations has increased as a function of increasing transparency and information from the spread of communication technologies in the latest wave of globalization\index{globalization}.

In the post-Cold War era, the US faces the prospect of a different form of power projection competition with China\index{China}. China\index{China} continues to increase its aid and economic investment in Asia\index{Asia}, Africa\index{Africa}, and Latin America\index{Latin America}. Trunkos shows that Russia\index{Russia} and China\index{China} have expanded their effort to build soft power\index{soft power} with other countries dramatically since 1995 and are in direct competition with US soft power in numerous regions.\autocite{Trunkos2020} The People's Republic of China's (PRC)\index{China} Belt and Road Initiative (BRI)\index{Belt and Road Initiative} is a massive economic push to create infrastructure in over 70 countries and connect much of the world to China\index{China}'s growing economic network.  The BRI\index{Belt and Road Initiative} is one of its more well-known and largest economic projects to expand its regional economic influence. Still, it is one of several projects to build Chinese economic capacity as an alternative to US financial and economic institutions.\autocite{Doshi2021}

The PRC is also expanding its military relationships in its near-abroad as a response to the perceived US efforts to contain China\index{China} in the Asia-Pacific\index{Asia} region.\autocite{Yao2019} Doshi argues that China\index{China} has gone through four distinct periods of Grand Strategy\index{Grand Strategy}.\autocite{Doshi2021} Before 1991, it was not seeking great power status, but the US dominance in the Gulf War\index{Gulf War} shocked it into a new phase of blunting the US' position. After the 2008 financial crisis\index{financial crisis}, it entered a stage of building its own regional and global position. Starting in 2017, China\index{China} sought to expand upon what it has created and is actively seeking ``national rejuvenation'' and the top spot in global power by 2050. Part of its military expansion involves expanding to new basing sites.

Regardless of whether the two countries come to compete over basing and access rights in multiple domains, it is increasingly likely that minor powers will confront an environment where the US and China\index{China} are simultaneously bidding for these rights---in some cases in the same country at the same time.\autocite{Campbell2019} Nor will such decisions be purely top-down elite-driven decisions. Particularly in democratic countries\index{democracy}, they will increasingly include domestic constituents in the decision-making process; the role of citizens in determining national-level security decisions has been increasing.\autocite{GoldsmithHoriuchi2012} Citizens, weighing the perceived costs and benefits of having US troops in their backyard, politically mobilize\index{mobilization} around that presence. As movements succeed, others have and will take similar routes to influence security policy. 

\section*{Peacetime Deployments and Host-Country Publics}

Public perceptions of major power governments\index{government} and the soldiers they deploy abroad are vital to maintaining existing deployments and allowing for new opportunities. Scholars have pointed to anti-US activism in the Philippines\index{Philippines} as a primary driver for removing the US base at Subic Bay\index{Philippines!Subic Bay}.\autocite{simbulan2009} Anti-base opposition has also added challenges for the United States in Germany,\index{Germany} South Korea,\index{South Korea} and Japan.\index{Japan} Alternatively, interactions between US personnel and their host communities can also be vital for creating population support for the presence of those troops and their mission.\autocite{Allen2020} Ultimately, as countries become wealthier and more economically developed, they can demand a greater say in basing and deployment arrangements. This greater demand arises both from greater state capacity and having the capacity to contribute more to an alliance and rely upon the United States less for overall security.\autocite{Lake2009a}  The US and China\index{China} are entering a new phase where soft power\index{soft power} and public diplomacy\index{public diplomacy} will be vital in ensuring the global feasibility of hard power projection. For the United States, its service members' daily and routine behavior form the microfoundations for building support for a US presence within host states. Consequently, US forces are not only engaged in winning the ``hearts and minds'' of people in combat zones but are actively engaged in similar tasks during peacetime deployments overseas. This book advances our understanding of this essential international relations\index{international relations} thread by examining the conditions in which service members generate support for and opposition against their presence, the United States government\index{government}, and the United States' people.  


In addition to the more general theoretical concerns, these issues directly impact contemporary policymaking challenges in US foreign policy. The United States military has focused on multi-domain competition with existing and rising challengers. Specifically, the Department of Defense\index{Department of Defense} is interested in evolving relations, competition, and conflict with Russia and the PRC\index{China}. The 2018 National Defense Strategy\index{National Defense Strategy} states, ``The reemergence of long-term strategic competition, rapid dispersion of technologies, and new concepts of warfare and competition that span the entire spectrum of conflict require a Joint Force structured to match this reality.''\autocite[p. 1]{Mattis2018} However, the summary only mentions the role of basing three times and focusing on drawing down the US presence to a leaner, smaller military footprint. This guiding document identifies competition with Russia\index{Russia} and China\index{China} and many critical dimensions to this competition. Still, we argue that it excludes one that will prove increasingly vital in determining US access: \textit{The Domain of Competitive Consent}\index{Domain of Competitive Consent}. The United States' de facto monopoly on basing is eroding, and other powers are expanding their networks. With its first permanent foreign military installation in Djibouti, China\index{China} has wholly entered the domain of overseas basing. Understanding the more democratic\index{democracy} nature of maintaining an overseas presence is vital to understanding how the US will maintain global influence.

We establish a new path from previous work on this topic by examining the unexplored microfoundations of support for the United States' military presence abroad\index{Domain of Competitive Consent}. We demonstrate the conditional nature of support for the United States by civilians in host states. Drawing on theoretical work from political science\index{political science}, international relations\index{international relations}, and sociology, we examine how several factors drive host state civilians' perception of the United States military and their behavior towards it. These factors include contact with troops, economic reliance on the US military, criminal activity\index{crime} involving US service members, and self-identification as an ethnic minority\index{minority!ethnic}. Understanding the conditions that generate support for and opposition to the United States military abroad is vital to understanding the changing shape of power politics on the international stage. It is also vital for understanding the contours of domestic political support within host countries. Particularly in democracies\index{democracy}, different political parties rely on the support of various societal coalitions. Understanding which groups are more likely to support or oppose the United States' military presence within a country can allow us to understand better how domestic political events are likely to affect the United States' relations with its allies\index{Domain of Competitive Consent}. Importantly, our work concerns itself with deployments generally and not just troops that deploy to permanent bases. While we speak to concerns about basing throughout our work, our purpose is more general as the US sends troops overseas for a wide variety of missions and configurations.\cite{Allen2011}

This book provides new theoretical and empirical insights supported by surveys and interviews\index{interview} from 2018 through 2021. We surveyed\index{survey} 14 countries with over 42,000 respondents and interviewed over 30 individuals, including military personnel, journalists\index{journalists}, policymakers, and activists\index{activists}, across six different countries. Our surveys\index{survey} focus on questions tailored to the United States presence and create a unique dataset. These data form the foundation in answering important questions about overseas perceptions of the United States, its military presence in a wide range of host countries, and its foreign policy more broadly. Using this data and qualitative interviews\index{interview} we conducted in Peru\index{Peru}, Panama\index{Panama}, Germany,\index{Germany} Great Britain\index{Great Britain}, Japan,\index{Japan} and South Korea,\index{South Korea} we provide the first generalizable study of the factors that shape mass attitudes towards the United States' vast network of overseas military deployments. By better understanding the sources of mass attitudes, we shed light on the microfoundations of US foreign policy and the origins of popular consent in other nations. Research tends to overlook the consent of foreign populations when exploring the causes and consequences of the United States' post-War military deployments. 

At its core, this book focuses on troop deployments during peacetime. Though a few of the deployments we study are a legacy of a US invasion or intervention (such as Germany,\index{Germany} Japan,\index{Japan} South Korea,\index{South Korea} and Kuwait), they have all been sustained for several decades by explicit consensual agreements between the United States and host countries\index{Domain of Competitive Consent}. Deployments and bases in Japan \index{Japan} and Germany \index{Germany} that originated with the US invasion and occupation are now at a stage where they regularly require consent from the host\index{Domain of Competitive Consent}. Both countries have Status of Forces Agreements\index{Status of Forces Agreements} that outline the rules and regulations associated with the US presence in their countries. They both engage in costly burden-sharing arrangements, with the host country helping to offset some of the costs of maintaining the US deployments and bases in their territories. Ultimately, each could choose to opt-out of their basing agreements with the United States if the arrangement was no longer within their interests or if the popular will were against the presence.\footnote{Even in recent cases like Iraq\index{Iraq} the period in which the presence of US military personnel was non-negotiable was, relatively speaking, reasonably short---after the United States military invaded Iraq\index{Iraq}\index{Iraq War} in 2003 the bulk of its personnel were required to leave in 2011 at request of the Iraqi Government\index{government} (though additional US forces did return to fight the Islamic State\index{Islamic State}). While the presence of US forces has \textit{technically} been consensual since that time, we also view this as a qualitatively different case. The close temporal proximity to the initial invasion and active combat operations means that the experiences many Iraq residents have are very likely to be informed by a highly unusual US military posture compared to most other large contemporary deployments.}


The United States began actively sending troops overseas in long-term deployments after the Spanish-American War\index{Spain}. The cessation of hostilities in 1898 saw the United States gain control and access in perpetuity to its first permanent base in Guant\'{a}namo Bay, Cuba\index{Cuba}. More than a century of troops overseas has produced some critical scholarship examining the effect of those troops, though it is more limited than one might imagine. First, although the study of armed conflict is an especially prominent part of political science\index{political science} and international relations\index{international relations}, studies focusing on the military itself have been relatively rare until more recently. Second, where scholars have debated the effect of such deployments in terms of conflict, security, and international relations\index{international relations}, these studies often treat deployments as a means of tapping into more abstract latent characteristics of the relationships between states.\autocite{Harkavy1989,Ikenberry2004,Lake2009a,Wohlforth1999,davis2011} For example, in his exploration of hierarchical relations\index{hierarchy} between states in the international system, Lake uses military deployments to construct an index of ``security hierarchy''\index{hierarchy} between the United States and other countries. While troop deployments serve as the basis for Lake's index, the nature of the deployments themselves---their causes and consequences---are of little importance in his theoretical and empirical analyses.\autocite{Lake2009a} 

In recent years a second wave of literature has started to seriously explore the causes and consequences of military deployments, treating the positioning of military personnel to overseas locations as the object of theoretical and empirical interest. In general, this research line has attempted to understand better the positive and negative effects of troop deployments on their host environments. For example, scholars have written extensive qualitative work on the negative externalities of military deployments in specific countries. These studies have pointed to issues such as increased criminality\index{crime}, environmental degradation\index{environment}, and other societal ills.\autocite{Hohn2010,Vine2015,Yeo2011} Cooley explores how the perceived roles of bases can be a focus for movements under periods of democratic transitions\index{democracy}.\autocite{Cooley2008} US bases do not become a point of contention in some countries, and the national dialogue tends to ignore their presence as a salient political issue. In other cases, US deployments become a focal point that rallies reformers and leads to the US's departure from the country. For Cooley, the difference is in how political elites can capture and use the benefits of bases domestically and how institutional change alters the distribution of those benefits. Calder sees a similar pattern with basing and national identity and argues that the longevity of US bases ties directly to the perceptions of those bases.\autocite{calder2007} If people see the base as a legacy of colonialism\index{colonialism}, it is more likely to be contested; however, bases that tie into a story of liberation are likely to endure in emerging democracies\index{democracy}. Additionally, for Calder, contact can inflame opposition, and bases in urban areas produce more resentment unless the base is enmeshed in the local culture\index{culture}.  

More recent quantitative work draws on the earlier theoretical work referenced above and combines it with advanced statistical analyses. Kane jumpstarted this body of work when he used Department of Defense\index{Department of Defense} reports to compile a database of annual US troop deployments around the globe.\autocite{Kane2004} Much of the early research studied the economic and security benefits of these deployments. Scholars found several positive economic effects\index{economic effect} associated with the US military presence. Biglaiser and DeRouen find that US deployments correlate with increased foreign direct investment in the host country.\autocite{biglaiserandderouen2007} Their subsequent study examines whether the relationship also holds for international trade\index{economic effect} activity and finds support for that hypothesis as well.\autocite{biglaiserandderouen2009} Kane and Jones find that a history of large, long-term US troop deployments positively correlates with economic and infrastructural growth in that country\index{economic effect}.\autocite{Kane2012,JonesKane2012}  Heo and Ye scrutinize these findings further through more complex statistical methods but also find that troops correlate with foreign direct investment, growth, and trade.\autocite{HeoYe2017}


A natural question about troop deployments is how they affect security in host states. After all, security concerns have been among the primary drivers of the United States' effort to expand its global military footprint. Importantly, this body of research has shown that the effects of US military deployments are often counter-intuitive. Machain and Morgan consider the existence of a free-rider effect from troop deployments and find that countries that host US troops do tend to spend less on their defense when compared to countries that host fewer or zero US troops.\autocite{machainandmorgan2013} However, they also find that larger military deployments correlate with a higher likelihood that the host country engages in militarized disputes with other states. Allen, Flynn, and VanDusky-Allen examine these issues further by splitting the population of host-states into NATO\index{North Atlantic Treaty Organization} allied states, non-NATO allies, and non-allies of the United States.\autocite{allenetal2016} Their research finds that the nature of the broader security relationships between the US and partner states conditions the effect of US military deployments on the host country's security policies. NATO\index{North Atlantic Treaty Organization} allies tend to increase their defense spending as a percentage of GDP as the size of the US troop deployment they host increases. Allen, Flynn, and VanDusky-Allen follow this study with a closer look at how regional deployments condition the effect of direct deployments on defense spending of the host-state.\autocite{allenetal2017} Most notably, among other findings, NATO\index{North Atlantic Treaty Organization} members tend to spend more on their security if \textit{neighboring} states host larger numbers of US troops. In contrast, non-NATO allies are more likely to free-ride by decreasing their defense expenditures as US deployments increase, as Machain and Morgain find.\autocite{machainandmorgan2013} 

A few authors explore other dimensions of security that do not directly deal with defense expenditures. Braithwaite and Kucik go beyond the United States and study how the presence of all foreign deployed troops affects internal stability within the host country.\autocite{braithwaiteandkucik2017} The authors find some interesting causal pathways between troops and government\index{government} duration. They argue that troops are a costly signal about a country's commitment to the security and stability of the host. This support for the host-state and its political institutions encourages rebel groups (latent or manifest) to find alternatives to fighting preferable; the authors find the presence of foreign troops in a country reduces the likelihood of a civil war\index{civil war} onset. Related to the argument of troops signaling support for a state, Bell, Clay, and Martinez Machain find that as ``non-invasion'' US troop deployments abroad provide the United States with influence over the host country; the presence of US troops correlates with lower respect for human rights\index{human rights} in host countries.\autocite{bell2017} Yet, they find that this effect only holds when the host is not strategically important to the US. The more strategically important hosts are less likely to acquiesce to US demands against the leader's self-interest. This finding comports with work from Stravers and El Kurd, who show that US forces in more strategic locations lead to a growth in autocracy\index{autocracy} in host governments\index{government}.\autocite{StraversElKurd2018} The authors show that the US presence provides more resources to repressive regimes and loses leverage over host regimes. The US needs to maintain access to the strategic location. Allen and Flynn look at whether increases in the size of US military deployments correlates with higher levels of crime\index{crime} in the host country.\autocite{allenandflynn2013} Ultimately, they find relatively little evidence that deployments systematically exacerbate criminal activity, though they do find some evidence of increases in property crimes.\index{crime!property} However, this analysis also struggles with data aggregation issues that make such conclusions tentative (a subject we return to later).


\subsection*{(Re)Connecting military deployments and theories of international order}

The second wave of research on military deployments sought to understand better the causes and consequences of deployments as a unique phenomenon worthy of study in their own right, rather than as a proxy for more abstract aspects of international orders or interstate relations. However, this research has also proven to be informative concerning the mechanics of the global security and economic orders that arose following World War II\index{World War II}. Such works have shown ways in which deployments can alter the behavior of host states, and the collection of these types of alterations can, in turn, change the nature of the international order. 

Beginning in the 1940s\index{Cold War}, the United States embarked upon ambitious efforts to promote multilateral engagement in security and economic policy domains. It expanded its global military presence rapidly and created what would become the current global basing network. Likewise, the establishment of the Bretton Woods System\index{Bretton Woods System}---the World Bank\index{World Bank}, the International Monetary Fund\index{International Monetary Fund}, and the Generalized Agreement on Tariffs and Trade (later the World Trade Organization\index{World Trade Organization})---served as an ambitious, liberalizing project designed to promote deeper integration between countries in the global economy. Scholars of hegemonic stability theory\index{hegemonic stability theory}\index{hierarchy} argued that such reshaping was natural to either the status of hegemony or became the result of hegemony if the state chose to become an international leader.\autocite{kindleberger1973,krasner1976}  Other scholars see this vast reshaping of economic ties as a novelty of the unique circumstance of the US situation and leadership in the post-War era\index{hierarchy}. In more contemporary research, Lake and Ikenberry argue that the current global US military presence is fundamental to maintaining the current international economic order.\autocite{Lake2009b,Ikenberry2011} At the very least, we agree that understanding troops and their effects is pivotal to understanding contemporary international relations\index{international relations} and the past 75 years of world politics. 

The more recent quantitative research on overseas military deployments is vital to understanding US foreign policy and US hegemony's\index{hierarchy}\index{hegemonic stability theory} political and economic underpinnings. For the last 70 years, the overseas positioning of military assets and the construction\index{construction} of military facilities have been vital to the projection and maintenance of the United States' political and economic power. However, while research on these subjects has come a long way, it is limited in a few crucial ways that we seek to help resolve. 

First, while statistical analyses have advanced our understanding of how military deployments affect host environments across many outcomes, those analyses tend to use more general cross-national time-series frameworks, relying on highly aggregated data (typically at the country-year level). In some cases, this approach is practical. Still, it can hinder establishing more direct linkages between the US military and outcomes of interest, like economic growth, anti-US protests,\index{protest} crime\index{crime}, or public opinion\index{public opinion}, as ecological inference problems often emerge from highly aggregated data. For example, Allen and Flynn use country-year data on US military deployments to look at variation in reported incidents of crime\index{crime} in the host country---also aggregated at the country-year level.\autocite{allenandflynn2013} In this example, it is possible to find that troop deployment numbers correlate with increases in reported crime\index{crime}. However, it is also possible that the increases in troop deployments geographically concentrate in one area of a country, while increases in crime\index{crime} may occur in a different region. While not impossible, such a pattern would make a causal link between the two trends more tenuous. The more general problem, in other words, is that it can be challenging to draw inferences about the behavior of individuals using data that measures country-level indicators. 

Second, existing studies cannot explore questions concerning the formation of individual attitudes towards the US presence within a country. This limitation is not trivial and has implications for policymaking as well as political science\index{political science} theory. If US military activities, or the behavior of individual service members, diminish public support in the host state, then the United States' ability to maintain a military presence is impaired. For example, domestic activists\index{activists} have long contested the US military presence throughout Asia\index{Asia}.\autocite{Chanlett-Avery2012} Even where US basing and deployments persist, declining public support can lead to increased restrictions or decreases in host-government\index{government} financial transfers that reduce maintenance costs for the US. Alternatively, US military deployments can create economic interests in support of the US presence.\autocite{Holmes2014} As we discussed above, the financial transfers associated with base contracts, routine individual spending, and employment for host-state contractors all conspire to create strong constituencies that benefit from the US deployments and support their continued presence. In short, military deployments have characteristics that plausibly lead to both more favorable \textit{and} more negative views of the US military presence among host-state residents. Further investigation is needed to ascertain the nature of these characteristics and determine their full impact.

The lack of systematic work on the relationship between US military deployments and mass attitudes also poses challenges for developing macro-level theories of international order. Lake argues that the United States established a series of ``contractual'' relationships with other states\index{hierarchy}.\autocite{Lake2013} In such cases, subordinate states cede authority over foreign policymaking to the US in exchange for security. Although these theoretical arguments advance our understanding of the international order, they leave the microfoundations of that order unexplored. The sustainability of these contractual relationships implicitly rests on the consent of the governed. The idea of governments\index{government} ceding foreign policymaking authority to the US in exchange for security guarantees affects domestic political processes. US security guarantees, and any accompanying policy concessions, typically require some level of domestic consent\index{Domain of Competitive Consent}, even if that consent comes in the form of simple acquiescence from the domestic population. The potential for public opinion\index{public opinion} to influence US security cooperation is particularly important given that democracies rank among the largest hosts of overseas US military personnel. Even in non-democracies, public opinion\index{public opinion} may limit how host governments\index{government} can cooperate with the United States. Where public opposition to a foreign military presence increases, the cost to host-state political elites for maintaining these relationships also increases. When this becomes untenable, the United States may see increasing costs, restrictions on operational freedoms, reduced access, or the complete ejection of US military personnel and the closure of US facilities. In the following section, we turn our attention to this issue of public perceptions and mass attitudes towards US military personnel.

\section*{Perceptions of the US Military Abroad}

Hosting a large foreign military deployment brings a wide range of positive and negative externalities that affect local, regional, and national environments within the host country. Construction\index{construction}, maintenance, and operations transform nearly every aspect of the environmental\index{environment}, social, economic, and political landscape of the areas surrounding US deployments. A US presence often requires substantial land to house service members, vehicles, and aircraft. US forces produce waste, and they can generate substantial increases in the volume of traffic activity\index{traffic}. Construction\index{construction} and providing services for US personnel can create economic opportunities for local labor and compete with local businesses for that labor. The presence of US forces can also create opportunities for social and economic advancement that might not be available to all members of the host country's society. More generally (and abstractly), foreign forces can also create outcomes beyond the material, as social and cultural influences blend to create new aesthetic, artistic, cultural, and culinary styles that defy the boundaries and labels of either US or host society alone.

Of course, not all military deployments involve the construction\index{construction} of an American base. In some cases, the local militaries host the US military, is allowed access to local military installations, or regularly deploys to train local forces. For example, in Latin America\index{Latin America} and Africa\index{Africa}, US military personnel often deploy on a smaller and more temporary basis than the deployments we have seen in Europe\index{Europe} and Asia\index{Asia} over the last 75 years. Yet, even in cases where a large physical base is not present, the effects of US deployments can often be profound. Deployed US personnel often spend substantial money on local businesses, and their income creates financial flows into the local economy. Such flows can take the form of normal economic consumption but may also fuel illicit market activity. In the case of large deployments, both labor and businesses become reliant upon the flow of American dollars into their paychecks and stores. Those that economically benefit from the bases pressure local mayors, governors, and other representatives to do what they can to maintain the American presence. Even in the case of smaller deployments, the presence of US forces can bring with it an influx of several million dollars for local firms---a substantial amount in less wealthy countries and communities. The demand for illicit goods (drugs\index{crime!drug-related}, stolen goods\index{crime!theft}, sex work in some jurisdictions) can also create social harms that require policy action, policing, or difficult community adjustments to a new normal. Additionally, service members can and do engage in illicit behavior ranging from minor traffic\index{traffic} offenses to major crimes\index{crime} like murder\index{crime!murder} and sexual assault\index{crime!sexual assault}. 

From the perspective of the host-state population, each deployment is a bundle of positive and negative political, economic, and social stimuli. In theory, it would be feasible to calculate the added benefits and costs of hosting a deployment to arrive at a neat positive or negative dollar amount. Yet, such an effort would ignore the human process of perceiving only parts of these costs and benefits or assigning higher salience to some of them when determining one's disposition to the new social environment. Additionally, most people will only experience some small fraction of the deployment's costs and benefits. While they may learn of a few others through the media, their exposure remains partial. As people experience or learn about multiple negative and positive events, they will choose (consciously or not) which to weigh more heavily in assessing the US military presence. For example, some people will consider the security benefits of a US presence more heavily than local and more personal consequences, whereas others might take the opposite view. Our task in this book is to examine the human experience filter and determine which interactions and experiences provoke negative or positive perceptions of US military deployments.

Additionally, we seek to understand if those direct experiences and assessments of the US military transfer to other related actors: the US government\index{government} and the US people. This project aims to establish the microfoundations of popular support for a state's foreign policy. It is thus pivotal to our task to understand individuals' perceptions of and responses to a US military presence.



\subsection*{Perceptions, Cycles, and Action}


According to our previous work and sociological research on contact theory, perceptions generated by American service members can create a vicious cycle. Criminal\index{crime} and anti-social behavior by service members leads to increased negative perceptions of them. If enough of those negative interactions occur, or if a single negative interaction causes enough harm to the community, this can lead locals to mobilize\index{mobilization} against the American military presence.  Mobilization\index{mobilization} can take the form of protest\index{protest} or political action against the military presence. Not every negative situation will lead to mobilization\index{mobilization}; however, political entrepreneurs, nativist, nationalist, or isolationist political parties, and anti-US activists\index{activists} can capitalize on increasing negative sentiment to build a coalition that threatens US basing efforts. When discontent becomes politically weaponized, base commanders or other high-ranking officials in charge of the deployment will have an incentive to quell the growing discontent with the military presence as a way to preserve long-term operations. This need, inherently, makes the basing process more expensive and likely leads to limitations in the base's operations. 

A typical response is to impose restrictions on troop interactions with the community. The intent is often to limit the opportunity for military personnel to become involved in any actions that might further damage the United States' image among host-country residents. While this inclination seems logical, it may also backfire. Our research demonstrates that consistent positive social interactions may generate positive views of and support for the mission of the US military.\autocite{Allen2020} By limiting contact between service members and the community, the US military misses out on the benefits that benign, regular interactions with service members can bring; benefits that could offset the harm caused by negative interactions by showing them to be the exception rather than the rule. This approach results in lingering grievances against previous wrongs and the schism between the military deployment and the local community continuing to grow.  

The post-9/11 era of basing has only amplified the protection of military installations against security threats. As bases have become fortified mini-cities, many service members can meet most, if not all, of their needs on base without having to venture into the community. Bases often prohibit overseas service members from wearing their uniforms off-duty and off-base. In some cases, military personnel attached to embassies often work in civilian clothes instead of military uniforms. The logic behind these decisions is reasonable. In part, such policies make individual service members less likely to target kidnapping\index{crime!kidnapping}, terrorist attacks\index{terrorism}, or other crimes\index{crime}. In other cases, these policies lower the risks of bad publicity---even something as trivial as a minor traffic accident\index{traffic} can have substantially larger implications if a uniformed service member is involved.  Yet, the less visible presence of the human element of the military, combined with fortified and publicly inaccessible military installations, has spawned distrust in neighboring communities. 

Across interviews\index{interview} in England\index{England}, Germany,\index{Germany} and Panama\index{Panama}, subjects shared a perception that US bases were shrouded in secrecy and were likely conducting operations against the popular consent of the locals. A journalist\index{journalists} in Panama\index{Panama} expressed serious concerns that the US had humanitarian\index{humanitarian} programs (such as Beyond the Horizon and New Horizon) were to spy on Latin America\index{Latin America}.\footnote{Beyond the Horizon\index{Beyond the Horizon} and New Horizons\index{New Horizons} are annual humanitarian\index{humanitarian} and civic-assistance missions that the United States Southern Command coordinates along with partner nations in Latin America. These exercises routinely involve US Army\index{Army, US} and Air Force\index{Air Force, US} National Guard\index{National Guard, US} units deploying on a short-term basis to work with their counterparts in various militaries throughout Latin America and sometimes also involve Canadian military units.}  He suggested that the US disguises the programs as humanitarian\index{humanitarian} help but that they are still a military presence, which is cause for concern. \footnote{The US has previously used the cover of humanitarian\index{humanitarian} or medical operations to gain intelligence on targets before, such as in the search for Osama bin Laden in Pakistan.} He further noted that US embassy personnel are not accessible or open with the press.  He told us that they provide information but do not hold press conferences. The information the embassy provides is superficial, which journalists\index{journalists} would be able to observe on their own.\autocite{journ20180713}

%The work that they do correspond to local government\index{government}s. Recently there was a leak that the US embassy had requested that military members be allowed to deploy with their weapons. Expressed that U
%\autocite{amb20180713}
%There was a common belief that the military was actively hiding some agenda or practice that would harm its reputation if the general public found out about it. A subject in Panama expressed that humanitarian efforts in the country were secretly programs to spy on the Panamanian population. 

This suspicion of US military deployments is both a cause and an effect of the cyclical dynamic we argue exists in post-9/11 deployments. It is a cause because if locals view the US military as being present for nefarious reasons, then American authorities have an incentive to keep the military inconspicuous, both for its security and not further alarm the locals. Yet, it is also an effect because it also limits American forces' desirable, support-building behaviors. For example, an Embassy Public Affairs Officer in Panama\index{Panama} noted that the attribution of humanitarian\index{humanitarian}, social service provision to the US military builds positive relationships with local populations.\autocite{embone20180712} Thus, socially insulating service members also perpetuates the vicious cycle. While seemingly meeting the demands of anti-base activists\index{activists} by decreasing the social harms of troops abroad, increased separation from society may build the public case against the military as people become increasingly wary of seemingly covert operations.

%Beyond perceptions, we also look at a range of other outcomes that serve as behavioral manifestations of mass attitudes towards US service personnel. In Chapter 5, we discuss crime, and in chapter 6, we examine protest behavior. Each chapter collected data for reported events of servicemember crime and protests against military bases globally. Each chapter employs our surveys to answer questions about criminal victimization (and its effect on perception) and whether people self-report involvement in protests against US bases. The additional use of event data allows us to compare what gets reported by the news media (and the police) versus what people self-report to surveys. We expect the news to under-report for both protests and crime, as not every crime nor protest is worth covering due to competing stories. In particular, small gatherings or minor offenses are less likely to be reported on in the media. Additionally, victims under-report crime to authorities due to institutional and social disincentives. As such, we can use our survey data to compare the mismatch between what becomes national news versus what people experience and are willing to share on a survey. 
%NOTE: Here's something good from subject 6 in Panama that we can quote here: 10. Concerns that the US has these programs in place to spy on Latin America.  They disguise programs as humanitarian help, but they are still a military presence. The work that they do correspond to local government\index{government}s. Recently there was a leak that the US embassy had requested that military members be allowed to deploy with their weapons. Expressed that US embassy personnel are not accessible or open with the press.  They provide information but do not hold press conferences. The information they provide is superficial, which you would be able to observe on your own.
%another graphic on DV
%explain illustration, 1-2 para

%The main question we asked our survey respondents was: ``In general, what is your opinion of the presence of American military forces in [your country]?'' While we seek to explain several different variables, this one question serves a vital purpose in our research in several chapters. Respondents answer on a five-point scale from ``very positive'' to ``very negative,'' or opt out of the question.\footnote{We cover this in more detail in Chapter 3 where we focus on the research design of our quantitative tests.} Specifically, we want to know how people view the US military and whether their other experiences correlate with positive or negative views. This question's aim is to distance respondents from their views about the United States military in general, as people may have wildly different views about the military given its humanitarian missions, engagement in wars in the Middle East and Central Asia, use of drone warfare, etc. The question instead localizes their views to the specific context of their local deployment. That is, people may be opposed to or in favor of the United States military as a global force but have the opposite view about the utility of the presence within their own country. For example, a German landlord who rents primarily to soldiers at above market rates may enjoy the presence of troops in Wiesbaden, but also be opposed to the use of drones for targeted killings in Central Asia. A Scottish citizen may favor the US military helping with vaccines and healthcare in Peru but be opposed to the base so close to their home given the seemingly weekly brawls that occur in nearby pubs. To hone in on this, and to try to get people's assessments of their own exposure and of the net benefits and harm they experience, we asked about how they view the presence of forces in their country.  

%footnote might be better for a later chapter.Agreed, greyed it out for now. CM
%\footnote{We did have some concern about this translating improperly across contexts as some respondents may see ``American'' as not a United States specific person but instead a resident of North or South America; after discussing this with language and regional experts, we were assured that people would contextualize the question as we intended. That is, they would specifically think the survey is asking about United States military forces. For example, in Spanish the adjective ``estadounidense'' refers unambiguously to someone from the United States, while ``americano'' could refer to the US or to the broader American continent. The broader context of the survey, and references to the United States, though, would provide the necessary disambiguation.}

%\caption\ref{figviewcountries}{Views on the United States military presence in fourteen countries across three different years, 2018-2020}
%Add figure on DV, as preview

%Figure \ref{figviewcountries} shows the variation in how people respond to the question, both across countries and across the three periods of time we survey. Generally speaking, we see high support for the US military presence in Kuwait, Poland, the Philippines, and South Korea \index{South Korea}. We see lower support for the presence of forces in Western European countries like Belgium Germany, the Netherlands, and Spain. Turkey, as well, has middling levels of support with high levels of negative views of the troops. We can compare this to similar questions we ask about how people view the government\index{government} and the people of the United States in Figures \ref{figviewgovernment\index{government}} and \ref{figviewpeople}.

%explain figure, 1 para

%Views of the American government\index{government} are far more negative than those of the troops. As we discussed at the beginning of this section, people may either associate or disassociate the military presence within their own country, and their individual interactions with United States servicemembers, with the policy, practices, and views of the US government\index{government} as a whole. More broadly, we are interested in whether interactions with US servicemembers transfer to the US government\index{government} and US people, but this preliminary view shows a stark difference. In contrast, people in countries that host the US military generally like, or at least do not actively hate, the American people. This likewise varies across countries and that variation, for both people and government\index{government}, seems to mirror the variation that occurs with troops.



\section*{From Unipolarity to Competitive Consent}

Understanding how interactions produce popular views of deployments underscores the basing and deployment process. Fundamentally, this process will only become more important as people demand an increasing voice in national security\index{national security} processes through democratic (or other) means. When maintaining a deployment abroad, the United States needs the consent of multiple populations\index{Domain of Competitive Consent}: the citizens in the United States and the government\index{government} or citizens of the host state. Maintaining a military presence overseas is a costly endeavor requiring massive logistic support to keep military installations properly supplied and support the service members stationed abroad. Beyond the direct cost, there are costs in leasing the property from another government\index{government} for military bases. As a population becomes increasingly unhappy with a military base, their demands may pressure their government\index{government} to extract more concessions from the United States in exchange for basing rights. Bases contain several fixed costs in their construction\index{construction} (even when adapted from existing facilities). Losing access to a built or modified base means bringing the troops home and losing that access point in the region or building a new presence in a neighboring state. As such, the United States has a few criteria that may appear ideal in selecting base and deployment sites. 

First, a country that is not responsive to the demands of its people decreases the likelihood of the government\index{government} demanding more in return for hosting a base or renewing the lease for that base. Democratic representation may make non-democratic regimes more attractive as base sites.\autocite{Allen2011} In such cases, authoritarian leaders may be able to absorb the benefits of hosting US bases and personnel while externalizing the costs onto various (and often disfavored) segments of the population.

Second, a government\index{government} that is overwhelmingly in favor of the base may be willing to pay for it. The governments\index{government} of both Japan \index{Japan} and South Korea \index{South Korea} contribute a large share of the financial burden of hosting US troops in their country. Finally, it may be ideal when there are no people to experience the negative externalities of basing. This situation is rare, but it occurred when the United Kingdom\index{United Kingdom} removed the native population of Diego Garcia\index{Diego Garcia}; the only inhabitants of the Indian Ocean\index{Indian Ocean} island are US and British\index{British} military service members.\autocite{Johnson2004,Vine2004,vinejeffery2009} 

The third condition is an anomaly and concerns our research less directly, while the government\index{government} types and the demand for a US military presence both play a role in why this research is essential. In autocratic\index{autocracy} states, when institutions become more democratic\index{democracy}, and people have more say in their government\index{government}, individual perceptions will matter more. In turn, the cost of maintaining a US base in that territory will increase if the population views the base negatively. Researchers have pointed to the role of democratic transitions in weakening the United States presence abroad in places like the Philippines or Spain\index{Spain}.\autocite{calder2007,Cooley2008} Many of these ideas rest more at the macro-level of how institutional change affects the viability of long-term bases. Beyond institutional change, our research facilitates changes in countries due to widespread support changes, despite institutions remaining static. Thus, even in countries that contribute substantial amounts to burden-sharing efforts with the United States, those contributions will dwindle if popular opposition to hosting US bases increases in the host state. Notably, support for basing during the Cold War\index{Cold War} did fluctuate with scandals in places like the United Kingdom\index{United Kingdom}, Germany,\index{Germany} Greece\index{Greece}, and Turkey\index{Turkey} and general opposition to negative externalities in Okinawa\index{Japan!Okinawa}, Bermuda\index{Bermuda}, and Trinidad\index{Trinidad}, but the context of the Cold War\index{Cold War} allowed the security concerns to trump domestic concerns.\autocite{cottrellmoorer1977,high2008} The United States is entering a new phase of basing where not only must it placate foreign populations to maintain its basing network, but it may be actively competing for popular support with the rise of a new potential basing power: the PRC\index{China}. 

The Chinese overseas base in Djibouti\index{Djibouti} is not an isolated event but rather part of an expanding campaign by the PRC\index{China} to increase its regional influence enough to become a global superpower. A list of recent developments that show the current extent of China\index{China}'s basing campaign:

\begin{quote}
	In 2017, the same year China\index{China} opened its facility in Djibouti, some accounts indicate that its negotiations for a ninety-nine-year lease on Sri Lanka's Hambantota port also included questions related to military access. In 2016 and 2017, a Chinese firm acquired a fifty-year lease over the island of Feydhoo Finolhu in the Maldives\index{Maldives}, paying only \$4 million for it, and then began land reclamation. Around this same time, there is evidence China\index{China} established an outpost in Tajikistan too. In 2018, a Chinese firm sought to fund and construct three airports in Greenland\index{Greenland}, a long-standing focus of its Arctic ambitions that came after an attempt to purchase an abandoned former US military base there. In 2019, China\index{China} negotiated a lease on a Cambodian naval facility and began construction\index{construction} on ports and airfields that could accommodate Chinese military vessels, and though these projects were nominally civilian, there were indications of discussions of military access between the two governments\index{government}. A Chinese conglomerate leased an entire island in the Solomon Islands that same year, though the decision was temporarily reversed. Admittedly, some of the details in these cases are difficult to corroborate, but the balance of evidence—particularly when juxtaposed with Chinese statements and Beijing's willingness to break its pledge never to station forces overseas—suggests a growing interest in global facilities.\autocite[p. 295]{Doshi2021}
\end{quote}


While some prospects for China\index{China} are unlikely to be suitable basing locations (which is equally true for the United States), a subset of countries could feasibly host either American or Chinese soldiers. Suppose American military superiority wanes (an almost certain prospect) and Chinese military strength increases enough to make China\index{China} a global competitor for the provision of security (something that is already happening). In that case, the United States will no longer serve as a default provider of defense. The Department of Defense\index{Department of Defense}'s 2020 report on ``Military and Security Developments Involving the People's Republic of China\index{China}'' lists the following countries as likely targets for expanding the People's Liberation Army\index{People's Liberation Army}'s basing: ``Myanmar\index{Myanmar}, Thailand\index{Thailand}, Singapore\index{Singapore}, Indonesia\index{Indonesia}, Pakistan\index{Pakistan}, Sri Lanka\index{Sri Lanka}, United Arab Emirates\index{United Arab Emirates}, Kenya\index{Kenya}, Seychelles\index{Seychelles}, Tanzania\index{Tanzania}, Angola\index{Angola}, and Tajikistan\index{Tajikistan}.''\autocite[p. x]{OSD2020} These countries represent a mix of polity types (both autocracies and democracies) and economic systems. Ideology does not guarantee who bases where, as even capitalist democracies are potential PRC host states.

As we point out throughout this book, public diplomacy\index{public diplomacy}, soft power\index{soft power}, and hard power intertwine. Each one can provide opportunities to expand the others. As China\index{China} increases its aid and economic investment to countries in Asia\index{Asia}, Africa\index{Africa}, and Latin America\index{Latin America}, subsequent expansion in military relationships makes likely a future where traditional allies may choose between hosting American or Chinese forces. Given the shaping nature of domestic politics and their influence on international relations\index{international relations}, such decisions will not be purely a top-down elite decision. Still, they will increasingly include domestic constituents in the decision-making process. The contemporary period has decreased avenues for the US to have total control over its bases and troops in Status of Forces Agreements\index{Status of Forces Agreements}. The majority of overseas US bases are in democratic\index{democracy} polities. When choosing base sites, China\index{China} will have some non-democratic options, some of which the United States may not have the capacity to pursue. Notably, anti-China protests\index{protest} in Vietnam\index{Vietnam} did not affect the country's foreign policy towards China\index{China} despite the enduring rivalry between the two countries; Vietnam\index{Vietnam} is willing to use the protests\index{protest} for leverage against China\index{China}, but was not pushed towards a more hostile position despite violent riots.\autocite{Hoang2019} If China\index{China} hopes to become a power outside of Asia\index{Asia} like its current basing effort in Djibouti suggests, it will have to consider host-states that have domestic input on foreign policy decisions. 

The perceptions of the United States, China\index{China}, and the soldiers of those countries are vital to maintaining deployments and allowing for new opportunities\index{Domain of Competitive Consent}. Research shows that US troops can create support for their deployments under various conditions.  The US and China\index{China} are entering a new phase where soft power\index{soft power} and public diplomacy\index{public diplomacy} will be vital in ensuring the global feasibility of hard power projection.\autocite{Trunkos2020} For the United States, its service members' daily and routine behavior are the micro-foundations for building support for a US presence within host-states; Chinese forces are likely to occupy a similar space. Consequently, US forces are not only engaged in winning the ``hearts and minds'' of people in combat zones but are actively engaged in such tasks during peacetime deployments overseas. Incorporating the strategic space of service member public diplomacy\index{public diplomacy} in bilateral competition is both an advance in International Relations\index{international relations} theory and important to the long-term strategic goals of the United States. The competitive domain of consent\index{Domain of Competitive Consent} is rapidly evolving, and the United States has not previously played a global ground game to secure peacetime deployments. Even during the Cold War\index{Cold War}, which saw spots of state-to-state competitive consent, the competition was not over people but over governments\index{government} and resources \cite{Harkavy1982,Harkavy1989,calder2007}. For all players involved, this will be a new game for which the rules are currently being written. 




\section*{Assessing Mass Attitudes Towards US Military Deployments}


We adopt a mixed-methods approach to understand better the foundations of people's beliefs about the US military in their country. The core of our research is a survey\index{survey} with just over fifty questions asking respondents basic demographic questions, followed by a series of questions related to defense, economics, foreign affairs, and the US presence. We deployed these surveys annually in 2018, 2019, and 2020 to 14 countries: Australia\index{Australia}, Belgium\index{Belgium}, Germany,\index{Germany} Italy\index{Italy}, Japan,\index{Japan} Kuwait\index{Kuwait}, Netherlands\index{Netherlands}, Philippines\index{Philippines}, Poland\index{Poland}, Portugal\index{Portugal}, South Korea,\index{South Korea} Spain\index{Spain}, Turkey\index{Turkey}, and the United Kingdom\index{United Kingdom}. We surveyed approximately 1,000 people in each country each year, giving us an annual sample of over 14,000 respondents and over 42,000 total respondents for all three years. We distributed the online survey to nationally representative samples across age, gender, and income using two different survey firms. 

\input{../Tables/Chapter-Intro/table-country-summary.tex}

To evaluate patterns in mass attitudes towards US military deployments, we estimate a series of statistical models that allow us to explore the relationships between various respondent characteristics and behaviors and their attitudes towards various US actors, policies, and actions. We provide more detail in chapter \ref{cha:meth} about our data collection efforts, variable selection, and estimation strategy. When available, we also use additional supplementary data sets to analyze a core concept differently. We collected novel events data about crimes\index{crime} and protests\index{protest} for chapters \ref{cha:crimes} and \ref{cha:protest}.

In assessing the relationship between our variables, we have adopted a Bayesian\index{Bayesian} approach to analyze our data. While computationally demanding and more complicated than traditional data analysis methods, the flexibility it allows in analyzing cross-cutting effects (for example, individuals within a province that's within a country) in a multilevel framework. Additionally, the models allow us to talk about certainty and uncertainty in logically consistent ways and communicate that graphically. Our modeling approach allows us to do some novel techniques as well. For example, in chapter \ref{cha:meth}, we can compare three years of surveys\index{survey} to an earlier paper we wrote that only had one year's worth of data. In chapter \ref{cha:protest}\index{protest}, we develop a causal model (as opposed to a purely correlational model) in assessing the influence of troop presence on the likelihood of a protest. Throughout this book, we present figures to tell the story of our data and facilitate easier interpretation of our results. Our longer output, including tables, additional figures, and relevant commands, resides partly in the appendix of this book and the remainder in a longer online appendix.
%flynn check this paragraph ^

In addition to collecting new cross-national survey and events data, we sought out additional qualitative information about individuals' perceptions of the US military and how its personnel interface with the communities in which they're embedded, as well as how they affect broader national politics, and how citizens of the host country perceive them. We traveled to six countries between 2018 and 2021 to talk to many individuals about their perspectives on troops and host-state civilian--US military relations. We conducted interviews\index{interview} in Panama\index{Panama}, Peru\index{Peru}, Germany,\index{Germany} England\index{England}, Japan (remotely),\index{Japan} and South Korea (remotely).\footnote{The COVID-19 pandemic ran through our two targetted interview periods of 2020 and 2021 and prevented us from traveling. However, we conducted interviews remotely to remedy our ability to travel.}\index{South Korea}\index{interview} We adopted a ``snowball'' strategy for recruiting interview subjects where one contact generally connected us with additional potential subjects, and we branched out throughout our visit. Through this process, we have talked to base commanders, newspaper journalists\index{journalists}, a mayor, military officers, and enlisted personnel, former ambassadors, diplomats\index{diplomats}, a former president, members of parliament, leaders and members of activist organizations, local elected leaders, and public relations officers. Importantly, we use the information obtained in these interviews\index{interview} to understand potential theoretical mechanisms that might determine individuals' views of US service personnel. This enables us to better understand some of the basic social, political, and economic issues affecting base-community relations in the states that we visited. 

\section*{Outline of the Book}
Chapter \ref{cha:theory} presents the main theoretical argument that underlies the whole of the book. We preview that argument in this chapter but explain it fully in Chapter \ref{cha:theory}. The theoretical argument centers on the microfoundations of hierarchy\index{hierarchy} in the international system. The liberal world order is maintained via the consent of states that make the policy concession of accepting US troops. Existing work studies hierarchy\index{hierarchy} as an agreement between countries. However, these studies neglect the role played by the mass publics. We thus establish the microfoundations of consent to American military deployments by focusing on the factors that influence individual views on foreign (in this case, the US) military presences. We do this by drawing from political psychology\index{psychology}, specifically, contact theory\index{contact theory}, and argue that direct and network contact with the US military, as well as direct and network economic benefits drawn from the US military presence, are more likely to lead to positive perceptions of the US military, and carry over to perceptions of the US government\index{government} and people.\autocite{Pettigrew1998,Woolcock2000,Putnam2001,Huckfeldt2001} This chapter links the individual-level theoretical argument (contact and economic benefits lead to more positive views of the US military) to the system-level theory (on hierarchy\index{hierarchy}) by explaining how it is that perceptions of the US military affect mobilization\index{mobilization} by the population (for example, protesting\index{protest} or voting behavior) and, thus, influence the preferences and actions of leaders in the host country. As such, it will connect the microfoundations with the broad patterns of behavior we have seen among American allies when it comes to US military basing.


Chapter \ref{cha:meth} begins by exploring the demographic ``profiles'' of the people who feel positively and negatively towards the US and describe what the ``typical'' supporter (and opponent) of a US military presence looks like in terms of factors such as age, education level, income, etc., along with reasons for why these factors impact views of US military forces. This chapter also introduces the methodology for the rest of the book. This introduction includes a description of the original surveys fielded and the qualitative fieldwork interviews\index{interview} conducted by the authors. Finally, this chapter presents results from analysis (using the data from the three years of surveys) that evaluates our main theoretical argument on direct and network contact and economic benefits derived from the US military presence correlate with more positive perceptions of the US military, government\index{government}, and population. Though we show a large positive effect of contact on perceptions, we also see that some individuals are more likely to have strong negative perceptions of the US military (or other US actors) if they have interacted with the US military (likely because of these interactions have been negative). 

Chapter \ref{cha:crimes} explores a focal point of dissatisfaction against the United States military: The crimes\index{crime} that US military personnel commit while stationed abroad. Hosting thousands of Americans is likely to come with some measure of crime,\index{crime} regardless of how well-regulated those forces are. Those crimes\index{crime} range from minor to serious, and we examine how prominent examples of criminal activity by US service members affect attitudes. Using new data on media reports of crimes,\index{crime} we use geospatial distance to analyze how reported crimes\index{crime} influence perspectives at the regional and national levels. This demonstrates exactly how negative interactions can alter narratives about US military forces and change attitudes in specific geographic locations. We then use both the statistical evidence and case study material to outline how these types of negative encounters can be minimized, along with how the US military can mitigate the fallout to the general basing relationship. 


We explore these negative interactions in more detail in Chapters \ref{cha:min}-\ref{cha:protest}.\index{protest} Most survey\index{survey} data and elite interviews\index{interview} tend to reflect the views of ethnic majorities. Yet, ethnic minorities\index{minority} within the host country often have different, more strained relationships with their national governments\index{government}--the national governments that have signed onto hosting American forces. In Chapter \ref{cha:min}, we argue that long-term peacetime deployments create direct and indirect paths toward minority discontent with the United States. Directly, Status of Forces Agreements\index{Status of Forces Agreements} between host states and the United States will place some of the basing burdens on minority groups not well represented by host-state governments. Indirectly, the very act of basing sends credible signals about US commitment to the host state government\index{government} and empowers it to maintain status quo policies towards disenfranchised groups. These combined forces compel minority groups to see the United States less positively than the dominant group.



Chapter \ref{cha:protest}\index{protest} focuses on the determinants of anti-military base protests\index{protest} abroad. Using newly collected events data from 1990 to 2018 on protest\index{protest} events around the world, we explain the circumstances that make anti-basing and anti-US protests\index{protest} more likely to occur. We argue that larger US military presences will lead to more anti-base demonstrations and more broad anti-US protests\index{protest} in host countries. We also use our survey data to understand the factors that make individuals more likely to protest\index{protest} the US military. We estimate a series of predictive models of protest\index{protest} participation. While predicting protest\index{protest} participation is difficult, we find that the best predictors are not demographic factors but rather the type of experiences individuals have when interacting with US military personnel. This finding carries important policy implications as it shows that the factors that can be influenced by policy are the most effective at creating positive civil-military relations between the US military and host country communities. We supplement this chapter's statistical evidence with material from first-hand interviews\index{interview} with anti-base activists\index{activists}, as well as US and host-country government\index{government} officials. 

Finally, in Chapter \ref{cha:conclusion} we conclude by discussing the policy implications of our argument and empirical findings for US basing and deployment in the future. We note how negative interactions between locals and US military members have led the US military to become more isolated from surrounding communities to avoid negative interactions. We argue that in these cases, the opportunity for positive interactions can be lost, which would produce a negative shift in perceptions of the US military in the host country over time. We also discuss the implications of our theory for the case of China\index{China}. We discussed China\index{China}'s expanding economic and military influence during all of our qualitative interviews\index{interview}, and the surveys also gathered quantitative public opinion\index{public opinion} on the issue of China\index{China} across all 14 countries. Thus, we can compare these perceptions to those of the US in this chapter and conclude with implications for US interactions with China\index{China} through the rise of great power basing and, perhaps, competition.
